% From mitthesis package
% Version: 1.01, 2023/06/19
% Documentation: https://ctan.org/pkg/mitthesis
%
% The abstract environment creates all the required headers and footnote. 
% You only need to add the text of the abstract itself.
%
% Approximately 500 words or less; try not to use formulas or special characters
% If you don't want an initial indentation, do \noindent at the start of the abstract

La evapotranspiración es uno de los principales componentes del ciclo
hidrológico y de vital importancia en la planeación y operación de zonas de riego,
pues de ella dependen en gran medida los cálculos para conocer las necesidades
hídricas de los cultivos para evitar la sub o sobreestimación de la lámina de riego
aplicada, sin embargo su estudio resulta complicado pues la medición depende
de dos procesos separados, evaporación y transpiración los cuales varían
espacial y temporalmente, por lo que existen métodos para la estimación de ésta
con ayuda de información meteorológica.
Con el objetivo de determinar la evapotranspiración de referencia (ETo) con
Machine Learning en la estación meteorológica Chapingo, México, se presentan
cuatro modelos para la predicción de la ETo, mediante machine learning usando
redes neuronales artificiales para el proceso de entrenamiento. Para entrenar el
modelo, se utilizaron 5119 datos diarios de la estación meteorológica automática
Chapingo, con los que se calcularon la ETo usando el método de FAO PenmanMonteith.
Se realizó un diagrama de correlaciones con el que se identificaron las variables
con mayor impacto en el cálculo de la ETo, sobresaliendo la radiación solar,
posteriormente la temperatura máxima, la humedad relativa, así como la
humedad relativa mínima y máxima. Con esta información se eligieron la
combinación de variables que sirvieron como datos de entrada a cada uno de los
modelos a entrenar.
En cada uno de los modelos se encontraron los parámetros de la red neuronal
que optimizaron el cálculo, tales como arquitectura de la red, capas ocultas y
número de neuronas en cada capa, así como el número de iteraciones, learning
rate y funciones de activación.
En el modelo 4, usando únicamente la radiación solar, se obtuvo un muy buen
ajuste del modelo con una R2 de 0.92, y un RSME de 0.0119, en los modelos 1 y
2, usando también la temperatura máxima, además la humedad relativa máxima
y mínima se mejoró en poca medida el ajuste del modelo, obteniendo una R2 de
0.93 en ambos casos y un RSME de 0.0082 y 0.0074 respectivamente.
Finalmente, el modelo 3, que no consideró la radiación solar no se ajustó
correctamente obteniendo una R2 de 0.66 y un RSME de 0.1946.
Palabras clave: Inteligencia Artificial, FAO Penman-Monteith, Redes Neuronales
Artificiales, Modelación, Agrometeorología.

\footnote{Text from Holman (1876): \doi{10.2307/25138434}.}  
