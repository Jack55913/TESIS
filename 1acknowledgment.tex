

\chapter*{AGRADECIMIENTOS}
\addcontentsline{toc}{chapter}{AGRADECIMIENTOS}
\begin{center}
    \texttt{``La educación agrícola es la base para una nación fuerte y autosuficiente''} - Marte R. Gómez, padre de la Universidad Autónoma Chapingo, ingeniero hidráulico hoy de irrigación.
\end{center}

Hago especial reconocimiento por las enseñanzas y valores que adquirí gracias a la \textbf{Universidad Autónoma Chapingo}: por ofrecerme la beca institucional, por la beca PROFONI referente a la investigación; por la beca SUBES y Benito Juarez del gobierno de México; al Comedor Central y Unidad Médica por su servicio de primera calidad; en especial estoy agradecido a causa de toda la inversión por concepto de:
\begin{itemize}
    \item Congreso Internacional en Santiago Chile, \item Intercambio Académico Internacional que fue en la Universidad de Agricultura de Tokio, 
    \item por la estancia pre-profesional en EcosueloLab Chile, 
    \item Viajes de estudio en México
    \item y por sus numerosos vínculos con instituciones prestigiosas, especialmente con el Colegio de Posgraduados de donde surgió este trabajo.
\end{itemize}

Hago un distinguido reconocimiento a las y los maestros que me han enseñaron desde el kinder hasta la universidad, todos ellos merecen nuestro profundo respeto, admiración y gratitud. Además es honorable el trabajo físico y arduo que realiza el personal administrativo o plantilla de trabajadores por mantener en funcionamiento las instalaciones.

Finalmente al \textbf{Proyecto Miyotl}, una app para preservar, difundir y enseñar las lenguas mexicanas para los pueblos indígenas.
\newpage

\chapter*{DEDICATORIA}
\addcontentsline{toc}{chapter}{AGRADECIMIENTOS}

A mi mamá \texttt{María Carolina Herrera Díaz} y a mi papá Agustín Álvarez Bautista, cuyos sacrificios y amor incondicional me han dado la fortaleza para alcanzar mis metas. Ustedes me enseñaron que la educación es el legado más valioso y que el esfuerzo constante siempre rinde frutos. Cada paso que doy es un reflejo de su dedicación y valores inculcados. 

A mis abuelos Laura Díaz Cruz - Mamá Aya, Mario Herrera Munguía - Papá Gogo$^\dag$, Luisa y Agustín, guardianes de la sabiduría y el cariño eterno. Aunque algunos ya no estén físicamente, sus enseñanzas y amor permanecen vivos en mi corazón. Sus historias y consejos me han guiado en los momentos más difíciles, dándome el coraje para persistir y superar obstáculos.

A mis hermanos Paulo Elías Fernández Herrera, Alan Yareth Álvarez Zarco y Aranza Ailín Álvarez Zarco, incondicionales de aventuras y desafíos. Gracias por ser mi apoyo en los días grises y mi celebración en los días de triunfo. Su confianza en mí ha sido una fuente de motivación constante.

A mis maestros Humberto López Chimil$^\dag$, Fernando Chávez León$^\dag$; a mis mentores Luis Tonatiuh Castellanos Serrano, la Dra. Teresa González Martínez, que con su sabiduría y paciencia han encendido en mí la llama del conocimiento. Sus enseñanzas han trascendido las aulas y han dejado una huella imborrable en mi formación personal y profesional. Gracias por creer en mi potencial y por inspirarme a ser mejor cada día.

A la Universidad Autónoma Chapingo y al Departamento de Irrigación que me llevaron tan lejos como a Sudamérica, Asia y a lo largo y ancho del mejor país del mundo: \textbf{México}.

Finalmente, dedico esta tesis a Dios, porque el me dio la voluntad de perseverar a pesar de las adversidades, por cada noche en vela y cada instante de duda superado. Este logro es el resultado de años de esfuerzo y dedicación, me recuerda que los sueños se alcanzan con determinación y pasión. Gracias a todos los que han sido parte de este viaje. Esta tesis es una manifestación de tu amor, apoyo y fe en mí.
