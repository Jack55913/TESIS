%% biography.tex
%% This section is optional

% From mitthesis package
% Version: 1.01, 2023/10/16
% Documentation: https://ctan.org/pkg/mitthesis

\chapter*{DATOS BIOGRÁFICOS}
\addcontentsline{toc}{chapter}{DATOS BIOGRÁFICOS}

\textbf{Luis Emilio Álvarez Herrera} nació el 7 de junio de 2002 en Texcoco de Mora, estado de México. Ingresó a la Universidad Autónoma Chapingo en 2017 y se incorporó al Departamento de Irrigación en 2020.

Durante su formación académica, realizó un intercambio en la Universidad de Agricultura de Tokio, Japón (2023) y efectuó sus prácticas profesionales en Ecosuelo Lab, Santiago Chile (2025).

Fue miembro del Programa de Formación de Nuevos Investigadores (PROFONI) de 2021 a 2025. 

Es autor de tres libros: \textit{Fundamentos de la Ingeniería en Irrigación} (ocho volúmenes), \textit{Matemáticas del Cubo Rubik} y \textit{Huertos Agroecológicos}.

Entre 2017 y 2022, se dedicó a la docencia de matemáticas en el Colegio Euro Texcoco, como asistente del maestro Fernando Chávez León. 

Se desempeñó como consejero departamental de la Preparatoria Agrícola en 2018; en 2020, nuevamente fue consejero, ahora en el Departamento de Irrigación; Y como presidente del Club de Ciencias Netzahualpilli de 2019 a 2022. 

Actualmente, es CEO del proyecto \textit{Miyotl: Aprende una lengua indígena}, CTO de \textit{Tláloc App: Ciencia ciudadana para el monitoreo de lluvia en el Monte Tláloc} (COLPOS) y CEO de la \textit{Olimpiada Mexicana de Agronomía}.



