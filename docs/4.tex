\chapter{MATERIALES Y MÉTODOS}
%  metodología estadística usada para probar las hipótesis

% diseño experimental, modelo estadístico y procedimiento de análisis


% IR ORDENANDO LOS MÉTODOS EN FUNCIÓN DE LOS OBJETIVOS. SIEMPRE LOS MÉTODOS SUELEN INICIAR CON LA DESCRIPCIÓN DEL SITIO O POBLACIÓN DE ESTUDIO. EL SITIO DE ESTUDIO ES EL MONTE TLÁLOC Y LA POBLACIÓN DE ESTUDIO SON  LAS PERSONAS QUÉ PARTICIPAN EN EL MONITOREO.

% MÉTODOS PARA CUBRIR TODOS LOS OBJETIVOS
% VER SI LA ENCUESTA ENTRA CÓMO OBJETIVO ADICIONAL O DENTRO DE EVALUACIÓN DEL NIVEL TEC.

\section{Materiales}
\subsection{Instrumentación y equipos}
\begin{itemize}
    \item Se utilizaron pluviómetros calibrados con botellas graduadas hechas de PET, instaladas en bases de metal y madera ubicadas en los sitios de monitoreo.

    \item Se utilizaron dispositivos móviles con cualquier tipo de sistema operativo, para realizar pruebas de la aplicación Tláloc App.
\end{itemize}

\subsection{Infraestructura tecnológica virtual}

\begin{itemize}
        \item \textbf{Instalaciones:} Se requiere descargar los siguientes programas en sus versiones actuales: \begin{enumerate}
        \item \textbf{Kit de Desarrollo de Software}: Flutter
        \item \textbf{Entorno de Desarrollo Integrado}: Visual Studio Code y Android Studio
        \item \textbf{Herramientas de desarrollo}: Git y Visual Studio 2022  
    \end{enumerate}
    \item \textbf{Almacenamiento del código:} Se usarán los servicios de GitHub\footnote{plataforma propietaria para desarrolladores que permite crear, almacenar, administrar y compartir su código.}. Esta plataforma utiliza Git para proporcionar control de versiones distribuido y, a su vez, GitHub proporciona control de acceso, seguimiento de errores, solicitudes de funciones de software, gestión de tareas, integración continua a través del programa Visual Studio Code.
    \item \textbf{Servicio de Base de datos:} Se creará un proyecto en Firebase para el Hosting en la web.
    \item \textbf{Google Play Console:} Se usará para publicar el archivo .bundle a Google Play Store, para su accesibilidad de descarga en los dispositivos Android.
\end{itemize}

\section{Método}
La presente sección describe la los pasos empleados esquematizadas en la figura \ref{t2}, para abordar el problema central del proyecto, el cual se profundiza en el diseño y desarrollo de un algoritmo capaz de calcular valores de precipitación reales a partir de registros acumulados. Para ello, se parte de una lista de datos de entrada proporcionada por los usuarios, y se genera como salida una lista de valores corregidos que representan las mediciones reales, obtenidas mediante la resta entre cada nuevo dato y el inmediatamente anterior.

Este proceso requiere, como paso previo, el diseño de un protocolo de monitoreo participativo que motive e instruya a los usuarios en el envío constante y preciso de datos. Posteriormente, se detalla el desarrollo del algoritmo dentro del entorno de la aplicación móvil, seguido de un análisis del estado actual del proyecto con base en el nivel de maduración tecnológica. En resumen, se elaborarán las siguientes actividades:
\begin{itemize}
  \item Protocolo de monitoreo participativo
  \item Desarrollo de la aplicación móvil
  \item Análisis del estado actual del proyecto
\end{itemize}






\begin{figure}[h!]
    \centering
    \begin{tikzpicture}[node distance=0.8cm]
    \node (p0) [draw=orange!50, rounded corners, minimum width=8cm, minimum height=0.7cm, dashed] {Inicio};
    \node (p1) [orangebox, below=of p0] {\textbf{1. Proceso participativo con ejidatarios}};
    \node (p2) [orangebox, below=of p1] {Presentación del proyecto a las autoridades ejidales};
    \node (p3) [orangebox, below=of p2] {Talleres participativos para la identificación de \textbf{actores} y \textbf{sitios} de monitoreo};
    
    \node (p4) [yellowbox, below=of p3] {\textbf{2. Diseño técnico de monitoreo}};
    \node (p5) [yellowbox, below=of p4] {Recorridos para la caracterización de los parajes};
    \node (p6) [yellowbox, below=of p5] {Elaboración de pluviómetros y programación de la app móvil};
    \node (p7) [yellowbox, below=of p6] {Instalación de pluviómetros};
    
    \node (p8) [cyanbox, below=of p7] {\textbf{3. Campaña de difusión con público en general}};
    \node (p9) [cyanbox, below=of p8] {Campañas de mediciones};
    \node (p10) [cyanbox, below=of p9] {Publicación de resultados y premiación};
    \node (p11) [draw=cyan!50, rounded corners, minimum width=8cm, minimum height=0.7cm, dashed, below=of p10] {Final};
    
    \foreach \i/\j in {p0/p1, p1/p2, p2/p3, p3/p4, p4/p5, p5/p6, p6/p7, p7/p8, p8/p9, p9/p10, p10/p11}
      \draw [arrow] (\i) -- (\j);
    \end{tikzpicture}
    \caption{Diagrama de flujo del Protocolo de monitoreo participativo. Autoría Propia, 2025.}
      \label{t2}
\end{figure}

\subsection{Protocolo de monitoreo participativo:}

El protocolo consiste en llevar a cabo un monitoreo de lluvia que involucró tres principales etapas las cuales son: 
\begin{enumerate}
  \item Proceso participativo de ejidatarios
  \item Diseño técnico de monitoreo
  \item Campaña de difusión con público en general
\end{enumerate}



\subsubsection{Procesos Participativos con Ejidatarios} 
El primer paso consiste en establecer contacto con los miembros de la Unión Ejidal del Monte (UEM) para presentarles el proyecto y generar alianzas para su desarrollo. Posteriormente llevar a cabo talleres participativos con cada grupo ejidal (Nativitas, San Pablo Ixayoc, San Dieguito, Tequexquinahuac, Santa Catarina del Monte) para identificar a las personas que potencialmente podrían participar en el monitoreo y los lugares para instalar sitios de monitoreo. Este paso se resume en estas dos actividades:
\begin{itemize}
  \item Contexto Geográfico
  \item Descripción de la población de estudio
\end{itemize}


\subsubsection*{Criterio de selección de sitios de monitoreo}
Luego de visitar los lugares propuestos por las autoridades, registrar los datos de coordenadas, altitud, pendiente, tipo de vegetación, superficie desprovista de árboles y tamaño de los árboles circundantes. Esta información permite identificar los sitios más adecuados para instalar los sitios de monitoreo, siguiendo las recomendaciones de la Organización Meteorológica Mundial (OMM, 2014). En una etapa posterior se realiza un proceso de capacitación con los Ejidatarios y público en general para el monitoreo de la lluvia. Asimismo, crear un protocolo para facilitar a los Ejidatarios el uso de la información generada en sus actividades de manejo de los bosques.


% Los sitios de monitoreo se seleccionan a partir de las características siguientes:
% \begin{itemize}
%   \item Pendiente
%   \item Distancia a obstáculos
%   \item Homogeneidad de barrera
%   \item Criterios de accesibilidad
%   \item Altitud
%   \item Tipo de vegetación
% \end{itemize}

\subsubsection*{Descripción de la población de estudio}

Encuestar a la población que sube a la montaña en los sitios de monitoreo los parámetros descritos en la tabla \ref{tabt3}
\begin{table}[h!]
\centering
\begin{tabular}{@{}cc@{}}
\toprule
Tipo de parámetro & Detalle                                           \\ \midrule
Sociodemográfico      & Edad, sexo, nivel educativo, lugar de origen/localidad          \\
Tecnológico       & Uso de celular, internet, manejo de apps          \\
Actitudinal       & Disponibilidad, disposición a participar, interés \\
Territorial/ecológico & Conocimiento del bosque, frecuencia de visita, actividades      \\
Institucional/laboral & Tipo de relación con la zona (trabajo, recreación, supervisión) \\ \bottomrule
\end{tabular}
\caption{parámetros a identificar de la población}
\label{tabt3}
\end{table}

\subsubsection{Diseño Técnico de Monitoreo}

\begin{enumerate}
    \item Construcción de los pluviómetros: con botellas de PET, y siguiendo los lineamientos de la Norma Mexicana NMX-AA-166/1-SCFI-2013 (\cite{se2013}) o de la Organización Meteorológica Mundial. La máxima capacidad de almacenamiento es de 153 mm y la escala tiene resolución de un milímetro, excepto por los primeros 5 mm que tienen resolución de 0.25 mm. Los pluviómetros se colocaron sobre bases de madera a un metro sobre el nivel del suelo cavando un hoyo de 50 centímetros de profundidad. Para evitar pérdidas de agua por evaporación se utilizan 5 mm de aceite comestible vegetal por pluviómetro.

    \item Los pluviómetros se vacían y registran por el equipo técnico con una frecuencia de un mes (más menos dos días), a menos que sea necesario vaciar con mayor frecuencia. Los participantes envían sus registros sin una frecuencia específica, por lo que sus observaciones son adicionales a las que realiza el equipo técnico. Cada estación de monitoreo cuenta con letreros que poseen la información necesaria para que las personas puedan participar aunque no se les haya dado una capacitación personal. Se cuenta con siete estaciones de monitoreo en un gradiente altitudinal que va de 2683 a 3870 m. 
\end{enumerate}


% TODO: FALTA PONER MÁS ROLLO






 
\subsubsection{Campaña de difusión con público en general}

Este último paso, consiste en crear una campaña permanente de difusión entre la gente que sube a la montaña. Implica generar material gráfico instalado en campo que invite a la población a participar, trípticos y carteles que se colocan en lugares estratégicos;  utilizar las redes sociales para dar a conocer el proyecto y mecanismos para premiar a los participantes activos con regalos. 


Se plantea utilizar diversos medios y plataformas de divulgación enfocados en cada audiencia objetivo, que incluye lonas impresas, carteles, trípticos, Facebook, correos institucional y pláticas informativas. En el cuadro \ref{tab1} se muestra la descripción de medios y plataformas.

\begin{table}[h!]
    \centering
    \resizebox{\columnwidth}{!}{%
    \begin{tabular}{@{}cccc@{}}
    \toprule
    Medios y plataformas &
      Objetivo &
      Distribución &
      Audiencia objetivo \\ \midrule
    Lonas impresas &
      \begin{tabular}[c]{@{}c@{}}Difundir información\\ en sitios estratégicos\\ para incentivar la\\ participación y dar a\\ conocer el procedimiento\\ de participación.\end{tabular} &
      \begin{tabular}[c]{@{}c@{}}Se van a colocar en la\\ entrada principal a la\\ montaña (pluma de\\ acceso ubicada en el\\ sitio conocido como el\\ venturero), así como en\\ las 6 oficinas ejidales de\\ los Ejidos de la Montaña.\end{tabular} &
      Todas las audiencias \\
    Carteles &
      \begin{tabular}[c]{@{}c@{}}Dar a conocer el\\ procedimiento para\\ realizar las\\ mediciones en cada\\ sitio de monitoreo\\ de la lluvia.\end{tabular} &
      \begin{tabular}[c]{@{}c@{}}Se van a colocar en cada\\ sitio de monitoreo.\end{tabular} &
      Todas las audiencias \\
    Trípticos &
      \begin{tabular}[c]{@{}c@{}}Dar a conocer el\\ proyecto y el procedimiento de\\ participación a las\\ personas que\\ ingresan a la\\ montaña.\end{tabular} &
      \begin{tabular}[c]{@{}c@{}}Se van a repartir en la\\ entrada principal a la\\ montaña.\end{tabular} &
      \begin{tabular}[c]{@{}c@{}}Visitantes externos\\ Visitantes internos\\ Miembros de\\ instituciones gubernamentales y\\ técnicos forestales\\ Miembros de la academia\end{tabular} \\
    Página de Facebook &
      \begin{tabular}[c]{@{}c@{}}Difundir de manera\\ masiva el proyecto.\end{tabular} &
      Red social Facebook &
      Todas las audiencias \\
    Correo institucional &
      \begin{tabular}[c]{@{}c@{}}Difundir el proyecto\\ en la comunidad\\ COLPOS.\end{tabular} &
      Correo Colpos &
      Miembros de la academia \\
    Pláticas informativas &
      \begin{tabular}[c]{@{}c@{}}Dar a conocer e\\  proyecto y el\\ procedimiento de\\ participación con\\ determinadas\\ audiencias objetivo.\end{tabular} &
      \begin{tabular}[c]{@{}c@{}}Se realizó en etapas\\ previas de preparación\\ del proyecto con cada\\ Comité Ejidal. También\\ se va a llevar a cabo una\\ reunión con académicos\\ que realizan trabajo en\\ el Monte Tláloc.\end{tabular} &
      \begin{tabular}[c]{@{}c@{}}Ejidatarios de la Unión\\ de Ejidos de la Montaña\\ y sus cuadrillas de trabajo\\ Miembros de la academia\end{tabular} \\ \bottomrule
    \end{tabular}%
    }
    \caption{Medios y plataformas de divulgación del proyecto ``Ciencia ciudadana para el monitoreo participativo de la lluvia en un gradiente altitudinal del Monte Tláloc, Texcoco, Estado de México''}
    \label{tab1}
    \end{table}


\subsubsection*{Descripción de información para los medios y plataformas de divulgación}

\begin{enumerate}
    \item \textbf{Lonas impresas:}
    \begin{enumerate}
        \item Título del proyecto:
        Proyecto ``Ciencia ciudadana para el monitoreo de la lluvia en un gradiente altitudinal del Monte Tláloc, Texcoco, Estado de México”
        \item	Slogan: 
        ``Ciencia para ti y para todos''
        \item Logo del proyecto
        \item Logo del COLPOS y Postgrado en Ciencias Forestales
        \item Frase: Unión de Ejidos de la Montaña (junto a los logos del COLPOS y PCF)
        \item Texto principal: ¡Te invitamos a colaborar en el monitoreo de la lluvia en el Monte Tláloc, es muy sencillo!
        \item Diagrama de flujo con imágenes: \begin{enumerate}
            \item Ubica un sitio de monitoreo.
            \item Observa cuánta lluvia está almacenada en el pluviómetro.
            \item Envíanos la información (nivel del agua, fecha y hora del día) y una fotografía, con la aplicación móvil Tláloc app o por WhatsApp.
            \item Croquis del monitoreo
            \item Información complementaria: Cada 30 días se premiará con un obsequio muy especial a los 3 participantes con más registros. Además, al            registrarte en Tláloc App podrás tener acceso a la información que            generemos entre todos. Descarga Tláloc App en (poner sitio de descarga). Consulta más información en (poner la página de Facebook) o mándanos un WhatsApp para asesorarte (poner número telefónico). ¡Ayúdanos a mantener en condiciones adecuadas los instrumentos de
            medición!
        \end{enumerate}
    \end{enumerate}
    \item \textbf{Cartel frontal} \begin{enumerate}
        \item Slogan: 
        Ciencia para ti y para todos (quizás rodeando el logo del proyecto, en letra pequeña)
        \item Logo del proyecto (que destaque más que los otros logos)
        \item Logo del COLPOS y Postgrado en Ciencias Forestales
        \item Frase: Unión de Ejidos de la Montaña (junto a los logos del COLPOS y PCF)
        \item Texto principal: ¡Te invitamos a colaborar en el monitoreo de la lluvia en el Monte Tláloc!
        \item Tutorial \begin{enumerate}
            \item Observa el pluviómetro agachándote hasta que el nivel del agua esté frente a tus ojos. 
            \item Ubica la línea más cercana al nivel del agua y registra tu medición. 
            Poner esquema de cómo observar y una ampliación a cómo se ve el nivel de agua y la escala de medición.
            \item Registra tu medición con Tláloc App:
            
            Abre la aplicación e inicia sesión (colaborador externo o monitor); Escanea el código QR ubicado en la base del Pluviómetro; Registra tu medición en el espacio “Precipitación en mm”; Verifica que la fecha  y hora de la aplicación son correctas o edítalas si es necesario (poner los íconos de fecha y hora);Toma una foto del pluviómetro en la que se vea el nivel del agua como una línea. (poner una foto correcta y una incorrecta)            
            \item 	Si no cuentas con Tláloc App, anota los siguientes datos y mándalos con Whats App: Clave del pluviómetro ubicada en la base del Pluviómetro; Resultado de tu medición (Precipitación en mm); Fecha y hora; Foto del pluviómetro en la que se vea el nivel del agua como una línea. (ver las indicaciones arriba); Nunca vacíes el pluviómetro, sólo personal autorizado puede hacerlo. ¡Muchas gracias por tu contribución!            
        \end{enumerate}
    \end{enumerate}
    \item \textbf{Cartel posterior} \begin{enumerate}
        \item Título del proyecto: Proyecto “Ciencia ciudadana para el monitoreo de la lluvia en un gradiente altitudinal del Monte Tláloc, Texcoco, Estado de México”
        \item Logo del COLPOS y Postgrado en Ciencias Forestales
        \item Frase: Unión de Ejidos de la Montaña
        \item Texto principal: Este es un pluviómetro. Tiene una escala de medición en milímetros que indica la cantidad de lluvia que cae por metro cuadrado de terreno.
        \item Esquema del pluviómetro y la equivalencia de un mm de lluvia (1 mm = a vaciar un litro de agua en cada metro cuadrado de terreno).
        \item Saber cuándo, dónde y cuánto llueve en la montaña ayuda a entender cómo conservar el bosque y el agua que viene de ella. Cada 30 días se premiará a los 3 participantes con más registros. Además, al registrarte en Tláloc App podrás tener acceso a la información que generemos entre todos. Descarga Tláloc App en (poner sitio de descarga). Consultar más información en Facebook o mándanos un WhatsApp para asesorarte. ¡Ayúdanos a cuidar este pluviómetro! Por favor reporta si encuentras dañado este sitio de monitoreo.
    \end{enumerate}
    \item Tríptico
    \item Página de Facebook
    \item Pláticas informativas
\end{enumerate}





































































\subsection{Desarrollo del código}

En este sistema de monitoreo participan tres elementos principales: el usuario, el pluviómetro y la aplicación multiplataforma. El usuario es la persona encargada de registrar las mediciones realizadas por el pluviómetro artesanal, instalado en un paraje específico. Esta interacción ocurre a través de un código QR asignado a cada pluviómetro, el cual permite vincularlo con su ubicación y sus datos de monitoreo.

La aplicación multiplataforma facilita esta interacción al estar disponible en diferentes dispositivos como teléfonos inteligentes, tabletas, laptops o computadoras de escritorio, y puede ejecutarse en sistemas operativos como Android, iOS, Huawei OS, Windows y navegadores web. De este modo, el usuario puede elegir el dispositivo de su preferencia para capturar y enviar los datos registrados por el pluviómetro al sistema.

La Figura \ref{t3}, muestra el diagrama de flujo que representa el comportamiento operativo del sistema y cómo se realiza el intercambio de información entre estos tres actores.

Finalmente, la metodología para el desarrollo del código, consiste en programar en lenguaje Dart, una aplicación multiplataforma, alojada en un sitio web y en Play Store, con el fin de enviar las mediciones a una base de datos pública, que con ayuda del diagrama \ref{t3}, cuente con las siguientes características:

\begin{enumerate}
  \item Backend \begin{enumerate}
    \item Alojamiento del código en GitHub
    \item Conexión de la base de datos y Hosting en la web en Firebase
  \end{enumerate}
  \item User Interface (UI) \begin{enumerate}
    \item \textbf{Registro de usuario}: Los usuarios podrán crear una cuenta y elegir el pluviómetro mediante el escaneo de códigos Qr del sitio de monitoreo 
    \item \textbf{Menú Principal}: Dispondrá de tutoriales, contador de mediciones, acerca de, tabulador de mediciones, mapas de las dos rutas, grupos para subir a la montaña y contacto
    \item \textbf{Envío de mediciones}: Campo de texto, pluviómetro interactivo, booleano de vaciado, cambio de paraje
    \item \textbf{Bitácora}: Disponibilidad de consulta, edición, difusión o eliminación de las mediciones propias y no de otros usuarios
    \item \textbf{Estadísticas}: Mostrará un gráfico interactivo en diferentes tiempo de interés, por ejemplo por semana, mes y año.
\end{enumerate}
  \item User Experience (UX) \begin{enumerate}
    \item Material Design
    \item Tienda de aplicaciones
  \end{enumerate}
\end{enumerate}

\begin{figure}[ht]
\centering
  \includegraphics[width=1\textwidth]{t3.pdf}
  \caption{Diagrama de flujo de trabajo del sistema de Pluviómetros con Tláloc App}
  \label{t3}
\end{figure}
















\subsection{Evaluación del nivel de maduración tecnológica}

El modelo del CONACyT, ahora SECIHTI, ha derivado la Norma Oficial Mexicana NMX-GT-004-IMNC-2012, identifica cada nivel de TRL por medio de una lista de verificación con preguntas dicotómicas con respuestas de ``si'' o ``no'', lo que permite que la herramienta sea eficiente.


El Nivel de Madurez Tecnológica (NMT o Technology Readiness Level, TRL, en inglés) 
Es una escala de medición usada para evaluar o medir el nivel de madurez de una
tecnología particular. Cada proyecto es evaluado frente a los parámetros de cada
nivel tecnológico y es asignado a una clasificación basada en el progreso del
proyecto.

Para la presente investigación, apoyándose de la figura \ref{m1} y \ref{m2}, se hará la evaluación del nivel de maduración tecnológica de Tlaloc App.

\begin{figure}[ht]
\centering
  \includegraphics[width=1\textwidth]{m1.png}
  \caption{Guía para el Diagnóstico del Nivel de Madurez Tecnológica (NMT o TRL, por sus siglas en inglés) Parte 1, Fuente: Secretaría de Ciencia, Humanidades, Tecnología e Innovación (Secihti),}
  \label{m1}
\end{figure}
% TODO; CITA
% https://secihti.mx/tecnologias-e-innovacion/convocatorias-desarrollo-tecnologico-vinculacion-e-innovacion/
% https://secihti.mx/sistema-nacional-de-investigadores/marco-legal/
\begin{figure}[ht]
\centering
  \includegraphics[width=1\textwidth]{m2.png}
  \caption{Guía para el Diagnóstico del Nivel de Madurez Tecnológica (NMT o TRL, por sus siglas en inglés) Parte 2}
  \label{m2}
\end{figure}

