\chapter{RESULTADOS Y DISCUSIÓN}
% comparativas con otras plataformas, aunque sean preliminares.


% tasas de participación de usuarios, número de lluvias registradas, precisión de mediciones    
\section{Protocolo de monitoreo participativo:}
  % \item Proceso participativo de ejidatarios
\subsection{Proceso participativo de ejidatarios}
\subsubsection{Descripción de los sitios de monitoreo}
% AQUÍ PONER:
El Monte Tláloc, es un volcán formado a partir de las capas de sucesivas erupciones basálticas fluidas; ubicado en el Eje Neovolcánico en el límite entre los municipios de Ixtapaluca y Texcoco al oriente del Estado de México. Forma parte de la Sierra Nevada y es el Área Natural Protegida “Parque Nacional Iztaccíhuatl-Popocatépetl” su ubicación hidrológica es al oriente de la cuenca de México. Con sus 4120 metros sobre el nivel del mar, el Tláloc es la novena cima más alta del país. Cuenta con un clima de montaña cuya designación oficial es semifrío subhúmedo con lluvias en verano, de humedad media (\cite{inegi_texcoco}).

Se definieron los sitios de monitoreo dentro del Monte Tláloc señalizados por los carteles e identificados con el código QR y su información está representada sistemáticamente en la tabla \ref{tabrsm1}: 

\begin{landscape}
\begin{table}[h!]
\centering
\resizebox{\columnwidth}{!}{%
\begin{tabular}{@{}ccccccccc@{}}
\toprule
Sitio /   Caract. &
  \textbf{El Venturero} &
  \textbf{El Jardín} &
  \textbf{Cabaña} &
  \textbf{Cruz de Atenco} &
  \textbf{Canoas altas} &
  \textbf{Los Manantiales} &
  \textbf{Tlaltlatlately} &
  \textbf{Agua de Chiqueros} \\ \midrule
Ejido &
  Nativitas &
  Nativitas &
  \begin{tabular}[c]{@{}c@{}}Sn Pablo\\ Ixayoc\end{tabular} &
  \begin{tabular}[c]{@{}c@{}}San\\ Dieguito\end{tabular} &
  \begin{tabular}[c]{@{}c@{}}San\\ Dieguito\end{tabular} &
  Tequexquinahuac &
  \begin{tabular}[c]{@{}c@{}}Santa Catarína\\ del Monte\end{tabular} &
  \begin{tabular}[c]{@{}c@{}}Santa Catarína\\ del Monte\end{tabular} \\
\textit{\begin{tabular}[c]{@{}c@{}}Tipo de \\  Vegetación\end{tabular}} &
  \begin{tabular}[c]{@{}c@{}}Encinos\\ y mixto\end{tabular} &
  \begin{tabular}[c]{@{}c@{}}Abies\\ religiosa\end{tabular} &
  \begin{tabular}[c]{@{}c@{}}Encino /\\ Abies religiosa\end{tabular} &
  \begin{tabular}[c]{@{}c@{}}Abies\\ religiosa\end{tabular} &
  \begin{tabular}[c]{@{}c@{}}Abies religiosa/\\ Pinus hartwegii\end{tabular} &
  \begin{tabular}[c]{@{}c@{}}Pinus\\ hartwegii\end{tabular} &
  \begin{tabular}[c]{@{}c@{}}Encinos y\\ mixto\end{tabular} &
  Pinus hartwegii \\
\multicolumn{9}{c}{\textbf{CARACTERÍSTICAS TOPOGRÁFICAS}} \\
\textit{\begin{tabular}[c]{@{}c@{}}Altitud\\      (msnm)\end{tabular}} &
  2683 &
  2977.3 &
  3064.3 &
  3435.4 &
  3515.1 &
  3625.3 &
  2980.5 &
  3727.3 \\
\textit{Latitud N} &
  19°27'44.40'' &
  19°26.6800' &
  19°25.3800' &
  19°25.0520' &
  19°23.6810' &
  19°23.6810' &
  19°27.8780' &
  19°26.3550' \\
\textit{Longitud   O} &
  98°47'28.66'' &
  98°46.3200' &
  98°45.8170' &
  98°45.4090' &
  98°45.0420' &
  98°43.6380' &
  98°45.9280' &
  98°43.1340' \\
\textit{\begin{tabular}[c]{@{}c@{}}Ancho\\ del   \\ paraje (m)\end{tabular}} &
  ? &
  35 &
  15 &
  70 &
  32 &
  125 &
  67 &
  110 \\
\textit{\begin{tabular}[c]{@{}c@{}}Largo\\ del   \\ paraje (m)\end{tabular}} &
  ? &
  45 &
  50 &
  79 &
  56 &
  189 &
  134 &
  191 \\
\begin{tabular}[c]{@{}c@{}}Condición\\ de la\\ barrera de\\ árboles\end{tabular} &
  ? &
  ? &
  ? &
  \begin{tabular}[c]{@{}c@{}}Barrera\\ completa\\ de árboles\\ de aprox\\ 20 m\end{tabular} &
  ? &
  \begin{tabular}[c]{@{}c@{}}Rodeado\\ de árboles\\ en el 75\%\\ del área,\\ hay mucho\\ viento.\end{tabular} &
  \begin{tabular}[c]{@{}c@{}}Árboles de\\ hasta 20 m,\\ cima de\\ un lomerío,\\ bordeado\\ de árboles\\ de manera\\ homogénea.\end{tabular} &
  \begin{tabular}[c]{@{}c@{}}Árboles de\\ hasta 30 m. \\ Masa forestal\\ homogénea.\\ Laderas\\ que rodean\\ todo el\\ paraje.\end{tabular} \\
Accesibilidad &
  \begin{tabular}[c]{@{}c@{}}Camino\\ accesible\end{tabular} &
  \begin{tabular}[c]{@{}c@{}}Camino\\ accesible\end{tabular} &
  \begin{tabular}[c]{@{}c@{}}Camino\\ accesible\end{tabular} &
  \begin{tabular}[c]{@{}c@{}}El camino está\\ aceptable y\\ se llega\\ en\\ camioneta\\ hasta\\ el sitio\end{tabular} &
  ? &
  \begin{tabular}[c]{@{}c@{}}El camino está\\ aceptable y\\ se llega en\\ camioneta hasta\\ el sitio,\\ pero se puede\\ poner feo\\ en lluvias\end{tabular} &
  \begin{tabular}[c]{@{}c@{}}Muy accesible,\\ a pie de\\ camino y\\ en la\\ parte baja\end{tabular} &
  \begin{tabular}[c]{@{}c@{}}La última loma\\ antes de llegar al\\ sitio donde se\\ deja la camioneta\\ es difícil de subir.\\ Se caminan 500 m\\ aprox para llegar al\\ sitio de monitoreo.\end{tabular} \\
\textit{Tipo de   monitoreo} &
  Ejidal &
  Mixto &
  Mixto &
  Mixto &
  Mixto &
  Mixto &
  Ejidal &
  Mixto \\
\multicolumn{9}{c}{Pendientes Menores al 19°} \\ \bottomrule
\end{tabular}%
}
\caption{Resultados de la selección de los sitios de monitoreo}
\label{tabrsm1}
\end{table}
\end{landscape}



\subsubsection{Descripción de la población de estudio}

La población de estudio para este trabajo, se define como toda persona que participe en el proceso del monitoreo; este se compone de los siguientes grupos identificados, con características muy contrastantes:

\begin{enumerate}
    \item \textbf{Ejidatarios de la montaña (Unión de Ejidos de la Montaña) y sus cuadrillas de trabajo}: mayoritariamente hombres de entre 20 y 70 años, con nivel de estudios muy variado que llega hasta licenciatura, pero principalmente personas con educación básica a educación media. Son personas que suben a la montaña a hacer actividades de aprovechamiento forestal (aprovechan la madera y algunas otras cosas como musgo, perlilla y heno), y de mantenimiento del bosque (reforestación, chaponeo, podas, control de plagas, control de incendios, tendido de cercas, construcción de obras para control de erosión, remoción de suelo, mantenimiento de caminos, etc.). Algunos pertenecen a las localidades dueñas de los terrenos forestales, y otros son contratados de otros sitios, principalmente de Río Frío. En general suelen tener mucho trabajo, pero están dispuestos a colaborar y son los participantes del monitoreo con los que se ha tenido un contacto más estrecho. A este grupo se le va a dar una capacitación personalizada sobre el procedimiento para tomar las lecturas de los pluviómetros y se van a tener compromisos para la periodicidad de las mediciones, por lo que no es necesario convencerlos de participar.
    \item \textbf{Visitantes externos}: son todas las personas que suben a la montaña pero que no provienen de los Ejidos de la Montaña. Principalmente adultos, con gusto por convivir en ambientes naturales y con las capacidades tecnológicas necesarias para participar (teléfono móvil, acceso a internet y facilidad para el manejo de aplicaciones). En este grupo se incluyen a personas que suben de manera frecuente y son una audiencia objetivo con mucho potencial de participación, como ciclistas, senderistas, campistas y guías de turistas de empresas privadas. Otros visitantes que suben cotidianamente, pero probablemente no estén interesados en participar, son grupos de personas con alto nivel socioeconómico que se dedican a subir en motocross, jeeps y racers, cuyo objetivo es la diversión sin considerar el bienestar de la naturaleza y el impacto que generan en la zona. Finalmente, también hay visitantes externos que suben muy esporádicamente o por ocasión única, algunos suben al evento de la montaña fantasma, otros vienen del interior de la república o simplemente no tienen la costumbre de subir continuamente. Estos tres subgrupos integran una audiencia que requiere más explicación sobre los objetivos del proyecto y de cómo pueden participar y beneficiarse.
    \item \textbf{Visitantes internos}: son personas que forman parte de las localidades de los Ejidos de la Montaña pero que no trabajan con los ejidatarios, suben a realizar actividades como colecta de hongos o caminar. Es un grupo muy heterogéneo que incluye desde niños hasta adultos mayores, con mucho conocimiento sobre la zona de estudio (caminos y rutas, parajes, uso de los recursos naturales del bosque), pero probablemente no cuentan con las capacidades tecnológicas necesarias para participar (teléfono móvil, acceso a internet y facilidad para el manejo de aplicaciones).  Este grupo integra una audiencia que también requiere mucha explicación sobre los objetivos del proyecto y de cómo pueden participar y beneficiarse.
    \item \textbf{Miembros de instituciones gubernamentales y técnicos forestales}: son profesionales encargados de supervisar las actividades de aprovechamiento y manejo forestal, de los recursos del agua y el estado del bosque. Incluye a empleados de Probosque (dependencia estatal), que supervisan constantemente los trabajos realizados en la zona y apoyan en las labores de combate de incendios. También incluye a empleados de otras entidades a nivel federal como (Comisión Nacional Forestal, Comisión Nacional de Áreas Naturales Protegidas, Secretaría de Recursos Naturales, Procuraduría Federal de Protección al Ambiente y Comisión Nacional del Agua). Es un grupo integrado por adultos de entre 30 y 50 años principalmente, con mucho conocimiento sobre la zona de estudio (caminos y rutas, parajes, uso de los recursos naturales del bosque), con las capacidades tecnológicas necesarias para participar (teléfono móvil, acceso a internet y facilidad para el manejo de aplicaciones).  Aunque algunos de ellos suben continuamente, no se sabe si van a tener la disponibilidad de participar aunque sea esporádicamente, ya que siempre tienen prisa.
    \item \textbf{Miembros de la academia}: son estudiantes, profesores e investigadores que realizan actividades de investigación de muy distinta índole en la zona. Algunos suben de manera esporádica y otros suben frecuentemente. Es un grupo integrado por adultos de entre 25 y 60 años principalmente, con mucho conocimiento sobre la zona de estudio (caminos y rutas, parajes, conocimiento sobre el bosque), con las capacidades tecnológicas necesarias para participar (teléfono móvil, acceso a internet y facilidad para el manejo de aplicaciones). Algunas de las instituciones con mayor presencia en la zona son el Colegio de Postgraduados, Universidad Autónoma Chapingo, UNAM, UAM. Aunque algunos de ellos suben continuamente, no se sabe si van a tener la disponibilidad de participar aunque sea esporádicamente, ya que siempre tienen prisa.
\end{enumerate}





\subsection{Diseño técnico de monitoreo}

% Construcción de los pluviómetros
% Imagenes del pluviómetro

% \begin{enumerate}
%     \item Construcción de los pluviómetros: con botellas de PET, y siguiendo los lineamientos de la Norma Mexicana NMX-AA-166/1-SCFI-2013 (\cite{se2013}) o de la Organización Meteorológica Mundial. La máxima capacidad de almacenamiento es de 153 mm y la escala tiene resolución de un milímetro, excepto por los primeros 5 mm que tienen resolución de 0.25 mm. Los pluviómetros se colocaron sobre bases de madera a un metro sobre el nivel del suelo cavando un hoyo de 50 centímetros de profundidad. Para evitar pérdidas de agua por evaporación se utilizan 5 mm de aceite comestible vegetal por pluviómetro.

%     \item Los pluviómetros se vacían y registran por el equipo técnico con una frecuencia de un mes (más menos dos días), a menos que sea necesario vaciar con mayor frecuencia. Los participantes envían sus registros sin una frecuencia específica, por lo que sus observaciones son adicionales a las que realiza el equipo técnico. Cada estación de monitoreo cuenta con letreros que poseen la información necesaria para que las personas puedan participar aunque no se les haya dado una capacitación personal. Se cuenta con siete estaciones de monitoreo en un gradiente altitudinal que va de 2683 a 3870 m. 
% \end{enumerate}

En la figura \ref{publicidad2}, puede observarse la instalación de la base

\begin{figure}[h!]
	\centering
	\begin{subfigure} 
		\centering
		\includegraphics[width=0.2\linewidth]{publicidad/2.jpg}
		\caption{Base para el pluviómetro}
		\label{publicidad2}
	\end{subfigure} 
	\begin{subfigure} 
		\centering
		\includegraphics[width=0.2\linewidth]{publicidad/4.jpg}
		\caption{Escala de graduación para precipitación.}
		\label{publicidad4}
	\end{subfigure}
	\caption{Resultados de la instalación de pluviómetros.}
	\label{publicidad24-}
\end{figure}


Así mismo, la distribución de los mismos está marcado en el mapa de la figura \ref{m3}

\begin{figure}[h!]
\centering
  \includegraphics[width=1\textwidth]{m3.jpg}
  \caption{Rutas de monitoreo de lluvia}
  \label{m3}
\end{figure}



\subsection{Campaña de difusión con público en general}
Como resultado, se diseñaron y publicaron los recursos ilustrados a continuación.


\subsubsection{Carteles}
La última versión de publicidad, contiene elementos cuidadosamente ordenados, en la figura \ref{publicidad1}.

\begin{figure}[ht]
\centering
  \includegraphics[width=0.5\textwidth]{publicidad/1.jpg}
  \caption{Carteles de publicidad}
  \label{publicidad1}
\end{figure}

\subsubsection{Lonas}
Se elaboró el cartel de la figura \ref{publicidad7}
\begin{figure}[ht]
\centering
  \includegraphics[width=0.5\textwidth]{publicidad/8.pdf}
  \caption{Carteles de publicidad}
  \label{publicidad7}
\end{figure}
% https://www.facebook.com/profile.php?id=100083233511805

\subsubsection{Trípticos}
Se elaboraron trípticos, mencionados en la figura \ref{publicidad9}
\begin{figure}[ht]
\centering
  \includegraphics[width=0.5\textwidth]{publicidad/9.pdf}
  \caption{Trípticos}
  \label{publicidad9}
\end{figure}

\subsubsection{Redes sociales}
La única red oficial de divulgación, está contenida en la página llamada ``Ciencia Ciudadana para el Monitoreo de Lluvia '', identificandolo con el url: \url{https://www.facebook.com/profile.php?id=100083233511805}

\subsubsection{Pláticas informativas}

A continuación, se adjuntan las evidencias fotográficas de pláticas informativas con los Ejidatarios:





\begin{itemize}
  % \item Lonas
  % \item Carteles
  % \item Trípticos
  % \item Facebook
  % \item Pláticas informativas
  \item Entrega de obsequios
\end{itemize}

% \begin{table}[h!]
%     \centering
%     \resizebox{\columnwidth}{!}{%
%     \begin{tabular}{@{}cccc@{}}
%     \toprule
%     Medios y plataformas &
%       Objetivo &
%       Distribución &
%       Audiencia objetivo \\ \midrule
%     Lonas impresas &
%       \begin{tabular}[c]{@{}c@{}}Difundir información\\ en sitios estratégicos\\ para incentivar la\\ participación y dar a\\ conocer el procedimiento\\ de participación.\end{tabular} &
%       \begin{tabular}[c]{@{}c@{}}Se van a colocar en la\\ entrada principal a la\\ montaña (pluma de\\ acceso ubicada en el\\ sitio conocido como el\\ venturero), así como en\\ las 6 oficinas ejidales de\\ los Ejidos de la Montaña.\end{tabular} &
%       Todas las audiencias \\
%     Carteles &
%       \begin{tabular}[c]{@{}c@{}}Dar a conocer el\\ procedimiento para\\ realizar las\\ mediciones en cada\\ sitio de monitoreo\\ de la lluvia.\end{tabular} &
%       \begin{tabular}[c]{@{}c@{}}Se van a colocar en cada\\ sitio de monitoreo.\end{tabular} &
%       Todas las audiencias \\
%     Trípticos &
%       \begin{tabular}[c]{@{}c@{}}Dar a conocer el\\ proyecto y el procedimiento de\\ participación a las\\ personas que\\ ingresan a la\\ montaña.\end{tabular} &
%       \begin{tabular}[c]{@{}c@{}}Se van a repartir en la\\ entrada principal a la\\ montaña.\end{tabular} &
%       \begin{tabular}[c]{@{}c@{}}Visitantes externos\\ Visitantes internos\\ Miembros de\\ instituciones gubernamentales y\\ técnicos forestales\\ Miembros de la academia\end{tabular} \\
%     Página de Facebook &
%       \begin{tabular}[c]{@{}c@{}}Difundir de manera\\ masiva el proyecto.\end{tabular} &
%       Red social Facebook &
%       Todas las audiencias \\
%     Correo institucional &
%       \begin{tabular}[c]{@{}c@{}}Difundir el proyecto\\ en la comunidad\\ COLPOS.\end{tabular} &
%       Correo Colpos &
%       Miembros de la academia \\
%     Pláticas informativas &
%       \begin{tabular}[c]{@{}c@{}}Dar a conocer e\\  proyecto y el\\ procedimiento de\\ participación con\\ determinadas\\ audiencias objetivo.\end{tabular} &
%       \begin{tabular}[c]{@{}c@{}}Se realizó en etapas\\ previas de preparación\\ del proyecto con cada\\ Comité Ejidal. También\\ se va a llevar a cabo una\\ reunión con académicos\\ que realizan trabajo en\\ el Monte Tláloc.\end{tabular} &
%       \begin{tabular}[c]{@{}c@{}}Ejidatarios de la Unión\\ de Ejidos de la Montaña\\ y sus cuadrillas de trabajo\\ Miembros de la academia\end{tabular} \\ \bottomrule
%     \end{tabular}%
%     }
%     \caption{Medios y plataformas de divulgación del proyecto ``Ciencia ciudadana para el monitoreo participativo de la lluvia en un gradiente altitudinal del Monte Tláloc, Texcoco, Estado de México''}
%     \label{tab1}
%     \end{table}


% \subsubsection*{Descripción de información para los medios y plataformas de divulgación}

% \begin{enumerate}
%     \item \textbf{Lonas impresas:}
%     \begin{enumerate}
%         \item Título del proyecto:
%         Proyecto ``Ciencia ciudadana para el monitoreo de la lluvia en un gradiente altitudinal del Monte Tláloc, Texcoco, Estado de México”
%         \item	Slogan: 
%         ``Ciencia para ti y para todos''
%         \item Logo del proyecto
%         \item Logo del COLPOS y Postgrado en Ciencias Forestales
%         \item Frase: Unión de Ejidos de la Montaña (junto a los logos del COLPOS y PCF)
%         \item Texto principal: ¡Te invitamos a colaborar en el monitoreo de la lluvia en el Monte Tláloc, es muy sencillo!
%         \item Diagrama de flujo con imágenes: \begin{enumerate}
%             \item Ubica un sitio de monitoreo.
%             \item Observa cuánta lluvia está almacenada en el pluviómetro.
%             \item Envíanos la información (nivel del agua, fecha y hora del día) y una fotografía, con la aplicación móvil Tláloc app o por WhatsApp.
%             \item Croquis del monitoreo
%             \item Información complementaria: Cada 30 días se premiará con un obsequio muy especial a los 3 participantes con más registros. Además, al            registrarte en Tláloc App podrás tener acceso a la información que            generemos entre todos. Descarga Tláloc App en (poner sitio de descarga). Consulta más información en (poner la página de Facebook) o mándanos un WhatsApp para asesorarte (poner número telefónico). ¡Ayúdanos a mantener en condiciones adecuadas los instrumentos de
%             medición!
%         \end{enumerate}
%     \end{enumerate}
%     \item \textbf{Cartel frontal} \begin{enumerate}
%         \item Slogan: 
%         Ciencia para ti y para todos (quizás rodeando el logo del proyecto, en letra pequeña)
%         \item Logo del proyecto (que destaque más que los otros logos)
%         \item Logo del COLPOS y Postgrado en Ciencias Forestales
%         \item Frase: Unión de Ejidos de la Montaña (junto a los logos del COLPOS y PCF)
%         \item Texto principal: ¡Te invitamos a colaborar en el monitoreo de la lluvia en el Monte Tláloc!
%         \item Tutorial \begin{enumerate}
%             \item Observa el pluviómetro agachándote hasta que el nivel del agua esté frente a tus ojos. 
%             \item Ubica la línea más cercana al nivel del agua y registra tu medición. 
%             Poner esquema de cómo observar y una ampliación a cómo se ve el nivel de agua y la escala de medición.
%             \item Registra tu medición con Tláloc App:
            
%             Abre la aplicación e inicia sesión (colaborador externo o monitor); Escanea el código QR ubicado en la base del Pluviómetro; Registra tu medición en el espacio “Precipitación en mm”; Verifica que la fecha  y hora de la aplicación son correctas o edítalas si es necesario (poner los íconos de fecha y hora);Toma una foto del pluviómetro en la que se vea el nivel del agua como una línea. (poner una foto correcta y una incorrecta)            
%             \item 	Si no cuentas con Tláloc App, anota los siguientes datos y mándalos con Whats App: Clave del pluviómetro ubicada en la base del Pluviómetro; Resultado de tu medición (Precipitación en mm); Fecha y hora; Foto del pluviómetro en la que se vea el nivel del agua como una línea. (ver las indicaciones arriba); Nunca vacíes el pluviómetro, sólo personal autorizado puede hacerlo. ¡Muchas gracias por tu contribución!            
%         \end{enumerate}
%     \end{enumerate}
%     \item \textbf{Cartel posterior} \begin{enumerate}
%         \item Título del proyecto: Proyecto “Ciencia ciudadana para el monitoreo de la lluvia en un gradiente altitudinal del Monte Tláloc, Texcoco, Estado de México”
%         \item Logo del COLPOS y Postgrado en Ciencias Forestales
%         \item Frase: Unión de Ejidos de la Montaña
%         \item Texto principal: Este es un pluviómetro. Tiene una escala de medición en milímetros que indica la cantidad de lluvia que cae por metro cuadrado de terreno.
%         \item Esquema del pluviómetro y la equivalencia de un mm de lluvia (1 mm = a vaciar un litro de agua en cada metro cuadrado de terreno).
%         \item Saber cuándo, dónde y cuánto llueve en la montaña ayuda a entender cómo conservar el bosque y el agua que viene de ella. Cada 30 días se premiará a los 3 participantes con más registros. Además, al registrarte en Tláloc App podrás tener acceso a la información que generemos entre todos. Descarga Tláloc App en (poner sitio de descarga). Consultar más información en Facebook o mándanos un WhatsApp para asesorarte. ¡Ayúdanos a cuidar este pluviómetro! Por favor reporta si encuentras dañado este sitio de monitoreo.
%     \end{enumerate}
%     \item Tríptico
%     \item Página de Facebook
%     \item Pláticas informativas
% \end{enumerate}
 

























% \subsection{Discusión}
% Descripción de la población:

% % lo de la gente que sube

% % ir de lo particular a lo general





















\newpage
\section{Desarrollo del código}


% \begin{enumerate}
%   \item Backend \begin{enumerate}
%     \item Alojamiento del código en GitHub
%     \item Conexión de la base de datos y Hosting en la web en Firebase
%   \end{enumerate}
%   \item User Interface (UI) \begin{enumerate}
%     \item \textbf{Registro de usuario}: Los usuarios podrán crear una cuenta y elegir el pluviómetro mediante el escaneo de códigos Qr del sitio de monitoreo 
%     \item \textbf{Menú Principal}: Dispondrá de tutoriales, contador de mediciones, acerca de, tabulador de mediciones, mapas de las dos rutas, grupos para subir a la montaña y contacto
%     \item \textbf{Envío de mediciones}: Campo de texto, pluviómetro interactivo, booleano de vaciado, cambio de paraje
%     \item \textbf{Bitácora}: Disponibilidad de consulta, edición, difusión o eliminación de las mediciones propias y no de otros usuarios
%     \item \textbf{Estadísticas}: Mostrará un gráfico interactivo en diferentes tiempo de interés, por ejemplo por semana, mes y año.
% \end{enumerate}
%   \item User Experience (UX) \begin{enumerate}
%     \item Material Design
%     \item Tienda de aplicaciones
%   \end{enumerate}
% \end{enumerate}


\subsection{Backend}

\subsubsection{Control de versiones y almacenamiento en GitHub}
 

El desarrollo de aplicaciones científicas requiere no solo un diseño estructurado del código, sino también un control riguroso de versiones y respaldos. GitHub es una plataforma ampliamente usada en el ámbito académico y profesional para alojar proyectos de software, permitiendo la colaboración, el seguimiento de cambios y el resguardo del historial del código fuente. Para este proyecto, el repositorio está disponible públicamente en:

\begin{center}
\url{https://github.com/Jack55913/TlalocApp.git}
\end{center}

\subsubsection*{Proceso para guardar código localmente en GitHub con Git}

El flujo básico para controlar versiones de un proyecto usando \texttt{git} desde la computadora incluye los siguientes pasos:

\begin{enumerate}
    \item Inicializar el repositorio en la carpeta del proyecto (solo la primera vez):
    \begin{minted}{bash}
    git init
    \end{minted}
    
    \item Configurar nombre y correo del autor (también solo una vez):
    \begin{minted}{bash}
    git config --global user.name "Luis Emilio Álvarez Herrera"
    git config --global user.email "ejemplo@dominio.com"
    \end{minted}
    
    \item Agregar el repositorio remoto de GitHub:
    \begin{minted}{bash}
    git remote add origin https://github.com/Jack55913/TlalocApp.git
    \end{minted}
    
    \item Verificar el estado del repositorio y los archivos modificados:
    \begin{minted}{bash}
    git status
    \end{minted}
    
    \item Añadir los archivos que se desean guardar (todos o algunos específicos):
    \begin{minted}{bash}
    git add .
    \end{minted}
    
    \item Registrar los cambios con un mensaje claro:
    \begin{minted}{bash}
    git commit -m "Primera versión funcional de Tláloc App"
    \end{minted}
    
    \item Subir los cambios al repositorio remoto en GitHub:
    \begin{minted}{bash}
    git push -u origin main
    \end{minted}
\end{enumerate}

Este flujo se puede repetir cada vez que se realicen avances importantes en el código. Gracias a Git, cada modificación queda registrada con fecha, autor y propósito, lo cual facilita el mantenimiento del software, el trabajo colaborativo y la trazabilidad científica del desarrollo.

\subsubsection*{Clonar el repositorio desde GitHub}

Si se desea obtener una copia exacta del proyecto alojado en GitHub en otra computadora, es posible usar el comando \texttt{git clone}. Este comando descarga todo el historial y los archivos del proyecto:

\begin{minted}{bash}
git clone https://github.com/Jack55913/TlalocApp.git
\end{minted}

Este comando crea automáticamente una carpeta llamada \texttt{TlalocApp} con todos los archivos y el historial del proyecto.

\subsubsection*{Subir cambios al repositorio remoto: \texttt{git push}}

Una vez realizados cambios en el código, es posible subirlos a GitHub con el siguiente flujo:

\begin{enumerate}
    \item Verificar el estado del repositorio local:
    \begin{minted}{bash}
    git status
    \end{minted}
    
    \item Agregar los archivos modificados al área de preparación:
    \begin{minted}{bash}
    git add .
    \end{minted}
    
    \item Registrar los cambios localmente con un mensaje:
    \begin{minted}{bash}
    git commit -m "Descripción clara del cambio realizado"
    \end{minted}
    
    \item Subir los cambios al servidor de GitHub:
    \begin{minted}{bash}
    git push origin main
    \end{minted}
\end{enumerate}

Este proceso permite mantener actualizado el repositorio en la nube, sirviendo como respaldo y facilitando el trabajo colaborativo.

\subsubsection*{Obtener actualizaciones del repositorio remoto: \texttt{git pull}}

Si se ha trabajado en el mismo repositorio desde otra computadora o por otros colaboradores, se deben sincronizar los cambios con:

\begin{minted}{bash}
git pull origin main
\end{minted}

Este comando fusiona los cambios realizados en GitHub con la copia local del proyecto. Es recomendable ejecutarlo antes de comenzar una nueva sesión de desarrollo para evitar conflictos.









\subsubsection{Backend en el código}

\subsubsection*{Función AppState}

En el anexo \ref{anexo:alg1}, el Listado de funcionalidades del código son:
\begin{enumerate}
    \item Gestión de Usuario y Configuración
    Autenticación con Google: Usa GoogleSignInProvider para manejar el inicio de sesión y obtiene el usuario actual (currentUser).
    
    Guardar Preferencias:
    
    changeParaje(): Cambia y guarda el paraje (ubicación) en SharedPreferences.
    
    changeRol(): Cambia y guarda el rol del usuario (ej: "Monitor") en SharedPreferences.
    
    Recupera datos guardados al inicializar (init()).
    
    \item Operaciones CRUD con Firebase Firestore
    Crear Mediciones:
    
    addMeasurement(): Guarda una nueva medición en la colección measurements y calcula automáticamente el valor real (real\_measurements).
    
    addRealMeasurement(): Guarda una medición directamente en real\_measurements.
    
    Leer Datos:
    
    getMeasurements(), getRealMeasurements(): Obtienen listas de mediciones.
    
    Streams en tiempo real: getMeasurementsStream(), getRealMeasurementsStream(), y variantes por paraje.
    
    Actualizar Mediciones:
    
    updateMeasurement(): Actualiza una medición existente en measurements.
    
    updateRealMeasurement(): Actualiza una medición en real\_measurements.
    
    Eliminar Mediciones:
    
    deleteMeasurement(), deleteRealMeasurement(): Borran documentos de Firestore.
    
    \item Manejo de Imágenes con Firebase Storage
    Sube imágenes a Storage desde la web o dispositivos móviles (en \_getMeasurementJson()).
    
    Convierte imágenes a URLs descargables o las guarda como base64 si no hay conexión.
    
    \item Lógica del algoritmo
    Cálculo de Precipitación Real: En addMeasurement(), determina si el pluviómetro fue vaciado y ajusta el valor de la precipitación.
    
    \item Gestión de Conexión
    Verifica conectividad con Connectivity().checkConnectivity() para decidir cómo guardar imágenes.
    
\end{enumerate}

Sobre los servicios de Firebase, fueron llamados en la tabla \ref{tabt2}


\begin{table}[h!]
    \centering
    \resizebox{\columnwidth}{!}{%
    \begin{tabular}{@{}ccc@{}}
    \toprule
    \textbf{Función}              & \textbf{Servicio de Firebase} & \textbf{Método/Acción}            \\ \midrule
    Actualizar medición           & Firestore                     & update() en documentos            \\
    Crear medición                & Firestore                     & add()                             \\
    Eliminar medición             & Firestore                     & delete()                          \\
    Subir imágenes                & Storage                       & putData(), putFile()              \\
    Escuchar datos en tiempo real & Firestore                     & snapshots()                       \\
    Autenticación                 & Auth                          & currentUser, uid, email, photoURL \\ \bottomrule
    \end{tabular}%
    }
    \caption{Funciones Clave de Firebase en el Código}
    \label{tabt2}
    \end{table}



\subsection{User Interface (UI)}




\subsection{User Experience (UX)}




 
















\section{Evaluación del nivel de maduración tecnológica}














\section{Alcance y limitaciones del estudio}
El principal alcance de este estudio es demostrar la viabilidad técnica de integrar la ciencia ciudadana y tecnologías móviles para fortalecer el acceso a datos meteorológicos en zonas de montaña, donde las redes oficiales presentan baja densidad.

El desarrollo de la aplicación incluye la recolección de datos mediante pluviómetros manuales, el registro georreferenciado de observaciones, el almacenamiento en bases de datos en la nube y la visualización de datos de manera intuitiva para los usuarios. Además, se evaluó preliminarmente el nivel de maduración tecnológica (TRL) de la herramienta.

Sin embargo, el estudio presenta limitaciones inherentes a su carácter exploratorio. La validación de los datos recopilados por los usuarios no fue exhaustiva, y la base de usuarios durante las pruebas piloto fue limitada en número y diversidad. Asimismo, las funcionalidades de descarga de estadísticas, algoritmos predictivos o de inteligencia artificial para la detección de anomalías quedaron identificadas como trabajo futuro, pero no fueron desarrolladas en esta primera fase.