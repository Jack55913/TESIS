\chapter{RESULTADOS Y DISCUSIÓN}
% comparativas con otras plataformas, aunque sean preliminares.


% tasas de participación de usuarios, número de lluvias registradas, precisión de mediciones    
\section{Protocolo de monitoreo participativo:}
  % \item Proceso participativo de ejidatarios
\subsection{Proceso participativo de ejidatarios}
\subsubsection{Descripción de los sitios de monitoreo}
% AQUÍ PONER:
El Monte Tláloc, es un volcán formado a partir de las capas de sucesivas erupciones basálticas fluidas; ubicado en el Eje Neovolcánico en el límite entre los municipios de Ixtapaluca y Texcoco al oriente del Estado de México. Forma parte de la Sierra Nevada y es el Área Natural Protegida “Parque Nacional Iztaccíhuatl-Popocatépetl” su ubicación hidrológica es al oriente de la cuenca de México. Con sus 4120 metros sobre el nivel del mar, el Tláloc es la novena cima más alta del país. Cuenta con un clima de montaña cuya designación oficial es semifrío subhúmedo con lluvias en verano, de humedad media (\cite{inegi_texcoco}).

Se definieron los sitios de monitoreo dentro del Monte Tláloc señalizados por los carteles e identificados con el código QR y su información está representada sistemáticamente en la tabla \ref{tabrsm1}: 

\begin{landscape}
\begin{table}[h!]
\centering
\resizebox{\columnwidth}{!}{%
\begin{tabular}{@{}ccccccccc@{}}
\toprule
Sitio /   Caract. &
  \textbf{El Venturero} &
  \textbf{El Jardín} &
  \textbf{Cabaña} &
  \textbf{Cruz de Atenco} &
  \textbf{Canoas altas} &
  \textbf{Los Manantiales} &
  \textbf{Tlaltlatlately} &
  \textbf{Agua de Chiqueros} \\ \midrule
Ejido &
  Nativitas &
  Nativitas &
  \begin{tabular}[c]{@{}c@{}}Sn Pablo\\ Ixayoc\end{tabular} &
  \begin{tabular}[c]{@{}c@{}}San\\ Dieguito\end{tabular} &
  \begin{tabular}[c]{@{}c@{}}San\\ Dieguito\end{tabular} &
  Tequexquinahuac &
  \begin{tabular}[c]{@{}c@{}}Santa Catarína\\ del Monte\end{tabular} &
  \begin{tabular}[c]{@{}c@{}}Santa Catarína\\ del Monte\end{tabular} \\
\textit{\begin{tabular}[c]{@{}c@{}}Tipo de \\  Vegetación\end{tabular}} &
  \begin{tabular}[c]{@{}c@{}}Encinos\\ y mixto\end{tabular} &
  \begin{tabular}[c]{@{}c@{}}Abies\\ religiosa\end{tabular} &
  \begin{tabular}[c]{@{}c@{}}Encino /\\ Abies religiosa\end{tabular} &
  \begin{tabular}[c]{@{}c@{}}Abies\\ religiosa\end{tabular} &
  \begin{tabular}[c]{@{}c@{}}Abies religiosa/\\ Pinus hartwegii\end{tabular} &
  \begin{tabular}[c]{@{}c@{}}Pinus\\ hartwegii\end{tabular} &
  \begin{tabular}[c]{@{}c@{}}Encinos y\\ mixto\end{tabular} &
  Pinus hartwegii \\
\multicolumn{9}{c}{\textbf{CARACTERÍSTICAS TOPOGRÁFICAS}} \\
\textit{\begin{tabular}[c]{@{}c@{}}Altitud\\      (msnm)\end{tabular}} &
  2683 &
  2977.3 &
  3064.3 &
  3435.4 &
  3515.1 &
  3625.3 &
  2980.5 &
  3727.3 \\
\textit{Latitud N} &
  19°27'44.40'' &
  19°26.6800' &
  19°25.3800' &
  19°25.0520' &
  19°23.6810' &
  19°23.6810' &
  19°27.8780' &
  19°26.3550' \\
\textit{Longitud   O} &
  98°47'28.66'' &
  98°46.3200' &
  98°45.8170' &
  98°45.4090' &
  98°45.0420' &
  98°43.6380' &
  98°45.9280' &
  98°43.1340' \\
\textit{\begin{tabular}[c]{@{}c@{}}Ancho\\ del   \\ paraje (m)\end{tabular}} &
  ? &
  35 &
  15 &
  70 &
  32 &
  125 &
  67 &
  110 \\
\textit{\begin{tabular}[c]{@{}c@{}}Largo\\ del   \\ paraje (m)\end{tabular}} &
  ? &
  45 &
  50 &
  79 &
  56 &
  189 &
  134 &
  191 \\
\begin{tabular}[c]{@{}c@{}}Condición\\ de la\\ barrera de\\ árboles\end{tabular} &
  ? &
  ? &
  ? &
  \begin{tabular}[c]{@{}c@{}}Barrera\\ completa\\ de árboles\\ de aprox\\ 20 m\end{tabular} &
  ? &
  \begin{tabular}[c]{@{}c@{}}Rodeado\\ de árboles\\ en el 75\%\\ del área,\\ hay mucho\\ viento.\end{tabular} &
  \begin{tabular}[c]{@{}c@{}}Árboles de\\ hasta 20 m,\\ cima de\\ un lomerío,\\ bordeado\\ de árboles\\ de manera\\ homogénea.\end{tabular} &
  \begin{tabular}[c]{@{}c@{}}Árboles de\\ hasta 30 m. \\ Masa forestal\\ homogénea.\\ Laderas\\ que rodean\\ todo el\\ paraje.\end{tabular} \\
Accesibilidad &
  \begin{tabular}[c]{@{}c@{}}Camino\\ accesible\end{tabular} &
  \begin{tabular}[c]{@{}c@{}}Camino\\ accesible\end{tabular} &
  \begin{tabular}[c]{@{}c@{}}Camino\\ accesible\end{tabular} &
  \begin{tabular}[c]{@{}c@{}}El camino está\\ aceptable y\\ se llega\\ en\\ camioneta\\ hasta\\ el sitio\end{tabular} &
  ? &
  \begin{tabular}[c]{@{}c@{}}El camino está\\ aceptable y\\ se llega en\\ camioneta hasta\\ el sitio,\\ pero se puede\\ poner feo\\ en lluvias\end{tabular} &
  \begin{tabular}[c]{@{}c@{}}Muy accesible,\\ a pie de\\ camino y\\ en la\\ parte baja\end{tabular} &
  \begin{tabular}[c]{@{}c@{}}La última loma\\ antes de llegar al\\ sitio donde se\\ deja la camioneta\\ es difícil de subir.\\ Se caminan 500 m\\ aprox para llegar al\\ sitio de monitoreo.\end{tabular} \\
\textit{Tipo de   monitoreo} &
  Ejidal &
  Mixto &
  Mixto &
  Mixto &
  Mixto &
  Mixto &
  Ejidal &
  Mixto \\
\multicolumn{9}{c}{Pendientes Menores al 19°} \\ \bottomrule
\end{tabular}%
}
\caption{Resultados de la selección de los sitios de monitoreo}
\label{tabrsm1}
\end{table}
\end{landscape}



\subsubsection{Descripción de la población de estudio}

La población de estudio para este trabajo, se define como toda persona que participe en el proceso del monitoreo; este se compone de los siguientes grupos identificados, con características muy contrastantes:

\begin{enumerate}
    \item \textbf{Ejidatarios de la montaña (Unión de Ejidos de la Montaña) y sus cuadrillas de trabajo}: mayoritariamente hombres de entre 20 y 70 años, con nivel de estudios muy variado que llega hasta licenciatura, pero principalmente personas con educación básica a educación media. Son personas que suben a la montaña a hacer actividades de aprovechamiento forestal (aprovechan la madera y algunas otras cosas como musgo, perlilla y heno), y de mantenimiento del bosque (reforestación, chaponeo, podas, control de plagas, control de incendios, tendido de cercas, construcción de obras para control de erosión, remoción de suelo, mantenimiento de caminos, etc.). Algunos pertenecen a las localidades dueñas de los terrenos forestales, y otros son contratados de otros sitios, principalmente de Río Frío. En general suelen tener mucho trabajo, pero están dispuestos a colaborar y son los participantes del monitoreo con los que se ha tenido un contacto más estrecho. A este grupo se le va a dar una capacitación personalizada sobre el procedimiento para tomar las lecturas de los pluviómetros y se van a tener compromisos para la periodicidad de las mediciones, por lo que no es necesario convencerlos de participar.
    \item \textbf{Visitantes externos}: son todas las personas que suben a la montaña pero que no provienen de los Ejidos de la Montaña. Principalmente adultos, con gusto por convivir en ambientes naturales y con las capacidades tecnológicas necesarias para participar (teléfono móvil, acceso a internet y facilidad para el manejo de aplicaciones). En este grupo se incluyen a personas que suben de manera frecuente y son una audiencia objetivo con mucho potencial de participación, como ciclistas, senderistas, campistas y guías de turistas de empresas privadas. Otros visitantes que suben cotidianamente, pero probablemente no estén interesados en participar, son grupos de personas con alto nivel socioeconómico que se dedican a subir en motocross, jeeps y racers, cuyo objetivo es la diversión sin considerar el bienestar de la naturaleza y el impacto que generan en la zona. Finalmente, también hay visitantes externos que suben muy esporádicamente o por ocasión única, algunos suben al evento de la montaña fantasma, otros vienen del interior de la república o simplemente no tienen la costumbre de subir continuamente. Estos tres subgrupos integran una audiencia que requiere más explicación sobre los objetivos del proyecto y de cómo pueden participar y beneficiarse.
    \item \textbf{Visitantes internos}: son personas que forman parte de las localidades de los Ejidos de la Montaña pero que no trabajan con los ejidatarios, suben a realizar actividades como colecta de hongos o caminar. Es un grupo muy heterogéneo que incluye desde niños hasta adultos mayores, con mucho conocimiento sobre la zona de estudio (caminos y rutas, parajes, uso de los recursos naturales del bosque), pero probablemente no cuentan con las capacidades tecnológicas necesarias para participar (teléfono móvil, acceso a internet y facilidad para el manejo de aplicaciones).  Este grupo integra una audiencia que también requiere mucha explicación sobre los objetivos del proyecto y de cómo pueden participar y beneficiarse.
    \item \textbf{Miembros de instituciones gubernamentales y técnicos forestales}: son profesionales encargados de supervisar las actividades de aprovechamiento y manejo forestal, de los recursos del agua y el estado del bosque. Incluye a empleados de Probosque (dependencia estatal), que supervisan constantemente los trabajos realizados en la zona y apoyan en las labores de combate de incendios. También incluye a empleados de otras entidades a nivel federal como (Comisión Nacional Forestal, Comisión Nacional de Áreas Naturales Protegidas, Secretaría de Recursos Naturales, Procuraduría Federal de Protección al Ambiente y Comisión Nacional del Agua). Es un grupo integrado por adultos de entre 30 y 50 años principalmente, con mucho conocimiento sobre la zona de estudio (caminos y rutas, parajes, uso de los recursos naturales del bosque), con las capacidades tecnológicas necesarias para participar (teléfono móvil, acceso a internet y facilidad para el manejo de aplicaciones).  Aunque algunos de ellos suben continuamente, no se sabe si van a tener la disponibilidad de participar aunque sea esporádicamente, ya que siempre tienen prisa.
    \item \textbf{Miembros de la academia}: son estudiantes, profesores e investigadores que realizan actividades de investigación de muy distinta índole en la zona. Algunos suben de manera esporádica y otros suben frecuentemente. Es un grupo integrado por adultos de entre 25 y 60 años principalmente, con mucho conocimiento sobre la zona de estudio (caminos y rutas, parajes, conocimiento sobre el bosque), con las capacidades tecnológicas necesarias para participar (teléfono móvil, acceso a internet y facilidad para el manejo de aplicaciones). Algunas de las instituciones con mayor presencia en la zona son el Colegio de Postgraduados, Universidad Autónoma Chapingo, UNAM, UAM. Aunque algunos de ellos suben continuamente, no se sabe si van a tener la disponibilidad de participar aunque sea esporádicamente, ya que siempre tienen prisa.
\end{enumerate}




\newpage
\subsection{Diseño técnico de monitoreo}

% Construcción de los pluviómetros
% Imagenes del pluviómetro

\begin{enumerate}
    \item Construcción de los pluviómetros: con botellas de PET, y siguiendo los lineamientos de la Norma Mexicana NMX-AA-166/1-SCFI-2013 (\cite{se2013}) o de la Organización Meteorológica Mundial. La máxima capacidad de almacenamiento es de 153 mm y la escala tiene resolución de un milímetro, excepto por los primeros 5 mm que tienen resolución de 0.25 mm. Los pluviómetros se colocaron sobre bases de madera a un metro sobre el nivel del suelo cavando un hoyo de 50 centímetros de profundidad. Para evitar pérdidas de agua por evaporación se utilizan 5 mm de aceite comestible vegetal por pluviómetro.

    \item Los pluviómetros se vacían y registran por el equipo técnico con una frecuencia de un mes (más menos dos días), a menos que sea necesario vaciar con mayor frecuencia. Los participantes envían sus registros sin una frecuencia específica, por lo que sus observaciones son adicionales a las que realiza el equipo técnico. Cada estación de monitoreo cuenta con letreros que poseen la información necesaria para que las personas puedan participar aunque no se les haya dado una capacitación personal. Se cuenta con siete estaciones de monitoreo en un gradiente altitudinal que va de 2683 a 3870 m. 
\end{enumerate}

En la figura \ref{publicidad2}, puede observarse la instalación de la base.

\begin{figure}[h!]
	\centering
	\begin{subfigure} 
		\centering
		\includegraphics[width=0.2\linewidth]{publicidad/2.jpg}
		\caption{Base para el pluviómetro}
		\label{publicidad2}
	\end{subfigure} 
	\begin{subfigure} 
		\centering
		\includegraphics[width=0.2\linewidth]{publicidad/4.jpg}
		\caption{Escala de graduación para precipitación.}
		\label{publicidad4}
	\end{subfigure}
	\caption{Resultados de la instalación de pluviómetros.}
	\label{publicidad24-}
\end{figure}


Así mismo, la distribución de los mismos está marcado en el mapa de la figura \ref{m3}

\begin{figure}[h!]
\centering
  \includegraphics[width=1\textwidth]{m3.jpg}
  \caption{Rutas de monitoreo de lluvia}
  \label{m3}
\end{figure}



\newpage
\subsection{Campaña de difusión con público en general}
Como resultado, se diseñaron y publicaron los recursos ilustrados a continuación.


\subsubsection{Carteles}
La última versión de publicidad, contiene elementos cuidadosamente ordenados, en la figura \ref{publicidad1}.

\begin{figure}[h!]
\centering
  \includegraphics[width=0.5\textwidth]{publicidad/1.jpg}
  \caption{Carteles de publicidad}
  \label{publicidad1}
\end{figure}

\subsubsection{Lonas}
Se elaboró el cartel de la figura \ref{publicidad7}
\begin{figure}[h!]
\centering
  \includegraphics[width=0.5\textwidth]{publicidad/8.pdf}
  \caption{Carteles de publicidad}
  \label{publicidad7}
\end{figure} 

\subsubsection{Trípticos}
Se elaboraron trípticos, mencionados en la figura \ref{publicidad9}
\begin{figure}[h!]
\centering
  \includegraphics[width=0.5\textwidth]{publicidad/9.pdf}
  \caption{Trípticos}
  \label{publicidad9}
\end{figure}

\subsubsection{Redes sociales}
La única red oficial de divulgación, está contenida en la página llamada ``Ciencia Ciudadana para el Monitoreo de Lluvia '', identificandolo con el url: 
\begin{center}
  \url{https://www.facebook.com/profile.php?id=100083233511805}
\end{center}

\subsubsection{Pláticas informativas}

A continuación, se adjuntan las evidencias fotográficas de pláticas informativas con los Ejidatarios:


\begin{figure}[h!]
	\centering
	\begin{subfigure} 
		\centering
		\includegraphics[width=0.5\linewidth]{ejidatarios/1.jpg}
		\caption{Reunión con el ejido de San Dieguito}
		\label{ejidatarios1}
	\end{subfigure} 
	\begin{subfigure} 
		\centering
		\includegraphics[width=0.5\linewidth]{ejidatarios/2.jpg}
		\caption{Reunión con ejido de Santa Catarina del Monte.}
		\label{ejidatarios2}
	\end{subfigure}
	\begin{subfigure} 
		\centering
		\includegraphics[width=0.5\linewidth]{ejidatarios/3.jpg}
		\caption{Reunión con ejido de Nativitas.}
		\label{ejidatarios3}
	\end{subfigure}
	\caption{Reuniones con ejidatarios}
	\label{ejidatarios123}
\end{figure}


\subsubsection{Entrega de obsequios}

\begin{figure}[h!]
\centering
  \includegraphics[width=0.5\textwidth]{publicidad/6.jpg}
  \caption{Cartel de publicidad de entrega de obsequios}
  \label{publicidad6}
\end{figure}


























































\newpage
\section{Desarrollo del código}
El proyecto fue organizado utilizando paquetes y carpetas temáticas dentro del directorio \texttt{lib/}, cada una con una responsabilidad específica. El estado global de la aplicación se gestionó mediante clases como \texttt{AppState}, que centraliza las operaciones CRUD con Firebase, la autenticación del usuario y el manejo de preferencias locales.

La presente sección documenta los resultados obtenidos durante el desarrollo del sistema, detallando tanto la arquitectura del código como las funciones principales implementadas. A través de diagramas, tablas y fragmentos del código fuente, se presenta de manera estructurada cómo se integraron los principios de la POO con las herramientas de Flutter y Firebase para construir una aplicación funcional, robusta y mantenible.


\subsection{Arquitectura de la app}

La estructura de archivos de un proyecto Flutter refleja la arquitectura y organización del desarrollo. A continuación se analiza la estructura general y luego se profundiza en la carpeta \texttt{lib/}, que contiene la lógica principal de la aplicación.

\subsubsection{Estructura general del proyecto}

La siguiente figura muestra la organización principal del proyecto:

\begin{center}
\begin{forest}
for tree={
    font=\ttfamily,
    grow'=0,
    child anchor=west,
    parent anchor=south,
    anchor=west,
    calign=first,
    edge path={
        \noexpand\path [draw, \forestoption{edge}]
        (!u.south west) ++(5pt,0) -- +(5pt,0) |- (.child anchor)\forestoption{edge label};
    },
    before typesetting nodes={
        if n=1
          {insert before={[,phantom]}}
          {}
    },
    fit=band,
    before computing xy={l=1.5em},
}
[TlalocApp
  [android/ Código específico para Android]
  [ios/ Código específico para iOS]
  [lib/ Código fuente principal en Dart
    [main.dart Punto de entrada de la app]
  ]
  [test/ Pruebas automatizadas]
  [pubspec.yaml Configuración (dependencias assets fonts)]
  [assets/ Recursos como imágenes e íconos]
  [build/ Archivos generados automáticamente]
]
\end{forest}
\end{center}

\subsubsection{Estructura detallada en \texttt{lib/}}

Dentro de \texttt{lib/}, se organiza el código fuente principal siguiendo un enfoque modular y funcional. Una estructura típica y aplicada en el proyecto es:

\begin{itemize}
  \item \texttt{main.dart}: punto de entrada de la aplicación.
  \item \texttt{app.dart}: configuración global y navegación.
  \item \texttt{models/}: definición de clases que representan los datos, así como la lógica de estado y control (ej. \texttt{AppState}). 
  \item \texttt{screens/}: pantallas principales de la interfaz.
  \item \texttt{widgets/}: componentes visuales reutilizables. 
\end{itemize}








\subsection{Backend}

\subsubsection{Conexión en GitHub}
Para facilitar el acceso al código fuente y fomentar la colaboración, se creó un repositorio en GitHub. Este repositorio abierto al público, contiene todo el código de la aplicación, incluyendo la estructura del proyecto, los archivos de configuración y las dependencias necesarias para su funcionamiento. La vista del repositorio es mostrado en la figura \ref{m6} y enlazado para su descarga en el siguiente link:
: 
\begin{center}
  \url{https://github.com/Jack55913/TlalocApp.git}
\end{center}

\begin{figure}[ht]
\centering
  \includegraphics[width=1\textwidth]{m6.png}
  \caption{Repositorio del código en GitHub}
  \label{m6}
\end{figure}






\subsubsection{Backend en el código}

\subsubsection*{Función AppState}

En el anexo \ref{anexo:alg1}, se adjunta el código completo de la función del estado de la aplicación, la explicación se detalla a continuación:
\begin{enumerate}
    \item \textbf{Gestión de Usuario y Configuración}
    Autenticación con Google: Usa GoogleSignInProvider para manejar el inicio de sesión y obtiene el usuario actual (currentUser).
    
    \begin{enumerate}
      \item Guardar Preferencias:
    
    \begin{enumerate}
      \item changeParaje(): Cambia y guarda el paraje (ubicación) en SharedPreferences.
    
    \item changeRol(): Cambia y guarda el rol del usuario (ej: "Monitor") en SharedPreferences.
    
   \item Recupera datos guardados al inicializar (init()).
    \end{enumerate}
    \end{enumerate}
    
    \item \textbf{Operaciones CRUD con Firebase Firestore}
    \begin{enumerate}
      \item Crear Mediciones:
    
    \begin{enumerate}
      \item addMeasurement(): Guarda una nueva medición en la colección measurements y calcula automáticamente el valor real (real\_measurements).
    
    \item addRealMeasurement(): 
    
    Guarda una medición directamente en real\_measurements.
    \end{enumerate}
    
    \item Leer Datos:
    
    \begin{enumerate}
    \item getMeasurements(), getRealMeasurements(): Obtienen listas de mediciones.
    
    \item Streams en tiempo real: getMeasurementsStream(), getRealMeasurementsStream(), y variantes por paraje.
    \end{enumerate}
    
    \item Actualizar Mediciones:
    
    \begin{enumerate}
      \item updateMeasurement(): Actualiza una medición existente en measurements.
    
    \item updateRealMeasurement(): Actualiza una medición en real\_measurements.
    \end{enumerate}
    
    \item Eliminar Mediciones:
    
    deleteMeasurement(), deleteRealMeasurement(): Borran documentos de Firestore.
    \end{enumerate}
    
    \item \textbf{Manejo de Imágenes con Firebase Storage}
    \begin{enumerate}
    \item Sube imágenes a Storage desde la web o dispositivos móviles (en \_getMeasurementJson()).
    
    \item Convierte imágenes a URLs descargables o las guarda como base64 si no hay conexión.
    \end{enumerate}
    
    \item \textbf{Lógica del algoritmo}
    Cálculo de Precipitación Real: En addMeasurement(), determina si el pluviómetro fue vaciado y ajusta el valor de la precipitación.
    
    \item \textbf{Gestión de Conexión}
    Verifica conectividad con Connectivity().checkConnectivity() para decidir cómo guardar imágenes.
\end{enumerate}

Sobre los servicios de Firebase, fueron llamados en la tabla \ref{tabt2}


\begin{table}[h!]
    \centering
    \resizebox{\columnwidth}{!}{%
    \begin{tabular}{@{}ccc@{}}
    \toprule
    \textbf{Función}              & \textbf{Servicio de Firebase} & \textbf{Método/Acción}            \\ \midrule
    Actualizar medición           & Firestore                     & update() en documentos            \\
    Crear medición                & Firestore                     & add()                             \\
    Eliminar medición             & Firestore                     & delete()                          \\
    Subir imágenes                & Storage                       & putData(), putFile()              \\
    Escuchar datos en tiempo real & Firestore                     & snapshots()                       \\
    Autenticación                 & Auth                          & currentUser, uid, email, photoURL \\ \bottomrule
    \end{tabular}%
    }
    \caption{Funciones Clave de Firebase en el Código}
    \label{tabt2}
    \end{table}






\subsubsection{Lógica y conexión de la base de datos a Firebase}

 

Para garantizar el almacenamiento eficiente y la disponibilidad continua de los datos registrados mediante \textit{Tláloc App}, se utilizó la plataforma \textbf{Firebase} de Google como solución integral en la nube. Firebase proporciona una infraestructura escalable para aplicaciones móviles y web, permitiendo la sincronización en tiempo real de la información entre los usuarios y la base de datos central.

\subsubsection*{Configuración de Firebase Firestore}

Una vez completada la inicialización descrita en el método, es posible realizar operaciones sobre la base de datos Firestore, como lecturas, escrituras y suscripciones en tiempo real. Por ejemplo, se puede acceder a una colección mediante \texttt{FirebaseFirestore.instance.collection('mediciones')}, y realizar inserciones usando \texttt{add}, lecturas con \texttt{get} o suscripciones usando \texttt{snapshots()}. En este punto, la aplicación Flutter está completamente conectada con Firestore, permitiendo aprovechar la escalabilidad, sincronización en tiempo real y persistencia que ofrece este servicio de Firebase.



La conexión con la base de datos se estableció mediante el servicio \textbf{Cloud Firestore}, el cual ofrece una base de datos NoSQL orientada a documentos. Esta elección permitió estructurar la información de forma flexible, utilizando colecciones y documentos que representan cada uno de los registros de lluvia capturados por los usuarios.

La integración con Firestore se llevó a cabo en el entorno de desarrollo de Flutter, utilizando el paquete oficial \texttt{cloud\_firestore}. La configuración inicial incluyó:

\begin{itemize}
    \item Creación del proyecto en la consola de Firebase.
    \item Registro de las plataformas (Android/iOS/Web) vinculadas a la app.
    \item Generación del archivo \texttt{google-services.json} e inclusión en el proyecto Flutter.
    \item Inicialización de Firebase dentro del archivo \texttt{main.dart} mediante la función:
    
    \texttt{Firebase.initializeApp()}.
\end{itemize}

Los datos almacenados incluyen fecha, valor de precipitación, ubicación geográfica, estado del pluviómetro y metadatos del usuario que realiza la captura mostrados en la figura \ref{m5}.

\begin{figure}[h!]
\centering
  \includegraphics[width=1\textwidth]{m5.png}
  \caption{Base de datos en Firebase Firestore}
  \label{m5}
\end{figure}

\newpage
\subsubsection*{Reglas de seguridad}

Para proteger el acceso a la base de datos, se definieron reglas de seguridad personalizadas en Firebase Security Rules. Estas reglas restringen las operaciones de lectura y escritura únicamente a usuarios autenticados con ciertos roles, como se muestra en el fragmento del anexo \ref{anexo:alg2}.


El sistema de reglas presentado combina tres modelos de control de acceso:

\begin{itemize}
\item Acceso basado en roles jerárquicos (ADMIN, EDITOR, VIEWER, OWNER)
\item Propiedad del dato (\textit{user ownership})
\item Permisos contextuales vinculados a la estructura organizacional
\end{itemize}



Sobre la autenticación, se establece el requisito mínimo para cualquier operación:

\begin{minted}{javascript}
match /{allPaths=**} {
allow read, write: if request.auth.token.roles.size() > 0;
}
\end{minted}

Las reglas específicas para Rowy, son que gestionan el acceso a la configuración administrativa:

\begin{minted}{javascript}
match /{collectionId}/{docId} {
allow read, create, update, delete: if colRule(["roles"],
["ADMIN","EDITOR","VIEWER","OWNER"]);

function colRule(collections, roles) {
return collectionId in collections && hasAnyRole(roles);
}
}
\end{minted}


Para controlar el acceso, edición o eliminación de los datos pluviométricos, se establece lo siguiente:

\begin{minted}{javascript}
match /roles/{rol}/parajes/{paraje}/measurements/{docId} {
allow read: if request.auth != null;
allow create: if request.auth != null;
allow update, delete: if
hasAnyRole(["ADMIN", "OWNER"]) ||
request.auth.uid == resource.data.uploader_id;
}
\end{minted}



El esquema implementado demuestra:

\begin{itemize}
\item Control granular mediante combinación de modelos RBAC y ABAC
\item Flexibilidad para administradores y autonomía para usuarios
\item Seguridad contra accesos no autorizados
\end{itemize}


 
\subsubsection{Implementación del Hosting Web}



El sitio web resultante está vinculado al mismo entorno de base de datos que la aplicación móvil, permitiendo la visualización en tiempo real de los datos recolectados en el Monte Tláloc.


Para habilitar la sincronización con la plataforma Firebase y desplegar la aplicación web en línea, se siguió el siguiente procedimiento, ejecutado desde la terminal del sistema:

\paragraph{1. Instalación de Firebase CLI}

Primero, se instaló la interfaz de línea de comandos de Firebase mediante el siguiente comando de \texttt{npm} (Node Package Manager):

\begin{verbatim}
npm install -g firebase-tools
\end{verbatim}

\paragraph{2. Inicio de sesión con la cuenta de Google}

Para autenticar el entorno de desarrollo con Firebase, se ejecutó:

\begin{verbatim}
firebase login
\end{verbatim}

Este comando abre una ventana en el navegador para iniciar sesión con una cuenta de Google autorizada.

\paragraph{3. Inicialización del proyecto}

Dentro del directorio del proyecto Flutter web compilado (por ejemplo, \texttt{build/web}), se corrió el siguiente comando para vincular el proyecto local con Firebase:

\begin{verbatim}
firebase init
\end{verbatim}

Durante este proceso interactivo, se seleccionaron las siguientes opciones:

\begin{itemize}
    \item \texttt{Hosting: Configure files for Firebase Hosting and (optionally) set up GitHub Action deploys}
    \item Proyecto de Firebase previamente creado en la consola.
    \item Directorio público: \texttt{build/web}
    \item Configuración como aplicación de una sola página (SPA): \texttt{Yes}
    \item Sobrescribir \texttt{index.html}: \texttt{No}
\end{itemize}

\paragraph{4. Compilación de la app Flutter para la web}

Para generar la versión web de la aplicación, se utilizó el siguiente comando:

\begin{verbatim}
flutter build web
\end{verbatim}

Este comando crea los archivos estáticos de la aplicación en la carpeta \texttt{build/web}, listos para ser desplegados.

\paragraph{5. Despliegue del sitio en Firebase Hosting}

Una vez compilado el proyecto, se procedió al despliegue con el comando:

\begin{verbatim}
firebase deploy
\end{verbatim}

Firebase genera una URL pública en el dominio \texttt{.web.app} o \texttt{.firebaseapp.com} donde la app queda disponible instantáneamente.

\paragraph{6. Configuración de dominio personalizado}

Para vincular un dominio propio (para este caso, \texttt{www.tlalocapp.web.app}), se utilizaron los siguientes pasos en la consola web de Firebase:

\begin{enumerate}
    \item Ir a la sección \textbf{Hosting} del proyecto.
    \item Seleccionar \textbf{Agregar dominio personalizado}.
    \item Ingresar el dominio comprado y verificar la propiedad agregando un registro \texttt{TXT} en la configuración DNS del proveedor.
    \item Una vez verificado, Firebase proporciona los registros \texttt{A} necesarios para redirigir el tráfico.
    \item Actualizar la configuración DNS del dominio apuntando a las direcciones IP que Firebase indica.
\end{enumerate}

En pocos minutos, el sitio se encuentra accesible desde un dominio personalizado con HTTPS habilitado automáticamente por Firebase Hosting, pudiéndose administrar sus versiones en el panel representado por la figura \ref{m8}.

\begin{figure}[ht]
\centering
  \includegraphics[width=1\textwidth]{m8.png}
  \caption{Panel de control del Hosting en Firebase}
  \label{m8}
\end{figure}

Nótese que en la configuración de almacenamiento de actualizaciones, se puso 1, para que solamente guarde una versión anterior en caso de un desastre y que pasó de ser tlaloc-3c65c.web.app a tlaloc.web.app.

\begin{center}
  \url{tlaloc.web.app}
\end{center}















\newpage
\subsection{Frontend}

\subsubsection{User Interface (UI)}
La Interfaz de Usuario (UI) fue diseñada con un enfoque centrado en el usuario, priorizando la accesibilidad, la usabilidad y la experiencia intuitiva. La estructura modular y responsiva permite una navegación fluida entre las diferentes funcionalidades, adaptándose a diversos dispositivos y contextos de uso. En la figura \ref{m10}, se detallan las principales secciones que componen la UI, destacando su lógica de implementación, componentes clave y flujos de interacción.

\begin{figure}[h!]
\centering
  \includegraphics[width=1\textwidth]{t4.pdf}
  \caption{Estructura del UI: división del Backend y UI; Login está dentro de UI. autoría propia}
  \label{m10}
\end{figure}

\subsubsection*{Registro de usuario (Onboarding y Log In)}

Cuando la aplicación es iniciada por primera vez, la función main (Anexo \ref{anexo:alg3}) llama al StatelessWidget \texttt{MyApp} (Anexo \ref{anexo:alg4}), que a su vez invoca el StatelessWidget \texttt{ConditionalOnboardingPage} (Anexo \ref{anexo:alg5}). Este ultimo widget es un condicional que determina si la pantalla inicial es el Onboarding (Anexo \ref{anexo:alg6}) o el HomePage (Anexo \ref{anexo:alg7}).

El Onboarding (figura \ref{pantallas1}) es importante porque da la bienvenida al usuario y es la carátula del proyecto. Una vez completada esta sección, se navega a la pantalla de iniciar sesión con cuenta Google, la cual se le llamó \texttt{SignUpWidget} (Anexo \ref{anexo:alg8}). Esta pantalla permite al usuario registrarse o iniciar sesión con su cuenta de Google, utilizando el paquete \texttt{google\_sign\_in} para la autenticación.

El algoritmo del anexo \ref{anexo:alg8}, utiliza dos herramientas clave para lograr un diseño responsivo:

\begin{enumerate}
    \item \textbf{MediaQuery:} Permite obtener el tamaño total de la pantalla.
    \item \textbf{LayoutBuilder:} Permite construir diferentes disposiciones visuales dependiendo del ancho máximo disponible en el contenedor.
\end{enumerate}

A continuación, se muestra el fragmento relevante del código:

\begin{minted}{dart}

final size = MediaQuery.of(context).size;

LayoutBuilder(
  builder: (context, constraints) {
    final isWide = constraints.maxWidth > 800;

    return isWide
      ? Row( ... )   // Para pantallas grandes (modo escritorio)
      : Column( ... ); // Para pantallas pequeñas (modo móvil)
  },
),
\end{minted}



\begin{itemize}
    \item \texttt{MediaQuery.of(context).size} obtiene el tamaño físico de la pantalla.
    \item \texttt{LayoutBuilder} evalúa el espacio disponible (\texttt{constraints.maxWidth}).
    \item Se define una condición lógica: \texttt{isWide}, que evalúa si el ancho disponible es mayor a 800 píxeles. Esta cifra es un umbral común para distinguir entre pantallas móviles y de escritorio.
    \item Si \texttt{isWide} es \texttt{true}, se muestra una interfaz en forma de \texttt{Row}, donde la imagen y el formulario aparecen lado a lado.
    \item Si \texttt{false}, se muestra en forma de \texttt{Column}, uno debajo del otro.
\end{itemize}

\subsubsection*{Inicio de sesión con Google}
El modelo de la pantalla en la figura \ref{pantallas2}, se muestra en el anexo \ref{anexo:alg9}, donde se observa que el usuario puede registrarse o iniciar sesión con su cuenta de Google, y una vez autenticado, se guarda su información en la base de datos de Firebase.


La clase \texttt{GoogleSignInProvider} que extiende de \texttt{ChangeNotifier}, permitiendo su uso con el patrón de arquitectura Provider para gestionar el estado de autenticación del usuario mediante Google y Firebase

\begin{itemize}
    \item Se importa \texttt{firebase\_auth}, \texttt{google\_sign\_in} y \texttt{foundation} para funcionalidades básicas, autenticación y estado reactivo.
    \item Se crean dos instancias privadas:
    \begin{itemize}
        \item \texttt{\_auth} para acceder a los métodos de Firebase Authentication.
        \item \texttt{\_googleSignIn} configurado con el \texttt{clientId} si la aplicación se ejecuta en entorno web (\texttt{kIsWeb}).
    \end{itemize}
\end{itemize}


Resumen del Flujo:

\begin{enumerate}
    \item El usuario inicia sesión con Google.
    \item Se obtienen los tokens y se genera la credencial de Firebase.
    \item Se autentica al usuario en Firebase.
    \item Se maneja el estado de carga para notificar a la interfaz.
    \item Si hay errores, se muestran mensajes específicos.
    \item El usuario puede cerrar sesión limpiamente.
\end{enumerate}

\begin{figure}[h!]
\centering
  \includegraphics[width=0.65\textwidth]{assets/pantallas/1.pdf}
  \caption{Pantalla de bienvenida}
  \label{pantallas1}
\end{figure}

\begin{figure}[h!]
\centering
  \includegraphics[width=0.65\textwidth]{assets/pantallas/3.pdf}
  \caption{Pantalla de registro mediante Google}
  \label{pantallas2}
\end{figure}

\begin{figure}[h!]
\centering
  \includegraphics[width=0.65\textwidth]{assets/pantallas/4.pdf}
  \caption{API de Google}
  \label{pantallas4}
\end{figure}


\newpage
\subsubsection*{Elección de paraje}


En la página CommonSelectPage() (Anexo \ref{anexo:alg10}), se permite al usuario seleccionar el paraje donde se encuentra el pluviómetro. Esta pantalla es accesible desde el menú principal y muestra una lista de los parajes disponibles, permitiendo al usuario elegir uno para registrar sus mediciones.


\paragraph{Lógica del Código QR:}

En la página \texttt{QrCodePage} (Anexo \ref{anexo:alg11}), se implementa la funcionalidad de escaneo de códigos QR para identificar el pluviómetro asociado al paraje seleccionado. Esta pantalla utiliza el paquete \texttt{qr\_code\_scanner} para capturar el código QR y extraer la información del pluviómetro.



El widget QrSelectWidget, encapsula el flujo completo desde la activación del escáner hasta la validación del contenido del código y la actualización del estado global de la aplicación. A continuación se desglosa el proceso y se explica detalladamente cada etapa relevante.

\begin{minted}{dart}
Future<void> _handleQrResult(BuildContext context, String? qrResult) async
\end{minted}

Esta función central se encarga de procesar el resultado del escaneo del código QR. Recibe como parámetro una cadena \texttt{qrResult}, la cual puede ser nula si el escaneo fue fallido o cancelado. La lógica se descompone en los siguientes pasos:

\begin{enumerate}
\item \textbf{Validación de nulo}: Si el valor del QR es nulo, se muestra un cuadro de diálogo de error notificando al usuario que el escaneo falló, sugiriéndole intentar nuevamente o seleccionar manualmente el paraje.

\begin{minted}{dart}
if (qrResult == null) {
await _showErrorDialog(context, title: 'Escaneo fallido', ...);
return;
}
\end{minted}

\item \textbf{Extracción del nombre del paraje}: Si el QR no es nulo, se invoca la función \texttt{\_parseQrResult}, la cual intenta extraer el nombre del paraje desde la URL contenida en el código QR.

\begin{minted}{dart}
final paraje = _parseQrResult(qrResult);
\end{minted}

Esta extracción ocurre únicamente si la cadena contiene el dominio \texttt{tlaloc.web.app}, evitando procesar códigos arbitrarios. Luego, se toma el último segmento del path y se decodifican espacios representados por guiones bajos (\_) o \texttt{\%20}, eliminando posibles espacios innecesarios:

\begin{minted}{dart}
String parseQrResult(String qrResult) {
if (!qrResult.contains('tlaloc.web.app')) return '';
return qrResult.split('/').last.replaceAll(RegExp(r'|%20'), ' ').trim();
}
\end{minted}

\item \textbf{Verificación del paraje}: Una vez extraído el nombre del paraje, se verifica si este existe en el diccionario global \texttt{parajes}. Si no existe, se notifica al usuario que el código no es válido o no corresponde a un paraje habilitado.

\begin{minted}{dart}
if (!parajes.containsKey(paraje)) {
await _showErrorDialog(context, title: 'Código inválido', ...);
return;
}
\end{minted}

\item \textbf{Asignación del nuevo paraje}: Si el paraje es válido, se procede a navegar a la pantalla principal de la aplicación (función \texttt{\_goHome}) y se actualiza el estado global con el nuevo paraje utilizando el patrón de proveedor (\texttt{Provider}) para sincronizarlo con toda la aplicación.

\begin{minted}{dart}
_goHome(context);
Provider.of<AppState>(context, listen: false).changeParaje(paraje);
\end{minted}
\end{enumerate}

El escaneo del QR se desencadena cuando el usuario pulsa sobre el componente visual que representa esta funcionalidad. Se utiliza una ventana emergente tipo diálogo que contiene el escáner mediante el paquete \texttt{mobile\_scanner}:

\begin{minted}{dart}
final qrResult = await showDialog<String>(
context: context,
builder: (context) => _QrScannerDialog(context),
);
\end{minted}

La clase \texttt{\_QrScannerDialog} encapsula el comportamiento del escáner. Este se configura para usar la cámara trasera, con una velocidad de detección normal y sin linterna activada:

\begin{minted}{dart}
MobileScanner(
controller: MobileScannerController(
detectionSpeed: DetectionSpeed.normal,
facing: CameraFacing.back,
torchEnabled: false,
),
onDetect: (barcode) {
if (barcode.barcodes.isNotEmpty) {
Navigator.pop(context, barcode.barcodes.first.rawValue ?? '');
}
},
);
\end{minted}

Cuando se detecta exitosamente un código, se extrae su valor en texto (\texttt{rawValue}) y se cierra el diálogo, devolviendo el resultado a la función principal de manejo, \texttt{\_handleQrResult}.

Esta estructura permite una interacción robusta, ágil y segura para seleccionar automáticamente el paraje correspondiente a un pluviómetro artesanal mediante un código QR, mejorando significativamente la experiencia del usuario y reduciendo errores de selección manual. Además, garantiza que solo se acepten códigos autorizados que correspondan a parajes reconocidos por la aplicación.

Finalmente, para generar los códigos Qr, se hace un mapa llamado ``parajes'', conteniendo el nombre y el ejido:

\begin{minted}{dart}
Map<String, String> parajes = {
  'El Venturero': 'Nativitas',
  'El Jardín': 'Nativitas',
  'Cabaña': 'San Pablo Ixayoc',
  'Cruz de Atenco': 'San Dieguito',
  'Canoas altas': 'San Dieguito',
  'Los Manantiales': 'Tequexquinahuac',
  'Tlaltlatlately': 'Santa Catarina del Monte',
  'Agua de Chiqueros': 'Santa Catarina del Monte',
  'Camino a las Trancas': 'Nativitas',
  'El Cedral': 'San Pablo Ixayoc',
  'Tlachichilpa': 'San Dieguito',
};
\end{minted}

Se pueden utilizar generadores de Qr en línea por ejemplo \url{https://es.qr-code-generator.com/}.

  Sea de ejemplo, la creación del código para el paraje ``El Venturero''. El link será (https://tlaloc.web.app/paraje):
  \begin{center}
    \url{https://tlaloc.web.app/El_Venturero}
  \end{center}
  El resultado de exportar el código Qr a través del generador, está mostrado en la figura \ref{m4}.
  \begin{figure}[h!]
  \centering
    \includegraphics[width=0.5\textwidth]{m4.png}
    \caption{Código Qr para el paraje ``El Venturero''}
    \label{m4}
  \end{figure} 


\begin{figure}[h!]
\centering
  \includegraphics[width=0.65\textwidth]{assets/pantallas/2.pdf}
  \caption{Pantalla de elección de paraje}
  \label{pantallas3}
\end{figure}
\newpage




\newpage
\subsubsection*{Navegación de pantallas principales}

HomePage es un StatefulWidget, lo que permite mantener un estado dinámico, crucial para la navegación. Dentro de su estado (\_HomePageState), se define el índice \_selectedIndex, que controla cuál pantalla debe mostrarse en todo momento. La lista \_screens contiene cinco widgets que representan las diferentes secciones de la app: 
\begin{enumerate}
  \item Pantalla principal (HomeScreen),
  \item Pantalla para agregar datos (AddScreen),
  \item Pantalla de datos (DataScreen), 
  \item Pantalla de gráficas (BarGraph) y
  \item Pantalla de configuración/perfil (ConfigureScreen).
\end{enumerate}

En el método ``initState'', se configura un listener en tiempo real a la colección notifications de Firebase Firestore. Se observa específicamente el documento globalCounter, y si éste existe y tiene un campo count, su valor se asigna a la variable globalNotificationCount mediante setState(). Esto permite reflejar en la interfaz cuántas notificaciones nuevas han llegado. También se define una bandera booleana hasSeenNotifications que se activa cuando el usuario abre la pestaña correspondiente.

En el método build, se construye una interfaz compuesta por dos partes principales: el cuerpo (body) y la barra de navegación inferior (bottomNavigationBar). El cuerpo utiliza un IndexedStack, que muestra solamente el widget cuyo índice coincide con \_selectedIndex, pero mantiene los demás en memoria para preservar su estado (ideal para rendimiento y persistencia de datos en pestañas).

La barra de navegación inferior se implementa con el paquete curved\_navigation\_bar.

Cuando el usuario toca un ítem de la barra, se dispara el método onTap, que actualiza \_selectedIndex y, si se accede a la pantalla de datos (índice 2), también marca que el usuario ya ha visto las notificaciones.

Finalmente, el último ítem de la barra es un CircleAvatar que muestra la foto de perfil del usuario autenticado, si está disponible mediante 
\begin{center}
  FirebaseAuth.instance.currentUser?.photoURL
\end{center}
Si no hay foto, se carga una imagen de respaldo desde internet.

\newpage
\subsubsection*{Pantalla de inicio} 

Está representada por el widget HomeScreen (figura \ref{pantallas16}), se integran diversos componentes visuales modulares cuyo propósito es organizar, presentar e incentivar la participación del usuario dentro de la aplicación Tláloc App. La estructura general de la interfaz se encuentra envuelta en un Scaffold, dentro de un SafeArea que protege los elementos de las zonas inseguras de la pantalla. Se utiliza LayoutBuilder en combinación con un Wrap y SingleChildScrollView para construir una disposición adaptable y desplazable, adecuada tanto para dispositivos móviles como para pantallas anchas.

La barra superior (AppBar) incluye el logotipo de la aplicación, su nombre estilizado mediante AutoSizeText, y dos botones: InfoButton2, que despliega información relevante del sistema, y FluidDialogWidget, que ofrece ventanas emergentes con contenido contextual. En la parte inferior se coloca un botón flotante (Fab) cuya visibilidad se ajusta dinámicamente dependiendo del desplazamiento del usuario, sirviendo como acceso rápido a funciones clave.

En cuanto a la participación activa del usuario, se incluye el widget OneTimeGoogleButton, diseñado para invitar al llenado de un formulario de retroalimentación mediante una cuenta de Google, así como QuickAddWidget, que permite agregar registros de precipitación de forma inmediata desde la pantalla de inicio.

Respecto a la visualización meteorológica, el widget TodayWeatherStyleCard muestra el clima del día, mientras que WeekRainMarker ofrece una representación gráfica del pronóstico de lluvia semanal. Ambas tarjetas permiten al usuario mantenerse informado sobre las condiciones actuales y futuras del tiempo en su región.

La sección educativa e informativa se refuerza con el widget TutorialWidget, el cual se presenta dentro de un GlassContainer, otorgándole una estética translúcida. Esta sección se complementa con un bloque que contiene la tabla de mediciones personales (PersonalMeasures) y generales (GeneralMeasures), también alojadas en un contenedor con estilo de vidrio, permitiendo al usuario consultar sus datos y compararlos con los de la comunidad.

Asimismo, se incorpora el componente TlalocMapData, que muestra las rutas disponibles mediante mapas, permitiendo la visualización de parajes o estaciones registrados. 

La dimensión social y participativa se fomenta a través del widget PhraseCard, que ofrece frases motivacionales o educativas;

TableButton, que dirige a la tabla completa de datos; 

CommunityButton, que enlaza con funciones colaborativas; 

y SocialLinksWidget, el cual facilita el acceso a las redes sociales del proyecto. Todos estos elementos están organizados de forma modular, responsiva y jerárquicamente clara, para favorecer una experiencia de usuario intuitiva y didáctica mostradas en la figura \ref{pantallas6}.
\begin{figure}[h!]
\centering
  \includegraphics[width=0.65\textwidth]{assets/pantallas/6.pdf}
  \caption{Pantalla de inicio}
  \label{pantallas6}
\end{figure}














\newpage
\subsubsection*{Envío de mediciones}


La clase \texttt{AddScreen} (anexo \ref{anexo:alg13}) representa la interfaz principal para el registro de una nueva medición de lluvia. Esta pantalla está diseñada como un formulario dinámico que permite a los usuarios capturar, modificar y enviar datos de precipitación, integrando funcionalidades de autenticación, selección de imágenes, validación de entradas, y almacenamiento en la nube.

Cuando se carga la pantalla, se inicializan los valores por defecto o, en caso de que se edite una medición existente, se precargan los datos previamente registrados. El formulario permite especificar el nombre del responsable de la medición, el valor de precipitación en milímetros, la fecha y hora del registro, así como el estado del pluviómetro (si fue vaciado o no). También se permite el envío opcional de una imagen como evidencia, tomada desde la cámara o seleccionada desde la galería, compatible tanto en entorno web como móvil.

La lógica del botón de guardado evalúa si se trata de una nueva medición o una edición. En el primer caso, llama a \texttt{addMeasurement()} desde el estado global (\texttt{AppState}), mientras que en el segundo ejecuta \texttt{updateMeasurement()} actualizando los campos correspondientes en Firestore y Storage. En ambos casos, se reproduce un sonido de confirmación al finalizar y se redirige al usuario a la pantalla principal.

El diseño de la interfaz emplea componentes personalizados como \texttt{MyTextFormField}, \texttt{RainInputWidget} y \texttt{Datetime}, con una estructura modular y responsiva. Adicionalmente, se incluye una tarjeta que facilita el envío de datos por WhatsApp, y una sección de reinicio que marca si el pluviómetro fue vaciado, funcionalidad crítica para el cálculo de datos reales. Esta pantalla está mostrada en la figura \ref{pantallas5}.

\begin{figure}[h!]
\centering
  \includegraphics[width=0.65\textwidth]{assets/pantallas/5.pdf}
  \caption{Pantalla envío de datos}
  \label{pantallas5}
\end{figure}

A continuación, se muestran los dos casos posibles al usar la app:
\begin{itemize}
  \item Cuando la medición anterior es menor a la actual
  \item Cuando la medición anterior es mayor a la actual
\end{itemize}
En el primer caso, la medición real será la resta del anterior y en el segundo caso, indica que se vació indirectamente, por lo que se conserva tal cuál en el volumen real.


\newpage
\subsubsection*{Bitácora}

La clase \texttt{DataScreen}  (anexo \ref{anexo:alg14}) constituye la interfaz central de la bitácora de mediciones dentro de Tláloc App, permitiendo al usuario consultar registros históricos de precipitación. Esta pantalla combina un diseño adaptable con navegación mediante pestañas y una experiencia optimizada tanto para dispositivos móviles como para pantallas anchas (escritorio o tabletas). Utiliza un \texttt{TabController} con dos secciones principales: ``Acumulados'' y ``Reales'', representando, respectivamente, los valores brutos de precipitación y los datos corregidos mediante un algoritmo basado en el vaciado del pluviómetro.

La interfaz está construida sobre un \texttt{NestedScrollView} con una \texttt{SliverAppBar} que incluye el logotipo de la aplicación, un título destacado, botones de información y una opción para exportar los registros en formato Excel (figura \ref{pantallas15}). El contenido de cada pestaña es renderizado mediante un \texttt{TabBarView}, que carga condicionalmente componentes diferentes según el ancho de la pantalla: una vista tipo lista detallada para escritorios (estilo Master-Detail), o bien listas verticales desplazables con controladores independientes en entornos móviles.

Cada lista accede a los datos a través del estado global (\texttt{AppState}) y se sincroniza con Firestore en tiempo real, permitiendo al usuario desplazarse fluidamente entre los registros disponibles. Además, la lógica de exportación se integra en el botón superior, ofreciendo retroalimentación visual tras completar o fallar la operación. La pantalla garantiza una navegación intuitiva, claridad visual y escalabilidad, consolidándose como el núcleo de consulta y análisis dentro de la plataforma. 

\begin{figure}[h!]
\centering
  \includegraphics[width=1\textwidth]{assets/pantallas/15.pdf}
  \caption{Exportación de datos a Excel}
  \label{pantallas15}
\end{figure}


\paragraph{Caso I}

Véase la figura \ref{pantallas7}, se sube una medición menor a la anterior, implicando que permanezca igual, mostrado en la ecuación \refeq{eq1}:
\begin{equation}
\text{Volumen Real, Caso I: } V_{a}<V_{b}\implies  V_{real} = V_{a}
\label{eq1}
\end{equation}
Donde $V_{a}$ es la medición actual y $V_{b}$ es la medición anterior

Esto da como resultado la pantalla de la figura \ref{pantallas8}.

\begin{figure}[h!]
\centering
  \includegraphics[width=0.65\textwidth]{assets/pantallas/7.pdf}
  \caption{Pantalla de bitácora en volúmenes acumulados (Caso I)}
  \label{pantallas7}
\end{figure}


\begin{figure}[h!]
\centering
  \includegraphics[width=0.65\textwidth]{assets/pantallas/8.pdf}
  \caption{Pantalla de bitácora en volúmenes reales (resultados del caso I)}
  \label{pantallas8}
\end{figure}

\paragraph{Caso II}

La figura \ref{pantallas10}, muestra que la medición actual es mayor que la anterior, esto acciona el algoritmo de la ecuación \refeq{eq2}:
\begin{equation}
\text{Volumen Real, Caso II: } V_{a}>V_{b}\implies V_{real} = V_{a} - V_{b}
\label{eq2}
\end{equation}
Donde $V_{a}$ es la medición actual y $V_{b}$ es la medición anterior.

En la figura \ref{pantallas11}, se muestra que la medición es diferente a la ingresada, esto porque se restó la medición actual de 119 con la medición anterior de 79 dando como resultado 40.


\begin{figure}[h!]
\centering
  \includegraphics[width=0.65\textwidth]{assets/pantallas/10.pdf}
  \caption{Pantalla de bitácora en volúmenes acumulados caso ii}
  \label{pantallas10}
\end{figure}


\begin{figure}[h!]
\centering
  \includegraphics[width=0.65\textwidth]{assets/pantallas/12.pdf}
  \caption{Pantalla de bitácora en volúmenes reales caso ii}
  \label{pantallas11}
\end{figure}






\newpage
\subsubsection*{Estadísticas} 

El widget \texttt{BarGraph} constituye el módulo principal de visualización estadística, proporcionando una gráfica de barras interactiva para analizar datos de precipitación registrados en distintos parajes. Esta pantalla permite comparar de forma clara y dinámica los volúmenes de lluvia registrados en un intervalo de fechas configurable por el usuario. La funcionalidad se adapta a dos modalidades principales: \texttt{acumulado}, donde se visualiza la precipitación total registrada por día sin corrección; y \texttt{real}, que muestra únicamente las mediciones corregidas mediante el procedimiento de reinicio del pluviómetro. 

Estas opciones están integradas mediante un interruptor interactivo que actualiza el origen de los datos en tiempo real, empleando \texttt{StreamBuilder} para obtener flujos de datos directamente desde Firestore.

El diseño visual emplea el paquete \texttt{fl\_chart} para representar las mediciones mediante barras verticales, con etiquetas rotadas para las fechas en el eje horizontal y valores en milímetros en el eje vertical. La gráfica se ajusta dinámicamente en tamaño y escala de acuerdo al número de datos disponibles, con colores adaptativos según el tema de la interfaz. Además, se incluyen herramientas de selección temporal, con opciones predefinidas como ``Esta semana'', ``Este mes'', ``Este año'' y ``Siempre'', o bien un modo \texttt{personalizado} que permite elegir manualmente un rango de fechas utilizando componentes tipo calendario.

Una funcionalidad destacada del componente es la capacidad de exportar la gráfica y sus datos en formato PDF. Para ello, se encapsula la gráfica en un widget \texttt{RepaintBoundary} que permite capturar su representación visual como imagen, la cual se inserta en el documento generado junto con una tabla de datos. Este proceso se complementa con el uso de la librería \texttt{pdf} para el diseño del informe y la librería \texttt{printing} o \texttt{file\_saver} para el guardado o impresión, dependiendo de la plataforma. En entornos web, la descarga se realiza automáticamente mediante un enlace temporal, mientras que en dispositivos móviles se ofrece la opción de previsualizar o guardar el archivo.

Desde una perspectiva funcional, este módulo constituye una herramienta de análisis elemental para el monitoreo participativo, ya que permite comparar de forma gráfica y temporal los registros recopilados en campo. A través del uso de \texttt{Provider}, la lógica se mantiene desacoplada del estado global de la aplicación, permitiendo una experiencia fluida, escalable y visualmente clara. La inclusión de esta herramienta fortalece el uso de la aplicación no solo como medio de recolección, sino también como instrumento de análisis y comunicación de resultados.



\begin{figure}[h!]
\centering
  \includegraphics[width=0.65\textwidth]{assets/pantallas/13.pdf}
  \caption{Pantalla de estadísticas: Gráfica de volúmenes reales }
  \label{pantallas13}
\end{figure}


\begin{figure}[h!]
\centering
  \includegraphics[width=0.65\textwidth]{assets/pantallas/14.pdf}
  \caption{Pantalla de estadísticas: Gráfica de volúmenes reales }
  \label{pantallas14}
\end{figure}

\newpage
\subsubsection*{Perfil} 

La pantalla de \texttt{ConfigureScreen} (figura \ref{pantallas16}) cumple la función de centro de control del usuario, integrando aspectos visuales, estadísticos y funcionales en una sola interfaz. Esta vista permite al usuario autenticado acceder a su información básica (nombre, correo electrónico y foto de perfil), consultar estadísticas clave sobre su participación en el monitoreo (mediciones propias, contribuciones globales y parajes monitoreados), así como modificar aspectos de configuración de la aplicación, compartirla o colaborar en su desarrollo.

La interfaz inicia con un encabezado visual tipo retrato, que integra la imagen de perfil del usuario obtenida mediante \texttt{FirebaseAuth}. En caso de no contar con una imagen, se muestra un ícono genérico de cuenta. Debajo de este encabezado, se despliega el nombre y correo electrónico del usuario en estilos personalizados para mantener coherencia visual con el resto de la aplicación. La obtención de estadísticas personalizadas se realiza mediante un \texttt{FutureBuilder}, el cual consulta a \texttt{AppState} para recuperar los totales de mediciones locales (realizadas por el usuario), contribuciones globales (registradas en la base de datos) y la cantidad de parajes distintos visitados en comparación con el total disponible. Estos valores se presentan mediante tarjetas visuales con diseño limpio y centrado.

La sección inferior se organiza en forma de lista categorizada bajo el título ``Configuración'', y agrupa una serie de acciones relevantes. Entre ellas se incluye la opción de compartir la aplicación mediante \texttt{share\_plus}, enviar retroalimentación por correo electrónico, consultar los términos legales y políticas de privacidad, así como acceder a un cuadro de diálogo “Acerca de” que enlaza con los créditos del equipo de desarrollo, preguntas frecuentes, redes sociales oficiales (Facebook, YouTube), contacto por correo electrónico y el repositorio público en GitHub. Esta modularidad permite tanto al usuario final como a los colaboradores potenciales interactuar con el ecosistema del proyecto.

Finalmente, la pantalla incluye un botón para cerrar sesión de forma segura. Al presionarlo, se ejecuta el método \texttt{logout()} del proveedor de autenticación con Google (\texttt{GoogleSignInProvider}), redirigiendo al usuario nuevamente a la pantalla de introducción \texttt{Onboarding}. Esta navegación se implementa utilizando \texttt{Navigator.pushReplacement} para asegurar que no se pueda volver atrás en la pila de navegación.

En conjunto, esta pantalla fortalece la experiencia del usuario al ofrecer un espacio personalizado, amigable y funcional. A nivel técnico, promueve buenas prácticas como la separación de responsabilidades, el uso eficiente del estado global a través de \texttt{Provider} y la integración fluida de servicios externos como Firebase, correo, redes sociales y repositorios comunitarios. Su diseño también mantiene coherencia estética con el resto de la aplicación mediante el uso consistente de estilos, colores y tipografías.



\begin{figure}[h!]
\centering
  \includegraphics[width=0.6\textwidth]{assets/pantallas/16.pdf}
  \caption{Pantalla de perfil}
  \label{pantallas16}
\end{figure}


























































% todo: Sobre la firma de aplicaciones
% https://developer.android.com/studio/publish/app-signing?hl=es-419

\newpage
\subsubsection{User Experience (UX)}
Se trabajaron en los cinco planos de la metodología

\textbf{Estrategia}

En la base del desarrollo se definió como estrategia fundamental el fortalecimiento del monitoreo participativo de lluvias en el Monte Tláloc, integrando saberes comunitarios con herramientas tecnológicas accesibles. La aplicación se diseñó pensando en los usuarios clave: ejidatarios, estudiantes, académicos y población interesada en los datos ambientales. Se priorizó una experiencia de uso sencilla, enfocada en registrar y consultar precipitaciones de forma colaborativa, permitiendo al mismo tiempo exportar los datos para análisis científicos.

\textbf{Alcance}

El alcance funcional de la aplicación comprende el registro de precipitaciones, la visualización de gráficas por paraje y por día, la exportación de datos en formatos Excel y PDF, así como la autenticación de usuarios mediante Google. En cuanto al contenido, se incluyeron elementos de ayuda, glosarios, créditos y vínculos a fuentes de información. Se limitó la complejidad para mantener la accesibilidad incluso en dispositivos con baja capacidad de procesamiento o conectividad intermitente.

\textbf{Estructura}

La estructura de la aplicación se organizó con base en una navegación tabular, que permite al usuario alternar entre las principales funciones: agregar mediciones, consultar gráficas, ver datos y acceder a la configuración. Esta estructura se alinea con un flujo lógico de uso: primero registrar una medición, luego visualizarla, compararla y finalmente personalizar preferencias o exportar los resultados. Los datos se estructuran por usuario y por paraje, y se sincronizan automáticamente con la nube mediante Firebase.

\textbf{Esqueleto}

El diseño del esqueleto priorizó la claridad y la jerarquía visual. Se emplearon elementos de interfaz como botones flotantes (FAB), tarjetas, listas y barras de navegación inferiores para ofrecer acceso directo a las funciones más usadas. Las gráficas utilizan componentes interactivos como \texttt{fl\_chart} para permitir al usuario explorar los datos acumulados o reales. La disposición de los elementos se adaptó mediante \texttt{MediaQuery} y condicionales de ancho de pantalla para una experiencia responsive en web y dispositivos móviles.

\textbf{Superficie}

La capa visual de la aplicación presenta una estética amigable, con paletas de color basadas en tonos azules, tipografía legible (\texttt{FredokaOne}) y el uso de íconos informativos. Se integró un modo claro y un modo oscuro, para mejorar la accesibilidad visual. El logotipo, los fondos e imágenes fueron diseñados para reforzar la identidad cultural y ecológica del proyecto. Los elementos de retroalimentación, como mensajes de éxito o error, se implementaron mediante \texttt{SnackBar}, asegurando una experiencia informativa y no intrusiva.
Los widgets anterior mencionados, se detallan a continuación en el material design.


\subsubsection*{Material Design}


\paragraph{Tema del dispositivo}

Para permitir al usuario alternar entre tema claro y tema oscuro, se implementa una gestión de estado centralizada con AppState, que expone el tema activo como una propiedad observable mediante ChangeNotifier. El cambio de tema afecta globalmente toda la aplicación y se guarda de forma persistente.

1. Definir los temas claro y oscuro
En el archivo constants.dart (o uno dedicado a temas), se define la apariencia visual para cada modo:


\begin{minted}{dart}
final ThemeData lightTheme = ThemeData(
  brightness: Brightness.light,
  colorScheme: ColorScheme.fromSeed(seedColor: Colors.blue),
  fontFamily: 'FredokaOne',
);

final ThemeData darkTheme = ThemeData(
  brightness: Brightness.dark,
  colorScheme: ColorScheme.fromSeed(
    seedColor: Colors.blue,
    brightness: Brightness.dark,
  ),
  fontFamily: 'FredokaOne',
);
\end{minted}
2. Agregar la propiedad isDarkMode a AppState
El estado global AppState debe incluir una propiedad booleana que indique si el modo oscuro está activo, así como un método para alternarlo:

\begin{minted}{dart}
class AppState extends ChangeNotifier {
  bool _isDarkMode = false;

  bool get isDarkMode => _isDarkMode;

  void toggleTheme() {
    _isDarkMode = !_isDarkMode;
    notifyListeners();
  }
}
\end{minted}
Se guarda esta preferencia en SharedPreferences si quieres mantenerla al reiniciar la app.

3. Aplicar el tema dinámicamente desde MaterialApp
En el archivo app.dart, se consume el AppState con Provider para asignar el tema actual según isDarkMode:

\begin{minted}{dart}
return ChangeNotifierProvider(
  create: (_) => AppState(),
  child: Consumer<AppState>(
    builder: (context, appState, _) {
      return MaterialApp(
        title: 'Tláloc App',
        theme: lightTheme,
        darkTheme: darkTheme,
        themeMode: appState.isDarkMode ? ThemeMode.dark : ThemeMode.light,
        home: const HomeScreen(),
      );
    },
  ),
);
\end{minted}
4. Añadir un Switch en la interfaz para cambiar el tema
En alguna pantalla de configuración (como ConfigureScreen), se puede incluir un interruptor que permita al usuario alternar el modo:

\begin{minted}{dart}
SwitchListTile(
  title: const Text('Modo oscuro'),
  value: appState.isDarkMode,
  onChanged: (_) => appState.toggleTheme(),
),
\end{minted}
Como resultado con esta estructura, el cambio entre modo claro y oscuro se aplica inmediatamente en toda la interfaz, utilizando el sistema de temas de Flutter y el patrón Provider para gestión reactiva del estado. Esta implementación es limpia, escalable y fácilmente extensible para incluir preferencias adicionales (como tamaño de texto, idioma o color personalizado), esto puede ser comparada en la figura \ref{pantallas14} para su modo claro y \ref{pantallas13} para el modo oscuro.



\subsubsection*{Tienda de aplicaciones} 

Como resultado tangible del desarrollo completo de la aplicación, se logró su publicación oficial en la tienda de aplicaciones de Google. Este hito representa la culminación técnica y administrativa del proyecto, permitiendo que cualquier usuario con un dispositivo Android pueda descargar e instalar la aplicación de manera gratuita. La disponibilidad pública también facilita la validación social y científica del sistema de monitoreo participativo, al abrirlo a un público más amplio y permitir la recolección continua de datos por parte de la ciudadanía. La aplicación puede consultarse y descargarse en la siguiente dirección electrónica:
\begin{center}
   \url{https://play.google.com/store/apps/details?id=com.TlalocApps.tlaloc&hl=es_US&gl=US}.
\end{center}

\begin{figure}[h!]
\centering
  \includegraphics[width=0.8\textwidth]{m9.png}
  \caption{Disponibilidad de la aplicación en PlayStore}
  \label{m9}
\end{figure}





















\section{Evaluación del nivel de maduración tecnológica}




























\newpage
\section{Alcance y limitaciones del estudio}
El principal alcance de este estudio es demostrar la viabilidad técnica de integrar la ciencia ciudadana y tecnologías móviles para fortalecer el acceso a datos meteorológicos en zonas de montaña, donde las redes oficiales presentan baja densidad.

El desarrollo de la aplicación incluye la recolección de datos mediante pluviómetros manuales, el registro georreferenciado de observaciones, el almacenamiento en bases de datos en la nube y la visualización de datos de manera intuitiva para los usuarios. Además, se evaluó preliminarmente el nivel de maduración tecnológica (TRL) de la herramienta.

Sin embargo, el estudio presenta limitaciones inherentes a su carácter exploratorio. La validación de los datos recopilados por los usuarios no fue exhaustiva, y la base de usuarios durante las pruebas piloto fue limitada en número y diversidad. Asimismo, las funcionalidades de descarga de estadísticas, algoritmos predictivos o de inteligencia artificial para la detección de anomalías quedaron identificadas como trabajo futuro, pero no fueron desarrolladas en esta primera fase.