\chapter{REVISIÓN DE LITERATURA}
\label{cap:3}



% \newpage





% \section{Acceso a datos meteorológicos de zonas de montaña en México}

% \subsection{Antecedentes en el Monte Tláloc}

% En mayo de 2012, la Universidad Autónoma del Estado de México (UAEM), en coordinación con el INAH y autoridades locales, impulsó la creación de una Estación de Investigación Ambiental y Monitoreo Meteorológico (EIAMM) en el Monte Tláloc, a más de 4,200 ms.n.m. El objetivo era analizar tendencias climáticas regionales y generar alertas tempranas ante eventos extremos, especialmente lluvias intensas que pudieran impactar el Valle de México (\cite{davila2012}). No obstante, no se encontraron publicaciones científicas que reportaran el funcionamiento o resultados de estaciones meteorológicas permanentes en el sitio, lo que indica que el proyecto no se consolidó operativamente.

% \subsection{Panorama a nivel nacional}

% El monitoreo meteorológico en zonas montañosas de México ha sido históricamente limitado. Sin embargo, en las últimas décadas, instituciones académicas y gubernamentales han instalado estaciones automáticas, sensores remotos y radares Doppler en sitios de alta altitud.

% Entre los casos más relevantes destaca el Observatorio Atmosférico Altzomoni del Centro de Ciencias de la Atmósfera de la UNAM, ubicado en el Parque Nacional Izta-Popo. También se han documentado estaciones en volcanes como el Nevado de Toluca, así como en observatorios astronómicos de alta elevación como Vallecitos y la Sierra de San Pedro Mártir.

% Además, la información satelital, aunque menos precisa a nivel local, ofrece una oportunidad para complementar y calibrar datos de superficie en zonas montañosas.

% \subsection{Observatorios Atmosféricos}

% \subsubsection*{Observatorio Atmosférico Altzomoni}

% Este observatorio, operado por la UNAM, se ubica a 4,000 m s.n.m. sobre el cerro Altzomoni, en las faldas del Iztaccíhuatl. Su objetivo es estudiar la composición de la atmósfera alta, el transporte de contaminantes y procesos convectivos entre la tropósfera y la estratósfera, así como el impacto volcánico sobre la atmósfera (\cite{sedema2025}).

% \paragraph{Observatorios astronómicos con estación meteorológica}

% En la Sierra Negra (Puebla), una estación fue instalada entre 2000 y 2008 junto al Gran Telescopio Milimétrico para registrar variables como temperatura, humedad, presión y radiación solar (\cite{granicus2009}).

% En Baja California, tanto el Observatorio Astronómico Nacional como el sitio Vallecitos cuentan con estaciones automáticas para registrar viento, nubes y precipitación (\cite{garcia2020, vallecitos2016}).

% \paragraph{Radares meteorológicos}

% La Red Nacional de Radares Meteorológicos del SMN incluye ocho sistemas Doppler. Entre los más avanzados se encuentran los de Sabancuy (Campeche) y El Mozotal (Chiapas), que utilizan doble polarización para mejorar la resolución de los datos (\cite{smn_radar65_2025}).

% \paragraph{Radar Catedral}

% Este radar Doppler se encuentra en la Sierra de las Cruces, entre 2,220 y 2,400 m s.n.m., y aporta cobertura indispensable para zonas complejas como el Monte Tláloc. Proporciona datos tridimensionales de precipitación, útiles para el análisis de tormentas convectivas y mejora de pronósticos (\cite{ConaguaRadar2025, RadarMeteoWiki2025, GarciaPalomo2008}).

% \begin{itemize}
%   \item CMAX / PPI / BASE – Reflectividad: 220 km
%   \item PROB. GRANIZO: 220 km
%   \item PPI - Velocidad: 120 km
%   \item EBASE / ETOPE – Reflectividad: 220 km
% \end{itemize}

% \paragraph{Estaciones climatológicas}

% La estación del Nevado de Toluca, ubicada a más de 4,000 m s.n.m., ha sido utilizada para calibrar modelos de temperatura por elevación y como referencia en modelación climática de alta montaña (\cite{soto_delgado_2020}).

% \paragraph{Sistemas integrados con satélites y modelos digitales}

% La integración de datos satelitales, sensores de superficie y modelos digitales del terreno permite mejorar las estimaciones climáticas en regiones montañosas. Estos sistemas híbridos ofrecen alta resolución espacial y temporal en variables como temperatura y precipitación (\cite{lei2022combining}).

% \section{Revisión de estudios previos sobre monitoreo ciudadano meteorológico}

% La ciencia ciudadana ha demostrado ser una herramienta eficaz para complementar redes oficiales de monitoreo meteorológico, especialmente en contextos con infraestructura limitada.

% En el Reino Unido, Illingworth et al. desarrollaron una red de monitoreo de precipitaciones con GoogleChart y voluntarios, evidenciando el potencial de las plataformas digitales para la recolección descentralizada de datos (\cite{illingworth2021ukprecipitation}).

% En Estados Unidos, Lute et al. presentaron la aplicación "Mountain Rain or Snow", desarrollada con apoyo de la NASA, que permite reportes ciudadanos en tiempo real sobre si está lloviendo o nevando, mejorando la precisión espacial de los modelos hidrometeorológicos (\cite{lute2021enhancing}).

% En Etiopía, Tedla et al. evaluaron la calidad de los datos recolectados por voluntarios en la cuenca Akaki, subrayando la importancia del entrenamiento para garantizar su utilidad científica (\cite{tedla2022evaluation}).

% En Brasil, Viegas et al. reportaron el uso de pluviómetros reciclados por comunidades rurales en Minas Gerais, destacando su potencial para la gestión local del agua y la agricultura (\cite{viegas2023citizen}).

% La iniciativa mPing, impulsada por la NOAA y el NSSL, ha generado miles de reportes en tiempo real sobre fenómenos meteorológicos, siendo útil para calibrar modelos de radar y mejorar pronósticos locales (\cite{elmore2014mping}).

% En Filipinas, el sistema COMET integró reportes ciudadanos y datos satelitales para fortalecer alertas tempranas ante inundaciones en zonas vulnerables (\cite{okada2019community}).

% Otros estudios resaltan aplicaciones similares para monitoreo de calidad del agua o aguas subterráneas, como las iniciativas descritas por McGinn et al. y Dennis et al., en el marco de los Objetivos de Desarrollo Sostenible (\cite{mcginn2021using, dennis2019groundwater}).


























\section{Acceso a datos meteorológicos de zonas de montaña en México}
% De lo general a lo particular.



\subsection{Panorama a nivel nacional}

Aunque el monitoreo meteorológico en zonas montañosas de México ha sido limitado, en las últimas décadas se han realizado diversos esfuerzos por parte de instituciones académicas, gubernamentales y ciudadanas para instalar estaciones meteorológicas automáticas, sensores remotos, radares Doppler y redes de observación en sitios de gran altitud.

Uno de los casos más conocidos es el Observatorio Atmosférico Altzomoni, propiedad del Centro de Ciencias de la Atmósfera de la UNAM, ubicado en el Parque Nacional Izta-Popo. Además, se han documentado estaciones automáticas y sensores especializados en volcanes como el Nevado de Toluca, así como instalaciones meteorológicas en sitios astronómicos de alta elevación como Vallecitos y el Observatorio Astronómico Nacional en la Sierra de San Pedro Mártir. Finalmente el uso de Información Satelital puede ser una opción inexacta pero representa una oportunidad para su calibración y obtención de datos útiles.



\subsection{Antecedentes en el Monte Tláloc}

En mayo de 2012, la Universidad Autónoma del Estado de México (UAEM), en coordinación con el INAH y autoridades locales, impulsó la creación de una Estación de Investigación Ambiental y Monitoreo Meteorológico (EIAMM) en el Monte Tláloc, a más de 4,120  ms.n.m. El objetivo era analizar tendencias climáticas regionales y generar alertas tempranas ante posibles eventos extremos, especialmente lluvias intensas que pudieran impactar el Valle de México (\cite{davila2012}). Sin embargo, no se encontraron publicaciones científicas posteriores que reportaran el funcionamiento de estaciones meteorológicas permanentes en ese sitio. Por tanto, este proyecto constituye un antecedente significativo, aunque no se llegó a consolidar con datos operativos documentados.

No se encontraron artículos científicos que reporten monitoreo climatológico temporal o permanente en Monte Tláloc mediante estaciones meteorológicas oficiales (CONAGUA, universidades, etc.).

\subsubsection*{Observatorios Atmosféricos}

\paragraph{Observatorio Atmosférico Altzomoni}

El Centro de Ciencias de la Atmósfera (CCA) de la Universidad Nacional Autónoma de México (UNAM) puso en marcha el Observatorio Atmosférico Altzomoni, ubicado a cuatro mil metros de altura sobre el nivel del mar. El observatorio se localiza sobre el cerro Altzomoni, en las faldas del volcán Iztaccíhuatl, dentro del Parque Nacional Izta-Popo, y tiene el propósito de estudiar con detalle la composición de la atmósfera alta, el transporte de contaminantes y los procesos convectivos entre la tropósfera y la estratósfera, así como el impacto de la actividad volcánica en la atmósfera  (\cite{sedema2025}).

\paragraph{Observatorios astronómicos con estación meteorológica}

En la Sierra Negra (Puebla), entre los años 2000 y 2008, se instaló una estación meteorológica de alta precisión en las inmediaciones del Gran Telescopio Milimétrico (LMT). Esta estación registraba temperatura, humedad relativa, presión atmosférica y radiación solar  (\cite{granicus2009}).

En Baja California, el Observatorio Astronómico Nacional (OAN) y el sitio Vallecitos (candidato del Cherenkov Telescope Array) cuentan con estaciones automáticas para medir condiciones locales como velocidad del viento, cobertura de nubes y precipitación (\cite{garcia2020, vallecitos2016}).

\paragraph{Radares meteorológicos}
Son instrumentos utilizados para localizar zonas con lluvia, granizo o nieve en la atmósfera. Además, permiten identificar la velocidad de desplazamiento de las tormentas, las regiones con posible formación de tornados y ayudan a localizar el centro de los ciclones tropicales. La información generada por los radares meteorológicos puede ser asimilada en los modelos numéricos para realizar mejores pronósticos de corto plazo (\cite{smn2025}).

La Red Nacional de Radares Meteorológicos está formada por ocho radares principales con tecnología Doppler. Algunos de ellos, como los ubicados en Sabancuy (Campeche) y El Mozotal (Chiapas), disponen de doble polarización, lo cual mejora la resolución de los datos obtenidos y permite un análisis más preciso de los tipos de partículas, volumen y distribución espacial de los fenómenos atmosféricos (\cite{smn_radar65_2025}).

\paragraph{Radar Catedral} 

El radar meteorológico en la Sierra de las Cruces llamado ``Catedral'' es parte de la Red Nacional de Radares Meteorológicos del SMN-Conagua y utiliza tecnología Doppler para detectar intensidad, tipo (lluvia, nieve, granizo) y movimiento de precipitaciones en tiempo casi real (\cite{ConaguaRadar2025}). Ubicado estratégicamente en la cadena montañosa que divide las cuencas del Valle de México y Toluca  elevaciones entre 2,220 y 2,400 m s.n.m. Este radar aporta cobertura indispensable para zonas complejas como el Monte Tláloc, donde las estaciones meteorológicas convencionales son escasas (\cite{RadarMeteoWiki2025}).

\begin{itemize}
\item Reflectividad máxima compuesta (CMAX): Representa la mayor reflectividad detectada en la columna vertical de la atmósfera, su alcance es de 220 km.
\item Indicador de Plano de Escaneo (PPI): Muestra la reflectividad horizontal a un ángulo fijo de elevación, con un alcance de 220 km.
\item Probabilidad de granizo (PROB. GRANIZO): Estimación basada en reflectividad de la posible presencia de granizo en el área, con un alcance de 220 km.
\item Intensidad de lluvia en la base (BASE): Representa la estimación de precipitación en el nivel más bajo del escaneo, con un alcance de 220 km.
\item Velocidad radial (PPI Velocidad): Muestra el movimiento del viento o la precipitación hacia o desde el radar, con un alcance de 120 km.
\item Eco de base (EBASE): Representa la reflectividad detectada en el barrido más bajo del radar, con un alcance de 220 km.
\item Eco tope (ETOPE): Indica la altura máxima alcanzada por las señales de reflectividad, útil para identificar el desarrollo vertical de tormentas. Su alcance es de 220 km.
\end{itemize}

Gracias a su alcance (100-400 km) y a tecnologías como la doble polarización, el radar permite estimar la forma, tamaño y concentración de partículas precipitantes, además de aportar datos tridimensionales útiles para analizar la estructura de tormentas convectivas, mejorar la predicción de eventos extremos y complementar registros locales (\cite{GarciaPalomo2008}). 



\paragraph{Estaciones climatológicas a gran altitud}
Se definen como un conjunto de instrumentos colocados a la intemperie que permiten medir las variaciones del clima, colocados en sitios estratégicos representativos de ambientes diversos (\cite{conagua_estaciones_climatologicas_2013}).
En México existen N estaciones climatológicas operando y N que dejaron de operarse

La estación climatológica del Nevado de Toluca es la más alta registrada en México, ubicada por encima de los 4,000 metros sobre el nivel del mar. Se utilizó para calibrar modelos de temperatura basada en elevación, aportando datos de referencia para la modelación climática en alta montaña  (\cite{soto_delgado_2020}).

\paragraph{Sistemas integrados con satélites y modelos digitales}

La combinación de datos instrumentales con modelos digitales del terreno y sensores satelitales permite desarrollar sistemas de interpolación espacial de alta precisión. Aunque los datos satelitales por sí solos carecen de la resolución y exactitud necesarias a nivel local, su integración con sensores de superficie mejora notablemente las estimaciones climáticas en zonas montañosas  (\cite{lei2022combining}). Estos sistemas híbridos permiten construir mapas de precipitación y temperatura con alta resolución temporal que aplican para el monitoreo en zonas de montaña.



\section{Monitoreo participativo meteorológico}

% TODO: EJEMPLO DE VERACRUZ DE LA DRA TERESA

Un artículo publicado en RMetS por Samuel Michael Illingworth, titulado ``Red de ciudadanos sobre precipitaciones del Reino Unido: un estudio piloto'', describe cómo se utilizó GoogleChart para llevar un registro colaborativo de las precipitaciones (\cite{illingworth2021ukprecipitation}).

Por otro lado, el artículo ``Enhancing Engagement of Citizen Scientists to Monitor Precipitation Phase'' menciona la aplicación Mountain Rain or Snow, una colaboración financiada por la NASA entre Lynker, Desert Research Institute y la Universidad de Nevada-Reno. Esta aplicación permite a los usuarios reportar si está lloviendo o nevando en un momento y lugar determinados (\cite{lute2021enhancing}).


En el contexto de África, el artículo ``Evaluation of Factors Affecting the Quality of Citizen Science Rainfall Data in Akaki Catchment, Addis Ababa, Ethiopia'' aborda los factores que influyen en la calidad de los datos sobre precipitaciones recolectados por científicos ciudadanos (\cite{tedla2022evaluation}).

Asimismo, la aplicación iFlood, mencionada en el estudio ``Coastal Flooding Generated by Ocean Wave- and Surge-Driven Groundwater Fluctuations on a Sandy Barrier Island'', tiene un enfoque similar, pero está diseñada específicamente para reportar inundaciones (\cite{elgar2021coastal}). 


Otras iniciativas destacan el uso de la ciencia ciudadana para monitorear la calidad del agua y llenar vacíos de datos para cumplir con los Objetivos de Desarrollo Sostenible de las Naciones Unidas, como se describe en el artículo ``Using Citizen Science to Understand River Water Quality While Filling Data Gaps to Meet United Nations Sustainable Development Goal 6 Objectives'' (\cite{mcginn2021using}).

En un enfoque relacionado, el desarrollo de aplicaciones móviles para el monitoreo de aguas subterráneas también ha sido promovido como una herramienta para involucrar a la ciencia ciudadana, según se menciona en el estudio ``Groundwater Mobile App Development to Engage Citizen Science'' (\cite{dennis2019groundwater}).
 

La ciencia ciudadana se ha consolidado como una herramienta eficaz para la recopilación de datos meteorológicos, en especial de precipitaciones, al involucrar a la población general en actividades de monitoreo ambiental.

En el Reino Unido, Illingworth et al. desarrollaron una red de monitoreo de precipitaciones con la participación de ciudadanos voluntarios. Este sistema piloto empleó GoogleChart para registrar colaborativamente datos de lluvia, mostrando que las plataformas digitales pueden facilitar la recolección descentralizada de información meteorológica (\cite{illingworth2021ukprecipitation}).

En Estados Unidos, Lute et al. presentaron la aplicación ``Mountain Rain or Snow'', una herramienta impulsada por la NASA y desarrollada en conjunto con Lynker, Desert Research Institute y la Universidad de Nevada-Reno. Esta aplicación permite a los usuarios reportar en tiempo real si en su ubicación está lloviendo o nevando, mejorando la precisión espacial de los modelos hidrometeorológicos  (\cite{lute2021enhancing}).

En el continente africano, Tedla et al. evaluaron los factores que afectan la calidad de los datos de lluvia recolectados por voluntarios en la cuenca Akaki, en Etiopía. Su estudio subraya la necesidad de entrenamiento y validación de datos para garantizar la utilidad de la ciencia ciudadana en contextos hidrológicos  (\cite{tedla2022evaluation}).

En Brasil, Viegas et al. reportaron una iniciativa de ciencia ciudadana basada en pluviómetros hechos con materiales reciclados para monitorear precipitaciones en comunidades rurales de Minas Gerais. El estudio demuestra que, mediante capacitación y diseño apropiado, los pobladores locales pueden generar datos útiles para la gestión del agua y la agricultura  (\cite{viegas2023citizen}).

Por otro lado, la iniciativa mPing (Meteorological Phenomena Identification Near the Ground), impulsada por la NOAA y el NSSL, ha permitido generar miles de reportes ciudadanos en tiempo real sobre condiciones climáticas en Estados Unidos. La aplicación móvil mPing ha sido estudiada como una herramienta útil para calibrar modelos de radar y mejorar pronósticos meteorológicos locales  (\cite{elmore2014mping}).

Finalmente, en Filipinas, la plataforma ``COMET'' (Community-Based Rainfall Observation for the Mitigation of Extreme Events) demostró que los reportes ciudadanos de lluvia, combinados con imágenes satelitales, pueden mejorar los sistemas de alerta temprana en zonas vulnerables a inundaciones (\cite{okada2019community}).

Estos estudios evidencian que el monitoreo ciudadano de precipitaciones no sólo es factible, sino que puede complementar efectivamente las redes oficiales.




\newpage
\section{Tecnologías actuales en monitoreo climático}
El monitoreo climático en regiones montañosas presenta desafíos particulares debido a su topografía accidentada, inaccesibilidad y variabilidad espacial del clima. En respuesta, se han desarrollado tecnologías instrumentales modernas que permiten una recopilación más precisa, continua y robusta de variables meteorológicas, incluso en condiciones extremas.



\subsection{Estaciones meteorológicas automáticas de alta resolución}

Las estaciones meteorológicas automáticas (EMA) modernas han evolucionado significativamente en los últimos años, incorporando sensores de alta precisión, sistemas de transmisión en tiempo real mediante redes celulares o satelitales, y capacidades de energía autónoma por medio de paneles solares. Estas estaciones permiten registrar variables como precipitación, temperatura, humedad relativa, presión atmosférica y velocidad del viento en intervalos de minutos, lo que las hace especialmente útiles para detectar eventos extremos en montaña (\cite{sabziparvar2019estimation}). Además, su diseño modular y bajo consumo energético las vuelve ideales para su instalación en zonas remotas.

\subsection{Redes de sensores inalámbricos (WSN)}

Las redes de sensores inalámbricos (Wireless Sensor Networks, WSN) permiten desplegar múltiples nodos interconectados en un área geográfica amplia para monitorear variables climáticas de forma distribuida. Estas redes pueden cubrir zonas montañosas de difícil acceso, transmitiendo los datos a una estación base para su análisis. Su capacidad para funcionar con baterías de larga duración y conectividad remota las hace una herramienta prometedora para la vigilancia continua del clima en ambientes hostiles (\cite{matese2009wireless}).

\subsection{Pluviómetros láser y disdrómetros ópticos}

Los pluviómetros láser y disdrómetros ópticos representan un avance significativo en la medición de precipitación. A diferencia de los pluviómetros convencionales, estas tecnologías permiten registrar no solo la cantidad, sino también el tamaño y la velocidad de las gotas, permitiendo caracterizar con mayor precisión la intensidad y tipo de lluvia. Son particularmente útiles en regiones donde la precipitación cambia rápidamente en cortos periodos de tiempo, como ocurre frecuentemente en la montaña (\cite{lenz2017optical}).

\subsection{Sistemas móviles de monitoreo y UAVs}

El uso de vehículos aéreos no tripulados (UAVs o drones) equipados con sensores meteorológicos ha comenzado a ser explorado para monitorear condiciones climáticas en terrenos complejos. Estos sistemas permiten obtener perfiles verticales de temperatura, humedad y velocidad del viento, así como datos puntuales en ubicaciones de difícil acceso. Aunque aún presentan limitaciones en autonomía y carga útil, representan una tecnología emergente de alto potencial  (\cite{villa2016uav}).


\subsection{Información satelital}

El monitoreo de precipitación desde satélites se basa en instrumentos geoestacionarios y polar, que permiten estimaciones en alta resolución espacial y temporal. Una fuente clave es la misión \emph{Global Precipitation Measurement} (GPM), gestionada por NASA y JAXA, que aporta datos cada 2 a 3 horas mediante radar DPR y microondas GMI, ofreciendo cobertura global detallada de estructura y tasa de lluvia (\cite{gpm2014}).

Los satélites geoestacionarios de la serie GOES-R, como GOES-16, equipado con el \emph{Advanced Baseline Imager} (ABI), han mejorado substancialmente las estimaciones cuantificativas de precipitación (QPE). Estudios recientes muestran que modelos de aprendizaje profundo como PERSIANN-cGAN, basados en imágenes multiespectrales del GOES-16 ABI, proporcionan mejoras en precisión mediante detección y estimación de lluvia casi en tiempo real \cite{hayatbini2019}.

Por otro lado, Landsat y Sentinel complementan estos datos, especialmente en zonas con cobertura intermitente. El sistema CHELSA, derivado de modelos reanalíticos y corregido por topografía, ha generado climatologías de precipitación a alta resolución (30 arcs) durante más de tres décadas (\cite{karger2016}). Asimismo, Sentinel-2 y Sentinel-3, aunque no enfocadas en precipitación directa, permiten estimar variables relevantes como humedad del suelo y detección de inundaciones, contribuyendo indirectamente al monitoreo climático (\cite{declaro2024}).

Estas tecnologías presentan limitaciones en zonas montañosas: la resolución temporal de satélites polar-orbitantes (cada varios días) y la interferencia de nubes en sensores ópticos dificultan la captación continua de eventos cortos. La integración de múltiples plataformas satelitales junto con técnicas de fusión y aprendizaje automático es esencial para compensar estas brechas y mejorar la precisión en territorios de difícil acceso como el Monte Tláloc.


\section{Generación de aplicaciones móviles}

Entre los avances más destacados está el proyecto Cooperative Open Online Landslide Repository (COOLR), que utiliza las aplicaciones \textbf{Landslide Reporter} y \textbf{Landslide Viewer}. Estas herramientas invitan a científicos ciudadanos de todo el mundo a contribuir con reportes de eventos de deslizamientos de tierra, mejorando la investigación y predicción de desastres.(\cite{coolr2021} )

Además, la aplicación \textbf{Sense-it} ofrece un kit de herramientas de sensores para la investigación ciudadana, funcionando como una herramienta educativa en dispositivos Android.(\cite{van2017senseit})


Otra categoría importante son los diarios de lluvia, como la aplicación \textbf{Rain Tracker} de Callum Hill, que permite a los usuarios gestionar sus propios datos de precipitaciones, aunque estos no son accesibles al público  (\cite{hill2021raintracker}).

Aplicaciones similares encontradas en el mercado de aplicaciones a junio de 2025, como \textbf{Pocket Rain Gauge}, \textbf{Rainlogger} y \textbf{Rain Recorder} registran las precipitaciones en función de la ubicación mediante GPS, pero tampoco ofrecen un sistema de registro público de los datos.








\section{Importancia hidrológica de las zonas de montaña}



Las zonas de montaña desempeñan múltiples funciones hidrológicas críticas. En primer lugar, actúan como \textbf{fuentes de agua natural}, captando precipitación y alimentando los principales ríos que abastecen a regiones aguas abajo, especialmente en zonas áridas o semiáridas  (\cite{viviroli2007mountain}). Además, la elevación y topografía compleja favorecen procesos de condensación y acumulación de nieve, que posteriormente se transforma en escorrentía estacional  (\cite{immerzeel2020importance}). 

En segundo lugar, estas zonas representan \textbf{reservorios de biodiversidad y hábitats} para especies sensibles al clima, cuya salud está directamente relacionada con la disponibilidad hídrica. También, los suelos y coberturas vegetales en montaña tienen un papel en la regulación del ciclo del agua, facilitando la infiltración y evitando escorrentías extremas  (\cite{buytaert2011mountain}).

Asimismo, se consideran áreas sensibles al cambio climático, donde las alteraciones en temperatura o precipitación pueden tener consecuencias desproporcionadas sobre la oferta de agua, tanto local como regional  (\cite{beniston2003climatic}). La dinámica hidrológica de estas regiones influye en los servicios ecosistémicos, la agricultura de laderas, y la seguridad hídrica de millones de personas.

Este estudio, centrado en el monitoreo de precipitaciones mediante ciencia ciudadana en zonas altas como el Monte Tláloc, contribuye indirectamente a la comprensión de estos procesos al generar datos valiosos para validar modelos hidrológicos y climáticos en áreas donde las estaciones meteorológicas tradicionales son escasas o inexistentes.



