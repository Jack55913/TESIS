\newgeometry{left=4cm, right=2.5cm, top=2.5cm, bottom=2.5cm, marginparwidth=0pt, headsep=0pt}
\chapter{INTRODUCCIÓN}
\pagenumbering{arabic}
\setcounter{page}{1}

Las montañas actúan como barreras orográficas que obligan a las nubes a elevarse y enfriarse, lo que genera precipitaciones más abundantes en comparación con los valles circundantes (\cite{CruzMiranda2021}). Sin embargo, la medición de estas lluvias en zonas montañosas suele ser limitada debido a su difícil acceso y a la falta de vigilancia para el mantenimiento de instrumentos de medición (\cite{aparicio1992}). Esta situación es crítica en ecosistemas como los bosques, donde la información sobre precipitación resulta fundamental para su conservación y manejo. Los bosques templados de montaña, como los del Monte Tláloc en México, enfrentan múltiples desafíos, entre ellos la deforestación, la fragmentación del hábitat y los efectos del cambio climático. Este último ha generado alteraciones en los patrones de precipitación y temperatura, impactando negativamente la biodiversidad, los ciclos hidrológicos y los servicios ecosistémicos que estos bosques proporcionan (\cite{gonzalez2016}).

Actualmente el ejido tiene participación en los programas forestales con 1628 hectáreas de superficie forestal en Monte Tláloc y ante la CONAFOR se tienen registradas 248 hectáreas para aprovechamiento forestal (\cite{nava2014}). Según (\cite{lopez2023}), en el Monte Tláloc, las extracciones por nivel altitudinal parecen estar relacionadas con la elevada mortalidad de árboles en las categorías más pequeñas, sin embargo, la intensidad y nivel de extracción de madera no parecen representar una amenaza que ponga en riesgo la viabilidad poblacional de Abies religiosa; la categoría diamétrica más pequeña parece beneficiarse de las aperturas debidas a las extracciones. El cambio climático repercute de diferente manera en el crecimiento de los bosques de montaña en los extremos altitudinales de su distribución; así como su relación con el proceso de migración de fauna, y finalmente los ecosistemas terminan siendo amenazados (\cite{hernandez2021}).



\newpage
\section{Planteamiento del problema}

Ante la falta de datos sobre precipitación en estas zonas, se requiere el desarrollo de estrategias innovadoras que permitan superar las limitaciones técnicas y logísticas, involucrando a las comunidades locales en la generación y uso de información. Es necesario recurrir a estrategias que incorporen a la población en la generación de información y en su utilización para el manejo de los ecosistemas (\cite{hubp1990}).


En México, las redes oficiales de monitoreo hidrometeorológico, como las operadas por la Comisión Nacional del Agua (CONAGUA), presentan una cobertura limitada en muchas regiones de montaña, donde los microclimas pueden variar significativamente en distancias cortas (\cite{rosas2021}).


El Monte Tláloc, ubicado en la zona montañosa del oriente del Valle de México, es un ejemplo de ello: su importancia ambiental, histórica y cultural contrasta con la escasa información climática precisa y en tiempo real disponible para la comunidad local, investigadores y tomadores de decisiones. Esta falta de datos puntuales dificulta la gestión sustentable del agua, la prevención de riesgos y el análisis del cambio climático a escala local.

Las aplicaciones disponibles para la recolección de datos meteorológicos suelen ser de uso profesional, poco accesibles o no están diseñadas para fomentar la participación ciudadana en contextos rurales o de baja conectividad. Esto genera una brecha entre el potencial de colaboración ciudadana y las herramientas disponibles para lograrlo.

Ante este panorama, surge la necesidad de desarrollar una aplicación multiplataforma intuitiva, accesible y robusta, que aproveche el poder de la ciencia ciudadana para llenar los vacíos de información sobre la precipitación en el Monte Tláloc. Dicha aplicación debe facilitar la recolección, visualización y validación de datos por parte de usuarios no expertos, promoviendo la generación de conocimiento colectivo, la educación ambiental y la participación activa de la comunidad en temas de gestión hídrica y climática.



\section{Justificación}
Se identifica la necesidad de crear un instrumento para la captura y envío de datos pluviales que sea accesible, participativo y que garantice la disponibilidad de la información obtenida para su análisis y toma de decisiones. Este instrumento debe ser sencillo de usar y estar diseñado específicamente para el público objetivo: los ejidatarios. Ellos, a través de su conocimiento del territorio y participación activa, pueden convertirse en aliados estratégicos para la recolección continua y precisa de datos.

La aplicación desarrollada se plantea como una solución innovadora que responde a esta necesidad. Su diseño intuitivo permite que usuarios con conocimientos tecnológicos básicos puedan capturar y enviar información sobre las precipitaciones de manera rápida y eficiente. Además, al integrar elementos de ciencia ciudadana, se fomenta la colaboración activa de las comunidades locales, fortaleciendo su empoderamiento y compromiso con la conservación de los recursos hídricos. Desde un enfoque técnico, el proyecto destaca por su carácter práctico y adaptable.

La app aprovecha tecnologías modernas para registrar datos de lluvia, optimizando la recopilación de información en tiempo real, y reduciendo costos asociados a equipos de medición tradicionales. Al centralizar y analizar estos datos en una plataforma digital, se genera un repositorio de información confiable que puede ser utilizado por investigadores, autoridades locales y los mismos ejidatarios para tomar
decisiones fundamentadas.

Por último, la disponibilidad de esta información en un formato accesible y visualmente comprensible contribuye a sensibilizar a los usuarios sobre la importancia de monitorear los patrones de lluvia, facilitando su uso en estrategias de manejo hídrico, planificación agrícola y mitigación de riesgos climáticos. De esta forma, el proyecto no solo soluciona un problema técnico, sino que también tiene un
impacto social y ambiental significativo.



\section{Hipótesis}
\subsection{Hipótesis general (H)}

La implementación de una aplicación multiplataforma basada en ciencia ciudadana incrementa significativamente la frecuencia y precisión de los reportes de lluvia en la región del Monte Tláloc, al promover la participación activa de los habitantes locales mediante herramientas digitales accesibles. Esto permite generar información meteorológica complementaria a la de las estaciones profesionales, mejorando la caracterización espacial y temporal de los eventos de precipitación.


\subsubsection{Hipótesis nula ($H_0$)}
La implementación de una aplicación multiplataforma basada en ciencia ciudadana \textbf{no tiene un efecto significativo} en la frecuencia ni en la precisión de los reportes de lluvia en la región del Monte Tláloc, ni contribuye sustancialmente a la caracterización de los eventos de precipitación.

\subsubsection{Hipótesis alternativa ($H_1$)}
La implementación de una aplicación multiplataforma basada en ciencia ciudadana \textbf{sí mejora significativamente} la frecuencia y precisión de los reportes de lluvia en la región del Monte Tláloc, y contribuye a una mejor caracterización de los eventos de precipitación respecto a los datos generados únicamente por estaciones profesionales.











% \section{Contribuciones de esta tesis}

% Este trabajo de tesis contribuye al campo del desarrollo tecnológico, la ciencia ciudadana y la meteorología local mediante la creación de una aplicación multiplataforma diseñada específicamente para el monitoreo participativo de lluvia en el Monte Tláloc. La solución propuesta integra tecnologías móviles modernas con servicios en la nube y diseño centrado en el usuario, permitiendo que cualquier ciudadano pueda registrar datos de precipitación de manera sencilla, segura y estructurada. Esta contribución tiene un impacto directo en la generación de datos alternativos en regiones donde la infraestructura meteorológica es escasa o limitada, y donde los fenómenos hidrometeorológicos presentan comportamientos complejos.

% Desde el punto de vista técnico, la tesis presenta una arquitectura modular desarrollada con Flutter, integrando funcionalidades clave como, sincronización con Firebase, visualización gráfica de estadísticas y un sistema para validar la veracidad de las mediciones con base en algoritmos desarrollados para la interpretación de datos de pluviómetros caseros. Se propone también una metodología de evaluación del nivel de maduración tecnológica (TRL) aplicada a aplicaciones de ciencia ciudadana, lo cual permite medir de forma objetiva el avance y aplicabilidad real del sistema desarrollado.

% Además, este trabajo representa un esfuerzo por brindar el acceso a las tecnologías de monitoreo ambiental, empoderando a las comunidades rurales al integrarlas como agentes activos en la recolección de datos climáticos, al tiempo que fortalece los vínculos entre el conocimiento científico y la sabiduría local. Finalmente, se generan aportes a futuras investigaciones en temas relacionados con aplicaciones móviles para monitoreo ambiental, ciencia abierta y educación en contextos rurales, abriendo camino a iniciativas de colaboración interdisciplinaria entre desarrolladores, científicos, comunidades y tomadores de decisiones.












\section{Alcance de la tesis}

Esta tesis se enmarca dentro del desarrollo tecnológico participativo aplicado al monitoreo climático en zonas rurales de difícil acceso. En particular, se concentra en el diseño, implementación y validación de una aplicación multiplataforma basada en principios de ciencia ciudadana, con el objetivo de mejorar la frecuencia, cobertura y trazabilidad de los reportes de lluvia en la región del Monte Tláloc.

El alcance del trabajo está delimitado a la creación de un sistema funcional compuesto por una herramienta digital (Tláloc App), un protocolo de monitoreo participativo validado con comunidades locales, y una primera aproximación de análisis de datos meteorológicos generados. No se abordan en esta tesis predicciones climáticas avanzadas, análisis hidrológicos derivados ni integración con sistemas nacionales meteorológicos, aunque se plantean como trabajo futuro.

Con base en lo anterior, esta investigación abarca los siguientes logros concretos:

\begin{itemize}
  \item \textbf{Demostración técnica de la viabilidad.} Se valida que es posible registrar mediciones meteorológicas confiables mediante dispositivos móviles de bajo costo y pluviómetros artesanales, sin requerir formación técnica previa por parte de los usuarios.
  
  \item \textbf{Desarrollo de una arquitectura modular y escalable.} Se implementa una aplicación desarrollada en Flutter que permite su ejecución en múltiples plataformas (Android y Web), integrando servicios en la nube como Firebase para autenticación, almacenamiento y sincronización en tiempo real.
  
  \item \textbf{Trazabilidad.} Cada medición capturada incluye coordenadas geográficas y marcas temporales, lo que permite realizar análisis espaciales y temporales sobre la distribución de la lluvia.
  
  \item \textbf{Visualización y exportación de datos.} Se incluyen herramientas gráficas interactivas, filtros temporales y funciones de exportación en formatos como Excel y PDF, pensadas para su integración en informes técnicos o investigaciones científicas.
  
  \item \textbf{Validación en campo y estimación de madurez tecnológica.} El prototipo fue probado exitosamente en condiciones reales con usuarios de la comunidad del Monte Tláloc, alcanzando un nivel de madurez tecnológica estimado en TRL 8, correspondiente a una solución validada y comercializable.
\end{itemize}

Este estudio representa un esfuerzo interdisciplinario que combina ingeniería de software, participación comunitaria y gestión ambiental, proponiendo una metodología replicable para la generación participativa de datos meteorológicos en regiones donde los sistemas tradicionales no tienen cobertura.




\section{Organización}

Este documento está estructurado de acuerdo con el proceso integral de investigación, desarrollo y validación de una aplicación multiplataforma orientada a la ciencia ciudadana, enfocada en el monitoreo de precipitaciones en el Monte Tláloc. La redacción mantiene una notación consistente en todo el texto, con excepciones claramente indicadas cuando es necesario. Al final del documento se presenta una bibliografía acumulativa con todas las referencias consultadas.

A continuación, se describe brevemente el contenido de cada capítulo:

\begin{itemize}
    \item \textbf{Capítulo 1: Introducción.} Presenta el planteamiento del problema, el contexto geográfico del Monte Tláloc, la justificación del proyecto, la hipótesis de trabajo, el esquema general del documento y las limitaciones del estudio.
    
    \item \textbf{Capítulo 2: Objetivos.} Define el objetivo general y los objetivos específicos que guiaron la realización del presente trabajo de investigación y desarrollo tecnológico.
    
    \item \textbf{Capítulo 3: Revisión de literatura.} Sistematiza los conceptos clave necesarios para comprender el proyecto, incluyendo el acceso a datos meteorológicos en zonas montañosas de México, experiencias previas en monitoreo ciudadano, tecnologías digitales aplicadas al monitoreo climático, y el papel de la ciencia ciudadana en el estudio ambiental.
    
    \item \textbf{Capítulo 4: Materiales y Métodos.} Detalla los recursos físicos empleados, la arquitectura tecnológica implementada, el protocolo de monitoreo participativo validado, el proceso de desarrollo de la aplicación Tláloc App y la metodología utilizada para estimar su nivel de madurez tecnológica.
    
    \item \textbf{Capítulo 5: Resultados.} Presenta los hallazgos derivados del monitoreo participativo, el desempeño funcional de la aplicación y la evaluación obtenida de su nivel de desarrollo tecnológico (TRL).
    
    \item \textbf{Capítulo 6: Conclusiones y trabajo futuro.} Resume los aportes de la tesis al campo del monitoreo ambiental, las principales contribuciones de la tesis, plantea escenarios de expansión hacia otras regiones y propone líneas futuras de desarrollo.
\end{itemize}


