\chapter{CONCLUSIONES FINALES Y TRABAJO FUTURO}



\section{Aportes de la tesis al monitoreo ambiental}
\subsection{Monitoreo ambiental}
\subsection{Desarrollo de apps}

\begin{itemize}
  \item La implementación del principio de diseño responsivo en Flutter, mediante el uso de \texttt{MediaQuery} y \texttt{LayoutBuilder}, que permite desarrollar interfaces flexibles y adaptativas. En el ejemplo analizado, esta lógica es fundamental para ofrecer una experiencia visual adecuada tanto en celulares como en computadoras. Esto demuestra cómo es posible construir aplicaciones universales a partir de una sola base de código.
  \item 

\end{itemize}
\section{Limitaciones encontradas}

En México, la instalación de (tecnología avanzada) no es viable por vandalismo (véase figura \ref{publicidad5}), ni incendios, ni económico.

% https://informativonacional.com.mx/ciencia_ciudadana_registra_lluvias_en_monte_tlaloc_piden_respeto_y_cuidado_de_los_equipos_en_area_forestal-e3Tcye3DY5e3A.html

\begin{figure}[h!]
\centering
  \includegraphics[width=0.8\textwidth]{publicidad/5.jpg}
  \caption{CIENCIA CIUDADANA REGISTRA LLUVIAS EN MONTE TLALOC, PIDEN RESPETO Y
  CUIDADO DE LOS EQUIPOS EN AREA FORESTAL, Fuente: Noticiero}
  \label{publicidad5}
\end{figure}

\section{Posibilidades de expansión a otras zonas}
\section{Trabajo futuro:}

\subsection{App offline}
\subsection{IA para detección de anomalías}
\subsection{Predicción climática}
\subsection{Comunidad activa de usuarios:} 








CÓMO MANTENER A LARGO PLAZO EL USO DE LA APP










