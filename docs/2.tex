\chapter{OBJETIVOS}
\section{Objetivo General}

Generar un instrumento tecnológico en forma de una aplicación multiplataforma que facilite la participación ciudadana en la recopilación de datos de precipitación y garantice el acceso abierto a esta información, fomentando la ciencia ciudadana en el Monte Tláloc.

\section{Objetivos Específicos}

\begin{itemize}
    \item Definir el protocolo de monitoreo participativo: Diseñar y establecer un protocolo claro y funcional para la recolección de datos de lluvia, utilizando una red de pluviómetros distribuidos estratégicamente en el Monte Tláloc, asegurando la
    precisión y confiabilidad de los datos recopilados.
    \item Desarrollar el código: Implementar el desarrollo de una aplicación móvil y una plataforma web que integren funcionalidades intuitivas, y herramientas interactivas que permitan a los usuarios registrar, consultar y analizar datos de precipitación de
    manera sencilla y segura.
    \item Evaluar el nivel de maduración tecnológica (Technology Readiness Level, TRL) de la aplicación desarrollada mediante el análisis de sus funcionalidades, estabilidad, precisión en la recolección de datos, rendimiento multiplataforma y experiencia del usuario, con el fin de determinar en qué etapa del desarrollo se encuentra y establecer su viabilidad para una implementación real en el entorno del Monte Tláloc, tomando como referencia la escala de TRL y utilizando pruebas piloto con participación ciudadana como evidencia de validación.
\end{itemize}