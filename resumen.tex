% From mitthesis package
% Version: 1.01, 2023/06/19
% Documentation: https://ctan.org/pkg/mitthesis
%
% The abstract environment creates all the required headers and footnote. 
% You only need to add the text of the abstract itself.
%
% Approximately 500 words or less; try not to use formulas or special characters
% If you don't want an initial indentation, do \noindent at the start of the abstract
\begin{abstract}
% use \input rather than \include because we're inside an environment

La evapotranspiración es uno de los principales componentes del ciclo
hidrológico y de vital importancia en la planeación y operación de zonas de riego,
pues de ella dependen en gran medida los cálculos para conocer las necesidades
hídricas de los cultivos para evitar la sub o sobreestimación de la lámina de riego
aplicada, sin embargo su estudio resulta complicado pues la medición depende
de dos procesos separados, evaporación y transpiración los cuales varían
espacial y temporalmente, por lo que existen métodos para la estimación de ésta
con ayuda de información meteorológica.

\textbf{Palabras-Clave:} Ecuaciones de Navier-Stokes en R3, Inteligencia Artificial, Hidropónia, Aeropónia, Agricultura Vertical
\end{abstract}

\renewcommand\abstractname{ABSTRACT}
\begin{abstract}
	Recent experiments show that active fluids stirred by swimming bacteria or ATPpowered microtubule networks can exhibit complex flow dynamics and emergent pattern scale selection. Here, I will investigate a simplified phenomenological approach
	to 'active turbulence', a chaotic non-equilibrium steady-state in which the solvent
	flow develops a dominant vortex size. This approach generalizes the incompressible
	Navier-Stokes equations by accounting for active stresses through a linear instability
	mechanism, in contrast to externally driven classical turbulence. This minimal model
	can reproduce experimentally observed velocity statistics and is analytically tractable
	in planar and curved geometry. Exact stationary bulk solutions include Abrikosovtype vortex lattices in 2D and chiral Beltrami fields in 3D. Numerical simulations
	for a plane Couette shear geometry predict a low viscosity phase mediated by stress
	defects, in qualitative agreement with recent experiments on bacterial suspensions.
	Considering the active analog of Stokes' second problem, our numerical analysis predicts that a periodically rotating ring will oscillate at a higher frequency in an active
	fluid than in a passive fluid, due to an activity-induced reduction of the fluid inertia.
	The model readily generalizes to curved geometries. On a two-sphere, we present exact stationary solutions and predict a new type of upward energy transfer mechanism
	realized through the formation of vortex chains, rather than the merging of vortices,
	as expected from classical 2D turbulence. In 3D simulations on periodic domains, we
	observe spontaneous mirror-symmetry breaking realized through Beltrami-like flows,
	which give rise to upward energy transfer, in contrast to the classical direct Richardson cascade. Our analysis of triadic interactions supports this numerical prediction
	by establishing an analogy with forced rigid body dynamics and reveals a previously
	unknown triad invariant for classical turbulence.
	
\textbf{Key-Words:} Navier-Stokes equations in R3, Artificial Intelligence, Hydroponics, Aeroponics, Vertical Agriculture
\end{abstract}