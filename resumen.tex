% From mitthesis package
% Version: 1.01, 2023/06/19
% Documentation: https://ctan.org/pkg/mitthesis
%
% The abstract environment creates all the required headers and footnote. 
% You only need to add the text of the abstract itself.
%
% Approximately 500 words or less; try not to use formulas or special characters
% If you don't want an initial indentation, do \noindent at the start of the abstract
\begin{abstract}
% use \input rather than \include because we're inside an environment

La evapotranspiración es uno de los principales componentes del ciclo
hidrológico y de vital importancia en la planeación y operación de zonas de riego,
pues de ella dependen en gran medida los cálculos para conocer las necesidades
hídricas de los cultivos para evitar la sub o sobreestimación de la lámina de riego
aplicada, sin embargo su estudio resulta complicado pues la medición depende
de dos procesos separados, evaporación y transpiración los cuales varían
espacial y temporalmente, por lo que existen métodos para la estimación de ésta
con ayuda de información meteorológica.

\textbf{Palabras-Clave:} Ecuaciones de Navier-Stokes en R3, Inteligencia Artificial, Hidropónia, Aeropónia, Agricultura Vertical
\end{abstract}

\renewcommand\abstractname{ABSTRACT}
\begin{abstract}
	La evapotranspiración es uno de los principales componentes del ciclo
hidrológico y de vital importancia en la planeación y operación de zonas de riego,
pues de ella dependen en gran medida los cálculos para conocer las necesidades
hídricas de los cultivos para evitar la sub o sobreestimación de la lámina de riego
aplicada, sin embargo su estudio resulta complicado pues la medición depende
de dos procesos separados, evaporación y transpiración los cuales varían
espacial y temporalmente, por lo que existen métodos para la estimación de ésta
con ayuda de información meteorológica.
Artificiales, Modelación, Agrometeorología. \footnote{Text from Holman (1876): \doi{10.2307/25138434}.}  
\textbf{Key-Words:} Ecuaciones de Navier-Stokes en R3, Inteligencia Artificial, Hidropónia, Aeropónia, Agricultura Vertical
\end{abstract}