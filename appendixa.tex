\chapter{ANEXO 1. Desarrollo del Código}

\section{Algoritmo del AppState}
\label{anexo:alg1}

\begin{minted}{dart}
import 'dart:convert';
import 'dart:io';
import 'package:connectivity_plus/connectivity_plus.dart';
import 'package:firebase_auth/firebase_auth.dart';
import 'package:firebase_storage/firebase_storage.dart';
import 'package:flutter/foundation.dart';
import 'package:shared_preferences/shared_preferences.dart';
import 'package:cloud_firestore/cloud_firestore.dart';
import 'package:tlaloc/src/models/google_sign_in.dart';

class Measurement {
  final String? uploader;
  final double? precipitation;
  final DateTime? dateTime;
  final String id;
  final String? imageUrl;
  final String? avatarUrl;
  final String? uploaderId;
  final bool? pluviometer;

  Measurement({
    this.uploader,
    this.uploaderId,
    this.precipitation,
    this.dateTime,
    required this.id,
    this.imageUrl,
    this.avatarUrl,
    this.pluviometer,
  });

  factory Measurement.fromJson(Map<String, dynamic> json, String id) {
    Timestamp timestamp = json['time'];
    return Measurement(
      uploader: json['uploader_name'],
      uploaderId: json['uploader_id'] as String? ?? 'unknown',
      precipitation: json['precipitation'],
      dateTime: timestamp.toDate(),
      id: id,
      imageUrl: json['image'],
      avatarUrl: json['avatar_url'],
      pluviometer: json['pluviometer_state'],
    );
  }
}

class AppState extends ChangeNotifier {
  Uint8List? _newWebImage;
  Uint8List? get newWebImage => _newWebImage;
  set newWebImage(Uint8List? value) {
    _newWebImage = value;
    notifyListeners(); 
  }

  final GoogleSignInProvider _authProvider;
  final db = FirebaseFirestore.instance;

  AppState(this._authProvider) {
    init();
  }

  String rol = 'Monitor';
  String paraje = 'El Venturero';
  bool loading = true;
  List<String> adminUIDs = [];
  bool isAdmin = false;

  User? get currentUser => _authProvider.currentUser;
  String? get currentUID => currentUser?.uid;

  DocumentReference get _parajeRef =>
      db.collection('roles').doc(rol).collection('parajes').doc(paraje);
  CollectionReference get _measurementsRef =>
      _parajeRef.collection('measurements');
  CollectionReference get _realMeasurementsRef =>
      _parajeRef.collection('real_measurements');

  Future<void> init() async {
    loading = true;
    notifyListeners();

    final prefs = await SharedPreferences.getInstance();
    rol = prefs.getString('rol') ?? 'Monitor';
    paraje = prefs.getString('paraje') ?? 'El Venturero';

    await _loadAdminUIDs();
    _checkAdminStatus();

    loading = false;
    notifyListeners();
  }

  Future<void> _loadAdminUIDs() async {
    try {
      final doc = await db.collection('admins').doc('adminUsers').get();
      if (doc.exists) {
        adminUIDs = List<String>.from(doc.data()?['uids'] ?? []);
      }
    } catch (e) {
      debugPrint("Error cargando admins: $e");
    }
  }

  void _checkAdminStatus() {
    isAdmin = currentUID != null && adminUIDs.contains(currentUID);
  }

  bool canEditMeasurement(String? uploaderId) =>
      currentUID == uploaderId || isAdmin;

  Future<void> changeParaje(String newParaje) async {
    paraje = newParaje;
    final prefs = await SharedPreferences.getInstance();
    prefs.setString('paraje', newParaje);
    prefs.setBool('hasFinishedOnboarding', true);
    notifyListeners();
  }

  Future<void> changeRol(String newRol) async {
    rol = newRol;
    final prefs = await SharedPreferences.getInstance();
    prefs.setString('rol', newRol);
    prefs.setBool('hasFinishedOnboarding', true);
    notifyListeners();
  }

  Future<Map<String, dynamic>> getCurrentParajeData() async {
    var snapshot = await _parajeRef.get();
    return (snapshot.data() as Map<String, dynamic>?) ?? {};
  }

  Future<String?> _uploadImage(
    String fileNameBase, {
    File? image,
    String? oldImage,
  }) async {
    final storageRef = FirebaseStorage.instance.ref();
    final connectivityResult = await Connectivity().checkConnectivity();

    if (kIsWeb && newWebImage != null) {
      if (connectivityResult != ConnectivityResult.none) {
        final imageRef = storageRef.child("measurements/$fileNameBase.png");
        final metadata = SettableMetadata(contentType: 'image/png');
        await imageRef.putData(newWebImage!, metadata);
        return await imageRef.getDownloadURL();
      } else {
        return base64Encode(newWebImage!);
      }
    } else if (image != null) {
      if (connectivityResult != ConnectivityResult.none) {
        final extension = image.path.split('.').last;
        final imageRef = storageRef.child(
          "measurements/$fileNameBase.$extension",
        );
        await imageRef.putFile(image);
        return await imageRef.getDownloadURL();
      } else {
        return await image.readAsString();
      }
    }

    return oldImage;
  }

  Future<Map<String, dynamic>> _getMeasurementJson({
    required num precipitation,
    required DateTime time,
    String? uploader,
    File? image,
    String? oldImage,
    bool? pluviometer,
  }) async {
    final fileNameBase =
        '${time.toIso8601String()}_$precipitation${currentUser?.email}';
    final imageUrl = await _uploadImage(
      fileNameBase,
      image: image,
      oldImage: oldImage,
    );

    return {
      'uploader_id': currentUID,
      'precipitation': precipitation,
      'uploader_name': uploader,
      'uploader_email': currentUser?.email,
      'time': time,
      'image': imageUrl,
      'avatar_url': currentUser?.photoURL,
      'pluviometer_state': pluviometer,
    };
  }

  Future<void> _saveMeasurement(
    String collectionName,
    Map<String, dynamic> data,
  ) async {
    await _parajeRef.collection(collectionName).add(data);
  }

  Future<num> _calculateRealValue(num current, bool? wasEmptied) async {
    final lastSnapshot =
        await _measurementsRef.orderBy('time', descending: true).limit(2).get();

    if (lastSnapshot.docs.length < 2 || wasEmptied == true) {
      return current;
    } else {
      final prevData = lastSnapshot.docs[1].data() as Map<String, dynamic>;
      final prevPrecip = prevData['precipitation'] as num? ?? 0;
      return current - prevPrecip;
    }
  }

  Future<void> updateGlobalCounter(int delta) async {
    final counterRef = db.collection('notifications').doc('globalCounter');
    await counterRef.set({
      'count': FieldValue.increment(delta),
      'timestamp': FieldValue.serverTimestamp(),
    }, SetOptions(merge: true));
  }

  Future<void> addMeasurement({
    required num precipitation,
    required DateTime time,
    String? uploader,
    File? image,
    bool? pluviometer,
  }) async {
    final data = await _getMeasurementJson(
      uploader: uploader,
      precipitation: precipitation,
      time: time,
      image: image,
      pluviometer: pluviometer,
    );
    await _saveMeasurement('measurements', data);

    final realValue = await _calculateRealValue(precipitation, pluviometer);
    final realData = await _getMeasurementJson(
      uploader: uploader,
      precipitation: realValue,
      time: time,
      image: image,
      pluviometer: pluviometer,
    );
    await _saveMeasurement('real_measurements', realData);

    await updateGlobalCounter(1);
  }

  Future<void> addRealMeasurement({
    required num precipitation,
    required DateTime time,
    num lastPrecipitation = 0,
    String? uploader,
    File? image,
    bool? pluviometer,
  }) async {
    final data = await _getMeasurementJson(
      uploader: uploader,
      precipitation: precipitation - lastPrecipitation,
      time: time,
      image: image,
      pluviometer: pluviometer,
    );
    await _saveMeasurement('real_measurements', data);
  }

  List<Measurement> getMeasurementsFromDocs(
    List<QueryDocumentSnapshot<Map<String, dynamic>>> docs,
  ) {
    final measurements =
        docs.map((doc) => Measurement.fromJson(doc.data(), doc.id)).toList();
    measurements.sort((a, b) => b.dateTime!.compareTo(a.dateTime!));
    return measurements;
  }

  Future<List<Measurement>> getMeasurements() async => getMeasurementsFromDocs(
    (await _measurementsRef.get() as QuerySnapshot<Map<String, dynamic>>).docs,
  );

  Future<List<Measurement>> getRealMeasurements() async =>
      getMeasurementsFromDocs(
        (await _realMeasurementsRef.get()
                as QuerySnapshot<Map<String, dynamic>>)
            .docs,
      );

  Stream<QuerySnapshot<Map<String, dynamic>>> _measurementStream(
    String collection, {
    String? parajeOverride,
  }) {
    final ref = db
        .collection('roles')
        .doc('Monitor')
        .collection('parajes')
        .doc(parajeOverride ?? paraje)
        .collection(collection);
    return ref.orderBy('time', descending: false).snapshots();
  }

  Stream<QuerySnapshot<Map<String, dynamic>>> getMeasurementsStream() =>
      _measurementStream('measurements');

  Stream<QuerySnapshot<Map<String, dynamic>>> getRealMeasurementsStream() =>
      _measurementStream('real_measurements');

  Stream<QuerySnapshot<Map<String, dynamic>>> getMeasurementsStreamForParaje(
    String name,
  ) => _measurementStream('measurements', parajeOverride: name);

  Stream<QuerySnapshot<Map<String, dynamic>>>
  getRealMeasurementsStreamForParaje(String name) =>
      _measurementStream('real_measurements', parajeOverride: name);

  Stream<QuerySnapshot<Map<String, dynamic>>> getAllUserMeasurementsStream() {
    if (currentUID == null) return const Stream.empty();
    return db
        .collectionGroup('measurements')
        .where('uploader_id', isEqualTo: currentUID)
        .snapshots();
  }

  Stream<QuerySnapshot<Map<String, dynamic>>> getAllMeasurementsStream() =>
      db.collectionGroup('measurements').snapshots();

  Future<void> updateMeasurement({
    required String id,
    required num precipitation,
    required DateTime time,
    String? uploader,
    File? image,
    bool? pluviometer,
    String? oldImage,
    required String uploaderId,
  }) async {
    if (!canEditMeasurement(uploaderId)) {
      throw Exception("No tiene permisos para editar esta medición");
    }
    final data = await _getMeasurementJson(
      uploader: uploader,
      precipitation: precipitation,
      time: time,
      image: image,
      oldImage: oldImage,
      pluviometer: pluviometer,
    );
    await _measurementsRef.doc(id).update(data);
  }

  Future<void> updateRealMeasurement({
    required String id,
    required num precipitation,
    required DateTime time,
    String? uploader,
    File? image,
    bool? pluviometer,
    String? oldImage,
  }) async {
    final data = await _getMeasurementJson(
      uploader: uploader,
      precipitation: precipitation,
      time: time,
      image: image,
      oldImage: oldImage,
      pluviometer: pluviometer,
    );
    await _realMeasurementsRef.doc(id).update(data);
  }

  Future<void> deleteMeasurement({required String id}) async {
    try {
      await _measurementsRef.doc(id).delete(); 
    } catch (e) {
      debugPrint("Error al borrar medición: $e");
    }
  }

  Future<void> deleteRealMeasurement({required String id}) async {
    try {
      await _realMeasurementsRef.doc(id).delete(); 
    } catch (e) {
      debugPrint("Error al borrar medición real: $e");
    }
  }

  Future<Map<String, dynamic>> getUserStats() async {
    try {
      if (currentUID == null) {
        return {
          'local': 0,
          'global': 0,
          'distinctParajes': 0,
          'totalParajes': 0,
        };
      }

      final localSnapshot = await _measurementsRef
          .where('uploader_id', isEqualTo: currentUID)
          .get(const GetOptions(source: Source.serverAndCache));

      final globalSnapshot = await db
          .collectionGroup('measurements')
          .where('uploader_id', isEqualTo: currentUID)
          .get(const GetOptions(source: Source.serverAndCache));

      final parajesContribuidos = <String>{};
      for (final doc in globalSnapshot.docs) {
        final segments = doc.reference.path.split('/');
        final parajeName =
            segments.contains('parajes')
                ? segments[segments.indexOf('parajes') + 1]
                : null;
        if (parajeName != null) parajesContribuidos.add(parajeName);
      }

      final totalParajesSnapshot = await db
          .collection('roles')
          .doc(rol)
          .collection('parajes')
          .get(const GetOptions(source: Source.serverAndCache));

      return {
        'local': localSnapshot.docs.length,
        'global': globalSnapshot.docs.length,
        'distinctParajes': parajesContribuidos.length,
        'totalParajes': totalParajesSnapshot.docs.length,
      };
    } catch (e) {
      debugPrint("Error en getUserStats: $e");
      return {
        'local': 0,
        'global': 0,
        'distinctParajes': 0,
        'totalParajes': 0,
        'error': e.toString(),};}}}


        \end{minted}


\newpage
\section{Reglas de Cloud FireStore}
\label{anexo:alg2}

\begin{minted}{javascript}

service cloud.firestore {
  
  // Permite leer/escribir si el usuario tiene algún rol (Rowy general)
  match /{allPaths=**} {
    allow read, write: if request.auth.token.roles.size() > 0;
  }

  match /databases/{database}/documents {

    // ----------------------------
    // Rowy rules start (NO MODIFICAR)
    // ----------------------------
    match /{collectionId}/{docId} {
      allow read, create, update, delete: 
      if colRule(["roles"], ["ADMIN","EDITOR","VIEWER","OWNER"]);
      
      function colRule(collections, roles) {
        return collectionId in collections && hasAnyRole(roles);
      }
    }
    // ----------------------------
    // Rowy rules end
    // ----------------------------

    // Permiso global a ADMIN y OWNER
    match /{document=**} {
      allow read, write: if hasAnyRole(["ADMIN", "OWNER"]);
    }

    // Configuración de Rowy (permitido a usuarios con rol)
    match /_rowy_/{docId} {
      allow read: if request.auth.token.roles.size() > 0;
      allow write: if hasAnyRole(["ADMIN", "OWNER"]);

      match /{document=**} {
        allow read: if request.auth.token.roles.size() > 0;
        allow write: if hasAnyRole(["ADMIN", "OWNER"]);
      }

      match /schema/{tableId} {
        allow update: if canModify(tableId,'pc');
        match /{document=**} {
          allow read, write: if canModify(tableId,'pc');
        }
      }

      match /groupSchema/{tableId} {
        allow update: if canModify(tableId,'cg');
        match /{document=**} {
          allow read, write: if canModify(tableId,'cg');
        }
      }
    }

    // Rowy: user management
    match /_rowy_/userManagement/users/{userId} {
      allow get, update, delete: if isDocOwner(userId);
      allow create: if request.auth.token.roles.size() > 0;
    }

    match /_rowy_/publicSettings {
      allow get: if true;
    }

    //PERMISOS PERSONALIZADOS PARA TUS MEDICIONES

    match /roles/{rol}/parajes/{paraje}/measurements/{docId} {
      allow read: if request.auth != null;
      allow create: if request.auth != null;
      allow update, delete: if 
        hasAnyRole(["ADMIN", "OWNER"]) || 
        request.auth.uid == resource.data.uploader_id;
    }

    match /roles/{rol}/parajes/{paraje}/real_measurements/{docId} {
      allow read: if request.auth != null;
      allow create: if request.auth != null;
      allow update, delete: if 
        hasAnyRole(["ADMIN", "OWNER"]) || 
        request.auth.uid == resource.data.uploader_id;
    }

    // Reglas por defecto: acceso propio
    match /{document=**} {
      allow read, write: if request.auth != null;
      allow create: if request.auth != null;
      allow update, delete: if request.auth.uid == resource.data.userId;
    }

    // UTILIDADES
    function isDocOwner(docId) {
      return request.auth != null &&
        (request.auth.uid == resource.id || request.auth.uid == docId);
    }

    function hasAnyRole(roles) {
      return request.auth != null &&
        request.auth.token.roles.hasAny(roles);
    }

    function canModify(tableId, tableType) {
      return hasAnyRole(get(/databases/$(database)/documents/_rowy_/settings)
        .data.tablesSettings[tableType][tableId].modifiableBy);
    }
  }
}
\end{minted}








% \newpage
% \section{Función main}
% \label{anexo:alg3}

% \begin{minted}{dart}

% import 'package:flutter/foundation.dart';
% import 'package:flutter/material.dart';
% import 'package:flutter/services.dart';
% import 'package:firebase_core/firebase_core.dart';
% import 'package:cloud_firestore/cloud_firestore.dart';
% import 'package:url_strategy/url_strategy.dart';
% import 'firebase_options.dart';
% import 'src/app.dart'; 

% void main() async {
  
%   WidgetsFlutterBinding.ensureInitialized();

%   // Configurar estrategia de URL limpia (sin #)
%   setPathUrlStrategy();

%   // Inicializar Firebase
%   await Firebase.initializeApp(
%     options: DefaultFirebaseOptions.currentPlatform,
%   );

%   // Configuración de Firestore: persistencia y caché ilimitado
%   FirebaseFirestore.instance.settings = const Settings(
%     persistenceEnabled: true,
%     cacheSizeBytes: Settings.CACHE_SIZE_UNLIMITED,
%   );

%   // Registrar licencias personalizadas (Google Fonts)
%   _registerLicenses();

%   // Iniciar la aplicación
%   runApp( const MyApp());
% }

% // Registrar licencias de fuentes y otros assets
% void _registerLicenses() {
%   LicenseRegistry.addLicense(() async* {
%     final license = await rootBundle.loadString('google_fonts/OFL.txt');
%     yield LicenseEntryWithLineBreaks(['google_fonts'], license);
%   });
% }
% \end{minted}


\newpage


% \section{Algoritmo de MyApp}
% \label{anexo:alg4}

% \begin{minted}{dart}

% import 'package:flutter/material.dart';
% import 'package:flutter_localizations/flutter_localizations.dart';
% import 'package:provider/provider.dart';
% import 'package:tlaloc/src/core/app_router.dart';
% import 'package:tlaloc/src/core/providers/app_providers.dart';
% import 'package:tlaloc/src/models/constants.dart';  

% class MyApp extends StatelessWidget {
%   const MyApp({super.key});

%   @override
%   Widget build(BuildContext context) {
%     return MultiProvider(
%       providers: appProviders,
%       child: Builder(
%         builder:
%             (context) => MaterialApp(
%               title: appName,
%               debugShowCheckedModeBanner: false,
%               theme: appLightTheme,
%               darkTheme: appDarkTheme,
%               themeMode: ThemeMode.system,
%               initialRoute: '/',
%               onGenerateRoute: generateRoute,
%               localizationsDelegates: const [
%                 GlobalMaterialLocalizations.delegate,
%                 GlobalWidgetsLocalizations.delegate,
%                 GlobalCupertinoLocalizations.delegate,
%               ],
%             ),
%       ),
%     );
%   }
% }

% \end{minted}



% \newpage

% \section{Algoritmo de ConditionalOnboardingPage}
% \label{anexo:alg5}

% \begin{minted}{dart}
% import 'package:flutter/material.dart';
% import 'package:provider/provider.dart';
% import 'package:shared_preferences/shared_preferences.dart';
% import 'package:tlaloc/src/models/google_sign_in.dart';
% import 'package:tlaloc/src/models/kernel.dart';
% import 'package:tlaloc/src/resources/onboarding/onbording.dart'; 
% import 'package:tlaloc/src/ui/widgets/backgrounds/empty_state.dart';
% import 'package:tlaloc/src/ui/widgets/backgrounds/splash.dart'; 

% class ConditionalOnboardingPage extends StatelessWidget {
%   const ConditionalOnboardingPage({super.key});

%   Future<Widget> _decideNextScreen(BuildContext context) async {
%     final prefs = await SharedPreferences.getInstance();
%     final hasFinishedOnboarding = prefs.getBool('hasFinishedOnboarding') ?? false;

%     final authProvider = Provider.of<GoogleSignInProvider>(context, listen: false);
%     final isLoggedIn = authProvider.currentUser != null;

%     if (hasFinishedOnboarding && isLoggedIn) {
%       return const HomePage();
%     } else {
%       return Onboarding();
%     }
%   }

%   @override
%   Widget build(BuildContext context) {
%     return FutureBuilder<Widget>(
%       future: _decideNextScreen(context),
%       builder: (context, snapshot) {
%         if (snapshot.hasError) {
%           return const _ErrorScreen();
%         } else if (snapshot.connectionState != ConnectionState.done) {
%           // Evita pantalla en blanco, mientras resuelve
%           return const SplashScreen(nextScreen: Scaffold());
%         } else {
%           return SplashScreen(nextScreen: snapshot.data!);
%         }
%       },
%     );
%   }
% }

% class _ErrorScreen extends StatelessWidget {
%   const _ErrorScreen();

%   @override
%   Widget build(BuildContext context) {
%     return Scaffold(
%       appBar: AppBar(title: const Text('Error de inicio')),
%       body: const EmptyState(
%         'No pudimos cargar la configuración inicial. '
%         'Por favor revisa tu conexión a internet o reinstala la aplicación.',
%       ),
%     );
%   }
% }
% \end{minted}






% \newpage
% \section{Pantalla Onboarding}
% \label{anexo:alg6}
% \begin{minted}{dart}
% import 'package:flutter/material.dart';
% import 'package:concentric_transition/concentric_transition.dart';
% import 'package:lottie/lottie.dart';
% import 'package:tlaloc/src/models/constants.dart';
% import 'package:tlaloc/src/ui/widgets/cards/onbording_cards.dart';
% import 'package:tlaloc/src/resources/onboarding/sign_in.dart';

% class Onboarding extends StatelessWidget {
%   Onboarding({super.key});

%   final List<CardPlanetData> data = [
%     CardPlanetData(
%       title: appName,
%       subtitle: "Ciencia para tí y para todos",
%       image: const AssetImage("assets/images/img-1.png"),
%       backgroundColor: AppColors.blue1,
%       titleColor: Colors.white,
%       subtitleColor: Colors.white,
%       background: LottieBuilder.asset("assets/animation/bg-1.json"),
%     ),
%     CardPlanetData(
%       title: "Te damos la bienvenida",
%       subtitle:
%           "Ya eres parte del proyecto ''Ciencia ciudadana para el monitoreo de la lluvia en el monte Tláloc'' ",
%       image: const AssetImage("assets/images/img-2.png"),
%       backgroundColor: Colors.white,
%       titleColor: AppColors.green1,
%       subtitleColor: const Color.fromRGBO(0, 10, 56, 1),
%       background: LottieBuilder.asset("assets/animation/bg-2.json"),
%     ),
%   ];

%   @override
%   Widget build(BuildContext context) {
%     return Scaffold(
%       body: ConcentricPageView(
%         direction: Axis.horizontal,
%         pageSnapping: true,

%         onFinish: () {
%           Navigator.push(
%             context,
%             MaterialPageRoute(builder: (context) => const SignUpWidget()),
%           );
%         },
%         colors: data.map((e) => e.backgroundColor).toList(),
%         itemCount: data.length,
%         itemBuilder: (int index) {
%           return CardPlanet(data: data[index]);
%         },
%       ),
%     );
%   }
% }

% \end{minted}






% \newpage
% \section{Pantalla HomePage}
% \label{anexo:alg7}
% \begin{minted}{dart}
% import 'package:curved_navigation_bar/curved_navigation_bar.dart';
% import 'package:firebase_auth/firebase_auth.dart';
% import 'package:flutter/material.dart';
% import 'package:tlaloc/src/models/constants.dart';
% import 'package:tlaloc/src/resources/statics/graphs/graph2.dart';
% import 'package:tlaloc/src/ui/screens/dir/add.dart';
% import 'package:tlaloc/src/ui/screens/dir/data.dart';
% import 'package:tlaloc/src/ui/screens/dir/home.dart';
% import 'package:tlaloc/src/ui/screens/home/profile_page.dart';
% import 'package:cloud_firestore/cloud_firestore.dart'; 

% class HomePage extends StatefulWidget {
%   const HomePage({super.key});

%   @override
%   State<HomePage> createState() => _HomePageState();
% }

% class _HomePageState extends State<HomePage> {
%   int _selectedIndex = 0;
%   final GlobalKey<CurvedNavigationBarState> _navKey = GlobalKey();

%   int globalNotificationCount = 0;
%   bool hasSeenNotifications = false;

%   late final List<Widget> _screens = const [
%     HomeScreen(),
%     AddScreen(),
%     DataScreen(),
%     BarGraph(),
%     ConfigureScreen(),
%   ];
%   @override
%   void initState() {
%     super.initState();

%     FirebaseFirestore.instance
%         .collection('notifications')
%         .doc('globalCounter')
%         .snapshots()
%         .listen((snapshot) {
%           if (snapshot.exists) {
%             setState(() {
%               globalNotificationCount = snapshot.data()?['count'] ?? 0;
%             });
%           }
%         });
%   }

%   @override
%   Widget build(BuildContext context) {
%     return Scaffold(
%       extendBody: true,
%       body: IndexedStack(index: _selectedIndex, children: _screens),
%       bottomNavigationBar: Theme(
%         data: Theme.of(context).copyWith(iconTheme: const IconThemeData()),
%         child: CurvedNavigationBar(
%           key: _navKey,
%           height: 60.0,
%           color: AppColors.blue1,
%           buttonBackgroundColor: AppColors.blue1,
%           backgroundColor: Colors.transparent,
%           animationCurve: Curves.easeInOut,
%           animationDuration: const Duration(milliseconds: 800),
%           items: _buildNavItems(),
%           index: _selectedIndex,
%           onTap: (index) {
%             setState(() {
%               _selectedIndex = index;
%               if (index == 2) hasSeenNotifications = true;
%             });
%           },
%         ),
%       ),
%     );
%   }

%   List<Widget> _buildNavItems() {
%     return [
%       const Icon(Icons.home, size: 30, color: Colors.white),
%       const Icon(Icons.add, size: 30, color: Colors.white), 
%       Stack(
%         children: [
%           const Icon(Icons.menu_book_rounded, size: 30, color: Colors.white),
%           if (globalNotificationCount > 0 && !hasSeenNotifications)
%             Positioned(
%               right: 0,
%               top: 0,
%               child: Container(
%                 padding: const EdgeInsets.all(2),
%                 decoration: const BoxDecoration(
%                   color: Colors.red,
%                   shape: BoxShape.circle,
%                 ),
%                 constraints: const BoxConstraints(minWidth: 16, minHeight: 16),
%                 child: Text(
%                   '$globalNotificationCount',
%                   style: const TextStyle(
%                     color: Colors.white,
%                     fontSize: 10,
%                     fontWeight: FontWeight.bold,
%                   ),
%                   textAlign: TextAlign.center,
%                 ),
%               ),
%             ),
%         ],
%       ),

%       const Icon(Icons.line_axis, size: 30, color: Colors.white),
%       CircleAvatar(
%         foregroundImage:
%             FirebaseAuth.instance.currentUser?.photoURL != null
%                 ? NetworkImage(FirebaseAuth.instance.currentUser!.photoURL!)
%                 : const NetworkImage(
%                   'https://s1.elespanol.com/2019/11/01/elandroidelibre/el_androide_libre_441218515_179632866_1024x576.jpg',
%                 ),
%       ),
%     ];
%   }
% }
% \end{minted}



% \newpage
% \section{Pantalla SignUpWidget}
% \label{anexo:alg8}

% \begin{minted}{dart}

% import 'package:flutter/material.dart';
% import 'package:font_awesome_flutter/font_awesome_flutter.dart';
% import 'package:lottie/lottie.dart';
% import 'package:provider/provider.dart';
% import 'package:tlaloc/src/models/constants.dart';
% import 'package:tlaloc/src/models/google_sign_in.dart';
% import 'package:tlaloc/src/resources/onboarding/common_select.dart';
% import 'package:url_launcher/url_launcher.dart';
% import 'dart:ui';

% class SignUpWidget extends StatefulWidget {
%   const SignUpWidget({super.key});

%   @override
%   State<SignUpWidget> createState() => _SignUpWidgetState();
% }

% class _SignUpWidgetState extends State<SignUpWidget>
%     with SingleTickerProviderStateMixin {
%   late AnimationController _controller;
%   bool _isLoading = false;

%   @override
%   void initState() {
%     super.initState();
%     _controller = AnimationController(
%       vsync: this,
%       duration: const Duration(seconds: 2),
%     )..repeat(reverse: true);
%   }

%   @override
%   void dispose() {
%     _controller.dispose();
%     super.dispose();
%   }

%   Future<void> _handleGoogleSignIn(BuildContext context) async {
%     setState(() => _isLoading = true);
%     final provider = Provider.of<GoogleSignInProvider>(context, listen: false);

%     try {
%       await provider.googleLogin();
%       if (provider.currentUser != null) {
%         Navigator.pushReplacement(
%           context,
%           PageRouteBuilder(
%             transitionDuration: const Duration(milliseconds: 1000),
%             pageBuilder: (_, __, ___) => const CommonSelectPage(),
%             transitionsBuilder:
%                 (_, a, __, c) => FadeTransition(opacity: a, child: c),
%           ),
%         );
%       }
%     } catch (e) {
%       _showErrorDialog(context, e.toString());
%     } finally {
%       if (mounted) setState(() => _isLoading = false);
%     }
%   }

%   void _showErrorDialog(BuildContext context, String error) {
%     showDialog(
%       context: context,
%       builder:
%           (context) => AlertDialog(
%             backgroundColor: AppColors.dark1.withOpacity(0.9),
%             shape: RoundedRectangleBorder(
%               borderRadius: BorderRadius.circular(20),
%             ),
%             title: Row(
%               children: [
%                 Lottie.asset('assets/animation/bg-3.json', width: 40),
%                 const SizedBox(width: 10),
%                 const Text('Error', style: TextStyle(color: Colors.white)),
%               ],
%             ),
%             content: Text(error, style: const TextStyle(color: Colors.white70)),
%             actions: [
%               TextButton(
%                 onPressed: () => Navigator.pop(context),
%                 child: const Text('OK', style: TextStyle(color: Colors.blue)),
%               ),
%             ],
%           ),
%     );
%   }

%   @override
%   Widget build(BuildContext context) {
%     final size = MediaQuery.of(context).size;

%     return Scaffold(
%       backgroundColor: AppColors.purple1,
%       body: Stack(
%         children: [
%           // Fondo animado
%           Positioned.fill(
%             child: Lottie.asset(
%               'assets/animation/bg-3.json',
%               fit: BoxFit.cover,
%             ),
%           ),

%           // Contenido principal
%           Center(
%             child: SafeArea(
%               child: LayoutBuilder(
%                 builder: (context, constraints) {
%                   final isWide = constraints.maxWidth > 800;

%                   return SingleChildScrollView(
%                     physics: const BouncingScrollPhysics(),
%                     child: Padding(
%                       padding: const EdgeInsets.all(20),
%                       child:
%                           isWide
%                               ? Row(
%                                 mainAxisAlignment: MainAxisAlignment.center,
%                                 children: [
%                                   Expanded(
%                                     child: Padding(
%                                       padding: const EdgeInsets.all(20),
%                                       child: AnimatedBuilder(
%                                         animation: _controller,
%                                         builder:
%                                             (context, child) =>
%                                                 Transform.translate(
%                                                   offset: Offset(
%                                                     0,
%                                                     10 * _controller.value,
%                                                   ),
%                                                   child: child,
%                                                 ),
%                                         child: Image.asset(
%                                           'assets/images/img-1-4.png',
%                                           width: size.width * 0.3,
%                                         ),
%                                       ),
%                                     ),
%                                   ),
%                                   Expanded(child: _buildLoginCard(size)),
%                                 ],
%                               )
%                               : Column(
%                                 children: [
%                                   // Logo animado
%                                   AnimatedBuilder(
%                                     animation: _controller,
%                                     builder:
%                                         (context, child) => Transform.translate(
%                                           offset: Offset(
%                                             0,
%                                             10 * _controller.value,
%                                           ),
%                                           child: child,
%                                         ),
%                                     child: Image.asset(
%                                       'assets/images/img-1-4.png',
%                                       width: size.width * 0.8,
%                                     ),
%                                   ),
%                                   const SizedBox(height: 40),
%                                   _buildLoginCard(size),
%                                 ],
%                               ),
%                     ),
%                   );
%                 },
%               ),
%             ),
%           ),
%         ],
%       ),
%     );
%   }

%   Widget _buildLoginCard(Size size) {
%     return ClipRRect(
%       borderRadius: BorderRadius.circular(30),
%       child: BackdropFilter(
%         filter: ImageFilter.blur(sigmaX: 10, sigmaY: 10),
%         child: Container(
%           padding: const EdgeInsets.all(30),
%           decoration: BoxDecoration(
%             color: Colors.white.withOpacity(0.1),
%             border: Border.all(color: Colors.white24),
%             borderRadius: BorderRadius.circular(30),
%           ),
%           child: Column(
%             mainAxisSize: MainAxisSize.min,
%             children: [
%               Text(
%                 'Iniciar sesión',
%                 style: const TextStyle(
%                   fontSize: 28,
%                   fontWeight: FontWeight.bold,
%                   color: Colors.white,
%                   fontFamily: 'FredokaOne',
%                 ),
%               ),
%               const SizedBox(height: 15),
%               const Text(
%                 'Conéctate para contribuir a la ciencia ciudadana',
%                 textAlign: TextAlign.center,
%                 style: TextStyle(
%                   fontSize: 16,
%                   color: Colors.white70,
%                   fontFamily: 'Poppins',
%                 ),
%               ),
%               const SizedBox(height: 30),
%               AnimatedSwitcher(
%                 duration: const Duration(milliseconds: 300),
%                 child:
%                     _isLoading
%                         ? const CircularProgressIndicator(
%                           valueColor: AlwaysStoppedAnimation<Color>(
%                             Colors.white,
%                           ),
%                         )
%                         : ElevatedButton.icon(
%                           icon: FaIcon(
%                             FontAwesomeIcons.google,
%                             color: Colors.red[400],
%                           ),
%                           label: const Text(
%                             'Continuar con Google',
%                             style: TextStyle(
%                               fontSize: 16,
%                               fontWeight: FontWeight.bold,
%                             ),
%                           ),
%                           style: ElevatedButton.styleFrom(
%                             backgroundColor: Colors.white.withOpacity(0.9),
%                             foregroundColor: Colors.black87,
%                             minimumSize: Size(size.width * 0.7, 55),
%                             shape: RoundedRectangleBorder(
%                               borderRadius: BorderRadius.circular(15),
%                             ),
%                             elevation: 5,
%                             shadowColor: Colors.black26,
%                           ),
%                           onPressed: () => _handleGoogleSignIn(context),
%                         ),
%               ),
%               const SizedBox(height: 30),
%               MouseRegion(
%                 cursor: SystemMouseCursors.click,
%                 child: GestureDetector(
%                   onTap:
%                       () => launchUrl(
%                         Uri.parse('https://tlaloc.web.app/privacy/'),
%                         mode: LaunchMode.inAppWebView,
%                       ),
%                   child: RichText(
%                     textAlign: TextAlign.center,
%                     text: TextSpan(
%                       style: const TextStyle(
%                         color: Colors.white70,
%                         fontSize: 13,
%                         height: 1.5,
%                       ),
%                       children: [
%                         const TextSpan(
%                           text: 'Al continuar, aceptas nuestros\n',
%                         ),
%                         TextSpan(
%                           text: 'Términos de servicio',
%                           style: TextStyle(
%                             color: Colors.blue[200],
%                             fontWeight: FontWeight.bold,
%                             decoration: TextDecoration.underline,
%                           ),
%                         ),
%                         const TextSpan(text: ' y '),
%                         TextSpan(
%                           text: 'Política de privacidad',
%                           style: TextStyle(
%                             color: Colors.blue[200],
%                             fontWeight: FontWeight.bold,
%                             decoration: TextDecoration.underline,
%                           ),
%                         ),
%                       ],
%                     ),
%                   ),
%                 ),
%               ),
%             ],
%           ),
%         ),
%       ),
%     );
%   }
% }

% \end{minted}






% \newpage
% \section{Modelo GoogleSignInProvider}
% \label{anexo:alg9}
% \begin{minted}{dart}
% import 'package:flutter/foundation.dart';
% import 'package:firebase_auth/firebase_auth.dart';
% import 'package:google_sign_in/google_sign_in.dart';

% class GoogleSignInProvider extends ChangeNotifier {
%   final FirebaseAuth _auth = FirebaseAuth.instance;
  
%   final GoogleSignIn _googleSignIn = GoogleSignIn(
%     scopes: ['email', 'profile'],
%     clientId: kIsWeb 
%         ? '228815382617-2rtslpepg048j80iuls7ilrc8ff9sn4l.apps.googleusercontent.com'
%         : null, // Para Android/iOS, Firebase maneja automáticamente el Client ID
%   );

%   User? get currentUser => _auth.currentUser;
%   bool _isLoading = false;
%   bool get isLoading => _isLoading;

%   Future<void> googleLogin() async {
%     try {
%       _isLoading = true;
%       notifyListeners();

%       final GoogleSignInAccount? googleUser = await _googleSignIn.signIn();
%       if (googleUser == null) return;

%       final GoogleSignInAuthentication googleAuth = 
%           await googleUser.authentication;

%       final AuthCredential credential = GoogleAuthProvider.credential(
%         accessToken: googleAuth.accessToken,
%         idToken: googleAuth.idToken,
%       );

%       await _auth.signInWithCredential(credential);
      
%     } on FirebaseAuthException catch (e) {
%       _handleAuthError(e);
%     } catch (e) {
%       debugPrint('Error inesperado: $e');
%       rethrow;
%     } finally {
%       _isLoading = false;
%       notifyListeners();
%     }
%   }

%   Future<void> logout() async {
%     try {
%       await _googleSignIn.signOut(); // Mejor que disconnect()
%       await _auth.signOut();
%     } catch (e) {
%       debugPrint('Error al cerrar sesión: $e');
%       rethrow;
%     } finally {
%       notifyListeners();
%     }
%   }

%   void _handleAuthError(FirebaseAuthException e) {
%     debugPrint('Código de error: ${e.code}');
%     String message = 'Error de autenticación';

%     switch (e.code) {
%       case 'account-exists-with-different-credential':
%         message = 'Cuenta ya existe con otro método de autenticación';
%         break;
%       case 'invalid-credential':
%         message = 'Credenciales inválidas';
%         break;
%       case 'operation-not-allowed':
%         message = 'Método de autenticación no habilitado';
%         break;
%       case 'user-disabled':
%         message = 'Cuenta deshabilitada';
%         break;
%       case 'user-not-found':
%         message = 'Usuario no encontrado';
%         break;
%     }

%     throw AuthException(message);
%   }
% }

% class AuthException implements Exception {
%   final String message;
%   AuthException(this.message);
  
%   @override
%   String toString() => message;
% }
% \end{minted}










% \section{Página de elección de paraje}
% \label{anexo:alg10}

% \begin{minted}{dart}
%   import 'package:flutter/material.dart';
% import 'package:tlaloc/src/models/constants.dart';
% import 'package:tlaloc/src/ui/widgets/cards/common_card.dart';
% import 'package:tlaloc/src/ui/widgets/cards/qr.dart';

% class CommonSelectPage extends StatelessWidget {
%   const CommonSelectPage({super.key});

%   @override
%   Widget build(BuildContext context) {
%     return Scaffold(
%       backgroundColor: AppColors.lightBlue,
%       body: SingleChildScrollView(
%         child: SafeArea(
%           child: Padding(
%             padding: EdgeInsets.all(16.0),
%             child: Center(
%               child: Column(
%                 children: [
%                   Text(
%                     '¿Qué pluviómetro estás observando?',
%                     style: TextStyle(
%                       fontSize: 32,
%                       fontFamily: 'FredokaOne',
%                       color: Colors.white,
%                     ),
%                     textAlign: TextAlign.center,
%                   ),
%                   SizedBox(height: 20),
%                   QrSelectWidget(),
%                   SizedBox(height: 20),
%                   Text(
%                     'Seleccionar manualmente',
%                     style: TextStyle(
%                       fontSize: 32,
%                       fontFamily: 'FredokaOne',
%                       color: Colors.white,
%                     ),
%                     textAlign: TextAlign.center,
%                   ),
%                   SizedBox(height: 20),
%                   CommonSelectWidget(),
%                 ],
%               ),
%             ),
%           ),
%         ),
%       ),
%     );
%   }
% }

% \end{minted}







\section{Modelo QR}
\label{anexo:alg11}
\begin{minted}{dart}
import 'dart:math';
import 'package:flutter/material.dart';
import 'package:mobile_scanner/mobile_scanner.dart';
import 'package:provider/provider.dart';
import 'package:tlaloc/src/models/app_state.dart';
import 'package:tlaloc/src/models/constants.dart';
import 'package:tlaloc/src/models/home_page.dart';

class QrSelectWidget extends StatelessWidget {
  const QrSelectWidget({super.key});

  void _goHome(BuildContext context) {
    Navigator.of(context).pushAndRemoveUntil(
      MaterialPageRoute<void>(builder: (context) => const HomePage()),
      (route) => false,
    );
  }

  Future<void> _handleQrResult(BuildContext context, String? qrResult) async {
    if (qrResult == null) {
      await _showErrorDialog(
        context,
        title: 'Escaneo fallido',
        content: 'Intenta de nuevo o selecciona tu paraje manualmente',
      );
      return;
    }

    final paraje = _parseQrResult(qrResult);
    if (!parajes.containsKey(paraje)) {
      await _showErrorDialog(
        context,
        title: 'Código inválido',
        content:
            paraje.isEmpty
                ? null
                : 'Tlaloc App no está disponible en el paraje "$paraje"',
      );
      return;
    }

    _goHome(context);
    Provider.of<AppState>(context, listen: false).changeParaje(paraje);
  }

  String _parseQrResult(String qrResult) {
    if (!qrResult.contains('tlaloc.web.app')) return '';
    return qrResult.split('/').last.replaceAll(RegExp(r'_|%20'), ' ').trim();
  }

  Future<void> _showErrorDialog(
    BuildContext context, {
    required String title,
    String? content,
  }) async {
    await showDialog(
      context: context,
      barrierDismissible: true,
      builder:
          (context) => AlertDialog(
            icon: const Icon(Icons.error_outline_rounded),
            iconColor: Theme.of(context).colorScheme.error,
            title: Text(title),
            content: content != null ? Text(content) : null,
            actions: [
              TextButton(
                onPressed: () => Navigator.pop(context),
                child: const Text('ENTENDIDO'),
              ),
            ],
          ),
    );
  }

  @override
  Widget build(BuildContext context) {
    final theme = Theme.of(context);
    final colors = theme.colorScheme;

    return LayoutBuilder(
      builder: (context, constraints) {
        final screenWidth = constraints.maxWidth;
        final isWide = screenWidth >= 600;

        // Igual que en las tarjetas comunes
        final cardWidth =
            isWide
                ? (screenWidth - 16 /* spacing */ - 32 /* padding */ ) / 2
                : screenWidth - 32;

        return ConstrainedBox(
          constraints: const BoxConstraints(
            maxWidth: 400, // Máximo recomendado para móviles
            minWidth: 280, // Mínimo para buena legibilidad
          ),
          child: Material(
            color: AppColors.dark2,
            borderRadius: BorderRadius.circular(
              12,
            ), // Reducido de 28 a 12 según MD
            clipBehavior: Clip.antiAlias,
            elevation: 1,
            child: InkWell(
              onTap: () async {
                final qrResult = await showDialog<String>(
                  context: context,
                  builder: (context) => _QrScannerDialog(context),
                );
                await _handleQrResult(context, qrResult);
              },
              splashColor: colors.primary.withOpacity(0.1),
              highlightColor: colors.primary.withOpacity(0.05),
              child: Padding(
                padding: const EdgeInsets.all(24), // Padding interno estándar
                child: Column(
                  mainAxisSize: MainAxisSize.min,
                  mainAxisAlignment: MainAxisAlignment.center,
                  children: [
                    Icon(
                      Icons.qr_code_scanner_rounded,
                      size: 48, // Reducido de 64 para mejor proporción
                      color: colors.primary,
                    ),
                    const SizedBox(height: 16),
                    Text(
                      'ESCANEAR QR',
                      style: theme.textTheme.titleLarge?.copyWith(
                        fontWeight: FontWeight.w600,
                        color: colors.onSurface,
                        fontFamily: 'FredokaOne',
                      ),
                    ),
                    const SizedBox(height: 8),
                    Padding(
                      padding: const EdgeInsets.symmetric(horizontal: 16),
                      child: Text(
                        'Detecta tu pluviómetro automáticamente',
                        textAlign: TextAlign.center,
                        style: theme.textTheme.bodyMedium?.copyWith(
                          color: colors.onSurfaceVariant,
                          fontFamily: 'Poppins',),),),],),),),),);},);}}

class _QrScannerDialog extends StatelessWidget {
  final BuildContext parentContext;

  const _QrScannerDialog(this.parentContext);

  @override
  Widget build(BuildContext context) {
    final size = min(
      MediaQuery.of(parentContext).size.width,
      MediaQuery.of(parentContext).size.height,
    );

    return Dialog(
      backgroundColor: Colors.black,
      insetPadding: const EdgeInsets.all(24),
      child: Column(
        mainAxisSize: MainAxisSize.min,
        children: [
          AppBar(
            title: const Text('Escanear código'),
            backgroundColor: Colors.transparent,
            automaticallyImplyLeading: false,
            actions: [
              IconButton(
                icon: const Icon(Icons.close),
                onPressed: () => Navigator.pop(context),
              ),
            ],
          ),
          ConstrainedBox(
            constraints: BoxConstraints(maxWidth: size, maxHeight: size),
            child: MobileScanner(
              controller: MobileScannerController(
                detectionSpeed: DetectionSpeed.normal,
                facing: CameraFacing.back,
                torchEnabled: false,
              ),
              onDetect: (barcode) {
                if (barcode.barcodes.isNotEmpty) {
                  Navigator.pop(context, barcode.barcodes.first.rawValue ?? 
                  ''
                  ); }}, ),),],),);}}
\end{minted}









% \section{Pantalla de inicio}
% \label{anexo:alg12}
% \begin{minted}{dart}
% import 'package:auto_size_text/auto_size_text.dart';
% import 'package:flutter/material.dart';
% import 'package:flutter/rendering.dart';
% import 'package:tlaloc/src/models/constants.dart';
% import 'package:tlaloc/src/ui/widgets/appbar/infobutton2.dart';
% import 'package:tlaloc/src/ui/widgets/backgrounds/container.dart';
% import 'package:tlaloc/src/ui/widgets/buttons/fab.dart';
% import 'package:tlaloc/src/ui/widgets/buttons/notebook.dart';
% import 'package:tlaloc/src/ui/widgets/cards/communitybutton.dart';
% import 'package:tlaloc/src/ui/widgets/cards/forms.dart';
% import 'package:tlaloc/src/ui/widgets/cards/personal_measures.dart';
% import 'package:tlaloc/src/ui/widgets/cards/phrase.dart';
% import 'package:tlaloc/src/ui/widgets/cards/tlalocmap.dart';
% import 'package:tlaloc/src/ui/widgets/cards/tutorials.dart';
% import 'package:tlaloc/src/ui/widgets/info/info_page.dart';
% import 'package:tlaloc/src/ui/widgets/objects/quickadd.dart';
% import 'package:tlaloc/src/ui/widgets/pluviometer/forecast.dart';
% import 'package:tlaloc/src/ui/widgets/pluviometer/header.dart';
% import 'package:tlaloc/src/ui/widgets/social/social_media.dart';

% class HomeScreen extends StatefulWidget {
%   const HomeScreen({super.key});

%   @override
%   State<HomeScreen> createState() => _HomeScreenState();
% }

% class _HomeScreenState extends State<HomeScreen> {
%   bool isFabVisable = true;
%   @override
%   Widget build(BuildContext context) {
%     return SafeArea(
%       child: Scaffold(
%         appBar: AppBar(
%           title: Row(
%             children: [
%               Image.asset('assets/images/tlaloc_logo.png', height: 32),
%               const SizedBox(width: 8),
%               AutoSizeText(
%                 appName,
%                 style: TextStyle(
%                   fontFamily: 'FredokaOne',
%                   fontSize: 24,
%                   letterSpacing: 2,
%                 ),
%               ),
%             ],
%           ),
%           actions: const <Widget>[InfoButton2(), FluidDialogWidget()],
%         ),
%         // drawer: DrawerApp(),
%         body: LayoutBuilder(
%           builder: (context, constraints) {
%             final isWide = constraints.maxWidth > 800;

%             final content = [
%               OneTimeGoogleButton(message: "Llena el formulario (1 min)"),
%               const SizedBox(height: 5),
%               QuickAddWidget(),
%               const Divider(height: 5, thickness: 4, color: Colors.black),

%               const TodayWeatherStyleCard(),
%               const WeekRainMarker(),

%               GlassContainer(child: TutorialWidget()),
%               GlassContainer(
%                 child: Column(
%                   children: [
%                     const Padding(
%                       padding: EdgeInsets.all(8.0),
%                       child: Text(
%                         'Tabla de mediciones',
%                         style: TextStyle(
%                           color: AppColors.blue1,
%                           fontFamily: 'FredokaOne',
%                           fontSize: 24,
%                           letterSpacing: 2,
%                         ),
%                       ),
%                     ),
%                     PersonalMeasures(),
%                     GeneralMeasures(),
%                   ],
%                 ),
%               ),

%               GlassContainer(child: TlalocMapData()),
%               Column(
%                 children: [
%                   Center(
%                     child: Row(
%                       children: const [
%                         PhraseCard(),
%                         Spacer(), // Espacio entre tarjetas
%                         TableButton(),
%                       ],
%                     ),
%                   ),
%                   const Divider(height: 20, thickness: 4, color: Colors.black),
%                   CommunityButton(),
%                   const Divider(height: 20, thickness: 4, color: Colors.black),
%                   SocialLinksWidget(),
%                   const Divider(height: 20, thickness: 4, color: Colors.black),
%                 ],
%               ),
%             ];

%             return NotificationListener<UserScrollNotification>(
%               onNotification: (notification) {
%                 if (notification.direction == ScrollDirection.forward) {
%                   if (!isFabVisable) setState(() => isFabVisable = true);
%                 } else if (notification.direction == ScrollDirection.reverse) {
%                   if (isFabVisable) setState(() => isFabVisable = false);
%                 }
%                 return true;
%               },
%               child: SingleChildScrollView(
%                 padding: const EdgeInsets.all(16),
%                 child: Wrap(
%                   runSpacing: 20,
%                   spacing: 20,
%                   alignment: WrapAlignment.center,
%                   children:
%                       content.map((widget) {
%                         return ConstrainedBox(
%                           constraints: BoxConstraints(
%                             maxWidth:
%                                 isWide
%                                     ? (constraints.maxWidth / 2) - 30
%                                     : constraints.maxWidth,
%                           ),
%                           child: widget,
%                         );
%                       }).toList(),
%                 ),
%               ),
%             );
%           },
%         ),

%         floatingActionButton: Visibility(visible: isFabVisable, child: Fab()),
%       ),
%     );
%   }
% }

% \end{minted}





% \section{Pantalla de envío de mediciones}
% \label{anexo:alg13}
% \begin{minted}{dart} 
% import 'dart:io';
% import 'package:firebase_auth/firebase_auth.dart';
% import 'package:flutter/material.dart';
% import 'package:flutter/services.dart';
% import 'package:image_picker/image_picker.dart';
% import 'package:provider/provider.dart';
% import 'package:tlaloc/src/models/constants.dart';
% import 'package:tlaloc/src/models/date.dart';
% import 'package:tlaloc/src/models/lluvia/send_rain.dart';
% import 'package:tlaloc/src/resources/onboarding/common_select.dart';
% import 'package:tlaloc/src/ui/screens/home/home_widget_classes.dart';
% import 'package:tlaloc/src/ui/widgets/measures/save_button.dart';
% import 'package:tlaloc/src/models/app_state.dart';
% import 'package:auto_size_text/auto_size_text.dart';
% import 'package:audioplayers/audioplayers.dart';
% import 'package:tlaloc/src/models/google_sign_in.dart';
% import 'package:tlaloc/src/models/home_page.dart';
% import 'package:tlaloc/src/ui/widgets/objects/text_field.dart';
% import 'package:flutter/foundation.dart' show kIsWeb;

% class AddScreen extends StatefulWidget {
%   final Measurement? measurement;

%   const AddScreen({super.key, this.measurement});

%   @override
%   State<AddScreen> createState() => _AddScreenState();
% }

% class _AddScreenState extends State<AddScreen> {
%   late TextEditingController _precipitationController;
%   bool pluviometer = false;

%   File? newImage;
%   Uint8List? newWebImage; // Para imagenes web
%   final ImagePicker picker = ImagePicker();

%   DateTime dateTime = DateTime.now();
%   num precipitation = 0; // Variable no-nullable
%   String? uploader = FirebaseAuth.instance.currentUser?.displayName;
%   String path = 'sounds/correcto.mp3';
%   var player = AudioPlayer();

%   @override
%   void initState() {
%     super.initState();
%     _precipitationController = TextEditingController(
%       text: widget.measurement?.precipitation?.toStringAsFixed(1) ?? '0',
%     );

%     if (widget.measurement != null) {
%       precipitation =
%           widget.measurement!.precipitation ?? 0; // Conversión segura
%       uploader = widget.measurement!.uploader;
%       dateTime = widget.measurement!.dateTime!;
%     }
%   }

%   @override
%   void dispose() {
%     _precipitationController.dispose();
%     player.dispose();
%     super.dispose();
%   }

%   Future<void> pickImage() async {
%     final pickedFile = await picker.pickImage(source: ImageSource.gallery);
%     if (pickedFile != null) {
%       if (kIsWeb) {
%         final bytes = await pickedFile.readAsBytes();
%         setState(() {
%           Provider.of<AppState>(context, listen: false).newWebImage = bytes;
%           newImage = null;
%         });
%       } else {
%         setState(() {
%           newImage = File(pickedFile.path);
%           Provider.of<AppState>(context, listen: false).newWebImage = null;
%         });
%       }
%     }
%   }

%   Future<void> pickImageC() async {
%     final pickedFile = await picker.pickImage(source: ImageSource.camera);
%     if (pickedFile != null) {
%       if (kIsWeb) {
%         final bytes = await pickedFile.readAsBytes();
%         setState(() {
%           Provider.of<AppState>(context, listen: false).newWebImage = bytes;
%           newImage = null;
%         });
%       } else {
%         setState(() {
%           newImage = File(pickedFile.path);
%           Provider.of<AppState>(context, listen: false).newWebImage = null;
%         });
%       }
%     }
%   }

%   @override
%   Widget build(BuildContext context) {  
%     return SafeArea(
%       child: Scaffold(
%         appBar: AppBar(
%           title: Consumer<GoogleSignInProvider>(
%             builder: (context, signIn, child) {
%               String place = Provider.of<AppState>(context).paraje;
%               return AutoSizeText(
%                 place,
%                 style: const TextStyle(fontSize: 24, fontFamily: 'FredokaOne'),
%               );
%             },
%           ),
%           actions: [
%             Padding(
%               padding: const EdgeInsets.all(8.0),
%               child: ButtonWidget(
%                 onClicked: () async {
%                   try {
%                     final state = Provider.of<AppState>(context, listen: false);
%                     if (widget.measurement == null) {
%                       state.addMeasurement(
%                         uploader: uploader!,
%                         precipitation: precipitation,
%                         time: dateTime,
%                         image: newImage,
%                         pluviometer: pluviometer,
%                       );
%                       // state.newWebImage = null;
%                       await player.play(AssetSource(path));
%                       Navigator.of(context).pushAndRemoveUntil(
%                         MaterialPageRoute<void>(
%                           builder: (BuildContext context) {
%                             return const HomePage();
%                           },
%                         ),
%                         (Route<dynamic> route) => false,
%                       );
%                     } else {
%                       state.updateMeasurement(
%                         uploaderId: widget.measurement!.uploaderId!,
%                         id: widget.measurement!.id,
%                         uploader: uploader!,
%                         precipitation: precipitation,
%                         time: dateTime,
%                         image: newImage,
%                         oldImage: widget.measurement!.imageUrl,
%                         pluviometer: pluviometer,
%                       );
%                       await player.play(AssetSource(path));
%                       Navigator.pop(context);
%                     }
%                   } catch (e) {
%                     showDialog(
%                       context: context,
%                       builder:
%                           (context) => AlertDialog(
%                             title: const Text('¡Error al guardar la medición!'),
%                             content: Text('$e'),
%                           ),
%                     );
%                   }
%                 },
%               ),
%             ),
%           ],
%         ),
%         body: SingleChildScrollView(
%           child: Padding(
%             padding: const EdgeInsets.all(16.0),
%             child: Column(
%               children: [
%                 Consumer<GoogleSignInProvider>(
%                   builder: (context, signIn, child) {
%                     final name =
%                         FirebaseAuth.instance.currentUser?.displayName ?? '';
%                     return MyTextFormField(
%                       initialValue: uploader ?? name,
%                       helperText: '1. Escriba nombre completo',
%                       hintText: 'Nombre',
%                       icon: const Icon(Icons.person, color: Colors.blueGrey),
%                       onChanged: (String value) {
%                         setState(() => uploader = value);
%                       },
%                       textInputType: TextInputType.name,
%                     );
%                   },
%                 ),
%                 const SizedBox(height: 20),

%                 Container(
%                   decoration: BoxDecoration(
%                     color:
%                         Theme.of(context).brightness == Brightness.dark
%                             ? AppColors.dark3
%                             : Colors.transparent,
%                     borderRadius: BorderRadius.circular(12.0),
%                   ),
%                   child: ListTile(
%                     leading: CircleAvatar(
%                       backgroundColor: Colors.red[300],
%                       child: Icon(Icons.place, color: Colors.red[900]),
%                     ),
%                     title: Text(
%                       'Estás ubicado en: "${Provider.of<AppState>(context).paraje}"',
%                       style: const TextStyle(
%                         fontSize: 18,
%                         fontFamily: 'FredokaOne',
%                       ),
%                     ),
%                     subtitle: const Text('2. Elige el paraje correctamente'),
%                     onTap: () {
%                       Navigator.push(
%                         context,
%                         MaterialPageRoute(
%                           builder: (context) => const CommonSelectPage(),
%                         ),
%                       );
%                     },
%                   ),
%                 ),
%                 const SizedBox(height: 20),
%                 Container(
%                   decoration: BoxDecoration(
%                     color:
%                         Theme.of(context).brightness == Brightness.dark
%                             ? AppColors.dark3
%                             : Colors.transparent,
%                     borderRadius: BorderRadius.circular(12.0),
%                   ),
%                   child: Padding(
%                     padding: const EdgeInsets.all(16.0),
%                     child: RainInputWidget(
%                       precipitation: precipitation,
%                       onChanged: (value) {
%                         setState(() {
%                           precipitation = value;
%                         });
%                       },
%                     ),
%                   ),
%                 ), 
%                 const SizedBox(height: 20),
%                 Container(
%                   decoration: BoxDecoration(
%                     color:
%                         Theme.of(context).brightness == Brightness.dark
%                             ? AppColors.dark3
%                             : Colors.transparent,
%                     borderRadius: BorderRadius.circular(12.0),
%                   ),
%                   child: Datetime(
%                     updateDateTime: (value) {
%                       dateTime = value;
%                     },
%                   ),
%                 ),
 
%                 const SizedBox(height: 20),
%                 Container(
%                   decoration: BoxDecoration(
%                     color:
%                         Theme.of(context).brightness == Brightness.dark
%                             ? AppColors.dark3
%                             : Colors.transparent,
%                     borderRadius: BorderRadius.circular(12.0),
%                   ),
%                   child: const ContactUsListTile(
%                     title: 'Mandar fotografía',
%                     title2:
%                         '5. Toma una foto del pluviómetro y mándala por WhatsApp',
%                     message:
%                         'https://api.whatsapp.com/send?phone=5630908507&text=%C2%A1Mira!%20en%20el%20paraje%20%22%20%22%20llovi%C3%B3%20%22%20%22mm,%20adjunto%20fotograf%C3%ADa%20del%20d%C3%ADa%20de%20hoy',
%                   ),
%                 ),
%                 SizedBox(height: 20),
%                 Container(
%                   decoration: BoxDecoration(
%                     color:
%                         Theme.of(context).brightness == Brightness.dark
%                             ? AppColors.dark3
%                             : Colors.transparent,
%                     borderRadius: BorderRadius.circular(12.0),
%                   ),
%                   child: SwitchListTile(
%                     title: const Text(
%                       'Reinicio de mediciones',
%                       style: TextStyle(fontSize: 18, fontFamily: 'FredokaOne'),
%                     ),
%                     value: pluviometer,
%                     secondary: CircleAvatar(
%                       backgroundColor: Colors.teal[300],
%                       child: Icon(Icons.output, color: Colors.teal[900]),
%                     ),
%                     subtitle: const Text(
%                       '6. Vaciar pluviómetro (sólo personal capacitado)',
%                     ),
%                     onChanged: (bool value) {
%                       setState(() => pluviometer = value);
%                     },
%                   ),
%                 ),
%                 const SizedBox(height: 20),
%               ],
%             ),
%           ),
%         ),
%       ),
%     );
%   }
% }
% \end{minted}



% \section{Pantalla de Bitácora}
% \label{anexo:alg14}
% \begin{minted}{dart}
% import 'package:flutter/material.dart';
% import 'package:provider/provider.dart';
% import 'package:tlaloc/src/models/app_state.dart';
% import 'package:tlaloc/src/models/constants.dart';
% import 'package:tlaloc/src/models/excel.dart';
% import 'package:tlaloc/src/ui/widgets/appbar/infobutton2.dart';
% import 'package:tlaloc/src/ui/widgets/data_screen_view.dart';
% import 'package:tlaloc/src/ui/widgets/data_widget.dart';
% import 'package:tlaloc/src/ui/widgets/info/info_page.dart';
% import 'package:tlaloc/src/ui/widgets/real_data_widget.dart';

% class DataScreen extends StatefulWidget {
%   const DataScreen({super.key});

%   @override
%   State<DataScreen> createState() => _DataScreenState();
% }

% class _DataScreenState extends State<DataScreen> with TickerProviderStateMixin {
%   late TabController _tabController;
%   final List<ScrollController> _scrollControllers = [
%     ScrollController(),
%     ScrollController(),
%   ];
%   bool isFabVisible = true;

%   @override
%   void initState() {
%     super.initState();
%     _tabController = TabController(length: 2, vsync: this);

%     // Escuchar cambios de pestaña
%     _tabController.addListener(_handleTabChange);
%   }

%   void _handleTabChange() {
%     // Hacer scroll al inicio cuando cambia la pestaña
%     _scrollToTop(_tabController.index);
%   }

%   void _scrollToTop(int index) {
%     final controller = _scrollControllers[index];
%     if (controller.hasClients) {
%       controller.animateTo(
%         0,
%         duration: const Duration(milliseconds: 300),
%         curve: Curves.easeOut,
%       );
%     }
%   }

%   @override
%   void dispose() {
%     _tabController.removeListener(_handleTabChange);
%     _tabController.dispose();
%     for (var controller in _scrollControllers) {
%       controller.dispose();
%     }
%     super.dispose();
%   }

%   @override
%   Widget build(BuildContext context) {
%     final appState = context.watch<AppState>();
%     bool isWideLayout = MediaQuery.of(context).size.width > 800;
%     return SafeArea(
%       child: NestedScrollView(
%         headerSliverBuilder: (context, value) {
%           return [
%             SliverAppBar(
%               title: Row(
%                 children: [
%                   Image.asset('assets/images/tlaloc_logo.png', height: 32),
%                   const SizedBox(width: 8),
%                   const Text(
%                     'Bitácora',
%                     textAlign: TextAlign.start,
%                     style: TextStyle(
%                       // color: AppColors.dark1,
%                       fontFamily: 'FredokaOne',
%                       fontSize: 24,
%                       letterSpacing: 2,
%                     ),
%                   ),
%                 ],
%               ),
%               floating: true,
%               pinned: true,
%               snap: false,
%               expandedHeight: 150.0,
%               actions: <Widget>[
%                 IconButton(
%                   icon: const Icon(Icons.file_download),
%                   onPressed: () async {
%                     try {
%                       await appState.exportMeasurements(context);
%                       ScaffoldMessenger.of(context).showSnackBar(
%                         SnackBar(
%                           content: Text(
%                             'Exportación completada',
%                             style: TextStyle(color: Colors.green),
%                           ),
%                         ),
%                       );
%                     } catch (e) {
%                       ScaffoldMessenger.of(context).showSnackBar(
%                         SnackBar(content: Text('Error al exportar: $e')),
%                       );
%                     }
%                   },
%                 ),
%                 InfoButton2(),
%                 FluidDialogWidget(),
%               ],
%               bottom: TabBar(
%                 controller: _tabController,
%                 onTap: (index) => _scrollToTop(index),
%                 labelColor: AppColors.blue1,
%                 unselectedLabelColor: Colors.grey,
%                 indicatorColor: AppColors.blue1,
%                 labelStyle: const TextStyle(fontWeight: FontWeight.bold),
%                 unselectedLabelStyle: const TextStyle(
%                   fontWeight: FontWeight.bold,
%                 ),
%                 tabs: const <Widget>[
%                   Tab(text: 'Acumulados', icon: Icon(Icons.cloud_outlined)),
%                   Tab(text: 'Reales', icon: Icon(Icons.cloud_done_outlined)),
%                 ],
%               ),
%             ),
%           ];
%         },
%         body: TabBarView(
%           controller: _tabController, // Asigna el mismo controlador
%           children: <Widget>[
%             // Pasa el ScrollController a cada widget hijo
%             isWideLayout
%                 ? MasterDetailScreen()
%                 : MyDataWidget(scrollController: _scrollControllers[0]),
%             isWideLayout
%                 ? MasterDetailRealScreen()
%                 : MyRealDataWidget(scrollController: _scrollControllers[0]),
%           ],
%         ),
%       ),
%     );
%   }
% }
% \end{minted}




% \section{Pantalla de estadísticas}
% \label{anexo:alg15}
% \begin{minted}{dart}
% import 'dart:typed_data';
% import 'dart:ui';

% import 'package:auto_size_text/auto_size_text.dart';
% import 'package:cloud_firestore/cloud_firestore.dart';
% import 'package:flutter/material.dart';
% import 'package:flutter/rendering.dart';
% import 'package:intl/intl.dart';
% import 'package:provider/provider.dart';
% import 'package:tlaloc/src/models/app_state.dart';
% import 'package:tlaloc/src/models/constants.dart';
% import 'package:tlaloc/src/models/datepicker.dart';
% import 'package:tlaloc/src/ui/widgets/backgrounds/empty_state.dart';
% import 'package:tlaloc/src/ui/widgets/appbar/infobutton2.dart';
% import 'package:tlaloc/src/ui/widgets/info/info_page.dart';
% import 'package:fl_chart/fl_chart.dart';
% import 'package:pdf/pdf.dart';
% import 'package:pdf/widgets.dart' as pw;
% import 'package:printing/printing.dart';
% import 'package:file_saver/file_saver.dart';
% import 'package:flutter/foundation.dart' show kIsWeb;
% import 'package:path_provider/path_provider.dart';
% import 'dart:io';

% import 'package:universal_html/html.dart' as html;

% class BarGraph extends StatefulWidget {
%   const BarGraph({super.key});

%   @override
%   State<BarGraph> createState() => _BarGraphState();
% }

% enum DateTimeMode { custom, week, month, year, always }

% enum DataMode { accumulated, real }

% class _BarGraphState extends State<BarGraph> {
%   DateTime initialDate = dateLongAgo;
%   DateTime finalDate = dateInALongTime;
%   DateTimeMode mode = DateTimeMode.always;
%   String? _currentParaje;
%   DataMode dataMode = DataMode.accumulated;
%   final GlobalKey chartKey = GlobalKey();

%   @override
%   Widget build(BuildContext context) {
%     return Scaffold(
%       appBar: AppBar(
%         title: Row(
%           children: [
%             Image.asset('assets/images/tlaloc_logo.png', height: 32),
%             const SizedBox(width: 8),
%             AutoSizeText(
%               dataMode == DataMode.real
%                   ? 'Volumen'
%                   : 'Acumulados',
%               style: const TextStyle(
%                 fontFamily: 'FredokaOne',
%                 fontSize: 18,
%                 letterSpacing: 2,
%               ),
%             ),
%           ],
%         ),
%         actions: <Widget>[
%           IconButton(
%             icon: const Icon(Icons.picture_as_pdf),
%             onPressed: () => _exportToPdf(context),
%             tooltip: 'Exportar a PDF',
%           ),
%           InfoButton2(),
%           FluidDialogWidget(),
%         ],
%       ),
%       body: SafeArea(
%         child: Column(
%           children: [
%             const SizedBox(height: 20),
%             SwitchListTile(
%               title: Text(
%                 dataMode == DataMode.real
%                     ? 'Mostrar datos reales'
%                     : 'Mostrar acumulados',
%                 style: const TextStyle(fontWeight: FontWeight.bold),
%               ),
%               value: dataMode == DataMode.real,
%               onChanged:
%                   (val) => setState(
%                     () => dataMode = val ? DataMode.real : DataMode.accumulated,
%                   ),
%             ),
%             const SizedBox(height: 10),
%             _buildDateControls(),
%             const SizedBox(height: 20),
%             _buildDatePickers(),
%             const SizedBox(height: 20),
%             Expanded(
%               child: _buildChartSection(isReal: dataMode == DataMode.real),
%             ),
%           ],
%         ),
%       ),
%     );
%   }

%   Widget _buildDateControls() {
%     return Padding(
%       padding: const EdgeInsets.symmetric(horizontal: 12),
%       child: Wrap(
%         spacing: 4,
%         children: [
%           _buildChoiceChip('Esta semana', DateTimeMode.week),
%           _buildChoiceChip('Este mes', DateTimeMode.month),
%           _buildChoiceChip('Este año', DateTimeMode.year),
%           _buildChoiceChip('Siempre', DateTimeMode.always),
%         ],
%       ),
%     );
%   }

%   Widget _buildDatePickers() {
%     return Padding(
%       padding: const EdgeInsets.symmetric(horizontal: 12),
%       child: Row(
%         children: [
%           const Text('Inicio: '),
%           DatePickerButton(
%             dateTime: initialDate,
%             onDateChanged: (date) => _updateDates(date, isStart: true),
%           ),
%           const Expanded(child: SizedBox()),
%           const Text('Fin: '),
%           DatePickerButton(
%             dateTime: finalDate,
%             onDateChanged: (date) => _updateDates(date, isStart: false),
%           ),
%         ],
%       ),
%     );
%   }

%   Widget _buildChartSection({bool isReal = false}) {
%     return Consumer<AppState>(
%       builder: (context, state, _) {
%         _handleParajeChange(state);
%         return StreamBuilder<QuerySnapshot<Map<String, dynamic>>>(
%           key: Key('${state.rol}-${state.paraje}'),
%           stream:
%               isReal
%                   ? state.getRealMeasurementsStream()
%                   : state.getMeasurementsStream(),
%           builder: (context, snapshot) {
%             if (snapshot.connectionState == ConnectionState.waiting) {
%               return _buildLoadingIndicator();
%             }
%             if (snapshot.hasError) {
%               return EmptyState('Error: ${snapshot.error}');
%             }
%             return _handleSnapshot(snapshot, state, isReal: isReal);
%           },
%         );
%       },
%     );
%   }

%   void _handleParajeChange(AppState state) {
%     if (_currentParaje != state.paraje) {
%       WidgetsBinding.instance.addPostFrameCallback((_) {
%         setState(() => _currentParaje = state.paraje);
%       });
%     }
%   }

%   Widget _handleSnapshot(
%     AsyncSnapshot<QuerySnapshot<Map<String, dynamic>>> snapshot,
%     AppState state, {
%     bool isReal = false,
%   }) {
%     if (!snapshot.hasData || snapshot.data!.docs.isEmpty) {
%       return EmptyState('No hay datos en ${state.paraje}');
%     }

%     final measurements = state.getMeasurementsFromDocs(snapshot.data!.docs);
%     final filteredMeasurements = _filterMeasurements(
%       isReal ? _filterOnlyReal(measurements) : measurements,
%     );

%     return filteredMeasurements.isEmpty
%         ? const EmptyState('No hay datos en el rango seleccionado')
%         : _buildChart(filteredMeasurements);
%   }

%   List<Measurement> _filterOnlyReal(List<Measurement> realValue) {
%     return realValue
%         .where(
%           (m) =>
%               m.dateTime != null &&
%               m.dateTime!.isAfter(initialDate) &&
%               m.dateTime!.isBefore(finalDate),
%         )
%         .toList()
%       ..sort((a, b) => a.dateTime!.compareTo(b.dateTime!));
%   }

%   List<Measurement> _filterMeasurements(List<Measurement> measurements) {
%     return measurements
%         .where(
%           (m) =>
%               m.dateTime != null &&
%               m.dateTime!.isAfter(initialDate) &&
%               m.dateTime!.isBefore(finalDate),
%         )
%         .toList()
%       ..sort((a, b) => a.dateTime!.compareTo(b.dateTime!));
%   }

%   Widget _buildLoadingIndicator() {
%     return const Center(
%       child: Column(
%         mainAxisAlignment: MainAxisAlignment.center,
%         children: [
%           CircularProgressIndicator(),
%           SizedBox(height: 20),
%           Text(
%             'Cargando datos...',
%             style: TextStyle(color: AppColors.blue1, fontSize: 16),
%           ),
%         ],
%       ),
%     );
%   }

%   Widget _buildChart(List<Measurement> measurements) {
%     final theme = Theme.of(context);
%     final primaryColor = theme.colorScheme.primary;
%     final surfaceVariant = theme.colorScheme.surfaceContainerHighest;
%     final onSurface = theme.colorScheme.onSurface;

%     // Definir ancho total dinámico
%     final chartWidth =
%         (measurements.length * 40).toDouble().clamp(300, 2000).toDouble();

%     return Padding(
%       padding: const EdgeInsets.all(16.0),
%       child: SizedBox(
%         height: 400,
%         child: SingleChildScrollView(
%           scrollDirection: Axis.horizontal,
%           child: SizedBox(
%             width: chartWidth,
%             child: RepaintBoundary(
%               key: chartKey,
%               child: BarChart(
%                 BarChartData(
%                   groupsSpace: 16,
%                   alignment: BarChartAlignment.spaceBetween,
%                   barTouchData: BarTouchData(
%                     enabled: true,
%                     touchTooltipData: BarTouchTooltipData(
%                       // tooltipBgColor: primaryColor.withOpacity(0.9),
%                       getTooltipItem: (group, groupIndex, rod, rodIndex) {
%                         final date = DateFormat(
%                           'dd/MM/yy',
%                         ).format(measurements[groupIndex].dateTime!);
%                         final value = rod.toY.toStringAsFixed(1);
%                         return BarTooltipItem(
%                           '$date\n$value mm',
%                           TextStyle(
%                             color: theme.colorScheme.onPrimary,
%                             fontWeight: FontWeight.bold,
%                             fontSize: 12,
%                           ),
%                         );
%                       },
%                     ),
%                   ),
%                   titlesData: FlTitlesData(
%                     leftTitles: AxisTitles(
%                       axisNameWidget: Text(
%                         'Precipitación (mm)',
%                         style: TextStyle(
%                           color: onSurface,
%                           fontWeight: FontWeight.bold,
%                           fontSize: 14,
%                         ),
%                       ),
%                       sideTitles: SideTitles(
%                         showTitles: true,
%                         reservedSize: 22,
%                         interval: _calculateYInterval(measurements),
%                         getTitlesWidget:
%                             (value, meta) => Text(
%                               '${value.toInt()}',
%                               style: TextStyle(color: onSurface, fontSize: 12),
%                             ),
%                       ),
%                     ),
%                     bottomTitles: AxisTitles(
%                       axisNameWidget: Padding(
%                         padding: const EdgeInsets.only(top: 12),
%                         child: Text(
%                           'Fecha',
%                           style: TextStyle(
%                             color: onSurface,
%                             fontWeight: FontWeight.bold,
%                             fontSize: 14,
%                           ),
%                         ),
%                       ),
%                       sideTitles: SideTitles(
%                         showTitles: true,
%                         getTitlesWidget: (value, meta) {
%                           final index = value.toInt();
%                           const maxLabels = 10;
%                           final total = measurements.length;
%                           if (total <= maxLabels ||
%                               index % (total ~/ maxLabels) == 0) {
%                             if (index >= 0 && index < total) {
%                               return _buildDateLabel(
%                                 measurements[index].dateTime!,
%                               );
%                             }
%                           }
%                           return const SizedBox.shrink();
%                         },
%                       ),
%                     ),
%                     topTitles: const AxisTitles(
%                       sideTitles: SideTitles(showTitles: false),
%                     ),
%                     rightTitles: const AxisTitles(
%                       sideTitles: SideTitles(showTitles: false),
%                     ),
%                   ),
%                   borderData: FlBorderData(
%                     show: true,
%                     border: const Border(
%                       left: BorderSide(width: 1, color: Colors.grey),
%                       bottom: BorderSide(width: 1, color: Colors.grey),
%                     ),
%                   ),
%                   gridData: FlGridData(
%                     show: true,
%                     horizontalInterval: _calculateYInterval(measurements),
%                     getDrawingHorizontalLine:
%                         (value) =>
%                             FlLine(color: surfaceVariant, strokeWidth: 1),
%                   ),
%                   barGroups:
%                       measurements.asMap().entries.map((entry) {
%                         final index = entry.key;
%                         final m = entry.value;
%                         return BarChartGroupData(
%                           x: index,
%                           barRods: [
%                             BarChartRodData(
%                               toY: (m.precipitation ?? 0).toDouble(),
%                               color: primaryColor,
%                               width: 5,
%                               borderRadius: BorderRadius.circular(6),
%                               gradient: LinearGradient(
%                                 colors: [
%                                   primaryColor.withOpacity(0.9),
%                                   primaryColor.withOpacity(0.5),
%                                 ],
%                                 begin: Alignment.topCenter,
%                                 end: Alignment.bottomCenter,
%                               ),
%                               backDrawRodData: BackgroundBarChartRodData(
%                                 show: true,
%                                 toY: (m.precipitation ?? 0) * 1.1,
%                                 color: surfaceVariant.withOpacity(0.3),
%                               ),
%                             ),
%                           ],
%                         );
%                       }).toList(),
%                   minY: 0,
%                   maxY: _calculateMaxY(measurements),
%                 ),
%                 duration: const Duration(milliseconds: 800),
%                 curve: Curves.easeOutQuart,
%               ),
%             ),
%           ),
%         ),
%       ),
%     );
%   }

%   double _calculateMaxY(List<Measurement> measurements) {
%     if (measurements.isEmpty) return 10;
%     final max = measurements
%         .map((m) => m.precipitation ?? 0)
%         .reduce((a, b) => a > b ? a : b);
%     return (max * 1.2).toDouble();
%   }

%   double _calculateYInterval(List<Measurement> measurements) {
%     if (measurements.isEmpty) return 10; // Manejo de lista vacía

%     final maxPrecip = measurements
%         .map((m) => m.precipitation?.toDouble() ?? 0.0)
%         .reduce((a, b) => a > b ? a : b); // Versión más eficiente

%     if (maxPrecip > 50) return 20;
%     if (maxPrecip > 20) return 10;
%     return 5;
%   }

%   Widget _buildDateLabel(DateTime date) {
%     return Transform.rotate(
%       angle: -0.4,
%       child: Text(
%         DateFormat('dd/MM').format(date),
%         style: const TextStyle(fontSize: 10),
%       ),
%     );
%   }

%   void _updateDates(DateTime date, {required bool isStart}) {
%     setState(() {
%       mode = DateTimeMode.custom;
%       if (isStart) {
%         initialDate = date;
%       } else {
%         finalDate = DateTime(
%           date.year,
%           date.month,
%           date.day,
%         ).add(const Duration(days: 1)).subtract(const Duration(seconds: 1));
%       }
%     });
%   }

%   Widget _buildChoiceChip(String label, DateTimeMode value) {
%     return ChoiceChip(
%       selectedColor: AppColors.blue1,
%       label: Text(label),
%       selected: mode == value,
%       onSelected: (val) => val ? _handleTimeModeChange(value) : null,
%     );
%   }

%   void _handleTimeModeChange(DateTimeMode value) {
%     final now = DateTime.now();
%     setState(() {
%       mode = value;
%       switch (value) {
%         case DateTimeMode.week:
%           final monday = now.subtract(Duration(days: now.weekday - 1));
%           initialDate = monday;
%           finalDate = monday.add(
%             const Duration(days: 6, hours: 23, minutes: 59, seconds: 59),
%           );
%           break;
%         case DateTimeMode.month:
%           initialDate = DateTime(now.year, now.month, 1);
%           finalDate = DateTime(now.year, now.month + 1, 0, 23, 59, 59);
%           break;
%         case DateTimeMode.year:
%           initialDate = DateTime(now.year, 1, 1);
%           finalDate = DateTime(now.year, 12, 31, 23, 59, 59);
%           break;
%         case DateTimeMode.always:
%           initialDate = dateLongAgo;
%           finalDate = dateInALongTime;
%           break;
%         case DateTimeMode.custom:
%           break;
%       }
%     });
%   }

%   Future<void> _exportToPdf(BuildContext context) async {
%     final appState = Provider.of<AppState>(context, listen: false);
%     final measurements = await _getCurrentMeasurements(appState);

%     if (measurements.isEmpty) {
%       ScaffoldMessenger.of(context).showSnackBar(
%         const SnackBar(content: Text('No hay datos para exportar')),
%       );
%       return;
%     }

%     final pdf = pw.Document();
%     final theme = Theme.of(context);
%     final title =
%         dataMode == DataMode.real
%             ? 'Datos Reales de Precipitación'
%             : 'Volúmenes Acumulados de Precipitación';
%     final dateRange =
%         '${DateFormat('dd/MM/yyyy').format(initialDate)} - ${DateFormat('dd/MM/yyyy').format(finalDate)}';
%     final paraje = appState.paraje;

%     final chartImage = await _generateChartImage(measurements, theme);

%     pdf.addPage(
%       pw.Page(
%         pageFormat: PdfPageFormat.a4,
%         build: (pw.Context context) {
%           return pw.Column(
%             crossAxisAlignment: pw.CrossAxisAlignment.start,
%             children: [
%               pw.Header(
%                 level: 0,
%                 child: pw.Text(title, style: pw.TextStyle(fontSize: 20)),
%               ),
%               pw.Text('Paraje: $paraje'),
%               pw.Text('Rango de fechas: $dateRange'),
%               pw.SizedBox(height: 20),
%               pw.Center(
%                 child: pw.Container(
%                   height: 300,
%                   child: pw.FittedBox(child: pw.Image(chartImage)),
%                 ),
%               ),
%               pw.SizedBox(height: 20),
%               _buildDataTable(measurements),
%             ],
%           );
%         },
%       ),
%     );

%     // Guardar o mostrar el PDF según la plataforma
%     final bytes = await pdf.save();
%     await _saveOrPrintPdf(context, bytes, title);
%   }

%   Future<void> _saveOrPrintPdf(
%     BuildContext context,
%     Uint8List bytes,
%     String title,
%   ) async {
%     if (kIsWeb) {
%       final blob = html.Blob([bytes], 'application/pdf');
%       final url = html.Url.createObjectUrlFromBlob(blob);
%       final anchor =
%           html.AnchorElement(href: url)
%             ..setAttribute('download', '$title.pdf')
%             ..click();
%       html.Url.revokeObjectUrl(url);
%     } else {
%       // Para móvil: mostrar diálogo de impresión/guardado
%       try {
%         await Printing.layoutPdf(
%           onLayout: (PdfPageFormat format) async => bytes,
%         );
%       } catch (e) {
%         // Si falla la impresión, guardar el archivo
%         final directory = await getApplicationDocumentsDirectory();
%         final file = File('${directory.path}/$title.pdf');
%         await file.writeAsBytes(bytes);

%         // Opcional: usar file_saver para mejor experiencia de usuario
%         try {
%           await FileSaver.instance.saveFile(
%             name: title,
%             bytes: bytes,
%             mimeType: MimeType.pdf,
%           );
%         } catch (e) {
%           ScaffoldMessenger.of(context).showSnackBar(
%             SnackBar(content: Text('PDF guardado en: ${file.path}')),
%           );
%         }
%       }
%     }
%   }

%   Future<List<Measurement>> _getCurrentMeasurements(AppState state) async {
%     final snapshot =
%         await (dataMode == DataMode.real
%             ? state.getRealMeasurementsStream().first
%             : state.getMeasurementsStream().first);

%     final measurements = state.getMeasurementsFromDocs(snapshot.docs);
%     return _filterMeasurements(measurements);
%   }

%   Future<pw.MemoryImage> _generateChartImage(
%     List<Measurement> measurements,
%     ThemeData theme,
%   ) async {
%     final boundary =
%         chartKey.currentContext!.findRenderObject() as RenderRepaintBoundary;
%     final image = await boundary.toImage(pixelRatio: 3.0);
%     final byteData = await image.toByteData(format: ImageByteFormat.png);
%     final bytes = byteData!.buffer.asUint8List();
%     return pw.MemoryImage(bytes);
%   }

%   pw.Widget _buildDataTable(List<Measurement> measurements) {
%     return pw.Table(
%       border: pw.TableBorder.all(),
%       children: [
%         pw.TableRow(
%           children: [
%             pw.Padding(
%               padding: const pw.EdgeInsets.all(4),
%               child: pw.Text(
%                 'Fecha',
%                 style: pw.TextStyle(fontWeight: pw.FontWeight.bold),
%                 textAlign: pw.TextAlign.center,
%               ),
%             ),
%             pw.Padding(
%               padding: const pw.EdgeInsets.all(4),
%               child: pw.Text(
%                 'Precipitación (mm)',
%                 style: pw.TextStyle(fontWeight: pw.FontWeight.bold),
%                 textAlign: pw.TextAlign.center,
%               ),
%             ),
%           ],
%         ),
%         ...measurements.map(
%           (m) => pw.TableRow(
%             children: [
%               pw.Padding(
%                 padding: const pw.EdgeInsets.all(4),
%                 child: pw.Text(
%                   DateFormat('dd/MM/yyyy').format(m.dateTime!),
%                   textAlign: pw.TextAlign.center,
%                 ),
%               ),
%               pw.Padding(
%                 padding: const pw.EdgeInsets.all(4),
%                 child: pw.Text(
%                   (m.precipitation ?? 0).toStringAsFixed(1),
%                   textAlign: pw.TextAlign.center,
%                 ),
%               ),
%             ],
%           ),
%         ),
%       ],
%     );
%   }
% }

% \end{minted}





% \section{Pantalla del perfil}
% \label{anexo:alg16}
% \begin{minted}{dart}
% import 'package:flutter/material.dart';
% import 'package:firebase_auth/firebase_auth.dart';
% import 'package:provider/provider.dart';
% import 'package:share_plus/share_plus.dart';
% import 'package:ionicons/ionicons.dart';
% import 'package:tlaloc/src/models/app_state.dart'; 
% import 'package:url_launcher/url_launcher.dart';
% import 'package:tlaloc/src/models/google_sign_in.dart';
% import 'package:tlaloc/src/resources/onboarding/onbording.dart';
% import 'package:tlaloc/src/ui/screens/settings/credits.dart';
% import 'package:tlaloc/src/ui/screens/settings/faq.dart';

% class ConfigureScreen extends StatefulWidget {
%   const ConfigureScreen({super.key});

%   @override
%   State<ConfigureScreen> createState() => _ConfigureScreenState();
% }

% class _ConfigureScreenState extends State<ConfigureScreen> {
%   @override
%   Widget build(BuildContext context) {
%     final appState = Provider.of<AppState>(context);

%     final user = FirebaseAuth.instance.currentUser;
%     final theme = Theme.of(context);
%     return Scaffold(
%       extendBodyBehindAppBar: true,
%       body: SingleChildScrollView(
%         child: Column(
%           children: [
%             Stack(
%               alignment: Alignment.bottomCenter,
%               children: [
%                 Image.asset('assets/images/portrate.jpg', fit: BoxFit.fitWidth),
%                 Container(
%                   width: 120,
%                   height: 120,
%                   decoration: BoxDecoration(
%                     shape: BoxShape.circle,
%                     border: Border.all(
%                       color: theme.colorScheme.surfaceContainerHighest,
%                       width: 4,
%                     ),
%                   ),
%                   child: CircleAvatar(
%                     radius: 56,
%                     backgroundImage:
%                         user?.photoURL != null
%                             ? NetworkImage(user!.photoURL!)
%                             : null,
%                     child:
%                         user?.photoURL == null
%                             ? Icon(
%                               Icons.account_circle,
%                               size: 60,
%                               color: theme.colorScheme.onSurface,
%                             )
%                             : null,
%                   ),
%                 ),
%               ],
%             ),

%             const SizedBox(height: 70),

%             // Información del usuario
%             Text(
%               user?.displayName ?? 'Usuario Tlaloc',
%               style: TextStyle(
%                 fontSize: 24,
%                 fontWeight: FontWeight.bold,
%                 color: theme.colorScheme.onSurface,
%                 fontFamily: 'FredokaOne',
%               ),
%             ),
%             const SizedBox(height: 8),
%             Text(
%               user?.email ?? 'correo@tlaloc.app',
%               style: TextStyle(
%                 color: theme.colorScheme.onSurfaceVariant,
%                 fontSize: 16,
%               ),
%             ),

%             // Estadísticas clave
%             Padding(
%               padding: const EdgeInsets.all(20),
%               child: FutureBuilder<Map<String, dynamic>>(
%                 future: appState.getUserStats(),
%                 builder: (context, snapshot) {
%                   if (!snapshot.hasData) {
%                     return const CircularProgressIndicator();
%                   }

%                   final stats = snapshot.data!;
%                   final local = stats['local'];
%                   final global = stats['global'];
%                   final parajes = stats['distinctParajes'];
%                   final total = stats['totalParajes'];

%                   return Row(
%                     mainAxisAlignment: MainAxisAlignment.spaceAround,
%                     children: [
%                       _buildStatCard(context, 'Mediciones', '$local'),
%                       _buildStatCard(context, 'Contribuciones', '$global'),
%                       _buildStatCard(context, 'Parajes', '$parajes/$total'),
%                     ],
%                   );
%                 },
%               ),
%             ),
 
%             _buildProfileSection(
%               context,
%               title: 'Configuración',
%               children: [ 
%                 _buildConfigItem(
%                   context,
%                   icon: Icons.share,
%                   title: 'Compartir aplicación',
%                   action: () {
%                     Share.share(
%                       '¡Próximamente podrás obtener varios datos de él!\n\nDescárgala en tlaloc.org',
%                       subject:
%                           '¿Sabías que hay una app donde puedes registrar los datos de la lluvia en el Monte Tláloc?',
%                     );
%                   },
%                 ),
%                 _buildConfigItem(
%                   context,
%                   icon: Icons.feedback,
%                   title: 'Enviar retroalimentación',
%                   action: () {
%                     launchUrl(
%                       Uri.parse(
%                         'mailto:tlloc-app@googlegroups.com?subject=Retroalimentación sobre Tláloc App',
%                       ),
%                     );
%                   },
%                 ),
%                 _buildConfigItem(
%                   context,
%                   icon: Icons.description,
%                   title: 'Términos y condiciones',
%                   action: () => Navigator.pushNamed(context, '/privacy'),
%                 ),
%                 _buildConfigItem(
%                   context,
%                   icon: Icons.security,
%                   title: 'Política de privacidad',
%                   action: () => Navigator.pushNamed(context, '/politics'),
%                 ),
%                 _buildConfigItem(
%                   context,
%                   icon: Icons.info,
%                   title: 'Acerca de',
%                   action:
%                       () => showAboutDialog(
%                         context: context,
%                         applicationIcon: CircleAvatar(
%                           backgroundImage: const AssetImage(
%                             'assets/images/img-1.png',
%                           ),
%                           backgroundColor: theme.colorScheme.surface,
%                         ),
%                         applicationLegalese: 'Con amor desde COLPOS',
%                         applicationVersion: 'versión inicial (beta)',
%                         children: [
%                           _buildDialogItem(
%                             context,
%                             icon: Icons.people,
%                             title: 'Ver créditos',
%                             action:
%                                 () => Navigator.push(
%                                   context,
%                                   MaterialPageRoute(
%                                     builder: (context) => const CreditsPage(),
%                                   ),
%                                 ),
%                           ),
%                           _buildDialogItem(
%                             context,
%                             icon: Icons.question_mark_rounded,
%                             title: 'Preguntas Frecuentes',
%                             action:
%                                 () => Navigator.push(
%                                   context,
%                                   MaterialPageRoute(
%                                     builder: (context) => const FaqPage(),
%                                   ),
%                                 ),
%                           ),
%                           _buildDialogItem(
%                             context,
%                             icon: Ionicons.logo_facebook,
%                             color: Colors.blue,
%                             title: 'Síguenos en Facebook',
%                             action:
%                                 () => launchUrl(
%                                   Uri.parse(
%                                     'https://www.facebook.com/Ciencia-Ciudadana-para-el-Monitoreo-de-Lluvia-100358326014423',
%                                   ),
%                                   mode: LaunchMode.externalApplication,
%                                 ),
%                           ),
%                           _buildDialogItem(
%                             context,
%                             icon: Ionicons.logo_youtube,
%                             color: Colors.red,
%                             title: 'Síguenos en YouTube',
%                             action:
%                                 () => launchUrl(
%                                   Uri.parse(
%                                     'https://www.youtube.com/channel/UC2wNEwvGEvnQVAX1Uv3qztA',
%                                   ),
%                                   mode: LaunchMode.externalApplication,
%                                 ),
%                           ),
%                           _buildDialogItem(
%                             context,
%                             icon: Icons.email,
%                             title: 'Mándanos un correo',
%                             action:
%                                 () => launchUrl(
%                                   Uri.parse(
%                                     'mailto:tlloc-app@googlegroups.com',
%                                   ),
%                                 ),
%                           ),
%                           _buildDialogItem(
%                             context,
%                             icon: Ionicons.logo_github,
%                             title: 'Colabora en GitHub',
%                             action:
%                                 () => launchUrl(
%                                   Uri.parse(
%                                     'https://github.com/Jack55913/TlalocApp',
%                                   ),
%                                   mode: LaunchMode.externalApplication,
%                                 ),
%                           ),
%                         ],
%                       ),
%                 ),
%                 ListTile(
%                   leading: Icon(Icons.logout, color: theme.colorScheme.error),
%                   title: Text(
%                     'Cerrar sesión',
%                     style: TextStyle(color: theme.colorScheme.onSurface),
%                   ),
%                   onTap: () {
%                     final provider = Provider.of<GoogleSignInProvider>(
%                       context,
%                       listen: false,
%                     );
%                     provider.logout();
%                     Navigator.pushReplacement(
%                       context,
%                       MaterialPageRoute(builder: (context) => Onboarding()),
%                     );
%                   },
%                 ),
%                 const SizedBox(height: 50),
%               ],
%             ),
%           ],
%         ),
%       ),
%     );
%   }

%   Widget _buildStatCard(BuildContext context, String title, String value) {
%     final theme = Theme.of(context);
%     return Column(
%       children: [
%         Text(
%           value,
%           style: TextStyle(
%             fontSize: 22,
%             fontWeight: FontWeight.bold,
%             color: theme.colorScheme.primary,
%           ),
%         ),
%         Text(
%           title,
%           style: TextStyle(color: theme.colorScheme.onSurfaceVariant),
%         ),
%       ],
%     );
%   }

%   Widget _buildProfileSection(
%     BuildContext context, {
%     required String title,
%     required List<Widget> children,
%   }) {
%     final theme = Theme.of(context);
%     return Padding(
%       padding: const EdgeInsets.symmetric(vertical: 15, horizontal: 20),
%       child: Column(
%         crossAxisAlignment: CrossAxisAlignment.start,
%         children: [
%           Text(
%             title,
%             style: TextStyle(
%               color: theme.colorScheme.onSurface,
%               fontSize: 18,
%               fontWeight: FontWeight.bold,
%             ),
%           ),
%           const SizedBox(height: 10),
%           ...children,
%         ],
%       ),
%     );
%   }

 

%   Widget _buildConfigItem(
%     BuildContext context, {
%     required IconData icon,
%     required String title,
%     required Function action,
%   }) {
%     final theme = Theme.of(context);
%     return ListTile(
%       leading: Icon(icon, color: theme.colorScheme.primary),
%       title: Text(title, style: TextStyle(color: theme.colorScheme.onSurface)),
%       trailing: Icon(
%         Icons.chevron_right,
%         color: theme.colorScheme.onSurfaceVariant,
%       ),
%       onTap: () => action(),
%     );
%   }

%   Widget _buildDialogItem(
%     BuildContext context, {
%     required IconData icon,
%     required String title,
%     required Function action,
%     Color? color,
%   }) {
%     final theme = Theme.of(context);
%     return ListTile(
%       leading: Icon(icon, color: color ?? theme.colorScheme.onSurface),
%       title: Text(title, style: TextStyle(color: theme.colorScheme.onSurface)),
%       onTap: () => action(),
%     );
%   }
% }

% \end{minted}