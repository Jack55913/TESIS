% !TEX encoding = UTF-8 Unicode
% !BIB TS-program = biber 
% !BIB program = biber    
%%%%%%%%%%%%%%%%%%%%%%%%%%%%%%%%%%%%%%%

\documentclass[twoside]{mitthesis} %,fontset=libertine, fontset=newtx-sans-text, fontset=heros-stix2, fontset=stix2
%
% option [twoside]		gives facing-page behavior for printing; omitting twoside will eliminate even-numbered blank pages.
% option [lineno]	 	provides line numbers, as for editing
% option [mydesign] 	loads packages for color, title and list formats, margins, or captions: edit mydesign.tex to change defaults.
% option [fontset] is a keyvalue which can be:
%					 	pdftex or unicode engines:  defaultfonts, libertine, lucida
%					 	pdftex only: 				fira-newtxsf, newtx, newtx-sans-text
%						unicode engines (luatex):	heros-stix2, stix2, termes, termes-stix2
%					 	if no key value is given, fonts default to CMR (pdftex) or LMR (unicode), i.e., "the LaTeX font".
%					 	You can edit the fontset files or you can write your own, myfonts.tex, and do [fontset=myfonts].
%						If you are using multiple languages, load the babel package in your fontset file, before the fonts.
\usepackage[spanish]{babel}

%%%%%%%%% Packages used in sample chapters (not otherwise required) %%%%%%%

%% Package for code listing in Appendix A.
\usepackage{listings}%   documentation is here https://ctan.org/pkg/listings

%% Set chemical formulas nicely
\usepackage[version=4]{mhchem}%   documentation is here https://ctan.org/pkg/mhchem

%% Latin filler used in Chapter 1, with a test for package version date. https://ctan.org/pkg/lipsum
\usepackage{lipsum}
\IfPackageAtLeastTF{lipsum}{2021/09/20}{\setlipsum{auto-lang=false}}{}


%%%%%%%%%  Graphics path (to figure files)  %%%%%%%%%%%%%%%%%%%%%%%%%%%%%%%%

%% Can set graphicspath to point to specific directories containing figures (the current directory is searched automatically)
%% For instance, to search a subdirectory of the current directory called "figures" and a parallel directory called "art", set:

\graphicspath{{assets/}}

%%%%%%%%%  Representative set-up for biblatex  %%%%%%%%%%%%%%%%%%%%%%%%%%%%%

\usepackage[style=ieee,maxbibnames=10,sorting=none]{biblatex}% style=ext-numeric-comp,articlein=false,giveninits=true
	\DefineBibliographyStrings{english}{url= \textsc{url} ,  }% replaces default "[Online]. Available" by "URL"


\addbibresource{mitthesis-sample.bib}%% <== change to YOUR bib file <= CHANGE

%% to avoid split urls and stretched white space, you can set the bibliography ragged-right:
%\appto{\bibsetup}{\raggedright}

% biblatex is very powerful, and you can customize most aspects the reference list and citations to suit your needs.
% documentation is here: https://ctan.org/pkg/biblatex


%%%%%%%%%%  Option to use natbib   %%%%%%%%%%%%%%%%%%%%%%%%%%%%%%%%%%%%%%%%%

%\RequirePackage[numbers,sort&compress]{natbib}
 
%%% add bibliography to table of contents
%\apptocmd{\bibliography}{\addcontentsline{toc}{chapter}{\protect\textbf{\bibname}}}{}{}

%%% You can use this to rename the bibliography section
%\renewcommand{\bibname}{References}

%%% Can adjust space between bibliography items (change 4pt to something else; don't drop last two lengths, they are stretchable "glue")
%\setlength\bibsep{4pt plus 1pt minus 1pt}


%%%%%%%%%%  Table related packages  %%%%%%%%%%%%%%%%%%%%%%%%%%%%%%%%%%%%%%%%

\usepackage{booktabs}% better quality tables, https://ctan.org/pkg/booktabs
\usepackage{array}%    additional options for table columns, https://ctan.org/pkg/array
%\usepackage{tabularx}%   https://ctan.org/pkg/tabularx

%\usepackage{dcolumn}%    alignment on decimal place, https://ctan.org/pkg/dcolumn
%\newcolumntype{d}[1]{D{.}{.}{#1}}

\usepackage{amsmath,amssymb,amsfonts}		
\newtheorem{notation}{Notación}[chapter]
\newtheorem{theorem}{Teorema}[chapter]

%%%%%%%%%%  Option for "double spacing" %%%%%%%%%%%%%%%%%%%%%%%%%%%%%%%%%%%%

%% Back in the typewriter era, double spaced lines were convenient for editing with a pencil. 
%% In typography, the separation between lines is called "leading", and it is usually set in 
%% proportion to the font size (i.e., when the font is loaded).  If you really feel the need 
%% to change the line separation, the most attractive results will be obtained by changing the
%% leading in proportion to the the current font size, rather than just doubling the space.

%% The setspace package provides a tool for changing line separation. Use these two commands here:
%
% \usepackage{setspace}%  documentation at https://ctan.org/pkg/setspace
% \setstretch{1.1}% you can choose some other value for the stretch of space between lines
%
%% Use one or more of the these commands AFTER the frontmatter
%
% \onehalfspacing
% \doublespacing
% \singlespacing  % will turn these effects off (you can use these anywhere in the document)

%% The best result may be to stay with leading selected by the typographer who set up the font.


%%%%%%%%%%%  Metadata  %%%%%%%%%%%%%%%%%%%%%%%%%%%%%%%%%%%%%%%%%%%%%%%%%%%%%%%

% Most of the document metadata is created automatically. 
% The following items should be adjusted to match your work. <================= !!!!!!!!!!

\hypersetup{%
	pdfsubject={Template for writing MIT theses with the mitthesis class},
	% Change this to briefly state topic of your thesis 
% 
	pdfkeywords={Massachusetts Institute of Technology, MIT},
	% Add keywords that will help search engines and libraries to find your work.
	% Includes the name[s] of the author[s] 
	% (If you have used \DocumentMetadata, at line 15, you can just put "\CopyrightAuthor," for the names.)
%
	pdfurl={},
	% If you have a url for the thesis, put it here. Otherwise delete this.
	% (MIT Libraries will put your thesis in DSPACE with a persistent url after you submit it.)
%	
	pdfcontactemail={},
	% You can put a [permanent] email address into the metadata, if you like.
	% Otherwise delete this.
%
	pdfauthortitle={},
	% If you have a title, you can include it here.
}

%%%%%%%%%%%%%%  End preamble %%%%%%%%%%%%%%%%%%%%%%%%%%%%%%%%%%%%%%%%%%%%%%%%%%%%%%%%%%%%%%%%%%%%%
%%%%%%%%%%%%%%%%%%%%%%%%%%%%%%%%%%%%%%%%%%%%%%%%%%%%%%%%%%%%%%%%%%%%%%%%%%%%%%%%%%%%%%%%%%%%%%%%%%

\begin{document}

\title{APLICACIONES DE LAS ECUACIONES DE NAVIER STOKES}

\Author{LUIS EMILIO ÁLVAREZ HERRERA}{DEPARTAMENTO DE IRRIGACIÓN}
% \Author{Luisa Hernández}{Department of Research}[B.S. Mechanical Engineering, UCLA, 2018][M.S. Stellar Interiors, Vulcan Science Academy, 2020]
% \Author{Thurston Howell III}{Department of Economics}[MBA, Ferengi School of Management, 2022]

% Use once for each degree fulfilled by thesis
% For two degrees from one department, leave the department argument blank for the second degree {}.
% \Degree{Bachelor of Science in Physics}{Department of Physics}
% \Degree{Master of Science in Physics}{}
\Degree{INGENIERO EN IRRIGACIÓN}{DEPARTAMENTO DE IRRIGACIÓN}

% If there is more than one supervisor, use the \Supervisor command for each.
\Supervisor{Edward C. Pickering}{Professor of Physics}
% \Supervisor{Secunda Castor}{Professor of Research}
% \Supervisor{Quintus Castor}{Professor of Log Dams}

% Professor who formally accepts theses for your department (e.g., the Graduate Officer, Professor Sméagol,...)
% If more than one department, use more than once
% **If you need to reduce vertical space, put the acceptor title in the second argument and leave the third blank {}.**
 \Acceptor{Primus Castor}{Professor of Wetlands Engineering}{Undergraduate Officer, Department of Physics}
% \Acceptor{Tertius Castor}{Professor of Log Dams}{Graduate Officer, Department of Research}
% \Acceptor{Quarta Castor}{Professor of Lodge Building}{Graduate Officer, Department of Mechanical Engineering}

% Usage: \DegreeDate{Month}{year}
% Valid degree months are September, February, or June
\DegreeDate{June}{1876}

% Date that final thesis is submitted to department
\ThesisDate{May 18, 1876}

%%%%%%  Choose whether to have a CREATIVE COMMONS License  %%%%%%%%%%%%%%%%%%%%%%%%%%%%%%%%%%%%%%
%
% If you are using a cc license, put details of your cc license here. 
% Omit this command if you are not using a cc license.
%
\CClicense{CC BY-NC-ND 4.0}{https://creativecommons.org/licenses/by-nc-nd/4.0/}
%

%%%%%%%  Solutions for overflowing titlepage  %%%%%%%%%%%%%%%%%%%%%%%%%%%%%%%%%%%%%%%%%%%%%%%%%%%

% If your title page is overflowing (from too many names, degrees, etc.):
%
% (a) you can reduce the 12pt and 18pt skips between various blocks to 6pt with this command:
%
% \Tighten
%
% (b)  you can scale down the Signature block at the bottom with this command:
%
% \SignatureBlockSize{\small}  %or this one \SignatureBlockSize{\footnotesize}
%
% (c) you can put the acceptor name and title onto two lines, rather than three like this:
%
% \Acceptor{Tertius Castor}{Professor and Graduate Officer, Department of Research}{}
% \Acceptor{Quarta Castor}{Professor and Graduate Officer, Department of Mechanical Engineering}{}
%
% (d) you can change the font size of the the author name[s] with
%
%	\AuthorNameSize{\normalsize}
%
% (e) and you can omit any previous degrees from the title page, instead mentioning them in the Biosketch

% Also, if you prefer to keep the text toward the top of the page with most white space at the bottom, you
% can you this command to squash all of the vertical glue (stretchy space) with this command:
%
% \Squash 
%
% This command is useful when the text has not already reach the bottom of the page, since the glue gets squashed automatically
% when the page is too full.

%%%%%%%%%%%%%%%%%%%%%%%%%%%%%%%%%%%%%%%%%%%%%%%%%%%%%%%%%%%%%%%%%%%%%%%%%%%%%%%%%%%%%%%%%%%%%%%%%


% \maketitle
%% acknowledgments.tex

% From mitthesis package
% Version: 1.01, 2023/10/16
% Documentation: https://ctan.org/pkg/mitthesis


\chapter*{AGRADECIMIENTOS}
\addcontentsline{toc}{chapter}{AGRADECIMIENTOS}
A lo largo de mi vida, he recorrido senderos llenos de desafíos, aprendizajes y gratitud. Hoy, al culminar esta etapa académica, deseo dedicar este trabajo a quienes han sido el faro y la brújula en mi viaje.

A mi mamá María Carolina Herrera Díaz y a mi papá Agustín Álvarez Bautista, cuyos sacrificios y amor incondicional me han dado la fortaleza para alcanzar mis metas. Ustedes me enseñaron que la educación es el legado más valioso y que el esfuerzo constante siempre rinde frutos. Cada paso que doy es un reflejo de su dedicación y valores inculcados. Mamá, papá, esta tesis es tan suya como mía.

A mis abuelos Mamá Aya, Papá Gogo, Luisa y Agustín, guardianes de la sabiduría y el cariño eterno. Aunque algunos ya no estén físicamente, sus enseñanzas y amor permanecen vivos en mi corazón. Sus historias y consejos me han guiado en los momentos más difíciles, dándome el coraje para persistir y superar obstáculos.

A mis hermanos Paulo Elías, Alan Yareth y Aranza Ailín, incondicionales de aventuras y desafíos. Gracias por ser mi apoyo en los días grises y mi celebración en los días de triunfo. Su confianza en mí ha sido una fuente de motivación constante.

A mis profesores Humberto López Chimil, Fernando Chávez León, Luis Castellanos y todos mis mentores, que con su sabiduría y paciencia han encendido en mí la llama del conocimiento. Sus enseñanzas han trascendido las aulas y han dejado una huella imborrable en mi formación personal y profesional. Gracias por creer en mi potencial y por inspirarme a ser mejor cada día.

Finalmente, dedico esta tesis a Dios, porque el me dió la voluntad de perseverar a pesar de las adversidades, por cada noche en vela y cada instante de duda superado. Este logro es el resultado de años de esfuerzo y dedicación, me recuerda que los sueños se alcanzan con determinación y pasión. Gracias a todos los que han sido parte de este viaje. Esta tesis es una manifestación de vuestro amor, apoyo y fe en mí. A todos ustedes, mi eterna gratitud.% .tex extension is presumed by \include 

% %% biography.tex
%% This section is optional

% From mitthesis package
% Version: 1.01, 2023/10/16
% Documentation: https://ctan.org/pkg/mitthesis

\chapter*{DATOS BIOGRÁFICOS}
\addcontentsline{toc}{chapter}{DATOS BIOGRÁFICOS}
Luis Emilio Álvarez Herrera nació en Texcoco de Mora, Estado de México el 7 de Junio de 2002. En 2017 ingresó a la Univerisdad Autónoma Chapingo y se unió al Departamento de Irrigación en 2020 con un intercambio en la Univerisdad de Agricultura de Tokio, Japón en 2023 y realizó prácticas profesionales en la Univerisdad de Arizona, USA 2025. Fue Miembro del programa de Nuevos Investigadores (PROFONI) de 2021-2025, autor de ters libros: ``Fundamentos de la Ingeniería en Irrigación'' con ocho volumenes, ``Matemáticas del Cubo Rubik'' y ``Huertos Agroecológicos''.

Se dedicó a la docencia en Matemáticas de 2017-2022 en el Colegio Euro Texcoco como asistente del Maestro Fernando Chávez León. Fue consejero departamental de la Preparatoria Agrícola en 2018 y presidente del Club de Ciencias Netzahualpilli de 2019-2022, nuevamente en 2020 fue consejero en el Departamento en Irrigación. Es CEO del Proyecto Miyotl:Aprende una lengua Indígena, CTO del proyecto Tláloc App: Ciencia ciudadana para el monitoreo de lluvia del Monte Tláloc del COLPOS y CEO de la Olimpiada Mexicana de Agronomía.




% optional, see MIT Libraries https://libraries.mit.edu/distinctive-collections/thesis-specs/#format

%%% Table of contents and lists of stuff (delete lists you don't need, e.g., if no tables) %%%%%%%%
\renewcommand{\contentsname}{CONTENIDO}
\tableofcontents
\renewcommand{\listtablename}{LISTA DE CUADROS}
\listoftables
\renewcommand{\listfigurename}{LISTA DE FIGURAS}
\listoffigures
\renewcommand\abstractname{RESUMEN}
\begin{abstract}
	% From mitthesis package
% Version: 1.01, 2023/06/19
% Documentation: https://ctan.org/pkg/mitthesis
%
% The abstract environment creates all the required headers and footnote. 
% You only need to add the text of the abstract itself.
%
% Approximately 500 words or less; try not to use formulas or special characters
% If you don't want an initial indentation, do \noindent at the start of the abstract

La evapotranspiración es uno de los principales componentes del ciclo
hidrológico y de vital importancia en la planeación y operación de zonas de riego,
pues de ella dependen en gran medida los cálculos para conocer las necesidades
hídricas de los cultivos para evitar la sub o sobreestimación de la lámina de riego
aplicada, sin embargo su estudio resulta complicado pues la medición depende
de dos procesos separados, evaporación y transpiración los cuales varían
espacial y temporalmente, por lo que existen métodos para la estimación de ésta
con ayuda de información meteorológica.
Con el objetivo de determinar la evapotranspiración de referencia (ETo) con
Machine Learning en la estación meteorológica Chapingo, México, se presentan
cuatro modelos para la predicción de la ETo, mediante machine learning usando
redes neuronales artificiales para el proceso de entrenamiento. Para entrenar el
modelo, se utilizaron 5119 datos diarios de la estación meteorológica automática
Chapingo, con los que se calcularon la ETo usando el método de FAO PenmanMonteith.
Se realizó un diagrama de correlaciones con el que se identificaron las variables
con mayor impacto en el cálculo de la ETo, sobresaliendo la radiación solar,
posteriormente la temperatura máxima, la humedad relativa, así como la
humedad relativa mínima y máxima. Con esta información se eligieron la
combinación de variables que sirvieron como datos de entrada a cada uno de los
modelos a entrenar.
En cada uno de los modelos se encontraron los parámetros de la red neuronal
que optimizaron el cálculo, tales como arquitectura de la red, capas ocultas y
número de neuronas en cada capa, así como el número de iteraciones, learning
rate y funciones de activación.
En el modelo 4, usando únicamente la radiación solar, se obtuvo un muy buen
ajuste del modelo con una R2 de 0.92, y un RSME de 0.0119, en los modelos 1 y
2, usando también la temperatura máxima, además la humedad relativa máxima
y mínima se mejoró en poca medida el ajuste del modelo, obteniendo una R2 de
0.93 en ambos casos y un RSME de 0.0082 y 0.0074 respectivamente.
Finalmente, el modelo 3, que no consideró la radiación solar no se ajustó
correctamente obteniendo una R2 de 0.66 y un RSME de 0.1946.
Palabras clave: Inteligencia Artificial, FAO Penman-Monteith, Redes Neuronales
Artificiales, Modelación, Agrometeorología. \footnote{Text from Holman (1876): \doi{10.2307/25138434}.}  


\textbf{Palabras clave:} Ecuaciones de Navier-Stokes en R3, Inteligencia Artificial, Hidropónia, Aeropónia, Agricultura Vertical% use \input rather than \include because we're inside an environment
\end{abstract}
\renewcommand\abstractname{ABSTRACT}
\begin{abstract}
	\input{abstract.tex}% use \input rather than \include because we're inside an environment
\end{abstract}

%%% Chapters of thesis  %%%%%%%%%%%%%%%%%%%%%%%%%%%%%%%%%%%%%%%%%%%%%%%%%%%%%%%%%%%%%%%%%%%%%%%%%%%

%% If you want to use "double spacing", you should start here...

\newgeometry{left=4cm, right=2.5cm, top=2.5cm, bottom=2.5cm, marginparwidth=0pt, headsep=0pt}
\chapter{INTRODUCCIÓN}
\pagenumbering{arabic}
\setcounter{page}{1}
%TODO: SE HACE AL FINAL problema que se pretende resolver, justificación, hipótesis, objetivos, .
% al final de cada capitulo poner conclusion

% Contextualización del tema
% DESARROLLO DEL CONTEXTO
% Planteamiento del problema

% Justificación



% Alcances y limitaciones

% Antecedentes  









% Contexto 

% \section{Flutter}
% \section{Firebase}
\section{Planteamiento del problema}

En México, las redes oficiales de monitoreo hidrometeorológico, como las operadas por la Comisión Nacional del Agua (CONAGUA), presentan una cobertura limitada en muchas regiones de montaña, donde los microclimas pueden variar significativamente en distancias cortas. % TODO: CITA BIBLIO


El Monte Tláloc, ubicado en la zona montañosa del oriente del Valle de México, es un ejemplo de ello: su importancia ambiental, histórica y cultural contrasta con la escasa información climática precisa y en tiempo real disponible para la comunidad local, investigadores y tomadores de decisiones. Esta falta de datos puntuales dificulta la \textbf{gestión sustentable del agua}, la prevención de riesgos y el análisis del cambio climático a escala local.
% TODO AGREGAR MÁS PROBLEMÁTICAS

Las aplicaciones disponibles para la recolección de datos meteorológicos suelen ser de uso profesional, poco accesibles o no están diseñadas para fomentar la participación ciudadana en contextos rurales o de baja conectividad. Esto genera una brecha entre el potencial de colaboración ciudadana y las herramientas disponibles para lograrlo.

Ante este panorama, surge la necesidad de desarrollar una aplicación multiplataforma intuitiva, accesible y robusta, que aproveche el poder de la ciencia ciudadana para llenar los vacíos de información sobre la precipitación en el Monte Tláloc. Dicha aplicación debe facilitar la recolección, visualización y validación de datos por parte de usuarios no expertos, promoviendo la generación de conocimiento colectivo, la educación ambiental y la participación activa de la comunidad en temas de gestión hídrica y climática.



% 2. Análisis del Movimiento de Nutrientes en Sistemas Hidropónicos Mediante Ecuaciones de Navier-Stokes

% Objetivo: Investigar cómo las ecuaciones de Navier-Stokes pueden mejorar la distribución de nutrientes en sistemas hidropónicos.
% Metodología: Modelado matemático y simulaciones para optimizar el diseño del sistema de riego.
% Impacto: Aumentar la eficiencia del uso de nutrientes y agua, promoviendo un crecimiento vegetal uniforme.


% 4. Estudio del Efecto de Diferentes Configuraciones de Riego en la Agricultura Vertical Mediante Navier-Stokes
% Objetivo: Analizar cómo distintas configuraciones de riego afectan la distribución de agua y nutrientes.
% Metodología: Simulaciones basadas en las ecuaciones de Navier-Stokes para evaluar la eficiencia de diferentes sistemas.
% Impacto: Optimización del riego, reducción del desperdicio de agua y mejora del crecimiento de las plantas.

%  8.Desarrollo de Algoritmos para el Control Automático del Riego en Agricultura Vertical Basados en Modelos CFD
% Objetivo: Crear algoritmos que automaticen el riego basándose en datos de simulaciones de flujo de fluidos.
% Metodología: Integración de modelos CFD con sistemas de control automático.
% Impacto: Incrementar la precisión del riego, optimizando el uso de agua y nutrientes.

\section{Contexto geográfico del Monte Tláloc}


El Monte Tláloc, es un volcán formado a partir de las capas de sucesivas erupciones basálticas fluidas; ubicado en el Eje Neovolcánico en el límite entre los municipios de Ixtapaluca y Texcoco al oriente del Estado de México. Forma parte de la Sierra Nevada y es el Área Natural Protegida “Parque Nacional Iztaccíhuatl-Popocatépetl” su ubicación hidrológica es al oriente de la cuenca de México. Con sus 4120 metros sobre el nivel del mar, el Tláloc es la novena cima más alta del país. Cuenta con un clima de montaña cuya designación oficial es semifrío subhúmedo con lluvias en verano, de humedad media \cite{inegi_texcoco}.


\section{Justificación}
Se identifica la necesidad de un instrumento para la captura y envío de datos pluviales que sea accesible, participativo y que garantice la disponibilidad de la información obtenida para su análisis y toma de decisiones. Este instrumento debe ser sencillo de usar y estar diseñado específicamente para el público objetivo: los ejidatarios. Ellos, a través de su conocimiento del territorio y participación activa, pueden convertirse en aliados estratégicos para la recolección continua y precisa de datos.
La aplicación desarrollada se plantea como una solución innovadora que responde a esta necesidad. Su diseño intuitivo permite que usuarios con conocimientos tecnológicos básicos puedan capturar y enviar información sobre las precipitaciones de manera rápida y eficiente. Además, al integrar elementos de ciencia ciudadana, se fomenta la colaboración activa de las comunidades locales, fortaleciendo su empoderamiento y compromiso con la conservación de los recursos hídricos.
Desde un enfoque técnico, el proyecto destaca por su carácter práctico y adaptable. La app aprovecha tecnologías modernas para registrar datos de lluvia, lo que no solo optimiza la recopilación de información en tiempo real, sino que también reduce los costos asociados a equipos de medición tradicionales. Al centralizar y analizar estos datos en una plataforma digital, se genera un repositorio de información confiable que puede ser utilizado por investigadores, autoridades locales y los mismos ejidatarios para tomar decisiones fundamentadas. 
Por último, la disponibilidad de esta información en un formato accesible y visualmente comprensible contribuye a sensibilizar a los usuarios sobre la importancia de monitorear los patrones de lluvia, facilitando su uso en estrategias de manejo hídrico, planificación agrícola y mitigación de riesgos climáticos. De esta forma, el proyecto no solo soluciona un problema técnico, sino que también tiene un impacto social y ambiental significativo.

\section{Hipótesis}

La implementación de una aplicación multiplataforma, basada en principios de ciencia ciudadana, mejora significativamente la precisión y frecuencia de los reportes de lluvia en la región del Monte Tláloc, al facilitar la participación activa de los habitantes locales mediante herramientas digitales accesibles e intuitivas; lo cual contribuye a la generación de datos meteorológicos complementarios a los obtenidos por estaciones profesionales, permitiendo una caracterización más detallada de los eventos de precipitación en zonas de difícil acceso.



\section{Contribuciones de este trabajo}

Este trabajo de tesis contribuye al campo del desarrollo tecnológico, la ciencia ciudadana y la meteorología local mediante la creación de una aplicación multiplataforma diseñada específicamente para el monitoreo participativo de lluvia en el Monte Tláloc. La solución propuesta integra tecnologías móviles modernas con servicios en la nube y diseño centrado en el usuario, permitiendo que cualquier ciudadano pueda registrar datos de precipitación de manera sencilla, segura y estructurada. Esta contribución tiene un impacto directo en la generación de datos alternativos en regiones donde la infraestructura meteorológica es escasa o limitada, y donde los fenómenos hidrometeorológicos presentan comportamientos complejos.

Desde el punto de vista técnico, la tesis presenta una arquitectura modular desarrollada con Flutter, integrando funcionalidades clave como, sincronización con Firebase, visualización gráfica de estadísticas y un sistema para validar la veracidad de las mediciones con base en algoritmos desarrollados para la interpretación de datos de pluviómetros caseros. Se propone también una metodología de evaluación del nivel de maduración tecnológica (TRL) aplicada a aplicaciones de ciencia ciudadana, lo cual permite medir de forma objetiva el avance y aplicabilidad real del sistema desarrollado.

Además, este trabajo representa un esfuerzo por brindar el acceso a las tecnologías de monitoreo ambiental, empoderando a las comunidades rurales al integrarlas como agentes activos en la recolección de datos climáticos, al tiempo que fortalece los vínculos entre el conocimiento científico y la sabiduría local. Finalmente, se generan aportes a futuras investigaciones en temas relacionados con aplicaciones móviles para monitoreo ambiental, ciencia abierta y educación en contextos rurales, abriendo camino a iniciativas de colaboración interdisciplinaria entre desarrolladores, científicos, comunidades y tomadores de decisiones.


\section{Esquema de la tesis}
Este trabajo está estructurado como sigue. La introducción va seguida de X capítulos independientes que se han ordenado de forma coherente con el proceso de desarrollo de las ecuaciones de Navier-Stokes. En gran medida, la notación es consistente a lo largo de esta tesis y cada excepción está claramente resaltada. Para facilitar la navegación, se incluyen apéndices al final de sus respectivos capítulos, mientras que la bibliografía acumulativa se adjunta al final de este documento. Los capítulos siguientes se resumen brevemente a continuación.
\begin{itemize}
    \item \textbf{Capítulo 1} Consiste en...
    \item \textbf{Capítulo 2} Consiste en...
    \item \textbf{Capítulo 3} Consiste en...
    \item \textbf{Capítulo 4} Consiste en...
    \item \textbf{Capítulo 5} Consiste en...
    \item \textbf{Capítulo 6} Consiste en...
\end{itemize}


\section{Alcance y limitaciones del estudio}









% \section{Estado del arte}
% \subsection{Una formulación canónica hamiltoniana del problema de Navier-Stokes}
% En marzo de 2024, un grupo de matemáticos de Carolina del Sur, propusieron una formulación canónica hamiltoniana \cite{sanders2024canonical}, a continuación se muestra la ecuación:
% \begin{equation}
%     \int\,dx_2\left[\frac{1}{2} \frac{1}{\rho^2} \frac{\delta S^*}{\delta u_1}\frac{\delta S^*}{\delta u_1} - \frac{1}{\rho}\left(p_{,1} - \mu u_{1,22} \frac{\delta S^*}{\delta u_1} \right) \right] + \frac{\delta S^*}{\delta \partial t} = 0
% \label{eqea1}
% \end{equation}
% con $\delta S^*/Sp=0$. La solución a la ecuación (\ref{eqea1}) proporcionaría una transformación canónica a una nueva conjunto de coordenadas, dando expresiones analíticas para $(u_1,p)$

% EXPLOSIÓN EN TIEMPO FINITO PARA UNA ECUACIÓN TRIDIMENSIONAL PROMEDIO DE NAVIER-STOKES 2015

% \subsection{Una formulación cuantitativa del problema de regularidad global para el periódico Sistema Navier-Stokes, Terence Tao}
% 2007



% La gran incógnita es la turbulencia. Para escalas finas en tres dimensiones es mucho más no lineal.
% NO EXISTE UNA EXPLICACIÓN MATEMÁTICA FORMAL DE CÓMO SE PASA DE UN FLUJO REGULAR A UN FLUJO TURBULENTO



% EL TRABAJO MÁS NUEVO. PUEDE SER EL DE LARENCE TAO



\chapter{OBJETIVOS}

\section{Objetivo General}
\begin{enumerate}
    \item Documentar el problema de las ecuaciones de Navier-Stokes en tercera dimensión
    \item Exponer las aplicaciones en el Ingeniería en Irrigación
    \item Aplicar algoritmos de Inteligencia Artificial y Machine Learning a los métodos numéricos
\end{enumerate}
\section{Objetivos Específicos}

\begin{enumerate}
    \item Modelar las ecuaciones de Navier-Stokes en un sistema de Agricultura Vertical Plant Factory Hidropónico con relación agua-planta
\end{enumerate}





\chapter{REVISIÓN DE LITERATURA}
\section{Definición de términos clave}
% \section{Ley de conservación de la masa}
La ley de conservación de masa es definida tal que:``\textit{En un sistema aislado, durante toda reacción química ordinaria, la masa total en el sistema permanece constante, es decir, la masa consumida de los reactivos es igual a la masa de los productos obtenidos}'' (Antoine Lavoisier, 1785 \cite{sterner2011conservation}).
\begin{definition}[Ley de conservación de la masa]
    a
\end{definition}
Se expresa esta ley mediante un sistema de ecuaciones diferenciales en derivadas parciales.

Conservación de la cantidad de movimiento
\begin{equation}
    F = m \cdot a = \frac{d}{dt}(mu)
\end{equation}
\begin{definition}[Fluido]
    Es una sustancia que puede ser líquido o gas.
\end{definition}
Se caracteriza un fluido por las siguientes funciones:
\begin{enumerate}
    \item \textbf{Campo de velocidades} $u(x,t) = \left(u_1(x,t),u_2(x,t),u_3(x,t)\right)$ que determina la velocidad que tiene una partícula en cada punto $x\in\Omega$ del dominio y en cada tiempo $t\in \mathbb{R}$
    \item \textbf{Presión}, $p=p(x,t)$, en el seno del fluido
    \item \textbf{Densidad}, $\rho= \rho(x,t)$ del fluido
\end{enumerate}
\begin{equation}
    \underbrace{\frac{\partial}{\partial t} u_i}_{\text{Campo de velocidad}} + \underbrace{\sum_{j = 1}^n u_j \frac{\partial u_i}{\partial x_j}}_{\text{Momentum del flujo}} = \underbrace{\nu \Delta u_i}_{\text{Viscosidad}} - \underbrace{ \frac{\partial p}{\partial x_i}}_{\text{Campo de presión}}+ \underbrace{f_i(x,t)}_{\text{Fuerzas externas}} 
\end{equation}
\begin{equation}
    \underbrace{\text{div}\, u}_{\text{Condición de incompresibilidad}} = \underbrace{\sum_{j = 1}^n \frac{\partial u_i}{\partial x_j}}_{parte\ 1} 
\end{equation}
\begin{equation}
    \underbrace{u(x,0)}_{parte 1} = \underbrace{u^{\circ}(x)}_{parte\ 1} 
\end{equation}

%%%%%%%%%%%%%%%%%%%%%%%%%%%%%%%%%%%%%%%%%%%%%%%%%%%%%%%%%%%%%%%%%%%%%%%%%%%%%%%%%%%%%%%%%%%%%%%%%%%%%%
% Revisión de literatura relevante
\section{Ecuaciones de Navier-Stokes}
Las ecuaciones de Euler y Navier-Stokes describen el movimiento de un fluido en $\mathbb{R}^n$ tal que $(n=2\text{ o }3)$. Estas ecuaciones se resuelven por un vector desconocido de velocidad $u(x,t)= \left(u_i(x,t)\right)_{1\leq i \leq n}\in \mathbb{R}^n$ y presión $p(x,t)\in \mathbb{R}$, definido por la posición $x\in \mathbb{R}^n$ y tiempo $t\geq 0$. Se restringe a fluidos incompresibles que están en $\mathbb{R}^n$, dichas ecuaciones son:
\begin{align}
    &\frac{\partial}{\partial t} u_i + \sum_{j = 1}^n u_j \frac{\partial u_i}{\partial x_j} = \nu \Delta u_i - \frac{\partial p}{\partial x_i} + f_i(x,t)&& \left(x\in \mathbb{R}^n, t\geq 0\right),\label{eq1}\\
    &\text{div}\, u = \sum_{j = 1}^n \frac{\partial u_i}{\partial x_j} && \left(x\in \mathbb{R}^n, t\geq 0\right) =0\label{eq2}\\
    &\text{Condiciones iniciales:}\quad u(x,0) = u^{\circ}(x) && \left(x\in \mathbb{R}^n\right)\label{eq3}
\end{align}
\begin{notation}
De las ecuaciones (\ref{eq1}), (\ref{eq2}) y (\ref{eq3})
    \begin{itemize}
        \item $u^{\circ}(x)$ está dado
        \item $C^{\infty}$  es un campo vectorial libre de divergencia en $\mathbb{R}^n$,
        \item $f_i(x,t)$ son los componentes de una fuerza externa aplicada (ej. gravedad),
        \item $\Delta = \sum_{i=1}^n \frac{\partial^2}{\partial x_i^2}$ es el Laplaciano en espacios variables.
        \item La viscosidad $\nu$, es un coeficiente positivo\footnote{Las ecuaciones de Eulier son las mismas que (\ref{eq1}), (\ref{eq2}) y (\ref{eq3}), pero con viscosidad $\nu=0$}
    \end{itemize}    
\end{notation}
\subsection{Descripción de las ecuaciones de Navier-Stokes}
\begin{itemize}
    \item La ecuación (\ref{eq1}), se basa en la Segunda Ley de Newton $f=m\cdot a$ para un fluido sujeto a una fuerza externa $f= \left(f_i(x,t)\right)_{1\leq i\leq n}$ y a las fuerzas que surgen de la presión y la fricción.
    \item La ecuación (\ref{eq2}), define que el fluido es incompresible.  $u(x, t)$ no debe crecer a medida que $\left\lvert x\right\rvert \to \infty$. Por lo tanto, se restringirá la atención en las fuerzas $f$ y las condiciones iniciales $u^{\circ}$ que satisfacen las ecuaciones (\ref{eq4}) y (\ref{eq5})
    \begin{align}
        &\left\lvert \partial_x^{\alpha}u^{\circ}(x) \right\rvert\leq C_{\alpha K}\left(1 + \left\lvert x \right\rvert  \right)^{-K}\in \mathbb{R}^n&& \forall\, \alpha\land K\label{eq4}\\
        &\left\lvert \partial_x^{\alpha}\partial_t^{m} f(x,t) \right\rvert\leq C_{\alpha m K}\left(1 + \left\lvert x \right\rvert + t \right)^{-K}\in \mathbb{R}^n\times \left[ 0,\infty \right)&& \forall\, \alpha, m \land K \label{eq5}
    \end{align}
\end{itemize}
Se acepta una solución de (\ref{eq1}), (\ref{eq2}), (\ref{eq3}) como físicamente razonables sólo si satisfacen las ecuaciones (\ref{eq6}) y (\ref{eq7})
\begin{align}
    &p,u\in C^{\infty}\left( \mathbb{R}^n \times \left[ 0, \infty\right) \right)\label{eq6}\\
    & \int_{\mathbb{R}^n} \left\lvert u(x,t)\right\rvert^2\, dx < C\quad \forall t\geq 0\label{eq7}  
\end{align}
Alternativamente, para descartar problemas en el infinito, se buscan espacios espacialmente soluciones periódicas de (\ref{eq1}), (\ref{eq2}), (\ref{eq3}). Por lo tanto, asumimos que $u^{\circ}(x), f(x,t)$ satisfacen
\begin{align}
    u^{\circ}(x + e_j) = u^{\circ}(x),\quad f(x + e_j,t) = f(x,t),\quad p(x + e_j,t) = p(x,t) && \forall\, 1\leq j\leq n\label{eq8}
\end{align}
\begin{notation} De la ecuación (\ref{eq8})
    \begin{itemize}
        \item $e_j = j^{th}$
    \end{itemize}
\end{notation}
En lugar de (\ref{eq4}) y (\ref{eq5}), se supone que $u^{\circ}$ es suave y que se cumple la ecuación (\ref{eq9})
\begin{align}
    &\left\lvert \partial_x^{\alpha}\partial_t^{m} f(x,t) \right\rvert\leq C_{\alpha m K}\left(1 + \left\lvert t \right\rvert \right)^{-K}\in \mathbb{R}^3\times \left[ 0,\infty \right)&& \forall\, \alpha, m \land K 
    \label{eq9}
\end{align}
Se acepta una solución de (\ref{eq1}), (\ref{eq2}), (\ref{eq3}), físicamente satisfacen las ecuaciones (\ref{eq10}) y (\ref{eq11})
\begin{align}
    &u(x + t) = u(x + e_j,t)\in \mathbb{R}^3 \times  \left[0,\infty \right)&& \forall\, 1\leq j\leq n\label{eq10}\\
    & p,u\in C^{\infty}\left(\mathbb{R}^n \times \left[0,\infty \right) \right)
    \label{eq11}
\end{align}
Desarrollando la ecuación (\ref{eq10}), se tiene:
\begin{align*}
&- \iint_{\mathbb{R}^n \times \mathbb{R}} u \cdot \Delta \frac{\partial \theta}{\partial t} \, dx\, dt - \sum_{ij} \iint_{\mathbb{R}^n \times \mathbb{R}} u_i u_j \frac{\partial \theta_i}{\partial x_j}\,dx\,dt\\
&= \nu \iint_{\mathbb{R}^n \times \mathbb{R}} u \cdot  \Delta \theta\, dx\,dt + \iint_{\mathbb{R}^n \times \mathbb{R}} f \cdot \theta \, dx\, dt + \iint_{\mathbb{R}^n \times \mathbb{R}} p \cdot (\text{div}\theta)\,dx\,dt
\end{align*}
\subsection{Demostraciones a resolver}
Se solicita una demostración de una de las siguientes cuatro afirmaciones.
\begin{enumerate}
    \item \textbf{Existencia y suavidad de las soluciones de Navier-Stokes en $\mathbb{R}^3$.}
    \begin{enumerate}
        \item Con $\nu>0$ y $n=3$, sea $u^{\circ}(x)$ cualquier campo vectorial suave y libre de divergencias que satisfaga la ecuación (\ref{eq4})
        \item Se toma $f(x,t)=0$, entonces existe una función suave $p(x,t),u_i(x,t)$ en $\mathbb{R}^3\times [0,\infty)$ que satisfagan las ecuaciones (\ref{eq1}),(\ref{eq2}),(\ref{eq3}),(\ref{eq6}),(\ref{eq7})
    \end{enumerate}
    \item \textbf{Existencia y fluidez de soluciones Navier-Stokes en $\mathbb{R}^3/\mathbb{Z}^3$} \begin{enumerate}
        \item Con $\nu > 0$ y $n = 3$. Sea $u^{\circ}(x)$ cualquier campo vectorial suave y libre de divergencia que satisfaga (\ref{eq8});
        \item Se toma $f(x, t)=0$. Entonces existen funciones suaves $p(x, t)$, $u_i(x, t)$ en $R^3\times [0,\infty)$ que satisfacen (\ref{eq1}),(\ref{eq2}),(\ref{eq3}),(\ref{eq10}),(\ref{eq11})
    \end{enumerate}
    \item \textbf{Desglose de las soluciones Navier-Stokes en $\mathbb{R}^3$} \begin{enumerate}
        \item  Con $\nu > 0$ y $n = 3$. Entonces existe un campo vectorial suave y libre de divergencia $u^{\circ}(x)$ en $\mathbb{R}^3$
        \item y existe un $f(x, t)$ suave en $\mathbb{R}^3 \times [0,\infty)$, que satisface (\ref{eq4}), (\ref{eq5}), para el cual no existen soluciones $(p, u)$ de (\ref{eq1}),(\ref{eq2}),(\ref{eq3}),(\ref{eq6}),(\ref{eq7}) en $\mathbb{R}^3\times [0,\infty)$
    \end{enumerate}
    \item \textbf{Desglose de las soluciones Navier-Stokes en $\mathbb{R}^3/\mathbb{Z}^3$.} \begin{enumerate}
        \item Con $\nu > 0$ y $n = 3$. Entonces existe un campo vectorial suave y libre de divergencia $u^{\circ}(x)$ en $\mathbb{R}^3$
        \item y existe un $f(x, t)$ suave en $\mathbb{R}^3 \times [0,\infty)$, que satisface (\ref{eq8}), (\ref{eq9}), para el cual no existen soluciones $(p, u)$ de (\ref{eq1}),(\ref{eq2}),(\ref{eq3}),(\ref{eq10}),(\ref{eq11}) en $\mathbb{R}^3 \times [0,\infty)$.
    \end{enumerate}
\end{enumerate}
\subsection{Resultados parciales conocidos}
\subsubsection{Dos dimensiones}
\textbf{Resuelto.} En la sección (\ref{sec321}), se demuestra para los análogos de las afirmaciones (1) y (2) por Ladyzhenskaya \cite{ladyzhenskaya1969mathematical}, también para el caso más difícil de las ecuaciones de Euler. 
\subsubsection{Tres dimensiones}
\textbf{Sin resolver.} En tres dimensiones, se sabe que (1) y (2) se mantienen siempre que la velocidad inicial $u^{\circ}$ satisfaga una condición de pequeñez. Para los datos iniciales $u^{\circ}(x)$ que no se supone que sean pequeños, se sabe que (1) y (2) se cumplen (también para $\nu=0$) si el intervalo de tiempo $[0,\infty)$ se reemplaza por un intervalo de tiempo pequeño $\left[0, T\right)$, dependiendo $T$ de los datos iniciales.

Para un $u^{\circ}(x)$ inicial dado, el $T$ máximo permitido se denomina ``tiempo de explosión''. (1) y (2) se cumplen, o bien hay un $u^{\circ}(x)$ suave y sin divergencia para el cual (\ref{eq1}), (\ref{eq2}), (\ref{eq3}) tienen una solución con un tiempo de explosión finito. Para las ecuaciones de Navier-Stokes $(\nu > 0)$, si hay una solución con un tiempo de explosión finito $T$, entonces la velocidad $\left(u_i(x,t)\right)_{1\leq i \leq n}\in \mathbb{R}^n$ se vuelve ilimitado cerca del momento de la explosión.

Se sabe que suceden otras cosas desagradables en el momento de explosión $T$, si $T < \infty$. Para las ecuaciones de Euler ($\nu = 0$), si hay una solución (con $f \equiv  0$) con un tiempo de explosión finito $T$, entonces la vorticidad $\omega(x, t) = curl_x u(x, t)$ satisface
\begin{align}
    &\text{Baele-Katp-Majda}&&\int_0^T \left\{ \sup_{x\in \mathbb{R}^3}\left\lvert \omega(x,t)\right\rvert\right\}\, dt = \infty
\end{align}
para que la vorticidad explote rápidamente.

Muchos cálculos numéricos parecen mostrar una explosión en las soluciones de las ecuaciones de Euler, pero la extrema inestabilidad numérica de las ecuaciones hace que sea muy difícil sacar conclusiones confiables.

Los resultados anteriores están muy bien tratados en el libro de Bertozzi y Majda \cite{majda2002vorticity}. A partir de Leray \cite{leray1934mouvement}, se han logrado importantes avances en la comprensión de las soluciones débiles de las ecuaciones de Navier-Stokes. Para llegar a la idea de una \texttt{solución débil} de las ecuaciones de Navier-Stokes.

Para llegar a la idea de una solución débil en una Ecuación Diferencial Parcial (EDP), se integra la ecuación con una función de prueba y luego se integra por partes (formalmente) para hacer que las derivadas caigan en la función de prueba. Por ejemplo, si (\ref{eq1}) y (\ref{eq2}) se cumplen, entonces, para cualquier campo vectorial suave $\theta(x, t) = \left(\theta_i(x, t)\right)_{1\leq i\leq n}$ soportado de forma compacta en $\mathbb{R}^3 \times (0,\infty)$ , una integración formal por partes produce
\begin{equation}
\begin{split}
    &\iint_{\mathbb{R}^3\times \mathbb{R}}u \frac{\partial \theta}{\partial t}\,dx\,dt - \sum_{ij} \iint_{\mathbb{R}^r\times \mathbb{R}}u_i u_j \frac{\partial \theta_i}{\partial x_j} \,dx\,dt\\
    &=\nu \iint_{\mathbb{R}^r\times \mathbb{R}} u \cdot \Delta \theta\,dx\,dt + \iint_{\mathbb{R}^3\times \mathbb{R}} f \cdot \theta \,dx\,dt - \iint_{R^3\times \mathbb{R}}p \cdot \left(div\theta\right)\,dx\,dt
    \label{eq12}
\end{split}
\end{equation}
La ecuación (12) tiene sentido para $u\in L^2, f\in L^1, p\in L^1$, mientras que (\ref{eq1}) tiene sentido sólo si $u(x, t)$ es dos veces diferenciable en x. De manera similar, si $\varphi (x, t)$ es una función suave, soportada de manera compacta en $\mathbb{R}^3 \times (0,\infty)$, entonces una integración formal por partes y (\ref{eq2}) implica:
\begin{equation}
    \iint_{\mathbb{R}^3\times \mathbb{R}} u \cdot \nabla_x \varphi \,dx\,dt = 0
    \label{eq13}
\end{equation}
Una solución de (\ref{eq12}), (\ref{eq13}) se llama solución débil de las ecuaciones de Navier Stokes.

Una idea establecida desde hace mucho tiempo en el análisis es demostrar la existencia y regularidad de las soluciones de una EDP construyendo primero una solución débil y luego demostrando que cualquier solución débil es suave. Este programa se ha probado para Navier-Stokes con éxito parcial.

Leray en \cite{leray1934mouvement} demostró que las ecuaciones de Navier-Stokes (\ref{eq1}), (\ref{eq2}), (\ref{eq3}) en tres dimensiones espaciales siempre tienen una solución débil $(p, u)$ con propiedades de crecimiento adecuadas. Se desconoce la unicidad de las soluciones débiles de la ecuación de Navier-Stokes. Para la ecuación de Euler, la unicidad de las soluciones débiles es sorprendentemente falsa. Scheffer \cite{scheffer1993inviscid} y, más tarde, Schnirelman \cite{shnirelman1997nonuniqueness} exhibieron soluciones débiles de las ecuaciones de Euler en $\mathbb{R}^2 \times \mathbb{R}$ con soporte compacto en el espacio-tiempo. Esto corresponde a un fluido que parte del reposo en el instante $t = 0$, comienza a moverse en el instante $t = 1$ sin estímulo externo y vuelve al reposo en el instante $t = 2$, con su movimiento siempre confinado a una bola $B \subset  \mathbb{R}^3$.

Scheffer \cite{scheffer2006turbulence} aplicó ideas de la teoría de la medida geométrica para demostrar un teorema de regularidad parcial para soluciones débiles adecuadas de las ecuaciones de Navier-Stokes.

Caffarelli-Kohn-Nirenberg \cite{caffarelli1982partial} mejoraron los resultados de Scheffer y F.-H. Lin \cite{lin1998new} simplificó las pruebas de los resultados en Caffarelli-Kohn-Nirenberg \cite{caffarelli1982partial}. El teorema de regularidad parcial de \cite{caffarelli1982partial}, \cite{lin1998new} se refiere a un análogo parabólico de la dimensión de Hausdorff del conjunto singular de una solución débil adecuada de Navier-Stokes. Aquí, el conjunto singular de una solución débil u consta de todos los puntos $(x^{\circ}, t^{\circ}) \in \mathbb{R}^3 \times \mathbb{R}$ tales que $u$ es ilimitado en todas las vecindades de $(x^{\circ}, t^{\circ})$. (Si la fuerza $f$ es suave, y si $(x^{\circ}, t^{\circ})$ no pertenece al conjunto singular, entonces no es difícil demostrar que u puede corregirse en un conjunto de medida cero para volverse suave en una vecindad de $(x^{\circ}, t^{\circ})$.)

Para definir el análogo parabólico de la dimensión de Hausdorff, utilizamos cilindros parabólicos $Q_r = B_r \times I_r \subset \mathbb{R}^3 \times \mathbb{R}$, donde $B_r \subset \mathbb{R}^3$ es una bola de radio $r$, e $I_r \subset \mathbb{R}$ es un intervalo de longitud $r^2$. Dado $E \subset \mathbb{R} \times \mathbb{R}$ y $\delta  > 0$, establecemos
\begin{equation*}
    \mathcal{P}_{K,\delta}(E) = inf \left\{ \sum_{i = 1}^{\infty} r_{i}^{K}:Q_{r1},Q_{r2},\dots\text{Cubre }E, \text{ y cada }r_i<\delta \right\} 
\end{equation*}
y luego definir
\begin{equation*}
    \mathcal{P}_K(E) = \lim_{\delta \to 0+} \mathcal{P}_{K,\delta}(E)
\end{equation*}
Los principales resultados de \cite{caffarelli1982partial}, \cite{lin1998new} pueden expresarse aproximadamente de la siguiente manera:
\begin{theorem}
    Regularidad Parcial de Caffarelli-Kohn-Nirenberg y F.-H. Lin
    \begin{enumerate}
        \item (A) Sea u una solución débil de las ecuaciones de Navier-Stokes, que satisfaga condiciones de crecimiento adecuadas. Sea $E$ el conjunto singular de $u$. Entonces $\mathcal{P}_1(E)=0$.
        \item (B) Dado un campo vectorial libre de divergencia $u^{\circ}(x)$ y una fuerza $f(x, t)$ que satisface (\ref{eq4}) y (\ref{eq5}), existe una solución débil de Navier-Stokes (\ref{eq1}), (\ref{eq2}), (\ref{eq3}) que satisfacen las condiciones de crecimiento en (A).
    \end{enumerate}
\end{theorem}
En particular, el conjunto singular de $u$ no puede contener una curva espacio-temporal de la forma ${(x, t) \in \mathbb{R}^3 \times \mathbb{R}: x = \phi (t)}$. Este es el mejor teorema de regularidad parcial conocido hasta ahora para la ecuación de Navier-Stokes.


%%%%%%%%%%%%%%%%%%%%%%%%%%%%%%%%%%%%%%%%%%%%%%%%%%%%%%%%%%%%%%%%%%%%%%%%%%%%%%%%%%%%%%%%%%%%%%%%%%%%%%





%%%%%%%%%%%%%%%%%%%%%%%%%%%%%%%%%%%%%%%%%%%%%%%%%%%%%%%%%%%%%%%%%%%%%%%%%%%%%%%%%%%%%%%%%%%%%%%%%%%%%%
\section{Las soluciones débiles}
En 1934, Leray introdujo el concepto de soluciones débiles en su libro ``Sobre el movimiento de un espacio de llenado de un líquido viscoso'' \cite{leray1934mouvement}. Este fue un avance significativo porque, aunque no siempre es posible encontrar soluciones clásicas (suaves) a las ecuaciones de Navier-Stokes, las soluciones débiles permiten un enfoque más general.

\subsection{Existencia global de estas soluciones en tres dimensiones}



%%%%%%%%%%%%%%%%%%%%%%%%%%%%%%%%%%%%%%%%%%%%%%%%%%%%%%%%%%%%%%%%%%%%%%%%%%%%%%%%%%%%%%%%%%%%%%%%%%%%%%
\section{Ecuación de Nevier-Stokes en 2D}
% ENCONTRAR FUENTE DE INFORMACIÓN Y CORREGIR EN CASO DE ERROR
En 1958, Olga A. Ladyzhenskaya \cite{ladyzhenskaya1969mathematical} extendió estos conceptos al trabajar tanto con soluciones débiles como con soluciones fuertes (más regulares). Mientras que Leray se centró en la existencia de soluciones débiles, Ladyzhenskaya se ocupó también de la unicidad y regularidad de las soluciones, particularmente en dos dimensiones.

La desigualdad multiplicativa:
\begin{equation}
    \left\lVert u \right\rVert_{L_4(\Omega)}^4\leq c \left\lVert u\right\rVert_{L_2(\Omega)}^2\left\lVert \nabla u\right\rVert_{L_2(\Omega)}^2  
\end{equation}
Que es válido para cualquier $u\in W_2^1(\Omega),\, \Omega\in \mathbb{R}^2$. Esta desigualdad dió la posibilidad de demostrar la existencia de una solución única global del sistema bidimensional Navier-Stokes
\subsection{Demostración de Olga A. Ladyzhenskaya}
\label{sec321}
Del artículo de Ladyzhenskaya: Solución ``en grande'' del problema de valores en la frontera no estacionarios para el sistema Navier-Stokes con dos variables espaciales; traducido del ruso se describe a continuación la demostración original:

Consideremos en la región $\Omega$ de cambios $x = (x_1, x_2)$ el sistema de ecuaciones de Navier-Stokes
\begin{align}
    \begin{split}
    &v_t -\nu \Delta v + \sum_{k = 1}^{2} \upsilon_k \cdot v_{x_k} = - grad\: p + f(x,t)\\
    &\text{div}\, v = 0
    \label{eql1}
    \end{split}
\end{align}
para funciones $v = \left(\upsilon_1(x, t), \upsilon_2(x, t)  \right)$ y $p(x, t)$ bajo condiciones iniciales y de frontera
\begin{align}
    v\mid_s = 0,&&v\mid_{t = 0} = a(x)\left(div\, a = 0 \right)
    \label{eql2}
\end{align}
\begin{theorem}
    El problema (\ref{eql1})-(\ref{eql2}) tiene solución única ``en general'' (es decir, para cualquier $t\geq 0$ para cualquier valor del número de Reynolds en el momento inicial y para $f$ arbitraria), si solo las integrales son finitas
    \begin{align*}
        \int_{\Omega} a^2\,dx,&&\int_{\Omega}\left(v_t(x,0)\right)^2\,dx, &&\int_0^t\int_{\Omega} f^2 + f_t^2\,dx\,dt
    \end{align*}
\end{theorem}
De los resultados obtenidos en el trabajo (a), se deduce que toda la cuestión de la existencia ``en general'' se reduce ahora a obtener una estimación a priori de la integral
\begin{equation}
    \int_0^t\int_{\Omega} v_t^2\,dx\,dt + \int_{\Omega} \sum_{k = 1}^2 \upsilon_k^4(x,t)\,dx
    \label{eql3}
\end{equation}
O $\max \left\lvert v \right\rvert $. En vista de esto, aquí hablaremos sólo de estimaciones a priori de las soluciones a los problemas (\ref{eql1}) a (\ref{eql2}). Se sabe que para soluciones del problema (\ref{eql1}) - (\ref{eql2}) la desigualdad se cumple
\begin{align}
    \begin{split}
        &\int_{\Omega}v^2(x,t)\,dx + 2\nu \int_0^t\int_{\Omega}\sum_{k =1}^2 \left(v_{x_k}\right)^2\,dx\,dt\leq\\
        &\leq \int_{\Omega}a^2\,dx + 2\left(\int_{\Omega}a^2\,dx\right)^{\frac{1}{2}}\left(\int_{\Omega} d^2\,dx\right)^{\frac{1}{2}}\,dt + 2\left[\int_0^t\left(\int f^2\,dx\right)^{\frac{1}{2}}\,dt\right]^2
        \label{eql4}
    \end{split}
\end{align}
Denotemos $\int_{\Omega}\sum_{k=1}^2\left[v_{x_k}(x,t)\right]^2\,dx=\varphi^2(t)$ De (\ref{eql4}) se deduce que conocemos la estimación de la integral $\int_0^t \varphi^2(t)\,dt $. Sea $\int_0^t\int_{\Omega} \left(f_t\right)^2\,dx\,dt<\infty$. vamos a diferenciar

(\ref{eql1}) por t, multiplicar el resultado escalarmente por (vt) e integrar
Después de transformaciones simples llegamos a la desigualdad.

\begin{align}
    \begin{split}
        &\frac{1}{2} \int_{\Omega}\left[v_t(x,t)\right]^2\,dx\mid_{t = 0}^{t = t} + \nu\int_0^t\int_{\Omega}\sum_{k = 1}^2\left(v_{tx_p}\right)^2\,dx\,dt + \int_0^t\int_{\Omega}\sum_{k = 1}^2 \upsilon_{xt} v_{x_k}v_t\,dx\,dt\\
        &= \int_0^t\int_{\Omega}f_tv_t\,dx\,dt
        \label{e}
    \end{split}
\end{align}
De donde se despeja:
\begin{equation}
    \phi^2(t)\mid_{t = 0}^{t = t} + 2\nu\int_0^t F^2(t)\,dt\leq c\int_0^t\phi(t)\left[\int_{\Omega}\sum_{k = 1}^2v^4_{kt}\,dx \right]^{\frac{1}{2}}\,dt +\int_0^t\phi(t)b(t)\,dt
    \label{eql6}
\end{equation}
Dónde
\begin{align*}
    \phi^2(t) = \int_{\Omega}\left[v_t(x,t)\right]^2\,dx&&F^2(t) = \int_{\Omega}\sum_{k = 1}^2\left[v_{tx_k}(x,t)\right]^2\,dx&& b^2(t) = 2\int_{\Omega} f^2_t(x,t)\,dx
\end{align*}
y $c$ aquí (y más) significa las constantes que conocemos.

Verifiquemos ahora que para cualquier función continuamente diferenciable soportada de forma compacta $u(x_1,x_2)$ dos variables espaciales se cumple la siguiente desigualdad:
\begin{equation}
    \iint u^4\left(x_1,x_2\right)\,dx_1\,dx_2\leq 2\iint u^2\left(x_1,x_2\right)\,dx_1\,dx_2\iint \left(u^{2}_{x_1} + u^{2}_{x_1} \right)\,dx_1\,dx_2
    \label{eql7}
\end{equation}
En el que la integración se realiza en todo el espacio $x_1, x_2$.Es obvio que.
\begin{equation*}
    u^2\left(x_1,x_2\right) = 2 \int_{ - \infty}^{x_k} uu_{x_k}\,dx_k,\quad k = 1,2
\end{equation*}
y por lo tanto
\begin{equation*}
    \max_{x_k}u^2 \left(x_1,x_2\right) \leq 2\int_{ - \infty}^{\infty}\left\lvert uu_{x_k}\right\rvert \,dx_k,\quad k = 1,2
\end{equation*}
Es por eso
\begin{align*}
    &\iint_{ -\infty}^{\infty}u^4\,dx_1dx_2 \leq \int_{ -\infty}^{\infty}dx_2\left(\max_{x_1}u^2\cdot\int_{ -\infty}^{\infty}u^2\,dx_1  \right)\leq  \\
    &\leq 2\int_{ -\infty }^{\infty}dx_2\left(\int_{ -\infty}^{\infty}\left\lvert uu_{x_1}\right\rvert \,dx_1 \max_{x_3}\int_{-\infty}^{\infty}u^2\,dx_1\right)\leq 4\int_{ -\infty}^{\infty}\left\lvert uu_{x_1} \right\rvert\,dx_1 \,dx_2\cdot \iint_{ -\infty}^{\infty}\left\lvert uu_{x_2}\right\rvert\,dx_1\,dx_2
\end{align*}
y esto implica desigualdad (\ref{eql7}).

Usemos la desigualdad (\ref{eql7}) Para estimaciones $\int_0^t\int_{\Omega}\sum_{k = 1}^2 \upsilon_{kt}\,dx$ a la (\ref{eql6}). Porque $v_{kt}$ son iguales a cero en la frontera $S$, entonces para ellos, en virtud de (\ref{eql7}), tenemos
\begin{equation*}
    \left(\int_{\infty} \upsilon_{kt}^4\,dx \right)^{\frac{1}{2}} \leq \sqrt{2}\phi(t)F(t)
\end{equation*}
y por lo tanto de (\ref{eql6}) se sigue
\begin{equation*}
    \phi^2(t)\mid_{t = 0}^{t = t} + 2 \nu\int_0^t F^2(t)\,dt \leq C_1\int_0^t \varphi(t) \phi(t) F(t)\,dt + \int_0^i \phi(t)b(t)\,dt
\end{equation*}
De aquí, a su vez, concluimos consistentemente sobre la validez de las desigualdades.
\begin{equation*}
    \phi^2(t)\mid_{t = 0}^{t = t} + 2 \nu\int_0^t F^2(t)\,dt \leq \nu\int_0^t F^2(t)\,dt + \frac{c_1}{2\nu}\int_0^t \varphi^2\phi^2\,dt + \int_0^t\phi b\,dt
\end{equation*}
\begin{equation}
    \phi^2(t)\leq c_2\int_0^t \left(\phi^2 + b^2\right)\phi^2\,dt + c_3,
    \label{eql8}
\end{equation}
\begin{equation}
    \nu \int_0^tF^2(t)\,ft\leq c_2\int_0^t\left(\phi^2 + b^2\right)\phi^2\,dt + c_3
    \label{eql9}
\end{equation}
Dado que la función $\phi^2(t) + b^2(t)$ es sumable en $[0, t]$, se deduce de (\ref{eql8}) que $\phi^2(t)\leq c_4$ y de (\ref{eql9}) $\int_0^t F^2(t)\,dt\leq c_5$

Estas desigualdades nos dan una estimación a priori de las soluciones, incluso más fuerte que (\ref{eql3}). De la prueba anterior se desprende claramente que el tamaño de la región, así como la suavidad de su límite, no afectan los valores de las constantes $c_k$. Estos últimos dependen únicamente de las integrales especificadas en el teorema.
Tenga en cuenta que para cualquier función no negativa soportada de forma compacta $u(x_1,x_2)$ de dos variables, junto con (\ref{eql7}), la desigualdad también es cierta
\begin{equation}
    \iint u^3\,dx_1\,dx_2 \leq \frac{9}{8} \iint u\,dx_1\,dx_2 \iint \left(u_{x_1}^2 + u_{x_2}^2 \right)\,dx_1\,dx_2
    \label{eql10}
\end{equation}
La prueba de desigualdad (\ref{eql7}) dada anteriormente es similar a la prueba de desigualdad (\ref{eql10}) de A. O. Gelfond.
% https://www.mathnet.ru/links/e51b7b04594d18eae89a4ecb64309f08/dan42515.pdf
\newpage
%%%%%%%%%%%%%%%%%%%%%%%%%%%%%%%%%%%%%%%%%%%%%%%%%%%%%%%%%%%%%%%%%%%%%%%%%%%%%%%%%%%%%%%%%%%%%%%%%%%%%%
\section{Ecuaciones de Saint Venant}
El sistema de ecuaciones de Saint-Venant está compuesto por dos conjuntos de ecuaciones: las ecuaciones de continuidad y las ecuaciones de cantidad de movimiento o momentum

\subsection{Ecuación de continuidad} 
La ecuación de continuidad tiene en cuenta un balance de masa sobre un volumen de control. En forma conservativa puede escribirse en términos del caudal $Q$ y del área $A$ de la siguiente manera:
\begin{equation}
    \frac{\partial Q}{\partial x} + \frac{\partial A}{\partial t} = 0
\end{equation}
De manera no conservativa en términos de la velocidad media longitudinal $V$ y la profundidad ($y$) así:
\begin{equation}
    V \frac{\partial y}{\partial x} + y \frac{\partial V}{\partial x} + \frac{\partial y}{\partial t} = 0
\end{equation}

\subsection{Ecuaciones de cantidad de movimiento}

La ecuación de momentum surge al igualar las fuerzas externas aplicadas al
volumen de control como la gravedad, la presión, la fricción, el viento entre
otras. En forma conservativa puede escribirse esta ecuación en términos del
caudal (Q) , área (A), profundidad (A), pendiente del canal ($S_0$),
pendiente de fricción ($S_f$)  y de la gravedad ($g$) de la siguiente manera:

\begin{equation}
    \frac{1}{A} \frac{\partial Q}{\partial t} + \frac{1}{A} \frac{\partial }{\partial x} \left(\frac{Q^2}{A}\right) + g \frac{\partial y}{\partial x} - g\left(S_0 - S_f \right) = 0
\end{equation}
O de manera no conservativa en términos de la velocidad media longitudinal ($V$):

\begin{equation}
    \frac{\partial V}{\partial t} + V \frac{\partial V}{\partial x} + g \frac{\partial y}{\partial x} - g \frac{\partial y}{\partial x} - g\left(S_0 - S_f \right) = 0
\end{equation}

La forma final de continuidad y momento es:
\begin{equation}
    \frac{\partial }{\partial t} \left(\frac{A}{Q}  \right) + \frac{\partial }{\partial x} 
\end{equation}


\textbf{Campo de aceleraciones:}
\begin{itemize}
    \item Campo de presión: $P=P(x,y,z,t)$
    \item Campo de velocidad: $\vec{V}= \vec{V}(x,y,z,t)$
    \item Campo de aceleración: $\vec{a}= \vec{a}(x,y,z,t)$
\end{itemize}
$\vec{V}$ se puede desarrollar en las coordenadas cartesianas 
\begin{equation}
    \vec{V} = (u,v,w) = u(x,y,z,t)\vec{y} + v(x,y,z,t)\vec{j} + w(x,y,z,t)\vec{k}
\end{equation}


Aceleración de una partícula de fluido expresada como una variable de campo:
\begin{equation}
\vec{a}\left(x,y,z,t\right) = \frac{d\vec{V}}{dt} = \underbrace{\frac{\partial \vec{V}}{\partial t}}_{\text{Aceleración local}} + \underbrace{\left(\vec{V} \cdot \vec{\nabla}\right)\vec{V}}_{\text{Aceleración convectiva}}
\end{equation}


El operador gradiente u operador nabla, es un operador vectorial que se define en coordenadas cartesianas como:

\begin{align}
    \text{Gradiente}&& \vec{\nabla} = \left( \frac{\partial}{\partial x} \frac{\partial}{\partial y} \frac{\partial}{\partial z}    \right) = \vec{i} \frac{\partial }{\partial x} + \vec{j} \frac{\partial}{\partial y} + \vec{k} \frac{\partial}{\partial z} 
\end{align}


% \begin{align}
%     \text{Derivada material}&& \frac{D}{DT} = \frac{d}{dt}= \frac{\partial}{\partial t} + \left(\vec{V} \cdot \vec{\nabla} \right)
% \end{align}


Ecuación para una línea de corriente:
\begin{align}
    \text{Línea de corriente}&& \frac{dr}{V} = \frac{dx}{u} = \frac{dy}{v} = \frac{dz}{w}
\end{align}



\section{Vorticidad}
el vector de vorticidad se define matemáticamente como el rotacional del vector de velocidad V
\begin{equation}
    \vec{\zeta} = \vec{\nabla} \times  \vec{V} = rot(\vec{V})
\end{equation}
Resulta que el vector de razón de rotación es igual a la mitad del vector de vorticidad:
\begin{equation}
\vec{\omega} = \frac{1}{2}\vec{\nabla} \times \vec{V} = \frac{1}{2} rot(\vec{V}) = \frac{\vec{\zeta}}{2}
\end{equation}
Por lo tanto, la vorticidad es una medida de la rotación de una partícula de fluido.

\section{Métodos Numéricos}
% :
\subsection{Métodos de Elementos Finitos (FEM)} 
\subsection{Métodos Espectrales}
\subsection{Elementos finitos estabilizados}
\subsection{Métodos de Volumen Finito (FVM)}
\subsection{Métodos Lattice Boltzmann}
% \subsection{Ecaciones de presión de Poisson}
% Galerkin

\section{Simulación de Turbulencia}
\subsection{Grandes Simulaciones de Turbulencia (LES)}
\subsection{Modelos de Turbulencia RANS (Reynolds-Averaged Navier-Stokes)}
%  RANS/URANS




\section{Agricultura Vertical}


% Teorías y modelos existentes
% Estudios previos
\chapter{MATERIALES Y MÉTODOS}
%  metodología estadística usada para probar las hipótesis
% diseño experimental, modelo estadístico y procedimiento de análisis


\section{Materiales}

\section{Método 1}

% \subsection{Resumen}

\subsection{Caso de estudio University of Arizona}


% \subsection{Apéndices}
% Esta sección incluye material complementario que respalda ciertos argumentos presentados en este capítulo.
% \subsubsection{Tema complementario}

\chapter{RESULTADOS}

\section{Protocolo de monitoreo participativo:}

\section{Desarrollo del código}
 

\section{Evaluación del nivel de maduración tecnológica}















\chapter{CONCLUSIONES FINALES Y TRABAJO FUTURO}



%%% Appendicies of thesis  %%%%%%%%%%%%%%%%%%%%%%%%%%%%%%%%%%%%%%%%%%%%%%%%%%%%%%%%%%%%%%%%%%%%%%%%

\appendix
\chapter{ANEXO 1. Desarrollo del Código}

\section{Algoritmo del AppState}
\label{anexo:alg1}

\begin{minted}{dart}
import 'dart:convert';
import 'dart:io';
import 'package:connectivity_plus/connectivity_plus.dart';
import 'package:firebase_auth/firebase_auth.dart';
import 'package:firebase_storage/firebase_storage.dart';
import 'package:flutter/foundation.dart';
import 'package:shared_preferences/shared_preferences.dart';
import 'package:cloud_firestore/cloud_firestore.dart';
import 'package:tlaloc/src/models/google_sign_in.dart';

class Measurement {
  final String? uploader;
  final double? precipitation;
  final DateTime? dateTime;
  final String id;
  final String? imageUrl;
  final String? avatarUrl;
  final String? uploaderId;
  final bool? pluviometer;

  Measurement({
    this.uploader,
    this.uploaderId,
    this.precipitation,
    this.dateTime,
    required this.id,
    this.imageUrl,
    this.avatarUrl,
    this.pluviometer,
  });

  factory Measurement.fromJson(Map<String, dynamic> json, String id) {
    Timestamp timestamp = json['time'];
    return Measurement(
      uploader: json['uploader_name'],
      uploaderId: json['uploader_id'] as String? ?? 'unknown',
      precipitation: json['precipitation'],
      dateTime: timestamp.toDate(),
      id: id,
      imageUrl: json['image'],
      avatarUrl: json['avatar_url'],
      pluviometer: json['pluviometer_state'],
    );
  }
}

class AppState extends ChangeNotifier {
  Uint8List? _newWebImage;
  Uint8List? get newWebImage => _newWebImage;
  set newWebImage(Uint8List? value) {
    _newWebImage = value;
    notifyListeners(); 
  }

  final GoogleSignInProvider _authProvider;
  final db = FirebaseFirestore.instance;

  AppState(this._authProvider) {
    init();
  }

  String rol = 'Monitor';
  String paraje = 'El Venturero';
  bool loading = true;
  List<String> adminUIDs = [];
  bool isAdmin = false;

  User? get currentUser => _authProvider.currentUser;
  String? get currentUID => currentUser?.uid;

  DocumentReference get _parajeRef =>
      db.collection('roles').doc(rol).collection('parajes').doc(paraje);
  CollectionReference get _measurementsRef =>
      _parajeRef.collection('measurements');
  CollectionReference get _realMeasurementsRef =>
      _parajeRef.collection('real_measurements');

  Future<void> init() async {
    loading = true;
    notifyListeners();

    final prefs = await SharedPreferences.getInstance();
    rol = prefs.getString('rol') ?? 'Monitor';
    paraje = prefs.getString('paraje') ?? 'El Venturero';

    await _loadAdminUIDs();
    _checkAdminStatus();

    loading = false;
    notifyListeners();
  }

  Future<void> _loadAdminUIDs() async {
    try {
      final doc = await db.collection('admins').doc('adminUsers').get();
      if (doc.exists) {
        adminUIDs = List<String>.from(doc.data()?['uids'] ?? []);
      }
    } catch (e) {
      debugPrint("Error cargando admins: $e");
    }
  }

  void _checkAdminStatus() {
    isAdmin = currentUID != null && adminUIDs.contains(currentUID);
  }

  bool canEditMeasurement(String? uploaderId) =>
      currentUID == uploaderId || isAdmin;

  Future<void> changeParaje(String newParaje) async {
    paraje = newParaje;
    final prefs = await SharedPreferences.getInstance();
    prefs.setString('paraje', newParaje);
    prefs.setBool('hasFinishedOnboarding', true);
    notifyListeners();
  }

  Future<void> changeRol(String newRol) async {
    rol = newRol;
    final prefs = await SharedPreferences.getInstance();
    prefs.setString('rol', newRol);
    prefs.setBool('hasFinishedOnboarding', true);
    notifyListeners();
  }

  Future<Map<String, dynamic>> getCurrentParajeData() async {
    var snapshot = await _parajeRef.get();
    return (snapshot.data() as Map<String, dynamic>?) ?? {};
  }

  Future<String?> _uploadImage(
    String fileNameBase, {
    File? image,
    String? oldImage,
  }) async {
    final storageRef = FirebaseStorage.instance.ref();
    final connectivityResult = await Connectivity().checkConnectivity();

    if (kIsWeb && newWebImage != null) {
      if (connectivityResult != ConnectivityResult.none) {
        final imageRef = storageRef.child("measurements/$fileNameBase.png");
        final metadata = SettableMetadata(contentType: 'image/png');
        await imageRef.putData(newWebImage!, metadata);
        return await imageRef.getDownloadURL();
      } else {
        return base64Encode(newWebImage!);
      }
    } else if (image != null) {
      if (connectivityResult != ConnectivityResult.none) {
        final extension = image.path.split('.').last;
        final imageRef = storageRef.child(
          "measurements/$fileNameBase.$extension",
        );
        await imageRef.putFile(image);
        return await imageRef.getDownloadURL();
      } else {
        return await image.readAsString();
      }
    }

    return oldImage;
  }

  Future<Map<String, dynamic>> _getMeasurementJson({
    required num precipitation,
    required DateTime time,
    String? uploader,
    File? image,
    String? oldImage,
    bool? pluviometer,
  }) async {
    final fileNameBase =
        '${time.toIso8601String()}_$precipitation${currentUser?.email}';
    final imageUrl = await _uploadImage(
      fileNameBase,
      image: image,
      oldImage: oldImage,
    );

    return {
      'uploader_id': currentUID,
      'precipitation': precipitation,
      'uploader_name': uploader,
      'uploader_email': currentUser?.email,
      'time': time,
      'image': imageUrl,
      'avatar_url': currentUser?.photoURL,
      'pluviometer_state': pluviometer,
    };
  }

  Future<void> _saveMeasurement(
    String collectionName,
    Map<String, dynamic> data,
  ) async {
    await _parajeRef.collection(collectionName).add(data);
  }

  Future<num> _calculateRealValue(num current, bool? wasEmptied) async {
    final lastSnapshot =
        await _measurementsRef.orderBy('time', descending: true).limit(2).get();

    if (lastSnapshot.docs.length < 2 || wasEmptied == true) {
      return current;
    } else {
      final prevData = lastSnapshot.docs[1].data() as Map<String, dynamic>;
      final prevPrecip = prevData['precipitation'] as num? ?? 0;
      return current - prevPrecip;
    }
  }

  Future<void> updateGlobalCounter(int delta) async {
    final counterRef = db.collection('notifications').doc('globalCounter');
    await counterRef.set({
      'count': FieldValue.increment(delta),
      'timestamp': FieldValue.serverTimestamp(),
    }, SetOptions(merge: true));
  }

  Future<void> addMeasurement({
    required num precipitation,
    required DateTime time,
    String? uploader,
    File? image,
    bool? pluviometer,
  }) async {
    final data = await _getMeasurementJson(
      uploader: uploader,
      precipitation: precipitation,
      time: time,
      image: image,
      pluviometer: pluviometer,
    );
    await _saveMeasurement('measurements', data);

    final realValue = await _calculateRealValue(precipitation, pluviometer);
    final realData = await _getMeasurementJson(
      uploader: uploader,
      precipitation: realValue,
      time: time,
      image: image,
      pluviometer: pluviometer,
    );
    await _saveMeasurement('real_measurements', realData);

    await updateGlobalCounter(1);
  }

  Future<void> addRealMeasurement({
    required num precipitation,
    required DateTime time,
    num lastPrecipitation = 0,
    String? uploader,
    File? image,
    bool? pluviometer,
  }) async {
    final data = await _getMeasurementJson(
      uploader: uploader,
      precipitation: precipitation - lastPrecipitation,
      time: time,
      image: image,
      pluviometer: pluviometer,
    );
    await _saveMeasurement('real_measurements', data);
  }

  List<Measurement> getMeasurementsFromDocs(
    List<QueryDocumentSnapshot<Map<String, dynamic>>> docs,
  ) {
    final measurements =
        docs.map((doc) => Measurement.fromJson(doc.data(), doc.id)).toList();
    measurements.sort((a, b) => b.dateTime!.compareTo(a.dateTime!));
    return measurements;
  }

  Future<List<Measurement>> getMeasurements() async => getMeasurementsFromDocs(
    (await _measurementsRef.get() as QuerySnapshot<Map<String, dynamic>>).docs,
  );

  Future<List<Measurement>> getRealMeasurements() async =>
      getMeasurementsFromDocs(
        (await _realMeasurementsRef.get()
                as QuerySnapshot<Map<String, dynamic>>)
            .docs,
      );

  Stream<QuerySnapshot<Map<String, dynamic>>> _measurementStream(
    String collection, {
    String? parajeOverride,
  }) {
    final ref = db
        .collection('roles')
        .doc('Monitor')
        .collection('parajes')
        .doc(parajeOverride ?? paraje)
        .collection(collection);
    return ref.orderBy('time', descending: false).snapshots();
  }

  Stream<QuerySnapshot<Map<String, dynamic>>> getMeasurementsStream() =>
      _measurementStream('measurements');

  Stream<QuerySnapshot<Map<String, dynamic>>> getRealMeasurementsStream() =>
      _measurementStream('real_measurements');

  Stream<QuerySnapshot<Map<String, dynamic>>> getMeasurementsStreamForParaje(
    String name,
  ) => _measurementStream('measurements', parajeOverride: name);

  Stream<QuerySnapshot<Map<String, dynamic>>>
  getRealMeasurementsStreamForParaje(String name) =>
      _measurementStream('real_measurements', parajeOverride: name);

  Stream<QuerySnapshot<Map<String, dynamic>>> getAllUserMeasurementsStream() {
    if (currentUID == null) return const Stream.empty();
    return db
        .collectionGroup('measurements')
        .where('uploader_id', isEqualTo: currentUID)
        .snapshots();
  }

  Stream<QuerySnapshot<Map<String, dynamic>>> getAllMeasurementsStream() =>
      db.collectionGroup('measurements').snapshots();

  Future<void> updateMeasurement({
    required String id,
    required num precipitation,
    required DateTime time,
    String? uploader,
    File? image,
    bool? pluviometer,
    String? oldImage,
    required String uploaderId,
  }) async {
    if (!canEditMeasurement(uploaderId)) {
      throw Exception("No tiene permisos para editar esta medición");
    }
    final data = await _getMeasurementJson(
      uploader: uploader,
      precipitation: precipitation,
      time: time,
      image: image,
      oldImage: oldImage,
      pluviometer: pluviometer,
    );
    await _measurementsRef.doc(id).update(data);
  }

  Future<void> updateRealMeasurement({
    required String id,
    required num precipitation,
    required DateTime time,
    String? uploader,
    File? image,
    bool? pluviometer,
    String? oldImage,
  }) async {
    final data = await _getMeasurementJson(
      uploader: uploader,
      precipitation: precipitation,
      time: time,
      image: image,
      oldImage: oldImage,
      pluviometer: pluviometer,
    );
    await _realMeasurementsRef.doc(id).update(data);
  }

  Future<void> deleteMeasurement({required String id}) async {
    try {
      await _measurementsRef.doc(id).delete(); 
    } catch (e) {
      debugPrint("Error al borrar medición: $e");
    }
  }

  Future<void> deleteRealMeasurement({required String id}) async {
    try {
      await _realMeasurementsRef.doc(id).delete(); 
    } catch (e) {
      debugPrint("Error al borrar medición real: $e");
    }
  }

  Future<Map<String, dynamic>> getUserStats() async {
    try {
      if (currentUID == null) {
        return {
          'local': 0,
          'global': 0,
          'distinctParajes': 0,
          'totalParajes': 0,
        };
      }

      final localSnapshot = await _measurementsRef
          .where('uploader_id', isEqualTo: currentUID)
          .get(const GetOptions(source: Source.serverAndCache));

      final globalSnapshot = await db
          .collectionGroup('measurements')
          .where('uploader_id', isEqualTo: currentUID)
          .get(const GetOptions(source: Source.serverAndCache));

      final parajesContribuidos = <String>{};
      for (final doc in globalSnapshot.docs) {
        final segments = doc.reference.path.split('/');
        final parajeName =
            segments.contains('parajes')
                ? segments[segments.indexOf('parajes') + 1]
                : null;
        if (parajeName != null) parajesContribuidos.add(parajeName);
      }

      final totalParajesSnapshot = await db
          .collection('roles')
          .doc(rol)
          .collection('parajes')
          .get(const GetOptions(source: Source.serverAndCache));

      return {
        'local': localSnapshot.docs.length,
        'global': globalSnapshot.docs.length,
        'distinctParajes': parajesContribuidos.length,
        'totalParajes': totalParajesSnapshot.docs.length,
      };
    } catch (e) {
      debugPrint("Error en getUserStats: $e");
      return {
        'local': 0,
        'global': 0,
        'distinctParajes': 0,
        'totalParajes': 0,
        'error': e.toString(),
      };
    }
  }
}


        \end{minted}


\newpage
\section{Reglas de CLoud FireStore}
\label{anexo:alg2}

\begin{minted}{javascript}

service cloud.firestore {
  
  // Permite leer/escribir si el usuario tiene algún rol (Rowy general)
  match /{allPaths=**} {
    allow read, write: if request.auth.token.roles.size() > 0;
  }

  match /databases/{database}/documents {

    // ----------------------------
    // Rowy rules start (NO MODIFICAR)
    // ----------------------------
    match /{collectionId}/{docId} {
      allow read, create, update, delete: if colRule(["roles"], ["ADMIN","EDITOR","VIEWER","OWNER"]);
      
      function colRule(collections, roles) {
        return collectionId in collections && hasAnyRole(roles);
      }
    }
    // ----------------------------
    // Rowy rules end
    // ----------------------------

    // Permiso global a ADMIN y OWNER
    match /{document=**} {
      allow read, write: if hasAnyRole(["ADMIN", "OWNER"]);
    }

    // Configuración de Rowy (permitido a usuarios con rol)
    match /_rowy_/{docId} {
      allow read: if request.auth.token.roles.size() > 0;
      allow write: if hasAnyRole(["ADMIN", "OWNER"]);

      match /{document=**} {
        allow read: if request.auth.token.roles.size() > 0;
        allow write: if hasAnyRole(["ADMIN", "OWNER"]);
      }

      match /schema/{tableId} {
        allow update: if canModify(tableId,'pc');
        match /{document=**} {
          allow read, write: if canModify(tableId,'pc');
        }
      }

      match /groupSchema/{tableId} {
        allow update: if canModify(tableId,'cg');
        match /{document=**} {
          allow read, write: if canModify(tableId,'cg');
        }
      }
    }

    // Rowy: user management
    match /_rowy_/userManagement/users/{userId} {
      allow get, update, delete: if isDocOwner(userId);
      allow create: if request.auth.token.roles.size() > 0;
    }

    match /_rowy_/publicSettings {
      allow get: if true;
    }

    //PERMISOS PERSONALIZADOS PARA TUS MEDICIONES

    match /roles/{rol}/parajes/{paraje}/measurements/{docId} {
      allow read: if request.auth != null;
      allow create: if request.auth != null;
      allow update, delete: if 
        hasAnyRole(["ADMIN", "OWNER"]) || 
        request.auth.uid == resource.data.uploader_id;
    }

    match /roles/{rol}/parajes/{paraje}/real_measurements/{docId} {
      allow read: if request.auth != null;
      allow create: if request.auth != null;
      allow update, delete: if 
        hasAnyRole(["ADMIN", "OWNER"]) || 
        request.auth.uid == resource.data.uploader_id;
    }

    // Reglas por defecto: acceso propio
    match /{document=**} {
      allow read, write: if request.auth != null;
      allow create: if request.auth != null;
      allow update, delete: if request.auth.uid == resource.data.userId;
    }

    // UTILIDADES
    function isDocOwner(docId) {
      return request.auth != null &&
        (request.auth.uid == resource.id || request.auth.uid == docId);
    }

    function hasAnyRole(roles) {
      return request.auth != null &&
        request.auth.token.roles.hasAny(roles);
    }

    function canModify(tableId, tableType) {
      return hasAnyRole(get(/databases/$(database)/documents/_rowy_/settings)
        .data.tablesSettings[tableType][tableId].modifiableBy);
    }
  }
}
\end{minted}








\newpage
\section{Función main}
\label{anexo:alg3}

\begin{minted}{dart}

import 'package:flutter/foundation.dart';
import 'package:flutter/material.dart';
import 'package:flutter/services.dart';
import 'package:firebase_core/firebase_core.dart';
import 'package:cloud_firestore/cloud_firestore.dart';
import 'package:url_strategy/url_strategy.dart';
import 'firebase_options.dart';
import 'src/app.dart'; 

void main() async {
  
  WidgetsFlutterBinding.ensureInitialized();

  // Configurar estrategia de URL limpia (sin #)
  setPathUrlStrategy();

  // Inicializar Firebase
  await Firebase.initializeApp(
    options: DefaultFirebaseOptions.currentPlatform,
  );

  // Configuración de Firestore: persistencia y caché ilimitado
  FirebaseFirestore.instance.settings = const Settings(
    persistenceEnabled: true,
    cacheSizeBytes: Settings.CACHE_SIZE_UNLIMITED,
  );

  // Registrar licencias personalizadas (Google Fonts)
  _registerLicenses();

  // Iniciar la aplicación
  runApp( const MyApp());
}

// Registrar licencias de fuentes y otros assets
void _registerLicenses() {
  LicenseRegistry.addLicense(() async* {
    final license = await rootBundle.loadString('google_fonts/OFL.txt');
    yield LicenseEntryWithLineBreaks(['google_fonts'], license);
  });
}
\end{minted}


\newpage


\section{Algoritmo de MyApp}
\label{anexo:alg4}

\begin{minted}{dart}

import 'package:flutter/material.dart';
import 'package:flutter_localizations/flutter_localizations.dart';
import 'package:provider/provider.dart';
import 'package:tlaloc/src/core/app_router.dart';
import 'package:tlaloc/src/core/providers/app_providers.dart';
import 'package:tlaloc/src/models/constants.dart';  

class MyApp extends StatelessWidget {
  const MyApp({super.key});

  @override
  Widget build(BuildContext context) {
    return MultiProvider(
      providers: appProviders,
      child: Builder(
        builder:
            (context) => MaterialApp(
              title: appName,
              debugShowCheckedModeBanner: false,
              theme: appLightTheme,
              darkTheme: appDarkTheme,
              themeMode: ThemeMode.system,
              initialRoute: '/',
              onGenerateRoute: generateRoute,
              localizationsDelegates: const [
                GlobalMaterialLocalizations.delegate,
                GlobalWidgetsLocalizations.delegate,
                GlobalCupertinoLocalizations.delegate,
              ],
            ),
      ),
    );
  }
}

\end{minted}



\newpage

\section{Algoritmo de ConditionalOnboardingPage}
\label{anexo:alg5}

\begin{minted}{dart}
import 'package:flutter/material.dart';
import 'package:provider/provider.dart';
import 'package:shared_preferences/shared_preferences.dart';
import 'package:tlaloc/src/models/google_sign_in.dart';
import 'package:tlaloc/src/models/kernel.dart';
import 'package:tlaloc/src/resources/onboarding/onbording.dart'; 
import 'package:tlaloc/src/ui/widgets/backgrounds/empty_state.dart';
import 'package:tlaloc/src/ui/widgets/backgrounds/splash.dart'; 

class ConditionalOnboardingPage extends StatelessWidget {
  const ConditionalOnboardingPage({super.key});

  Future<Widget> _decideNextScreen(BuildContext context) async {
    final prefs = await SharedPreferences.getInstance();
    final hasFinishedOnboarding = prefs.getBool('hasFinishedOnboarding') ?? false;

    final authProvider = Provider.of<GoogleSignInProvider>(context, listen: false);
    final isLoggedIn = authProvider.currentUser != null;

    if (hasFinishedOnboarding && isLoggedIn) {
      return const HomePage();
    } else {
      return Onboarding();
    }
  }

  @override
  Widget build(BuildContext context) {
    return FutureBuilder<Widget>(
      future: _decideNextScreen(context),
      builder: (context, snapshot) {
        if (snapshot.hasError) {
          return const _ErrorScreen();
        } else if (snapshot.connectionState != ConnectionState.done) {
          // Evita pantalla en blanco, mientras resuelve
          return const SplashScreen(nextScreen: Scaffold());
        } else {
          return SplashScreen(nextScreen: snapshot.data!);
        }
      },
    );
  }
}

class _ErrorScreen extends StatelessWidget {
  const _ErrorScreen();

  @override
  Widget build(BuildContext context) {
    return Scaffold(
      appBar: AppBar(title: const Text('Error de inicio')),
      body: const EmptyState(
        'No pudimos cargar la configuración inicial. '
        'Por favor revisa tu conexión a internet o reinstala la aplicación.',
      ),
    );
  }
}
\end{minted}






\newpage
\section{Pantalla Onboarding}
\label{anexo:alg6}
\begin{minted}{dart}
import 'package:flutter/material.dart';
import 'package:concentric_transition/concentric_transition.dart';
import 'package:lottie/lottie.dart';
import 'package:tlaloc/src/models/constants.dart';
import 'package:tlaloc/src/ui/widgets/cards/onbording_cards.dart';
import 'package:tlaloc/src/resources/onboarding/sign_in.dart';

class Onboarding extends StatelessWidget {
  Onboarding({super.key});

  final List<CardPlanetData> data = [
    CardPlanetData(
      title: appName,
      subtitle: "Ciencia para tí y para todos",
      image: const AssetImage("assets/images/img-1.png"),
      backgroundColor: AppColors.blue1,
      titleColor: Colors.white,
      subtitleColor: Colors.white,
      background: LottieBuilder.asset("assets/animation/bg-1.json"),
    ),
    CardPlanetData(
      title: "Te damos la bienvenida",
      subtitle:
          "Ya eres parte del proyecto ''Ciencia ciudadana para el monitoreo de la lluvia en el monte Tláloc'' ",
      image: const AssetImage("assets/images/img-2.png"),
      backgroundColor: Colors.white,
      titleColor: AppColors.green1,
      subtitleColor: const Color.fromRGBO(0, 10, 56, 1),
      background: LottieBuilder.asset("assets/animation/bg-2.json"),
    ),
  ];

  @override
  Widget build(BuildContext context) {
    return Scaffold(
      body: ConcentricPageView(
        direction: Axis.horizontal,
        pageSnapping: true,

        onFinish: () {
          Navigator.push(
            context,
            MaterialPageRoute(builder: (context) => const SignUpWidget()),
          );
        },
        colors: data.map((e) => e.backgroundColor).toList(),
        itemCount: data.length,
        itemBuilder: (int index) {
          return CardPlanet(data: data[index]);
        },
      ),
    );
  }
}

\end{minted}






\newpage
\section{Pantalla HomePage}
\label{anexo:alg7}
\begin{minted}{dart}
import 'package:curved_navigation_bar/curved_navigation_bar.dart';
import 'package:firebase_auth/firebase_auth.dart';
import 'package:flutter/material.dart';
import 'package:tlaloc/src/models/constants.dart';
import 'package:tlaloc/src/resources/statics/graphs/graph2.dart';
import 'package:tlaloc/src/ui/screens/dir/add.dart';
import 'package:tlaloc/src/ui/screens/dir/data.dart';
import 'package:tlaloc/src/ui/screens/dir/home.dart';
import 'package:tlaloc/src/ui/screens/home/profile_page.dart';
import 'package:cloud_firestore/cloud_firestore.dart'; 

class HomePage extends StatefulWidget {
  const HomePage({super.key});

  @override
  State<HomePage> createState() => _HomePageState();
}

class _HomePageState extends State<HomePage> {
  int _selectedIndex = 0;
  final GlobalKey<CurvedNavigationBarState> _navKey = GlobalKey();

  int globalNotificationCount = 0;
  bool hasSeenNotifications = false;

  late final List<Widget> _screens = const [
    HomeScreen(),
    AddScreen(),
    DataScreen(),
    BarGraph(),
    ConfigureScreen(),
  ];
  @override
  void initState() {
    super.initState();

    FirebaseFirestore.instance
        .collection('notifications')
        .doc('globalCounter')
        .snapshots()
        .listen((snapshot) {
          if (snapshot.exists) {
            setState(() {
              globalNotificationCount = snapshot.data()?['count'] ?? 0;
            });
          }
        });
  }

  @override
  Widget build(BuildContext context) {
    return Scaffold(
      extendBody: true,
      body: IndexedStack(index: _selectedIndex, children: _screens),
      bottomNavigationBar: Theme(
        data: Theme.of(context).copyWith(iconTheme: const IconThemeData()),
        child: CurvedNavigationBar(
          key: _navKey,
          height: 60.0,
          color: AppColors.blue1,
          buttonBackgroundColor: AppColors.blue1,
          backgroundColor: Colors.transparent,
          animationCurve: Curves.easeInOut,
          animationDuration: const Duration(milliseconds: 800),
          items: _buildNavItems(),
          index: _selectedIndex,
          onTap: (index) {
            setState(() {
              _selectedIndex = index;
              if (index == 2) hasSeenNotifications = true;
            });
          },
        ),
      ),
    );
  }

  List<Widget> _buildNavItems() {
    return [
      const Icon(Icons.home, size: 30, color: Colors.white),
      const Icon(Icons.add, size: 30, color: Colors.white), 
      Stack(
        children: [
          const Icon(Icons.menu_book_rounded, size: 30, color: Colors.white),
          if (globalNotificationCount > 0 && !hasSeenNotifications)
            Positioned(
              right: 0,
              top: 0,
              child: Container(
                padding: const EdgeInsets.all(2),
                decoration: const BoxDecoration(
                  color: Colors.red,
                  shape: BoxShape.circle,
                ),
                constraints: const BoxConstraints(minWidth: 16, minHeight: 16),
                child: Text(
                  '$globalNotificationCount',
                  style: const TextStyle(
                    color: Colors.white,
                    fontSize: 10,
                    fontWeight: FontWeight.bold,
                  ),
                  textAlign: TextAlign.center,
                ),
              ),
            ),
        ],
      ),

      const Icon(Icons.line_axis, size: 30, color: Colors.white),
      CircleAvatar(
        foregroundImage:
            FirebaseAuth.instance.currentUser?.photoURL != null
                ? NetworkImage(FirebaseAuth.instance.currentUser!.photoURL!)
                : const NetworkImage(
                  'https://s1.elespanol.com/2019/11/01/elandroidelibre/el_androide_libre_441218515_179632866_1024x576.jpg',
                ),
      ),
    ];
  }
}
\end{minted}



\newpage
\section{Pantalla SignUpWidget}
\label{anexo:alg8}

\begin{minted}{dart}

import 'package:flutter/material.dart';
import 'package:font_awesome_flutter/font_awesome_flutter.dart';
import 'package:lottie/lottie.dart';
import 'package:provider/provider.dart';
import 'package:tlaloc/src/models/constants.dart';
import 'package:tlaloc/src/models/google_sign_in.dart';
import 'package:tlaloc/src/resources/onboarding/common_select.dart';
import 'package:url_launcher/url_launcher.dart';
import 'dart:ui';

class SignUpWidget extends StatefulWidget {
  const SignUpWidget({super.key});

  @override
  State<SignUpWidget> createState() => _SignUpWidgetState();
}

class _SignUpWidgetState extends State<SignUpWidget>
    with SingleTickerProviderStateMixin {
  late AnimationController _controller;
  bool _isLoading = false;

  @override
  void initState() {
    super.initState();
    _controller = AnimationController(
      vsync: this,
      duration: const Duration(seconds: 2),
    )..repeat(reverse: true);
  }

  @override
  void dispose() {
    _controller.dispose();
    super.dispose();
  }

  Future<void> _handleGoogleSignIn(BuildContext context) async {
    setState(() => _isLoading = true);
    final provider = Provider.of<GoogleSignInProvider>(context, listen: false);

    try {
      await provider.googleLogin();
      if (provider.currentUser != null) {
        Navigator.pushReplacement(
          context,
          PageRouteBuilder(
            transitionDuration: const Duration(milliseconds: 1000),
            pageBuilder: (_, __, ___) => const CommonSelectPage(),
            transitionsBuilder:
                (_, a, __, c) => FadeTransition(opacity: a, child: c),
          ),
        );
      }
    } catch (e) {
      _showErrorDialog(context, e.toString());
    } finally {
      if (mounted) setState(() => _isLoading = false);
    }
  }

  void _showErrorDialog(BuildContext context, String error) {
    showDialog(
      context: context,
      builder:
          (context) => AlertDialog(
            backgroundColor: AppColors.dark1.withOpacity(0.9),
            shape: RoundedRectangleBorder(
              borderRadius: BorderRadius.circular(20),
            ),
            title: Row(
              children: [
                Lottie.asset('assets/animation/bg-3.json', width: 40),
                const SizedBox(width: 10),
                const Text('Error', style: TextStyle(color: Colors.white)),
              ],
            ),
            content: Text(error, style: const TextStyle(color: Colors.white70)),
            actions: [
              TextButton(
                onPressed: () => Navigator.pop(context),
                child: const Text('OK', style: TextStyle(color: Colors.blue)),
              ),
            ],
          ),
    );
  }

  @override
  Widget build(BuildContext context) {
    final size = MediaQuery.of(context).size;

    return Scaffold(
      backgroundColor: AppColors.purple1,
      body: Stack(
        children: [
          // Fondo animado
          Positioned.fill(
            child: Lottie.asset(
              'assets/animation/bg-3.json',
              fit: BoxFit.cover,
            ),
          ),

          // Contenido principal
          Center(
            child: SafeArea(
              child: LayoutBuilder(
                builder: (context, constraints) {
                  final isWide = constraints.maxWidth > 800;

                  return SingleChildScrollView(
                    physics: const BouncingScrollPhysics(),
                    child: Padding(
                      padding: const EdgeInsets.all(20),
                      child:
                          isWide
                              ? Row(
                                mainAxisAlignment: MainAxisAlignment.center,
                                children: [
                                  Expanded(
                                    child: Padding(
                                      padding: const EdgeInsets.all(20),
                                      child: AnimatedBuilder(
                                        animation: _controller,
                                        builder:
                                            (context, child) =>
                                                Transform.translate(
                                                  offset: Offset(
                                                    0,
                                                    10 * _controller.value,
                                                  ),
                                                  child: child,
                                                ),
                                        child: Image.asset(
                                          'assets/images/img-1-4.png',
                                          width: size.width * 0.3,
                                        ),
                                      ),
                                    ),
                                  ),
                                  Expanded(child: _buildLoginCard(size)),
                                ],
                              )
                              : Column(
                                children: [
                                  // Logo animado
                                  AnimatedBuilder(
                                    animation: _controller,
                                    builder:
                                        (context, child) => Transform.translate(
                                          offset: Offset(
                                            0,
                                            10 * _controller.value,
                                          ),
                                          child: child,
                                        ),
                                    child: Image.asset(
                                      'assets/images/img-1-4.png',
                                      width: size.width * 0.8,
                                    ),
                                  ),
                                  const SizedBox(height: 40),
                                  _buildLoginCard(size),
                                ],
                              ),
                    ),
                  );
                },
              ),
            ),
          ),
        ],
      ),
    );
  }

  Widget _buildLoginCard(Size size) {
    return ClipRRect(
      borderRadius: BorderRadius.circular(30),
      child: BackdropFilter(
        filter: ImageFilter.blur(sigmaX: 10, sigmaY: 10),
        child: Container(
          padding: const EdgeInsets.all(30),
          decoration: BoxDecoration(
            color: Colors.white.withOpacity(0.1),
            border: Border.all(color: Colors.white24),
            borderRadius: BorderRadius.circular(30),
          ),
          child: Column(
            mainAxisSize: MainAxisSize.min,
            children: [
              Text(
                'Iniciar sesión',
                style: const TextStyle(
                  fontSize: 28,
                  fontWeight: FontWeight.bold,
                  color: Colors.white,
                  fontFamily: 'FredokaOne',
                ),
              ),
              const SizedBox(height: 15),
              const Text(
                'Conéctate para contribuir a la ciencia ciudadana',
                textAlign: TextAlign.center,
                style: TextStyle(
                  fontSize: 16,
                  color: Colors.white70,
                  fontFamily: 'Poppins',
                ),
              ),
              const SizedBox(height: 30),
              AnimatedSwitcher(
                duration: const Duration(milliseconds: 300),
                child:
                    _isLoading
                        ? const CircularProgressIndicator(
                          valueColor: AlwaysStoppedAnimation<Color>(
                            Colors.white,
                          ),
                        )
                        : ElevatedButton.icon(
                          icon: FaIcon(
                            FontAwesomeIcons.google,
                            color: Colors.red[400],
                          ),
                          label: const Text(
                            'Continuar con Google',
                            style: TextStyle(
                              fontSize: 16,
                              fontWeight: FontWeight.bold,
                            ),
                          ),
                          style: ElevatedButton.styleFrom(
                            backgroundColor: Colors.white.withOpacity(0.9),
                            foregroundColor: Colors.black87,
                            minimumSize: Size(size.width * 0.7, 55),
                            shape: RoundedRectangleBorder(
                              borderRadius: BorderRadius.circular(15),
                            ),
                            elevation: 5,
                            shadowColor: Colors.black26,
                          ),
                          onPressed: () => _handleGoogleSignIn(context),
                        ),
              ),
              const SizedBox(height: 30),
              MouseRegion(
                cursor: SystemMouseCursors.click,
                child: GestureDetector(
                  onTap:
                      () => launchUrl(
                        Uri.parse('https://tlaloc.web.app/privacy/'),
                        mode: LaunchMode.inAppWebView,
                      ),
                  child: RichText(
                    textAlign: TextAlign.center,
                    text: TextSpan(
                      style: const TextStyle(
                        color: Colors.white70,
                        fontSize: 13,
                        height: 1.5,
                      ),
                      children: [
                        const TextSpan(
                          text: 'Al continuar, aceptas nuestros\n',
                        ),
                        TextSpan(
                          text: 'Términos de servicio',
                          style: TextStyle(
                            color: Colors.blue[200],
                            fontWeight: FontWeight.bold,
                            decoration: TextDecoration.underline,
                          ),
                        ),
                        const TextSpan(text: ' y '),
                        TextSpan(
                          text: 'Política de privacidad',
                          style: TextStyle(
                            color: Colors.blue[200],
                            fontWeight: FontWeight.bold,
                            decoration: TextDecoration.underline,
                          ),
                        ),
                      ],
                    ),
                  ),
                ),
              ),
            ],
          ),
        ),
      ),
    );
  }
}

\end{minted}






\newpage
\section{Modelo GoogleSignInProvider}
\label{anexo:alg9}
\begin{minted}{dart}
import 'package:flutter/foundation.dart';
import 'package:firebase_auth/firebase_auth.dart';
import 'package:google_sign_in/google_sign_in.dart';

class GoogleSignInProvider extends ChangeNotifier {
  final FirebaseAuth _auth = FirebaseAuth.instance;
  
  final GoogleSignIn _googleSignIn = GoogleSignIn(
    scopes: ['email', 'profile'],
    clientId: kIsWeb 
        ? '228815382617-2rtslpepg048j80iuls7ilrc8ff9sn4l.apps.googleusercontent.com'
        : null, // Para Android/iOS, Firebase maneja automáticamente el Client ID
  );

  User? get currentUser => _auth.currentUser;
  bool _isLoading = false;
  bool get isLoading => _isLoading;

  Future<void> googleLogin() async {
    try {
      _isLoading = true;
      notifyListeners();

      final GoogleSignInAccount? googleUser = await _googleSignIn.signIn();
      if (googleUser == null) return;

      final GoogleSignInAuthentication googleAuth = 
          await googleUser.authentication;

      final AuthCredential credential = GoogleAuthProvider.credential(
        accessToken: googleAuth.accessToken,
        idToken: googleAuth.idToken,
      );

      await _auth.signInWithCredential(credential);
      
    } on FirebaseAuthException catch (e) {
      _handleAuthError(e);
    } catch (e) {
      debugPrint('Error inesperado: $e');
      rethrow;
    } finally {
      _isLoading = false;
      notifyListeners();
    }
  }

  Future<void> logout() async {
    try {
      await _googleSignIn.signOut(); // Mejor que disconnect()
      await _auth.signOut();
    } catch (e) {
      debugPrint('Error al cerrar sesión: $e');
      rethrow;
    } finally {
      notifyListeners();
    }
  }

  void _handleAuthError(FirebaseAuthException e) {
    debugPrint('Código de error: ${e.code}');
    String message = 'Error de autenticación';

    switch (e.code) {
      case 'account-exists-with-different-credential':
        message = 'Cuenta ya existe con otro método de autenticación';
        break;
      case 'invalid-credential':
        message = 'Credenciales inválidas';
        break;
      case 'operation-not-allowed':
        message = 'Método de autenticación no habilitado';
        break;
      case 'user-disabled':
        message = 'Cuenta deshabilitada';
        break;
      case 'user-not-found':
        message = 'Usuario no encontrado';
        break;
    }

    throw AuthException(message);
  }
}

class AuthException implements Exception {
  final String message;
  AuthException(this.message);
  
  @override
  String toString() => message;
}
\end{minted}










\section{Página de elección de paraje}
\label{anexo:alg10}

\begin{minted}{dart}
  import 'package:flutter/material.dart';
import 'package:tlaloc/src/models/constants.dart';
import 'package:tlaloc/src/ui/widgets/cards/common_card.dart';
import 'package:tlaloc/src/ui/widgets/cards/qr.dart';

class CommonSelectPage extends StatelessWidget {
  const CommonSelectPage({super.key});

  @override
  Widget build(BuildContext context) {
    return Scaffold(
      backgroundColor: AppColors.lightBlue,
      body: SingleChildScrollView(
        child: SafeArea(
          child: Padding(
            padding: EdgeInsets.all(16.0),
            child: Center(
              child: Column(
                children: [
                  Text(
                    '¿Qué pluviómetro estás observando?',
                    style: TextStyle(
                      fontSize: 32,
                      fontFamily: 'FredokaOne',
                      color: Colors.white,
                    ),
                    textAlign: TextAlign.center,
                  ),
                  SizedBox(height: 20),
                  QrSelectWidget(),
                  SizedBox(height: 20),
                  Text(
                    'Seleccionar manualmente',
                    style: TextStyle(
                      fontSize: 32,
                      fontFamily: 'FredokaOne',
                      color: Colors.white,
                    ),
                    textAlign: TextAlign.center,
                  ),
                  SizedBox(height: 20),
                  CommonSelectWidget(),
                ],
              ),
            ),
          ),
        ),
      ),
    );
  }
}

\end{minted}







\section{Modelo QR}
\label{anexo:alg11}
\begin{minted}{dart}
import 'dart:math';
import 'package:flutter/material.dart';
import 'package:mobile_scanner/mobile_scanner.dart';
import 'package:provider/provider.dart';
import 'package:tlaloc/src/models/app_state.dart';
import 'package:tlaloc/src/models/constants.dart';
import 'package:tlaloc/src/models/home_page.dart';

class QrSelectWidget extends StatelessWidget {
  const QrSelectWidget({super.key});

  void _goHome(BuildContext context) {
    Navigator.of(context).pushAndRemoveUntil(
      MaterialPageRoute<void>(builder: (context) => const HomePage()),
      (route) => false,
    );
  }

  Future<void> _handleQrResult(BuildContext context, String? qrResult) async {
    if (qrResult == null) {
      await _showErrorDialog(
        context,
        title: 'Escaneo fallido',
        content: 'Intenta de nuevo o selecciona tu paraje manualmente',
      );
      return;
    }

    final paraje = _parseQrResult(qrResult);
    if (!parajes.containsKey(paraje)) {
      await _showErrorDialog(
        context,
        title: 'Código inválido',
        content:
            paraje.isEmpty
                ? null
                : 'Tlaloc App no está disponible en el paraje "$paraje"',
      );
      return;
    }

    _goHome(context);
    Provider.of<AppState>(context, listen: false).changeParaje(paraje);
  }

  String _parseQrResult(String qrResult) {
    if (!qrResult.contains('tlaloc.web.app')) return '';
    return qrResult.split('/').last.replaceAll(RegExp(r'_|%20'), ' ').trim();
  }

  Future<void> _showErrorDialog(
    BuildContext context, {
    required String title,
    String? content,
  }) async {
    await showDialog(
      context: context,
      barrierDismissible: true,
      builder:
          (context) => AlertDialog(
            icon: const Icon(Icons.error_outline_rounded),
            iconColor: Theme.of(context).colorScheme.error,
            title: Text(title),
            content: content != null ? Text(content) : null,
            actions: [
              TextButton(
                onPressed: () => Navigator.pop(context),
                child: const Text('ENTENDIDO'),
              ),
            ],
          ),
    );
  }

  @override
  Widget build(BuildContext context) {
    final theme = Theme.of(context);
    final colors = theme.colorScheme;

    return LayoutBuilder(
      builder: (context, constraints) {
        final screenWidth = constraints.maxWidth;
        final isWide = screenWidth >= 600;

        // Igual que en las tarjetas comunes
        final cardWidth =
            isWide
                ? (screenWidth - 16 /* spacing */ - 32 /* padding */ ) / 2
                : screenWidth - 32;

        return ConstrainedBox(
          constraints: const BoxConstraints(
            maxWidth: 400, // Máximo recomendado para móviles
            minWidth: 280, // Mínimo para buena legibilidad
          ),
          child: Material(
            color: AppColors.dark2,
            borderRadius: BorderRadius.circular(
              12,
            ), // Reducido de 28 a 12 según MD
            clipBehavior: Clip.antiAlias,
            elevation: 1,
            child: InkWell(
              onTap: () async {
                final qrResult = await showDialog<String>(
                  context: context,
                  builder: (context) => _QrScannerDialog(context),
                );
                await _handleQrResult(context, qrResult);
              },
              splashColor: colors.primary.withOpacity(0.1),
              highlightColor: colors.primary.withOpacity(0.05),
              child: Padding(
                padding: const EdgeInsets.all(24), // Padding interno estándar
                child: Column(
                  mainAxisSize: MainAxisSize.min,
                  mainAxisAlignment: MainAxisAlignment.center,
                  children: [
                    Icon(
                      Icons.qr_code_scanner_rounded,
                      size: 48, // Reducido de 64 para mejor proporción
                      color: colors.primary,
                    ),
                    const SizedBox(height: 16),
                    Text(
                      'ESCANEAR QR',
                      style: theme.textTheme.titleLarge?.copyWith(
                        fontWeight: FontWeight.w600,
                        color: colors.onSurface,
                        fontFamily: 'FredokaOne',
                      ),
                    ),
                    const SizedBox(height: 8),
                    Padding(
                      padding: const EdgeInsets.symmetric(horizontal: 16),
                      child: Text(
                        'Detecta tu pluviómetro automáticamente',
                        textAlign: TextAlign.center,
                        style: theme.textTheme.bodyMedium?.copyWith(
                          color: colors.onSurfaceVariant,
                          fontFamily: 'Poppins',
                        ),
                      ),
                    ),
                  ],
                ),
              ),
            ),
          ),
        );
      },
    );
  }
}

class _QrScannerDialog extends StatelessWidget {
  final BuildContext parentContext;

  const _QrScannerDialog(this.parentContext);

  @override
  Widget build(BuildContext context) {
    final size = min(
      MediaQuery.of(parentContext).size.width,
      MediaQuery.of(parentContext).size.height,
    );

    return Dialog(
      backgroundColor: Colors.black,
      insetPadding: const EdgeInsets.all(24),
      child: Column(
        mainAxisSize: MainAxisSize.min,
        children: [
          AppBar(
            title: const Text('Escanear código'),
            backgroundColor: Colors.transparent,
            automaticallyImplyLeading: false,
            actions: [
              IconButton(
                icon: const Icon(Icons.close),
                onPressed: () => Navigator.pop(context),
              ),
            ],
          ),
          ConstrainedBox(
            constraints: BoxConstraints(maxWidth: size, maxHeight: size),
            child: MobileScanner(
              controller: MobileScannerController(
                detectionSpeed: DetectionSpeed.normal,
                facing: CameraFacing.back,
                torchEnabled: false,
              ),
              onDetect: (barcode) {
                if (barcode.barcodes.isNotEmpty) {
                  Navigator.pop(context, barcode.barcodes.first.rawValue ?? '');
                }
              },
            ),
          ),
        ],
      ),
    );
  }
}
\end{minted}









\section{Pantalla de inicio}
\label{anexo:alg12}
\begin{minted}{dart}
import 'package:auto_size_text/auto_size_text.dart';
import 'package:flutter/material.dart';
import 'package:flutter/rendering.dart';
import 'package:tlaloc/src/models/constants.dart';
import 'package:tlaloc/src/ui/widgets/appbar/infobutton2.dart';
import 'package:tlaloc/src/ui/widgets/backgrounds/container.dart';
import 'package:tlaloc/src/ui/widgets/buttons/fab.dart';
import 'package:tlaloc/src/ui/widgets/buttons/notebook.dart';
import 'package:tlaloc/src/ui/widgets/cards/communitybutton.dart';
import 'package:tlaloc/src/ui/widgets/cards/forms.dart';
import 'package:tlaloc/src/ui/widgets/cards/personal_measures.dart';
import 'package:tlaloc/src/ui/widgets/cards/phrase.dart';
import 'package:tlaloc/src/ui/widgets/cards/tlalocmap.dart';
import 'package:tlaloc/src/ui/widgets/cards/tutorials.dart';
import 'package:tlaloc/src/ui/widgets/info/info_page.dart';
import 'package:tlaloc/src/ui/widgets/objects/quickadd.dart';
import 'package:tlaloc/src/ui/widgets/pluviometer/forecast.dart';
import 'package:tlaloc/src/ui/widgets/pluviometer/header.dart';
import 'package:tlaloc/src/ui/widgets/social/social_media.dart';

class HomeScreen extends StatefulWidget {
  const HomeScreen({super.key});

  @override
  State<HomeScreen> createState() => _HomeScreenState();
}

class _HomeScreenState extends State<HomeScreen> {
  bool isFabVisable = true;
  @override
  Widget build(BuildContext context) {
    return SafeArea(
      child: Scaffold(
        appBar: AppBar(
          title: Row(
            children: [
              Image.asset('assets/images/tlaloc_logo.png', height: 32),
              const SizedBox(width: 8),
              AutoSizeText(
                appName,
                style: TextStyle(
                  fontFamily: 'FredokaOne',
                  fontSize: 24,
                  letterSpacing: 2,
                ),
              ),
            ],
          ),
          actions: const <Widget>[InfoButton2(), FluidDialogWidget()],
        ),
        // drawer: DrawerApp(),
        body: LayoutBuilder(
          builder: (context, constraints) {
            final isWide = constraints.maxWidth > 800;

            final content = [
              OneTimeGoogleButton(message: "Llena el formulario (1 min)"),
              const SizedBox(height: 5),
              QuickAddWidget(),
              const Divider(height: 5, thickness: 4, color: Colors.black),

              const TodayWeatherStyleCard(),
              const WeekRainMarker(),

              GlassContainer(child: TutorialWidget()),
              GlassContainer(
                child: Column(
                  children: [
                    const Padding(
                      padding: EdgeInsets.all(8.0),
                      child: Text(
                        'Tabla de mediciones',
                        style: TextStyle(
                          color: AppColors.blue1,
                          fontFamily: 'FredokaOne',
                          fontSize: 24,
                          letterSpacing: 2,
                        ),
                      ),
                    ),
                    PersonalMeasures(),
                    GeneralMeasures(),
                  ],
                ),
              ),

              GlassContainer(child: TlalocMapData()),
              Column(
                children: [
                  Center(
                    child: Row(
                      children: const [
                        PhraseCard(),
                        Spacer(), // Espacio entre tarjetas
                        TableButton(),
                      ],
                    ),
                  ),
                  const Divider(height: 20, thickness: 4, color: Colors.black),
                  CommunityButton(),
                  const Divider(height: 20, thickness: 4, color: Colors.black),
                  SocialLinksWidget(),
                  const Divider(height: 20, thickness: 4, color: Colors.black),
                ],
              ),
            ];

            return NotificationListener<UserScrollNotification>(
              onNotification: (notification) {
                if (notification.direction == ScrollDirection.forward) {
                  if (!isFabVisable) setState(() => isFabVisable = true);
                } else if (notification.direction == ScrollDirection.reverse) {
                  if (isFabVisable) setState(() => isFabVisable = false);
                }
                return true;
              },
              child: SingleChildScrollView(
                padding: const EdgeInsets.all(16),
                child: Wrap(
                  runSpacing: 20,
                  spacing: 20,
                  alignment: WrapAlignment.center,
                  children:
                      content.map((widget) {
                        return ConstrainedBox(
                          constraints: BoxConstraints(
                            maxWidth:
                                isWide
                                    ? (constraints.maxWidth / 2) - 30
                                    : constraints.maxWidth,
                          ),
                          child: widget,
                        );
                      }).toList(),
                ),
              ),
            );
          },
        ),

        floatingActionButton: Visibility(visible: isFabVisable, child: Fab()),
      ),
    );
  }
}

\end{minted}





\section{Pantalla de envío de mediciones}
\label{anexo:alg13}
\begin{minted}{dart} 
import 'dart:io';
import 'package:firebase_auth/firebase_auth.dart';
import 'package:flutter/material.dart';
import 'package:flutter/services.dart';
import 'package:image_picker/image_picker.dart';
import 'package:provider/provider.dart';
import 'package:tlaloc/src/models/constants.dart';
import 'package:tlaloc/src/models/date.dart';
import 'package:tlaloc/src/models/lluvia/send_rain.dart';
import 'package:tlaloc/src/resources/onboarding/common_select.dart';
import 'package:tlaloc/src/ui/screens/home/home_widget_classes.dart';
import 'package:tlaloc/src/ui/widgets/measures/save_button.dart';
import 'package:tlaloc/src/models/app_state.dart';
import 'package:auto_size_text/auto_size_text.dart';
import 'package:audioplayers/audioplayers.dart';
import 'package:tlaloc/src/models/google_sign_in.dart';
import 'package:tlaloc/src/models/home_page.dart';
import 'package:tlaloc/src/ui/widgets/objects/text_field.dart';
import 'package:flutter/foundation.dart' show kIsWeb;

class AddScreen extends StatefulWidget {
  final Measurement? measurement;

  const AddScreen({super.key, this.measurement});

  @override
  State<AddScreen> createState() => _AddScreenState();
}

class _AddScreenState extends State<AddScreen> {
  late TextEditingController _precipitationController;
  bool pluviometer = false;

  File? newImage;
  Uint8List? newWebImage; // Para imagenes web
  final ImagePicker picker = ImagePicker();

  DateTime dateTime = DateTime.now();
  num precipitation = 0; // Variable no-nullable
  String? uploader = FirebaseAuth.instance.currentUser?.displayName;
  String path = 'sounds/correcto.mp3';
  var player = AudioPlayer();

  @override
  void initState() {
    super.initState();
    _precipitationController = TextEditingController(
      text: widget.measurement?.precipitation?.toStringAsFixed(1) ?? '0',
    );

    if (widget.measurement != null) {
      precipitation =
          widget.measurement!.precipitation ?? 0; // Conversión segura
      uploader = widget.measurement!.uploader;
      dateTime = widget.measurement!.dateTime!;
    }
  }

  @override
  void dispose() {
    _precipitationController.dispose();
    player.dispose();
    super.dispose();
  }

  Future<void> pickImage() async {
    final pickedFile = await picker.pickImage(source: ImageSource.gallery);
    if (pickedFile != null) {
      if (kIsWeb) {
        final bytes = await pickedFile.readAsBytes();
        setState(() {
          Provider.of<AppState>(context, listen: false).newWebImage = bytes;
          newImage = null;
        });
      } else {
        setState(() {
          newImage = File(pickedFile.path);
          Provider.of<AppState>(context, listen: false).newWebImage = null;
        });
      }
    }
  }

  Future<void> pickImageC() async {
    final pickedFile = await picker.pickImage(source: ImageSource.camera);
    if (pickedFile != null) {
      if (kIsWeb) {
        final bytes = await pickedFile.readAsBytes();
        setState(() {
          Provider.of<AppState>(context, listen: false).newWebImage = bytes;
          newImage = null;
        });
      } else {
        setState(() {
          newImage = File(pickedFile.path);
          Provider.of<AppState>(context, listen: false).newWebImage = null;
        });
      }
    }
  }

  @override
  Widget build(BuildContext context) {  
    return SafeArea(
      child: Scaffold(
        appBar: AppBar(
          title: Consumer<GoogleSignInProvider>(
            builder: (context, signIn, child) {
              String place = Provider.of<AppState>(context).paraje;
              return AutoSizeText(
                place,
                style: const TextStyle(fontSize: 24, fontFamily: 'FredokaOne'),
              );
            },
          ),
          actions: [
            Padding(
              padding: const EdgeInsets.all(8.0),
              child: ButtonWidget(
                onClicked: () async {
                  try {
                    final state = Provider.of<AppState>(context, listen: false);
                    if (widget.measurement == null) {
                      state.addMeasurement(
                        uploader: uploader!,
                        precipitation: precipitation,
                        time: dateTime,
                        image: newImage,
                        pluviometer: pluviometer,
                      );
                      // state.newWebImage = null;
                      await player.play(AssetSource(path));
                      Navigator.of(context).pushAndRemoveUntil(
                        MaterialPageRoute<void>(
                          builder: (BuildContext context) {
                            return const HomePage();
                          },
                        ),
                        (Route<dynamic> route) => false,
                      );
                    } else {
                      state.updateMeasurement(
                        uploaderId: widget.measurement!.uploaderId!,
                        id: widget.measurement!.id,
                        uploader: uploader!,
                        precipitation: precipitation,
                        time: dateTime,
                        image: newImage,
                        oldImage: widget.measurement!.imageUrl,
                        pluviometer: pluviometer,
                      );
                      await player.play(AssetSource(path));
                      Navigator.pop(context);
                    }
                  } catch (e) {
                    showDialog(
                      context: context,
                      builder:
                          (context) => AlertDialog(
                            title: const Text('¡Error al guardar la medición!'),
                            content: Text('$e'),
                          ),
                    );
                  }
                },
              ),
            ),
          ],
        ),
        body: SingleChildScrollView(
          child: Padding(
            padding: const EdgeInsets.all(16.0),
            child: Column(
              children: [
                Consumer<GoogleSignInProvider>(
                  builder: (context, signIn, child) {
                    final name =
                        FirebaseAuth.instance.currentUser?.displayName ?? '';
                    return MyTextFormField(
                      initialValue: uploader ?? name,
                      helperText: '1. Escriba nombre completo',
                      hintText: 'Nombre',
                      icon: const Icon(Icons.person, color: Colors.blueGrey),
                      onChanged: (String value) {
                        setState(() => uploader = value);
                      },
                      textInputType: TextInputType.name,
                    );
                  },
                ),
                const SizedBox(height: 20),

                Container(
                  decoration: BoxDecoration(
                    color:
                        Theme.of(context).brightness == Brightness.dark
                            ? AppColors.dark3
                            : Colors.transparent,
                    borderRadius: BorderRadius.circular(12.0),
                  ),
                  child: ListTile(
                    leading: CircleAvatar(
                      backgroundColor: Colors.red[300],
                      child: Icon(Icons.place, color: Colors.red[900]),
                    ),
                    title: Text(
                      'Estás ubicado en: "${Provider.of<AppState>(context).paraje}"',
                      style: const TextStyle(
                        fontSize: 18,
                        fontFamily: 'FredokaOne',
                      ),
                    ),
                    subtitle: const Text('2. Elige el paraje correctamente'),
                    onTap: () {
                      Navigator.push(
                        context,
                        MaterialPageRoute(
                          builder: (context) => const CommonSelectPage(),
                        ),
                      );
                    },
                  ),
                ),
                const SizedBox(height: 20),
                Container(
                  decoration: BoxDecoration(
                    color:
                        Theme.of(context).brightness == Brightness.dark
                            ? AppColors.dark3
                            : Colors.transparent,
                    borderRadius: BorderRadius.circular(12.0),
                  ),
                  child: Padding(
                    padding: const EdgeInsets.all(16.0),
                    child: RainInputWidget(
                      precipitation: precipitation,
                      onChanged: (value) {
                        setState(() {
                          precipitation = value;
                        });
                      },
                    ),
                  ),
                ), 
                const SizedBox(height: 20),
                Container(
                  decoration: BoxDecoration(
                    color:
                        Theme.of(context).brightness == Brightness.dark
                            ? AppColors.dark3
                            : Colors.transparent,
                    borderRadius: BorderRadius.circular(12.0),
                  ),
                  child: Datetime(
                    updateDateTime: (value) {
                      dateTime = value;
                    },
                  ),
                ),
 
                const SizedBox(height: 20),
                Container(
                  decoration: BoxDecoration(
                    color:
                        Theme.of(context).brightness == Brightness.dark
                            ? AppColors.dark3
                            : Colors.transparent,
                    borderRadius: BorderRadius.circular(12.0),
                  ),
                  child: const ContactUsListTile(
                    title: 'Mandar fotografía',
                    title2:
                        '5. Toma una foto del pluviómetro y mándala por WhatsApp',
                    message:
                        'https://api.whatsapp.com/send?phone=5630908507&text=%C2%A1Mira!%20en%20el%20paraje%20%22%20%22%20llovi%C3%B3%20%22%20%22mm,%20adjunto%20fotograf%C3%ADa%20del%20d%C3%ADa%20de%20hoy',
                  ),
                ),
                SizedBox(height: 20),
                Container(
                  decoration: BoxDecoration(
                    color:
                        Theme.of(context).brightness == Brightness.dark
                            ? AppColors.dark3
                            : Colors.transparent,
                    borderRadius: BorderRadius.circular(12.0),
                  ),
                  child: SwitchListTile(
                    title: const Text(
                      'Reinicio de mediciones',
                      style: TextStyle(fontSize: 18, fontFamily: 'FredokaOne'),
                    ),
                    value: pluviometer,
                    secondary: CircleAvatar(
                      backgroundColor: Colors.teal[300],
                      child: Icon(Icons.output, color: Colors.teal[900]),
                    ),
                    subtitle: const Text(
                      '6. Vaciar pluviómetro (sólo personal capacitado)',
                    ),
                    onChanged: (bool value) {
                      setState(() => pluviometer = value);
                    },
                  ),
                ),
                const SizedBox(height: 20),
              ],
            ),
          ),
        ),
      ),
    );
  }
}
\end{minted}



\section{Pantalla de Bitácora}
\label{anexo:alg14}
\begin{minted}{dart}
import 'package:flutter/material.dart';
import 'package:provider/provider.dart';
import 'package:tlaloc/src/models/app_state.dart';
import 'package:tlaloc/src/models/constants.dart';
import 'package:tlaloc/src/models/excel.dart';
import 'package:tlaloc/src/ui/widgets/appbar/infobutton2.dart';
import 'package:tlaloc/src/ui/widgets/data_screen_view.dart';
import 'package:tlaloc/src/ui/widgets/data_widget.dart';
import 'package:tlaloc/src/ui/widgets/info/info_page.dart';
import 'package:tlaloc/src/ui/widgets/real_data_widget.dart';

class DataScreen extends StatefulWidget {
  const DataScreen({super.key});

  @override
  State<DataScreen> createState() => _DataScreenState();
}

class _DataScreenState extends State<DataScreen> with TickerProviderStateMixin {
  late TabController _tabController;
  final List<ScrollController> _scrollControllers = [
    ScrollController(),
    ScrollController(),
  ];
  bool isFabVisible = true;

  @override
  void initState() {
    super.initState();
    _tabController = TabController(length: 2, vsync: this);

    // Escuchar cambios de pestaña
    _tabController.addListener(_handleTabChange);
  }

  void _handleTabChange() {
    // Hacer scroll al inicio cuando cambia la pestaña
    _scrollToTop(_tabController.index);
  }

  void _scrollToTop(int index) {
    final controller = _scrollControllers[index];
    if (controller.hasClients) {
      controller.animateTo(
        0,
        duration: const Duration(milliseconds: 300),
        curve: Curves.easeOut,
      );
    }
  }

  @override
  void dispose() {
    _tabController.removeListener(_handleTabChange);
    _tabController.dispose();
    for (var controller in _scrollControllers) {
      controller.dispose();
    }
    super.dispose();
  }

  @override
  Widget build(BuildContext context) {
    final appState = context.watch<AppState>();
    bool isWideLayout = MediaQuery.of(context).size.width > 800;
    return SafeArea(
      child: NestedScrollView(
        headerSliverBuilder: (context, value) {
          return [
            SliverAppBar(
              title: Row(
                children: [
                  Image.asset('assets/images/tlaloc_logo.png', height: 32),
                  const SizedBox(width: 8),
                  const Text(
                    'Bitácora',
                    textAlign: TextAlign.start,
                    style: TextStyle(
                      // color: AppColors.dark1,
                      fontFamily: 'FredokaOne',
                      fontSize: 24,
                      letterSpacing: 2,
                    ),
                  ),
                ],
              ),
              floating: true,
              pinned: true,
              snap: false,
              expandedHeight: 150.0,
              actions: <Widget>[
                IconButton(
                  icon: const Icon(Icons.file_download),
                  onPressed: () async {
                    try {
                      await appState.exportMeasurements(context);
                      ScaffoldMessenger.of(context).showSnackBar(
                        SnackBar(
                          content: Text(
                            'Exportación completada',
                            style: TextStyle(color: Colors.green),
                          ),
                        ),
                      );
                    } catch (e) {
                      ScaffoldMessenger.of(context).showSnackBar(
                        SnackBar(content: Text('Error al exportar: $e')),
                      );
                    }
                  },
                ),
                InfoButton2(),
                FluidDialogWidget(),
              ],
              bottom: TabBar(
                controller: _tabController,
                onTap: (index) => _scrollToTop(index),
                labelColor: AppColors.blue1,
                unselectedLabelColor: Colors.grey,
                indicatorColor: AppColors.blue1,
                labelStyle: const TextStyle(fontWeight: FontWeight.bold),
                unselectedLabelStyle: const TextStyle(
                  fontWeight: FontWeight.bold,
                ),
                tabs: const <Widget>[
                  Tab(text: 'Acumulados', icon: Icon(Icons.cloud_outlined)),
                  Tab(text: 'Reales', icon: Icon(Icons.cloud_done_outlined)),
                ],
              ),
            ),
          ];
        },
        body: TabBarView(
          controller: _tabController, // Asigna el mismo controlador
          children: <Widget>[
            // Pasa el ScrollController a cada widget hijo
            isWideLayout
                ? MasterDetailScreen()
                : MyDataWidget(scrollController: _scrollControllers[0]),
            isWideLayout
                ? MasterDetailRealScreen()
                : MyRealDataWidget(scrollController: _scrollControllers[0]),
          ],
        ),
      ),
    );
  }
}
\end{minted}




\section{Pantalla de estadísticas}
\label{anexo:alg15}
\begin{minted}{dart}
import 'dart:typed_data';
import 'dart:ui';

import 'package:auto_size_text/auto_size_text.dart';
import 'package:cloud_firestore/cloud_firestore.dart';
import 'package:flutter/material.dart';
import 'package:flutter/rendering.dart';
import 'package:intl/intl.dart';
import 'package:provider/provider.dart';
import 'package:tlaloc/src/models/app_state.dart';
import 'package:tlaloc/src/models/constants.dart';
import 'package:tlaloc/src/models/datepicker.dart';
import 'package:tlaloc/src/ui/widgets/backgrounds/empty_state.dart';
import 'package:tlaloc/src/ui/widgets/appbar/infobutton2.dart';
import 'package:tlaloc/src/ui/widgets/info/info_page.dart';
import 'package:fl_chart/fl_chart.dart';
import 'package:pdf/pdf.dart';
import 'package:pdf/widgets.dart' as pw;
import 'package:printing/printing.dart';
import 'package:file_saver/file_saver.dart';
import 'package:flutter/foundation.dart' show kIsWeb;
import 'package:path_provider/path_provider.dart';
import 'dart:io';

import 'package:universal_html/html.dart' as html;

class BarGraph extends StatefulWidget {
  const BarGraph({super.key});

  @override
  State<BarGraph> createState() => _BarGraphState();
}

enum DateTimeMode { custom, week, month, year, always }

enum DataMode { accumulated, real }

class _BarGraphState extends State<BarGraph> {
  DateTime initialDate = dateLongAgo;
  DateTime finalDate = dateInALongTime;
  DateTimeMode mode = DateTimeMode.always;
  String? _currentParaje;
  DataMode dataMode = DataMode.accumulated;
  final GlobalKey chartKey = GlobalKey();

  @override
  Widget build(BuildContext context) {
    return Scaffold(
      appBar: AppBar(
        title: Row(
          children: [
            Image.asset('assets/images/tlaloc_logo.png', height: 32),
            const SizedBox(width: 8),
            AutoSizeText(
              dataMode == DataMode.real
                  ? 'Volumen'
                  : 'Acumulados',
              style: const TextStyle(
                fontFamily: 'FredokaOne',
                fontSize: 18,
                letterSpacing: 2,
              ),
            ),
          ],
        ),
        actions: <Widget>[
          IconButton(
            icon: const Icon(Icons.picture_as_pdf),
            onPressed: () => _exportToPdf(context),
            tooltip: 'Exportar a PDF',
          ),
          InfoButton2(),
          FluidDialogWidget(),
        ],
      ),
      body: SafeArea(
        child: Column(
          children: [
            const SizedBox(height: 20),
            SwitchListTile(
              title: Text(
                dataMode == DataMode.real
                    ? 'Mostrar datos reales'
                    : 'Mostrar acumulados',
                style: const TextStyle(fontWeight: FontWeight.bold),
              ),
              value: dataMode == DataMode.real,
              onChanged:
                  (val) => setState(
                    () => dataMode = val ? DataMode.real : DataMode.accumulated,
                  ),
            ),
            const SizedBox(height: 10),
            _buildDateControls(),
            const SizedBox(height: 20),
            _buildDatePickers(),
            const SizedBox(height: 20),
            Expanded(
              child: _buildChartSection(isReal: dataMode == DataMode.real),
            ),
          ],
        ),
      ),
    );
  }

  Widget _buildDateControls() {
    return Padding(
      padding: const EdgeInsets.symmetric(horizontal: 12),
      child: Wrap(
        spacing: 4,
        children: [
          _buildChoiceChip('Esta semana', DateTimeMode.week),
          _buildChoiceChip('Este mes', DateTimeMode.month),
          _buildChoiceChip('Este año', DateTimeMode.year),
          _buildChoiceChip('Siempre', DateTimeMode.always),
        ],
      ),
    );
  }

  Widget _buildDatePickers() {
    return Padding(
      padding: const EdgeInsets.symmetric(horizontal: 12),
      child: Row(
        children: [
          const Text('Inicio: '),
          DatePickerButton(
            dateTime: initialDate,
            onDateChanged: (date) => _updateDates(date, isStart: true),
          ),
          const Expanded(child: SizedBox()),
          const Text('Fin: '),
          DatePickerButton(
            dateTime: finalDate,
            onDateChanged: (date) => _updateDates(date, isStart: false),
          ),
        ],
      ),
    );
  }

  Widget _buildChartSection({bool isReal = false}) {
    return Consumer<AppState>(
      builder: (context, state, _) {
        _handleParajeChange(state);
        return StreamBuilder<QuerySnapshot<Map<String, dynamic>>>(
          key: Key('${state.rol}-${state.paraje}'),
          stream:
              isReal
                  ? state.getRealMeasurementsStream()
                  : state.getMeasurementsStream(),
          builder: (context, snapshot) {
            if (snapshot.connectionState == ConnectionState.waiting) {
              return _buildLoadingIndicator();
            }
            if (snapshot.hasError) {
              return EmptyState('Error: ${snapshot.error}');
            }
            return _handleSnapshot(snapshot, state, isReal: isReal);
          },
        );
      },
    );
  }

  void _handleParajeChange(AppState state) {
    if (_currentParaje != state.paraje) {
      WidgetsBinding.instance.addPostFrameCallback((_) {
        setState(() => _currentParaje = state.paraje);
      });
    }
  }

  Widget _handleSnapshot(
    AsyncSnapshot<QuerySnapshot<Map<String, dynamic>>> snapshot,
    AppState state, {
    bool isReal = false,
  }) {
    if (!snapshot.hasData || snapshot.data!.docs.isEmpty) {
      return EmptyState('No hay datos en ${state.paraje}');
    }

    final measurements = state.getMeasurementsFromDocs(snapshot.data!.docs);
    final filteredMeasurements = _filterMeasurements(
      isReal ? _filterOnlyReal(measurements) : measurements,
    );

    return filteredMeasurements.isEmpty
        ? const EmptyState('No hay datos en el rango seleccionado')
        : _buildChart(filteredMeasurements);
  }

  List<Measurement> _filterOnlyReal(List<Measurement> realValue) {
    return realValue
        .where(
          (m) =>
              m.dateTime != null &&
              m.dateTime!.isAfter(initialDate) &&
              m.dateTime!.isBefore(finalDate),
        )
        .toList()
      ..sort((a, b) => a.dateTime!.compareTo(b.dateTime!));
  }

  List<Measurement> _filterMeasurements(List<Measurement> measurements) {
    return measurements
        .where(
          (m) =>
              m.dateTime != null &&
              m.dateTime!.isAfter(initialDate) &&
              m.dateTime!.isBefore(finalDate),
        )
        .toList()
      ..sort((a, b) => a.dateTime!.compareTo(b.dateTime!));
  }

  Widget _buildLoadingIndicator() {
    return const Center(
      child: Column(
        mainAxisAlignment: MainAxisAlignment.center,
        children: [
          CircularProgressIndicator(),
          SizedBox(height: 20),
          Text(
            'Cargando datos...',
            style: TextStyle(color: AppColors.blue1, fontSize: 16),
          ),
        ],
      ),
    );
  }

  Widget _buildChart(List<Measurement> measurements) {
    final theme = Theme.of(context);
    final primaryColor = theme.colorScheme.primary;
    final surfaceVariant = theme.colorScheme.surfaceContainerHighest;
    final onSurface = theme.colorScheme.onSurface;

    // Definir ancho total dinámico
    final chartWidth =
        (measurements.length * 40).toDouble().clamp(300, 2000).toDouble();

    return Padding(
      padding: const EdgeInsets.all(16.0),
      child: SizedBox(
        height: 400,
        child: SingleChildScrollView(
          scrollDirection: Axis.horizontal,
          child: SizedBox(
            width: chartWidth,
            child: RepaintBoundary(
              key: chartKey,
              child: BarChart(
                BarChartData(
                  groupsSpace: 16,
                  alignment: BarChartAlignment.spaceBetween,
                  barTouchData: BarTouchData(
                    enabled: true,
                    touchTooltipData: BarTouchTooltipData(
                      // tooltipBgColor: primaryColor.withOpacity(0.9),
                      getTooltipItem: (group, groupIndex, rod, rodIndex) {
                        final date = DateFormat(
                          'dd/MM/yy',
                        ).format(measurements[groupIndex].dateTime!);
                        final value = rod.toY.toStringAsFixed(1);
                        return BarTooltipItem(
                          '$date\n$value mm',
                          TextStyle(
                            color: theme.colorScheme.onPrimary,
                            fontWeight: FontWeight.bold,
                            fontSize: 12,
                          ),
                        );
                      },
                    ),
                  ),
                  titlesData: FlTitlesData(
                    leftTitles: AxisTitles(
                      axisNameWidget: Text(
                        'Precipitación (mm)',
                        style: TextStyle(
                          color: onSurface,
                          fontWeight: FontWeight.bold,
                          fontSize: 14,
                        ),
                      ),
                      sideTitles: SideTitles(
                        showTitles: true,
                        reservedSize: 22,
                        interval: _calculateYInterval(measurements),
                        getTitlesWidget:
                            (value, meta) => Text(
                              '${value.toInt()}',
                              style: TextStyle(color: onSurface, fontSize: 12),
                            ),
                      ),
                    ),
                    bottomTitles: AxisTitles(
                      axisNameWidget: Padding(
                        padding: const EdgeInsets.only(top: 12),
                        child: Text(
                          'Fecha',
                          style: TextStyle(
                            color: onSurface,
                            fontWeight: FontWeight.bold,
                            fontSize: 14,
                          ),
                        ),
                      ),
                      sideTitles: SideTitles(
                        showTitles: true,
                        getTitlesWidget: (value, meta) {
                          final index = value.toInt();
                          const maxLabels = 10;
                          final total = measurements.length;
                          if (total <= maxLabels ||
                              index % (total ~/ maxLabels) == 0) {
                            if (index >= 0 && index < total) {
                              return _buildDateLabel(
                                measurements[index].dateTime!,
                              );
                            }
                          }
                          return const SizedBox.shrink();
                        },
                      ),
                    ),
                    topTitles: const AxisTitles(
                      sideTitles: SideTitles(showTitles: false),
                    ),
                    rightTitles: const AxisTitles(
                      sideTitles: SideTitles(showTitles: false),
                    ),
                  ),
                  borderData: FlBorderData(
                    show: true,
                    border: const Border(
                      left: BorderSide(width: 1, color: Colors.grey),
                      bottom: BorderSide(width: 1, color: Colors.grey),
                    ),
                  ),
                  gridData: FlGridData(
                    show: true,
                    horizontalInterval: _calculateYInterval(measurements),
                    getDrawingHorizontalLine:
                        (value) =>
                            FlLine(color: surfaceVariant, strokeWidth: 1),
                  ),
                  barGroups:
                      measurements.asMap().entries.map((entry) {
                        final index = entry.key;
                        final m = entry.value;
                        return BarChartGroupData(
                          x: index,
                          barRods: [
                            BarChartRodData(
                              toY: (m.precipitation ?? 0).toDouble(),
                              color: primaryColor,
                              width: 5,
                              borderRadius: BorderRadius.circular(6),
                              gradient: LinearGradient(
                                colors: [
                                  primaryColor.withOpacity(0.9),
                                  primaryColor.withOpacity(0.5),
                                ],
                                begin: Alignment.topCenter,
                                end: Alignment.bottomCenter,
                              ),
                              backDrawRodData: BackgroundBarChartRodData(
                                show: true,
                                toY: (m.precipitation ?? 0) * 1.1,
                                color: surfaceVariant.withOpacity(0.3),
                              ),
                            ),
                          ],
                        );
                      }).toList(),
                  minY: 0,
                  maxY: _calculateMaxY(measurements),
                ),
                duration: const Duration(milliseconds: 800),
                curve: Curves.easeOutQuart,
              ),
            ),
          ),
        ),
      ),
    );
  }

  double _calculateMaxY(List<Measurement> measurements) {
    if (measurements.isEmpty) return 10;
    final max = measurements
        .map((m) => m.precipitation ?? 0)
        .reduce((a, b) => a > b ? a : b);
    return (max * 1.2).toDouble();
  }

  double _calculateYInterval(List<Measurement> measurements) {
    if (measurements.isEmpty) return 10; // Manejo de lista vacía

    final maxPrecip = measurements
        .map((m) => m.precipitation?.toDouble() ?? 0.0)
        .reduce((a, b) => a > b ? a : b); // Versión más eficiente

    if (maxPrecip > 50) return 20;
    if (maxPrecip > 20) return 10;
    return 5;
  }

  Widget _buildDateLabel(DateTime date) {
    return Transform.rotate(
      angle: -0.4,
      child: Text(
        DateFormat('dd/MM').format(date),
        style: const TextStyle(fontSize: 10),
      ),
    );
  }

  void _updateDates(DateTime date, {required bool isStart}) {
    setState(() {
      mode = DateTimeMode.custom;
      if (isStart) {
        initialDate = date;
      } else {
        finalDate = DateTime(
          date.year,
          date.month,
          date.day,
        ).add(const Duration(days: 1)).subtract(const Duration(seconds: 1));
      }
    });
  }

  Widget _buildChoiceChip(String label, DateTimeMode value) {
    return ChoiceChip(
      selectedColor: AppColors.blue1,
      label: Text(label),
      selected: mode == value,
      onSelected: (val) => val ? _handleTimeModeChange(value) : null,
    );
  }

  void _handleTimeModeChange(DateTimeMode value) {
    final now = DateTime.now();
    setState(() {
      mode = value;
      switch (value) {
        case DateTimeMode.week:
          final monday = now.subtract(Duration(days: now.weekday - 1));
          initialDate = monday;
          finalDate = monday.add(
            const Duration(days: 6, hours: 23, minutes: 59, seconds: 59),
          );
          break;
        case DateTimeMode.month:
          initialDate = DateTime(now.year, now.month, 1);
          finalDate = DateTime(now.year, now.month + 1, 0, 23, 59, 59);
          break;
        case DateTimeMode.year:
          initialDate = DateTime(now.year, 1, 1);
          finalDate = DateTime(now.year, 12, 31, 23, 59, 59);
          break;
        case DateTimeMode.always:
          initialDate = dateLongAgo;
          finalDate = dateInALongTime;
          break;
        case DateTimeMode.custom:
          break;
      }
    });
  }

  Future<void> _exportToPdf(BuildContext context) async {
    final appState = Provider.of<AppState>(context, listen: false);
    final measurements = await _getCurrentMeasurements(appState);

    if (measurements.isEmpty) {
      ScaffoldMessenger.of(context).showSnackBar(
        const SnackBar(content: Text('No hay datos para exportar')),
      );
      return;
    }

    final pdf = pw.Document();
    final theme = Theme.of(context);
    final title =
        dataMode == DataMode.real
            ? 'Datos Reales de Precipitación'
            : 'Volúmenes Acumulados de Precipitación';
    final dateRange =
        '${DateFormat('dd/MM/yyyy').format(initialDate)} - ${DateFormat('dd/MM/yyyy').format(finalDate)}';
    final paraje = appState.paraje;

    final chartImage = await _generateChartImage(measurements, theme);

    pdf.addPage(
      pw.Page(
        pageFormat: PdfPageFormat.a4,
        build: (pw.Context context) {
          return pw.Column(
            crossAxisAlignment: pw.CrossAxisAlignment.start,
            children: [
              pw.Header(
                level: 0,
                child: pw.Text(title, style: pw.TextStyle(fontSize: 20)),
              ),
              pw.Text('Paraje: $paraje'),
              pw.Text('Rango de fechas: $dateRange'),
              pw.SizedBox(height: 20),
              pw.Center(
                child: pw.Container(
                  height: 300,
                  child: pw.FittedBox(child: pw.Image(chartImage)),
                ),
              ),
              pw.SizedBox(height: 20),
              _buildDataTable(measurements),
            ],
          );
        },
      ),
    );

    // Guardar o mostrar el PDF según la plataforma
    final bytes = await pdf.save();
    await _saveOrPrintPdf(context, bytes, title);
  }

  Future<void> _saveOrPrintPdf(
    BuildContext context,
    Uint8List bytes,
    String title,
  ) async {
    if (kIsWeb) {
      final blob = html.Blob([bytes], 'application/pdf');
      final url = html.Url.createObjectUrlFromBlob(blob);
      final anchor =
          html.AnchorElement(href: url)
            ..setAttribute('download', '$title.pdf')
            ..click();
      html.Url.revokeObjectUrl(url);
    } else {
      // Para móvil: mostrar diálogo de impresión/guardado
      try {
        await Printing.layoutPdf(
          onLayout: (PdfPageFormat format) async => bytes,
        );
      } catch (e) {
        // Si falla la impresión, guardar el archivo
        final directory = await getApplicationDocumentsDirectory();
        final file = File('${directory.path}/$title.pdf');
        await file.writeAsBytes(bytes);

        // Opcional: usar file_saver para mejor experiencia de usuario
        try {
          await FileSaver.instance.saveFile(
            name: title,
            bytes: bytes,
            mimeType: MimeType.pdf,
          );
        } catch (e) {
          ScaffoldMessenger.of(context).showSnackBar(
            SnackBar(content: Text('PDF guardado en: ${file.path}')),
          );
        }
      }
    }
  }

  Future<List<Measurement>> _getCurrentMeasurements(AppState state) async {
    final snapshot =
        await (dataMode == DataMode.real
            ? state.getRealMeasurementsStream().first
            : state.getMeasurementsStream().first);

    final measurements = state.getMeasurementsFromDocs(snapshot.docs);
    return _filterMeasurements(measurements);
  }

  Future<pw.MemoryImage> _generateChartImage(
    List<Measurement> measurements,
    ThemeData theme,
  ) async {
    final boundary =
        chartKey.currentContext!.findRenderObject() as RenderRepaintBoundary;
    final image = await boundary.toImage(pixelRatio: 3.0);
    final byteData = await image.toByteData(format: ImageByteFormat.png);
    final bytes = byteData!.buffer.asUint8List();
    return pw.MemoryImage(bytes);
  }

  pw.Widget _buildDataTable(List<Measurement> measurements) {
    return pw.Table(
      border: pw.TableBorder.all(),
      children: [
        pw.TableRow(
          children: [
            pw.Padding(
              padding: const pw.EdgeInsets.all(4),
              child: pw.Text(
                'Fecha',
                style: pw.TextStyle(fontWeight: pw.FontWeight.bold),
                textAlign: pw.TextAlign.center,
              ),
            ),
            pw.Padding(
              padding: const pw.EdgeInsets.all(4),
              child: pw.Text(
                'Precipitación (mm)',
                style: pw.TextStyle(fontWeight: pw.FontWeight.bold),
                textAlign: pw.TextAlign.center,
              ),
            ),
          ],
        ),
        ...measurements.map(
          (m) => pw.TableRow(
            children: [
              pw.Padding(
                padding: const pw.EdgeInsets.all(4),
                child: pw.Text(
                  DateFormat('dd/MM/yyyy').format(m.dateTime!),
                  textAlign: pw.TextAlign.center,
                ),
              ),
              pw.Padding(
                padding: const pw.EdgeInsets.all(4),
                child: pw.Text(
                  (m.precipitation ?? 0).toStringAsFixed(1),
                  textAlign: pw.TextAlign.center,
                ),
              ),
            ],
          ),
        ),
      ],
    );
  }
}

\end{minted}





\section{Pantalla del perfil}
\label{anexo:alg16}
\begin{minted}{dart}
import 'package:flutter/material.dart';
import 'package:firebase_auth/firebase_auth.dart';
import 'package:provider/provider.dart';
import 'package:share_plus/share_plus.dart';
import 'package:ionicons/ionicons.dart';
import 'package:tlaloc/src/models/app_state.dart'; 
import 'package:url_launcher/url_launcher.dart';
import 'package:tlaloc/src/models/google_sign_in.dart';
import 'package:tlaloc/src/resources/onboarding/onbording.dart';
import 'package:tlaloc/src/ui/screens/settings/credits.dart';
import 'package:tlaloc/src/ui/screens/settings/faq.dart';

class ConfigureScreen extends StatefulWidget {
  const ConfigureScreen({super.key});

  @override
  State<ConfigureScreen> createState() => _ConfigureScreenState();
}

class _ConfigureScreenState extends State<ConfigureScreen> {
  @override
  Widget build(BuildContext context) {
    final appState = Provider.of<AppState>(context);

    final user = FirebaseAuth.instance.currentUser;
    final theme = Theme.of(context);
    return Scaffold(
      extendBodyBehindAppBar: true,
      body: SingleChildScrollView(
        child: Column(
          children: [
            Stack(
              alignment: Alignment.bottomCenter,
              children: [
                Image.asset('assets/images/portrate.jpg', fit: BoxFit.fitWidth),
                Container(
                  width: 120,
                  height: 120,
                  decoration: BoxDecoration(
                    shape: BoxShape.circle,
                    border: Border.all(
                      color: theme.colorScheme.surfaceContainerHighest,
                      width: 4,
                    ),
                  ),
                  child: CircleAvatar(
                    radius: 56,
                    backgroundImage:
                        user?.photoURL != null
                            ? NetworkImage(user!.photoURL!)
                            : null,
                    child:
                        user?.photoURL == null
                            ? Icon(
                              Icons.account_circle,
                              size: 60,
                              color: theme.colorScheme.onSurface,
                            )
                            : null,
                  ),
                ),
              ],
            ),

            const SizedBox(height: 70),

            // Información del usuario
            Text(
              user?.displayName ?? 'Usuario Tlaloc',
              style: TextStyle(
                fontSize: 24,
                fontWeight: FontWeight.bold,
                color: theme.colorScheme.onSurface,
                fontFamily: 'FredokaOne',
              ),
            ),
            const SizedBox(height: 8),
            Text(
              user?.email ?? 'correo@tlaloc.app',
              style: TextStyle(
                color: theme.colorScheme.onSurfaceVariant,
                fontSize: 16,
              ),
            ),

            // Estadísticas clave
            Padding(
              padding: const EdgeInsets.all(20),
              child: FutureBuilder<Map<String, dynamic>>(
                future: appState.getUserStats(),
                builder: (context, snapshot) {
                  if (!snapshot.hasData) {
                    return const CircularProgressIndicator();
                  }

                  final stats = snapshot.data!;
                  final local = stats['local'];
                  final global = stats['global'];
                  final parajes = stats['distinctParajes'];
                  final total = stats['totalParajes'];

                  return Row(
                    mainAxisAlignment: MainAxisAlignment.spaceAround,
                    children: [
                      _buildStatCard(context, 'Mediciones', '$local'),
                      _buildStatCard(context, 'Contribuciones', '$global'),
                      _buildStatCard(context, 'Parajes', '$parajes/$total'),
                    ],
                  );
                },
              ),
            ),
 
            _buildProfileSection(
              context,
              title: 'Configuración',
              children: [ 
                _buildConfigItem(
                  context,
                  icon: Icons.share,
                  title: 'Compartir aplicación',
                  action: () {
                    Share.share(
                      '¡Próximamente podrás obtener varios datos de él!\n\nDescárgala en tlaloc.org',
                      subject:
                          '¿Sabías que hay una app donde puedes registrar los datos de la lluvia en el Monte Tláloc?',
                    );
                  },
                ),
                _buildConfigItem(
                  context,
                  icon: Icons.feedback,
                  title: 'Enviar retroalimentación',
                  action: () {
                    launchUrl(
                      Uri.parse(
                        'mailto:tlloc-app@googlegroups.com?subject=Retroalimentación sobre Tláloc App',
                      ),
                    );
                  },
                ),
                _buildConfigItem(
                  context,
                  icon: Icons.description,
                  title: 'Términos y condiciones',
                  action: () => Navigator.pushNamed(context, '/privacy'),
                ),
                _buildConfigItem(
                  context,
                  icon: Icons.security,
                  title: 'Política de privacidad',
                  action: () => Navigator.pushNamed(context, '/politics'),
                ),
                _buildConfigItem(
                  context,
                  icon: Icons.info,
                  title: 'Acerca de',
                  action:
                      () => showAboutDialog(
                        context: context,
                        applicationIcon: CircleAvatar(
                          backgroundImage: const AssetImage(
                            'assets/images/img-1.png',
                          ),
                          backgroundColor: theme.colorScheme.surface,
                        ),
                        applicationLegalese: 'Con amor desde COLPOS',
                        applicationVersion: 'versión inicial (beta)',
                        children: [
                          _buildDialogItem(
                            context,
                            icon: Icons.people,
                            title: 'Ver créditos',
                            action:
                                () => Navigator.push(
                                  context,
                                  MaterialPageRoute(
                                    builder: (context) => const CreditsPage(),
                                  ),
                                ),
                          ),
                          _buildDialogItem(
                            context,
                            icon: Icons.question_mark_rounded,
                            title: 'Preguntas Frecuentes',
                            action:
                                () => Navigator.push(
                                  context,
                                  MaterialPageRoute(
                                    builder: (context) => const FaqPage(),
                                  ),
                                ),
                          ),
                          _buildDialogItem(
                            context,
                            icon: Ionicons.logo_facebook,
                            color: Colors.blue,
                            title: 'Síguenos en Facebook',
                            action:
                                () => launchUrl(
                                  Uri.parse(
                                    'https://www.facebook.com/Ciencia-Ciudadana-para-el-Monitoreo-de-Lluvia-100358326014423',
                                  ),
                                  mode: LaunchMode.externalApplication,
                                ),
                          ),
                          _buildDialogItem(
                            context,
                            icon: Ionicons.logo_youtube,
                            color: Colors.red,
                            title: 'Síguenos en YouTube',
                            action:
                                () => launchUrl(
                                  Uri.parse(
                                    'https://www.youtube.com/channel/UC2wNEwvGEvnQVAX1Uv3qztA',
                                  ),
                                  mode: LaunchMode.externalApplication,
                                ),
                          ),
                          _buildDialogItem(
                            context,
                            icon: Icons.email,
                            title: 'Mándanos un correo',
                            action:
                                () => launchUrl(
                                  Uri.parse(
                                    'mailto:tlloc-app@googlegroups.com',
                                  ),
                                ),
                          ),
                          _buildDialogItem(
                            context,
                            icon: Ionicons.logo_github,
                            title: 'Colabora en GitHub',
                            action:
                                () => launchUrl(
                                  Uri.parse(
                                    'https://github.com/Jack55913/TlalocApp',
                                  ),
                                  mode: LaunchMode.externalApplication,
                                ),
                          ),
                        ],
                      ),
                ),
                ListTile(
                  leading: Icon(Icons.logout, color: theme.colorScheme.error),
                  title: Text(
                    'Cerrar sesión',
                    style: TextStyle(color: theme.colorScheme.onSurface),
                  ),
                  onTap: () {
                    final provider = Provider.of<GoogleSignInProvider>(
                      context,
                      listen: false,
                    );
                    provider.logout();
                    Navigator.pushReplacement(
                      context,
                      MaterialPageRoute(builder: (context) => Onboarding()),
                    );
                  },
                ),
                const SizedBox(height: 50),
              ],
            ),
          ],
        ),
      ),
    );
  }

  Widget _buildStatCard(BuildContext context, String title, String value) {
    final theme = Theme.of(context);
    return Column(
      children: [
        Text(
          value,
          style: TextStyle(
            fontSize: 22,
            fontWeight: FontWeight.bold,
            color: theme.colorScheme.primary,
          ),
        ),
        Text(
          title,
          style: TextStyle(color: theme.colorScheme.onSurfaceVariant),
        ),
      ],
    );
  }

  Widget _buildProfileSection(
    BuildContext context, {
    required String title,
    required List<Widget> children,
  }) {
    final theme = Theme.of(context);
    return Padding(
      padding: const EdgeInsets.symmetric(vertical: 15, horizontal: 20),
      child: Column(
        crossAxisAlignment: CrossAxisAlignment.start,
        children: [
          Text(
            title,
            style: TextStyle(
              color: theme.colorScheme.onSurface,
              fontSize: 18,
              fontWeight: FontWeight.bold,
            ),
          ),
          const SizedBox(height: 10),
          ...children,
        ],
      ),
    );
  }

 

  Widget _buildConfigItem(
    BuildContext context, {
    required IconData icon,
    required String title,
    required Function action,
  }) {
    final theme = Theme.of(context);
    return ListTile(
      leading: Icon(icon, color: theme.colorScheme.primary),
      title: Text(title, style: TextStyle(color: theme.colorScheme.onSurface)),
      trailing: Icon(
        Icons.chevron_right,
        color: theme.colorScheme.onSurfaceVariant,
      ),
      onTap: () => action(),
    );
  }

  Widget _buildDialogItem(
    BuildContext context, {
    required IconData icon,
    required String title,
    required Function action,
    Color? color,
  }) {
    final theme = Theme.of(context);
    return ListTile(
      leading: Icon(icon, color: color ?? theme.colorScheme.onSurface),
      title: Text(title, style: TextStyle(color: theme.colorScheme.onSurface)),
      onTap: () => action(),
    );
  }
}

\end{minted}

%%% Bibliography  %%%%%%%%%%%%%%%%%%%%%%%%%%%%%%%%%%%%%%%%%%%%%%%%%%%%%%%%%%%%%%%%%%%%%%%%%%%%%%%%%

\printbibliography[title={LITERATURA CONSULTADA},heading=bibintoc]
\end{document} 