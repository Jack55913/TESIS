% !TEX encoding = UTF-8 Unicode
% !BIB TS-program = biber 
% !BIB program = biber    
%%%%%%%%%%%%%%%%%%%%%%%%%%%%%%%%%%%%%%%

\documentclass[twoside]{mitthesis} %,fontset=libertine, fontset=newtx-sans-text, fontset=heros-stix2, fontset=stix2
%
% option [twoside]		gives facing-page behavior for printing; omitting twoside will eliminate even-numbered blank pages.
% option [lineno]	 	provides line numbers, as for editing
% option [mydesign] 	loads packages for color, title and list formats, margins, or captions: edit mydesign.tex to change defaults.
% option [fontset] is a keyvalue which can be:
%					 	pdftex or unicode engines:  defaultfonts, libertine, lucida
%					 	pdftex only: 				fira-newtxsf, newtx, newtx-sans-text
%						unicode engines (luatex):	heros-stix2, stix2, termes, termes-stix2
%					 	if no key value is given, fonts default to CMR (pdftex) or LMR (unicode), i.e., "the LaTeX font".
%					 	You can edit the fontset files or you can write your own, myfonts.tex, and do [fontset=myfonts].
%						If you are using multiple languages, load the babel package in your fontset file, before the fonts.
\usepackage[spanish]{babel}

%%%%%%%%% Packages used in sample chapters (not otherwise required) %%%%%%%

%% Package for code listing in Appendix A.
\usepackage{listings}%   documentation is here https://ctan.org/pkg/listings

%% Set chemical formulas nicely
\usepackage[version=4]{mhchem}%   documentation is here https://ctan.org/pkg/mhchem

%% Latin filler used in Chapter 1, with a test for package version date. https://ctan.org/pkg/lipsum
\usepackage{lipsum}
\IfPackageAtLeastTF{lipsum}{2021/09/20}{\setlipsum{auto-lang=false}}{}


%%%%%%%%%  Graphics path (to figure files)  %%%%%%%%%%%%%%%%%%%%%%%%%%%%%%%%

%% Can set graphicspath to point to specific directories containing figures (the current directory is searched automatically)
%% For instance, to search a subdirectory of the current directory called "figures" and a parallel directory called "art", set:

\graphicspath{{assets/}}

%%%%%%%%%  Representative set-up for biblatex  %%%%%%%%%%%%%%%%%%%%%%%%%%%%%

\usepackage[style=ieee,maxbibnames=10,sorting=none]{biblatex}% style=ext-numeric-comp,articlein=false,giveninits=true
	\DefineBibliographyStrings{english}{url= \textsc{url} ,  }% replaces default "[Online]. Available" by "URL"


\addbibresource{mitthesis-sample.bib}%% <== change to YOUR bib file <= CHANGE

%% to avoid split urls and stretched white space, you can set the bibliography ragged-right:
%\appto{\bibsetup}{\raggedright}

% biblatex is very powerful, and you can customize most aspects the reference list and citations to suit your needs.
% documentation is here: https://ctan.org/pkg/biblatex


%%%%%%%%%%  Option to use natbib   %%%%%%%%%%%%%%%%%%%%%%%%%%%%%%%%%%%%%%%%%

%\RequirePackage[numbers,sort&compress]{natbib}
 
%%% add bibliography to table of contents
%\apptocmd{\bibliography}{\addcontentsline{toc}{chapter}{\protect\textbf{\bibname}}}{}{}

%%% You can use this to rename the bibliography section
%\renewcommand{\bibname}{References}

%%% Can adjust space between bibliography items (change 4pt to something else; don't drop last two lengths, they are stretchable "glue")
%\setlength\bibsep{4pt plus 1pt minus 1pt}


%%%%%%%%%%  Table related packages  %%%%%%%%%%%%%%%%%%%%%%%%%%%%%%%%%%%%%%%%

\usepackage{booktabs}% better quality tables, https://ctan.org/pkg/booktabs
\usepackage{array}%    additional options for table columns, https://ctan.org/pkg/array
%\usepackage{tabularx}%   https://ctan.org/pkg/tabularx

%\usepackage{dcolumn}%    alignment on decimal place, https://ctan.org/pkg/dcolumn
%\newcolumntype{d}[1]{D{.}{.}{#1}}

\usepackage{amsmath,amssymb,amsfonts}		
\newtheorem{notation}{Notación}[chapter]
\newtheorem{theorem}{Teorema}[chapter]

%%%%%%%%%%  Option for "double spacing" %%%%%%%%%%%%%%%%%%%%%%%%%%%%%%%%%%%%

%% Back in the typewriter era, double spaced lines were convenient for editing with a pencil. 
%% In typography, the separation between lines is called "leading", and it is usually set in 
%% proportion to the font size (i.e., when the font is loaded).  If you really feel the need 
%% to change the line separation, the most attractive results will be obtained by changing the
%% leading in proportion to the the current font size, rather than just doubling the space.

%% The setspace package provides a tool for changing line separation. Use these two commands here:
%
% \usepackage{setspace}%  documentation at https://ctan.org/pkg/setspace
% \setstretch{1.1}% you can choose some other value for the stretch of space between lines
%
%% Use one or more of the these commands AFTER the frontmatter
%
% \onehalfspacing
% \doublespacing
% \singlespacing  % will turn these effects off (you can use these anywhere in the document)

%% The best result may be to stay with leading selected by the typographer who set up the font.


%%%%%%%%%%%  Metadata  %%%%%%%%%%%%%%%%%%%%%%%%%%%%%%%%%%%%%%%%%%%%%%%%%%%%%%%

% Most of the document metadata is created automatically. 
% The following items should be adjusted to match your work. <================= !!!!!!!!!!

\hypersetup{%
	pdfsubject={Template for writing MIT theses with the mitthesis class},
	% Change this to briefly state topic of your thesis 
% 
	pdfkeywords={Massachusetts Institute of Technology, MIT},
	% Add keywords that will help search engines and libraries to find your work.
	% Includes the name[s] of the author[s] 
	% (If you have used \DocumentMetadata, at line 15, you can just put "\CopyrightAuthor," for the names.)
%
	pdfurl={},
	% If you have a url for the thesis, put it here. Otherwise delete this.
	% (MIT Libraries will put your thesis in DSPACE with a persistent url after you submit it.)
%	
	pdfcontactemail={},
	% You can put a [permanent] email address into the metadata, if you like.
	% Otherwise delete this.
%
	pdfauthortitle={},
	% If you have a title, you can include it here.
}

%%%%%%%%%%%%%%  End preamble %%%%%%%%%%%%%%%%%%%%%%%%%%%%%%%%%%%%%%%%%%%%%%%%%%%%%%%%%%%%%%%%%%%%%
%%%%%%%%%%%%%%%%%%%%%%%%%%%%%%%%%%%%%%%%%%%%%%%%%%%%%%%%%%%%%%%%%%%%%%%%%%%%%%%%%%%%%%%%%%%%%%%%%%

\begin{document}

\title{APLICACIONES DE LAS ECUACIONES DE NAVIER STOKES}

\Author{LUIS EMILIO ÁLVAREZ HERRERA}{DEPARTAMENTO DE IRRIGACIÓN}
% \Author{Luisa Hernández}{Department of Research}[B.S. Mechanical Engineering, UCLA, 2018][M.S. Stellar Interiors, Vulcan Science Academy, 2020]
% \Author{Thurston Howell III}{Department of Economics}[MBA, Ferengi School of Management, 2022]

% Use once for each degree fulfilled by thesis
% For two degrees from one department, leave the department argument blank for the second degree {}.
% \Degree{Bachelor of Science in Physics}{Department of Physics}
% \Degree{Master of Science in Physics}{}
\Degree{INGENIERO EN IRRIGACIÓN}{DEPARTAMENTO DE IRRIGACIÓN}

% If there is more than one supervisor, use the \Supervisor command for each.
\Supervisor{Edward C. Pickering}{Professor of Physics}
% \Supervisor{Secunda Castor}{Professor of Research}
% \Supervisor{Quintus Castor}{Professor of Log Dams}

% Professor who formally accepts theses for your department (e.g., the Graduate Officer, Professor Sméagol,...)
% If more than one department, use more than once
% **If you need to reduce vertical space, put the acceptor title in the second argument and leave the third blank {}.**
 \Acceptor{Primus Castor}{Professor of Wetlands Engineering}{Undergraduate Officer, Department of Physics}
% \Acceptor{Tertius Castor}{Professor of Log Dams}{Graduate Officer, Department of Research}
% \Acceptor{Quarta Castor}{Professor of Lodge Building}{Graduate Officer, Department of Mechanical Engineering}

% Usage: \DegreeDate{Month}{year}
% Valid degree months are September, February, or June
\DegreeDate{June}{1876}

% Date that final thesis is submitted to department
\ThesisDate{May 18, 1876}

%%%%%%  Choose whether to have a CREATIVE COMMONS License  %%%%%%%%%%%%%%%%%%%%%%%%%%%%%%%%%%%%%%
%
% If you are using a cc license, put details of your cc license here. 
% Omit this command if you are not using a cc license.
%
\CClicense{CC BY-NC-ND 4.0}{https://creativecommons.org/licenses/by-nc-nd/4.0/}
%

%%%%%%%  Solutions for overflowing titlepage  %%%%%%%%%%%%%%%%%%%%%%%%%%%%%%%%%%%%%%%%%%%%%%%%%%%

% If your title page is overflowing (from too many names, degrees, etc.):
%
% (a) you can reduce the 12pt and 18pt skips between various blocks to 6pt with this command:
%
% \Tighten
%
% (b)  you can scale down the Signature block at the bottom with this command:
%
% \SignatureBlockSize{\small}  %or this one \SignatureBlockSize{\footnotesize}
%
% (c) you can put the acceptor name and title onto two lines, rather than three like this:
%
% \Acceptor{Tertius Castor}{Professor and Graduate Officer, Department of Research}{}
% \Acceptor{Quarta Castor}{Professor and Graduate Officer, Department of Mechanical Engineering}{}
%
% (d) you can change the font size of the the author name[s] with
%
%	\AuthorNameSize{\normalsize}
%
% (e) and you can omit any previous degrees from the title page, instead mentioning them in the Biosketch

% Also, if you prefer to keep the text toward the top of the page with most white space at the bottom, you
% can you this command to squash all of the vertical glue (stretchy space) with this command:
%
% \Squash 
%
% This command is useful when the text has not already reach the bottom of the page, since the glue gets squashed automatically
% when the page is too full.

%%%%%%%%%%%%%%%%%%%%%%%%%%%%%%%%%%%%%%%%%%%%%%%%%%%%%%%%%%%%%%%%%%%%%%%%%%%%%%%%%%%%%%%%%%%%%%%%%


% \maketitle
%% acknowledgments.tex

% From mitthesis package
% Version: 1.01, 2023/10/16
% Documentation: https://ctan.org/pkg/mitthesis


\chapter*{AGRADECIMIENTOS}
\addcontentsline{toc}{chapter}{AGRADECIMIENTOS}
A lo largo de mi vida, he recorrido senderos llenos de desafíos, aprendizajes y gratitud. Hoy, al culminar esta etapa académica, deseo dedicar este trabajo a quienes han sido el faro y la brújula en mi viaje.

A mi mamá María Carolina Herrera Díaz y a mi papá Agustín Álvarez Bautista, cuyos sacrificios y amor incondicional me han dado la fortaleza para alcanzar mis metas. Ustedes me enseñaron que la educación es el legado más valioso y que el esfuerzo constante siempre rinde frutos. Cada paso que doy es un reflejo de su dedicación y valores inculcados. Mamá, papá, esta tesis es tan suya como mía.

A mis abuelos Mamá Aya, Papá Gogo, Luisa y Agustín, guardianes de la sabiduría y el cariño eterno. Aunque algunos ya no estén físicamente, sus enseñanzas y amor permanecen vivos en mi corazón. Sus historias y consejos me han guiado en los momentos más difíciles, dándome el coraje para persistir y superar obstáculos.

A mis hermanos Paulo Elías, Alan Yareth y Aranza Ailín, incondicionales de aventuras y desafíos. Gracias por ser mi apoyo en los días grises y mi celebración en los días de triunfo. Su confianza en mí ha sido una fuente de motivación constante.

A mis profesores Humberto López Chimil, Fernando Chávez León, Luis Castellanos y todos mis mentores, que con su sabiduría y paciencia han encendido en mí la llama del conocimiento. Sus enseñanzas han trascendido las aulas y han dejado una huella imborrable en mi formación personal y profesional. Gracias por creer en mi potencial y por inspirarme a ser mejor cada día.

A la Univerisdad Autónoma Chapingo

Finalmente, dedico esta tesis a Dios, porque el me dió la voluntad de perseverar a pesar de las adversidades, por cada noche en vela y cada instante de duda superado. Este logro es el resultado de años de esfuerzo y dedicación, me recuerda que los sueños se alcanzan con determinación y pasión. Gracias a todos los que han sido parte de este viaje. Esta tesis es una manifestación de vuestro amor, apoyo y fe en mí. A todos ustedes, mi eterna gratitud.% .tex extension is presumed by \include 

% %% biography.tex
%% This section is optional

% From mitthesis package
% Version: 1.01, 2023/10/16
% Documentation: https://ctan.org/pkg/mitthesis

\chapter*{Biographical Sketch}
\addcontentsline{toc}{chapter}{Biographical Sketch}

Silas Whitcomb Holman was born in Harvard, Massachusetts on January 20, 1856. He received his S.B. degree in Physics from MIT in 1876, and then joined the MIT Department of Physics as an Assistant. He became Instructor in Physics in 1880, Assistant Professor in 1882, Associate Professor in 1885, and Full Professor in 1893. Throughout this period, he struggled with increasingly severe rheumatoid arthritis. At length, he was defeated, becoming Professor Emeritus in 1897 and dying on April 1, 1900.

Holman's light burned brilliantly before his tragic and untimely death. He published extensively in thermal physics, and authored textbooks on precision measurement, fundamental mechanics, and other subjects. He established the original Heat Measurements Laboratory. Holman was a much admired teacher among both his students and his colleagues. The reports of his department and of the Institute itself refer to him frequently in the 1880's and 1890's, in tones that gradually shift from the greatest respect to the deepest sympathy.

Holman was a student of Professor Edward C. Pickering, then head of the Physics department. Holman himself became second in command of Physics, under Professor Charles R. Cross, some years later. Among Holman's students, several went on to distinguish themselves, including: the astronomer George E. Hale ('90) who organized the Yerkes and Mt. Wilson observatories and who designed the 200 inch telescope on Mt. Palomar; Charles G. Abbot ('94), also an astrophysicist and later Secretary of the Smithsonian Institution; and George K. Burgess ('96), later Director of the Bureau of Standards. % optional, see MIT Libraries https://libraries.mit.edu/distinctive-collections/thesis-specs/#format

%%% Table of contents and lists of stuff (delete lists you don't need, e.g., if no tables) %%%%%%%%
\renewcommand{\contentsname}{CONTENIDO}
\tableofcontents
\renewcommand{\listtablename}{LISTA DE CUADROS}
\listoftables
\renewcommand{\listfigurename}{LISTA DE FIGURAS}
\listoffigures
\renewcommand\abstractname{RESUMEN}
\begin{abstract}
	% From mitthesis package
% Version: 1.01, 2023/06/19
% Documentation: https://ctan.org/pkg/mitthesis
%
% The abstract environment creates all the required headers and footnote. 
% You only need to add the text of the abstract itself.
%
% Approximately 500 words or less; try not to use formulas or special characters
% If you don't want an initial indentation, do \noindent at the start of the abstract
\begin{abstract}
% use \input rather than \include because we're inside an environment

La evapotranspiración es uno de los principales componentes del ciclo
hidrológico y de vital importancia en la planeación y operación de zonas de riego,
pues de ella dependen en gran medida los cálculos para conocer las necesidades
hídricas de los cultivos para evitar la sub o sobreestimación de la lámina de riego
aplicada, sin embargo su estudio resulta complicado pues la medición depende
de dos procesos separados, evaporación y transpiración los cuales varían
espacial y temporalmente, por lo que existen métodos para la estimación de ésta
con ayuda de información meteorológica.

\textbf{Palabras-Clave:} Ecuaciones de Navier-Stokes en R3, Inteligencia Artificial, Hidropónia, Aeropónia, Agricultura Vertical
\end{abstract}

\renewcommand\abstractname{ABSTRACT}
\begin{abstract}
	Recent experiments show that active fluids stirred by swimming bacteria or ATPpowered microtubule networks can exhibit complex flow dynamics and emergent pattern scale selection. Here, I will investigate a simplified phenomenological approach
	to 'active turbulence', a chaotic non-equilibrium steady-state in which the solvent
	flow develops a dominant vortex size. This approach generalizes the incompressible
	Navier-Stokes equations by accounting for active stresses through a linear instability
	mechanism, in contrast to externally driven classical turbulence. This minimal model
	can reproduce experimentally observed velocity statistics and is analytically tractable
	in planar and curved geometry. Exact stationary bulk solutions include Abrikosovtype vortex lattices in 2D and chiral Beltrami fields in 3D. Numerical simulations
	for a plane Couette shear geometry predict a low viscosity phase mediated by stress
	defects, in qualitative agreement with recent experiments on bacterial suspensions.
	Considering the active analog of Stokes' second problem, our numerical analysis predicts that a periodically rotating ring will oscillate at a higher frequency in an active
	fluid than in a passive fluid, due to an activity-induced reduction of the fluid inertia.
	The model readily generalizes to curved geometries. On a two-sphere, we present exact stationary solutions and predict a new type of upward energy transfer mechanism
	realized through the formation of vortex chains, rather than the merging of vortices,
	as expected from classical 2D turbulence. In 3D simulations on periodic domains, we
	observe spontaneous mirror-symmetry breaking realized through Beltrami-like flows,
	which give rise to upward energy transfer, in contrast to the classical direct Richardson cascade. Our analysis of triadic interactions supports this numerical prediction
	by establishing an analogy with forced rigid body dynamics and reveals a previously
	unknown triad invariant for classical turbulence.
	
\textbf{Key-Words:} Navier-Stokes equations in R3, Artificial Intelligence, Hydroponics, Aeroponics, Vertical Agriculture
\end{abstract}% use \input rather than \include because we're inside an environment
\end{abstract}
\renewcommand\abstractname{ABSTRACT}
\begin{abstract}
	% From mitthesis package
% Version: 1.01, 2023/06/19
% Documentation: https://ctan.org/pkg/mitthesis
%
% The abstract environment creates all the required headers and footnote. 
% You only need to add the text of the abstract itself.
%
% Approximately 500 words or less; try not to use formulas or special characters
% If you don't want an initial indentation, do \noindent at the start of the abstract

La evapotranspiración es uno de los principales componentes del ciclo
hidrológico y de vital importancia en la planeación y operación de zonas de riego,
pues de ella dependen en gran medida los cálculos para conocer las necesidades
hídricas de los cultivos para evitar la sub o sobreestimación de la lámina de riego
aplicada, sin embargo su estudio resulta complicado pues la medición depende
de dos procesos separados, evaporación y transpiración los cuales varían
espacial y temporalmente, por lo que existen métodos para la estimación de ésta
con ayuda de información meteorológica.
Con el objetivo de determinar la evapotranspiración de referencia (ETo) con
Machine Learning en la estación meteorológica Chapingo, México, se presentan
cuatro modelos para la predicción de la ETo, mediante machine learning usando
redes neuronales artificiales para el proceso de entrenamiento. Para entrenar el
modelo, se utilizaron 5119 datos diarios de la estación meteorológica automática
Chapingo, con los que se calcularon la ETo usando el método de FAO PenmanMonteith.
Se realizó un diagrama de correlaciones con el que se identificaron las variables
con mayor impacto en el cálculo de la ETo, sobresaliendo la radiación solar,
posteriormente la temperatura máxima, la humedad relativa, así como la
humedad relativa mínima y máxima. Con esta información se eligieron la
combinación de variables que sirvieron como datos de entrada a cada uno de los
modelos a entrenar.
En cada uno de los modelos se encontraron los parámetros de la red neuronal
que optimizaron el cálculo, tales como arquitectura de la red, capas ocultas y
número de neuronas en cada capa, así como el número de iteraciones, learning
rate y funciones de activación.
En el modelo 4, usando únicamente la radiación solar, se obtuvo un muy buen
ajuste del modelo con una R2 de 0.92, y un RSME de 0.0119, en los modelos 1 y
2, usando también la temperatura máxima, además la humedad relativa máxima
y mínima se mejoró en poca medida el ajuste del modelo, obteniendo una R2 de
0.93 en ambos casos y un RSME de 0.0082 y 0.0074 respectivamente.
Finalmente, el modelo 3, que no consideró la radiación solar no se ajustó
correctamente obteniendo una R2 de 0.66 y un RSME de 0.1946.
Palabras clave: Inteligencia Artificial, FAO Penman-Monteith, Redes Neuronales
Artificiales, Modelación, Agrometeorología.

\footnote{Text from Holman (1876): \doi{10.2307/25138434}.}  
% use \input rather than \include because we're inside an environment
\end{abstract}

%%% Chapters of thesis  %%%%%%%%%%%%%%%%%%%%%%%%%%%%%%%%%%%%%%%%%%%%%%%%%%%%%%%%%%%%%%%%%%%%%%%%%%%

%% If you want to use "double spacing", you should start here...

\newgeometry{left=4cm, right=2.5cm, top=2.5cm, bottom=2.5cm, marginparwidth=0pt, headsep=0pt}
\chapter{INTRODUCCIÓN}
\pagenumbering{arabic}
\setcounter{page}{1}
%TODO: SE HACE AL FINAL problema que se pretende resolver, justificación, hipótesis, objetivos, .
% al final de cada capitulo poner conclusion

% Contextualización del tema
% DESARROLLO DEL CONTEXTO
% Planteamiento del problema

% Justificación



% Alcances y limitaciones

% Antecedentes  









% Contexto 

% \section{Flutter}
% \section{Firebase}
\section{Planteamiento del problema}

En México, las redes oficiales de monitoreo hidrometeorológico, como las operadas por la Comisión Nacional del Agua (CONAGUA), presentan una cobertura limitada en muchas regiones de montaña, donde los microclimas pueden variar significativamente en distancias cortas. % TODO: CITA BIBLIO


El Monte Tláloc, ubicado en la zona montañosa del oriente del Valle de México, es un ejemplo de ello: su importancia ambiental, histórica y cultural contrasta con la escasa información climática precisa y en tiempo real disponible para la comunidad local, investigadores y tomadores de decisiones. Esta falta de datos puntuales dificulta la \textbf{gestión sustentable del agua}, la prevención de riesgos y el análisis del cambio climático a escala local.
% TODO AGREGAR MÁS PROBLEMÁTICAS

Las aplicaciones disponibles para la recolección de datos meteorológicos suelen ser de uso profesional, poco accesibles o no están diseñadas para fomentar la participación ciudadana en contextos rurales o de baja conectividad. Esto genera una brecha entre el potencial de colaboración ciudadana y las herramientas disponibles para lograrlo.

Ante este panorama, surge la necesidad de desarrollar una aplicación multiplataforma intuitiva, accesible y robusta, que aproveche el poder de la ciencia ciudadana para llenar los vacíos de información sobre la precipitación en el Monte Tláloc. Dicha aplicación debe facilitar la recolección, visualización y validación de datos por parte de usuarios no expertos, promoviendo la generación de conocimiento colectivo, la educación ambiental y la participación activa de la comunidad en temas de gestión hídrica y climática.



% 2. Análisis del Movimiento de Nutrientes en Sistemas Hidropónicos Mediante Ecuaciones de Navier-Stokes

% Objetivo: Investigar cómo las ecuaciones de Navier-Stokes pueden mejorar la distribución de nutrientes en sistemas hidropónicos.
% Metodología: Modelado matemático y simulaciones para optimizar el diseño del sistema de riego.
% Impacto: Aumentar la eficiencia del uso de nutrientes y agua, promoviendo un crecimiento vegetal uniforme.


% 4. Estudio del Efecto de Diferentes Configuraciones de Riego en la Agricultura Vertical Mediante Navier-Stokes
% Objetivo: Analizar cómo distintas configuraciones de riego afectan la distribución de agua y nutrientes.
% Metodología: Simulaciones basadas en las ecuaciones de Navier-Stokes para evaluar la eficiencia de diferentes sistemas.
% Impacto: Optimización del riego, reducción del desperdicio de agua y mejora del crecimiento de las plantas.

%  8.Desarrollo de Algoritmos para el Control Automático del Riego en Agricultura Vertical Basados en Modelos CFD
% Objetivo: Crear algoritmos que automaticen el riego basándose en datos de simulaciones de flujo de fluidos.
% Metodología: Integración de modelos CFD con sistemas de control automático.
% Impacto: Incrementar la precisión del riego, optimizando el uso de agua y nutrientes.

\section{Contexto geográfico del Monte Tláloc}


El Monte Tláloc, es un volcán formado a partir de las capas de sucesivas erupciones basálticas fluidas; ubicado en el Eje Neovolcánico en el límite entre los municipios de Ixtapaluca y Texcoco al oriente del Estado de México. Forma parte de la Sierra Nevada y es el Área Natural Protegida “Parque Nacional Iztaccíhuatl-Popocatépetl” su ubicación hidrológica es al oriente de la cuenca de México. Con sus 4120 metros sobre el nivel del mar, el Tláloc es la novena cima más alta del país. Cuenta con un clima de montaña cuya designación oficial es semifrío subhúmedo con lluvias en verano, de humedad media \cite{inegi_texcoco}.


\section{Justificación}
Se identifica la necesidad de un instrumento para la captura y envío de datos pluviales que sea accesible, participativo y que garantice la disponibilidad de la información obtenida para su análisis y toma de decisiones. Este instrumento debe ser sencillo de usar y estar diseñado específicamente para el público objetivo: los ejidatarios. Ellos, a través de su conocimiento del territorio y participación activa, pueden convertirse en aliados estratégicos para la recolección continua y precisa de datos.
La aplicación desarrollada se plantea como una solución innovadora que responde a esta necesidad. Su diseño intuitivo permite que usuarios con conocimientos tecnológicos básicos puedan capturar y enviar información sobre las precipitaciones de manera rápida y eficiente. Además, al integrar elementos de ciencia ciudadana, se fomenta la colaboración activa de las comunidades locales, fortaleciendo su empoderamiento y compromiso con la conservación de los recursos hídricos.
Desde un enfoque técnico, el proyecto destaca por su carácter práctico y adaptable. La app aprovecha tecnologías modernas para registrar datos de lluvia, lo que no solo optimiza la recopilación de información en tiempo real, sino que también reduce los costos asociados a equipos de medición tradicionales. Al centralizar y analizar estos datos en una plataforma digital, se genera un repositorio de información confiable que puede ser utilizado por investigadores, autoridades locales y los mismos ejidatarios para tomar decisiones fundamentadas. 
Por último, la disponibilidad de esta información en un formato accesible y visualmente comprensible contribuye a sensibilizar a los usuarios sobre la importancia de monitorear los patrones de lluvia, facilitando su uso en estrategias de manejo hídrico, planificación agrícola y mitigación de riesgos climáticos. De esta forma, el proyecto no solo soluciona un problema técnico, sino que también tiene un impacto social y ambiental significativo.

\section{Hipótesis}

La implementación de una aplicación multiplataforma, basada en principios de ciencia ciudadana, mejora significativamente la precisión y frecuencia de los reportes de lluvia en la región del Monte Tláloc, al facilitar la participación activa de los habitantes locales mediante herramientas digitales accesibles e intuitivas; lo cual contribuye a la generación de datos meteorológicos complementarios a los obtenidos por estaciones profesionales, permitiendo una caracterización más detallada de los eventos de precipitación en zonas de difícil acceso.



\section{Contribuciones de este trabajo}

Este trabajo de tesis contribuye al campo del desarrollo tecnológico, la ciencia ciudadana y la meteorología local mediante la creación de una aplicación multiplataforma diseñada específicamente para el monitoreo participativo de lluvia en el Monte Tláloc. La solución propuesta integra tecnologías móviles modernas con servicios en la nube y diseño centrado en el usuario, permitiendo que cualquier ciudadano pueda registrar datos de precipitación de manera sencilla, segura y estructurada. Esta contribución tiene un impacto directo en la generación de datos alternativos en regiones donde la infraestructura meteorológica es escasa o limitada, y donde los fenómenos hidrometeorológicos presentan comportamientos complejos.

Desde el punto de vista técnico, la tesis presenta una arquitectura modular desarrollada con Flutter, integrando funcionalidades clave como, sincronización con Firebase, visualización gráfica de estadísticas y un sistema para validar la veracidad de las mediciones con base en algoritmos desarrollados para la interpretación de datos de pluviómetros caseros. Se propone también una metodología de evaluación del nivel de maduración tecnológica (TRL) aplicada a aplicaciones de ciencia ciudadana, lo cual permite medir de forma objetiva el avance y aplicabilidad real del sistema desarrollado.

Además, este trabajo representa un esfuerzo por brindar el acceso a las tecnologías de monitoreo ambiental, empoderando a las comunidades rurales al integrarlas como agentes activos en la recolección de datos climáticos, al tiempo que fortalece los vínculos entre el conocimiento científico y la sabiduría local. Finalmente, se generan aportes a futuras investigaciones en temas relacionados con aplicaciones móviles para monitoreo ambiental, ciencia abierta y educación en contextos rurales, abriendo camino a iniciativas de colaboración interdisciplinaria entre desarrolladores, científicos, comunidades y tomadores de decisiones.


\section{Esquema de la tesis}
Este trabajo está estructurado como sigue. La introducción va seguida de X capítulos independientes que se han ordenado de forma coherente con el proceso de desarrollo de las ecuaciones de Navier-Stokes. En gran medida, la notación es consistente a lo largo de esta tesis y cada excepción está claramente resaltada. Para facilitar la navegación, se incluyen apéndices al final de sus respectivos capítulos, mientras que la bibliografía acumulativa se adjunta al final de este documento. Los capítulos siguientes se resumen brevemente a continuación.
\begin{itemize}
    \item \textbf{Capítulo 1} Consiste en...
    \item \textbf{Capítulo 2} Consiste en...
    \item \textbf{Capítulo 3} Consiste en...
    \item \textbf{Capítulo 4} Consiste en...
    \item \textbf{Capítulo 5} Consiste en...
    \item \textbf{Capítulo 6} Consiste en...
\end{itemize}


\section{Alcance y limitaciones del estudio}









% \section{Estado del arte}
% \subsection{Una formulación canónica hamiltoniana del problema de Navier-Stokes}
% En marzo de 2024, un grupo de matemáticos de Carolina del Sur, propusieron una formulación canónica hamiltoniana \cite{sanders2024canonical}, a continuación se muestra la ecuación:
% \begin{equation}
%     \int\,dx_2\left[\frac{1}{2} \frac{1}{\rho^2} \frac{\delta S^*}{\delta u_1}\frac{\delta S^*}{\delta u_1} - \frac{1}{\rho}\left(p_{,1} - \mu u_{1,22} \frac{\delta S^*}{\delta u_1} \right) \right] + \frac{\delta S^*}{\delta \partial t} = 0
% \label{eqea1}
% \end{equation}
% con $\delta S^*/Sp=0$. La solución a la ecuación (\ref{eqea1}) proporcionaría una transformación canónica a una nueva conjunto de coordenadas, dando expresiones analíticas para $(u_1,p)$

% EXPLOSIÓN EN TIEMPO FINITO PARA UNA ECUACIÓN TRIDIMENSIONAL PROMEDIO DE NAVIER-STOKES 2015

% \subsection{Una formulación cuantitativa del problema de regularidad global para el periódico Sistema Navier-Stokes, Terence Tao}
% 2007



% La gran incógnita es la turbulencia. Para escalas finas en tres dimensiones es mucho más no lineal.
% NO EXISTE UNA EXPLICACIÓN MATEMÁTICA FORMAL DE CÓMO SE PASA DE UN FLUJO REGULAR A UN FLUJO TURBULENTO



% EL TRABAJO MÁS NUEVO. PUEDE SER EL DE LARENCE TAO



\chapter{OBJETIVOS}
\section{Objetivo General}
\begin{enumerate}
    \item Exponer las aplicaciones en el Ingeniería en Irrigación mediante las Ecuaciones de Navier Stokes
\end{enumerate}
\section{Objetivos Específicos}
\begin{enumerate}
        \item Documentar el problema de las ecuaciones de Navier-Stokes en tercera dimensión
        \item Modelar las ecuaciones de Navier-Stokes en un sistema de Agricultura Vertical Plant Factory Hidropónico con relación agua-planta
        \item Aplicar algoritmos de Inteligencia Artificial y Machine Learning a los métodos numéricos
\end{enumerate}
% \chapter{HIPÓTESIS}
% % Disponer de más agua, puede ser un incentivo para el desarrollo de sistemas hidropónicos y de agricultura vertical 

% % Hablar sobre la calidad del agua de lluvia y su tratamiento. saber si sirve pa regar
% \chapter{JUSTIFICACIÓN}

\chapter{REVISIÓN DE LITERATURA}
% s antecedentes relevantes al estudio en orden histórico
\section{Ecuación de Nevier-Stokes 2d, Ladyzhenskaya}
Hay dos direcciones principales en la vida científica del prof. Ladyzhenskaya. La primera: la existencia, unicidad y regularidad de las soluciones de las ecuaciones de Navier-Stokes. La segunda: teoría de la regularidad para ecuaciones elípticas y parabólicas no lineales.

En 1951, Olga A. Ladyzhenskaya demostró la segunda desigualdad fundamental para operadores elípticos L de segundo orden con coeficientes suaves y para cualquier condición de frontera homogénea clásica.
\begin{equation}
    \left\lVert u \right\rVert W_2^2 (\Omega)\leq c_{\Omega} \left( \left\lVert Lu\right\rVert_{L_2(\Omega)} +\left\lVert u\right\rVert_{L_2(\Omega)} \right)
\end{equation}
que es válido para cualquier $u\in W_2^2(\Omega)$.

En cuanto a la primera dirección, en 1958 en [3] Olga A. Ladyzhenskaya demostró la desigualdad multiplicativa
\begin{equation}
    \left\lVert u \right\rVert_{L_4(\Omega)}^4\leq c \left\lVert u\right\rVert_{L_2(\Omega)}^2\left\lVert \nabla u\right\rVert_{L_2(\Omega)}^2  
\end{equation}
que es válido para cualquier $u\in W_2^1(\Omega),\, \Omega\in \mathbb{R}^2$.

Esta desigualdad dio la posibilidad de demostrar la existencia de una solución única global. del sistema bidimensional Navier-Stokes
\chapter{MATERIALES Y MÉTODOS}

\section{Método 1}
\subsection{Resumen}



% \subsection{Apéndices}
% Esta sección incluye material complementario que respalda ciertos argumentos presentados en este capítulo.
% \subsubsection{Tema complementario}

\chapter{RESULTADOS}

\section{Protocolo de monitoreo participativo:}

\section{Desarrollo del código}
 

\section{Evaluación del nivel de maduración tecnológica}















\chapter{CONCLUSIONES FINALES Y TRABAJO FUTURO}


















%%% Appendicies of thesis  %%%%%%%%%%%%%%%%%%%%%%%%%%%%%%%%%%%%%%%%%%%%%%%%%%%%%%%%%%%%%%%%%%%%%%%%

\appendix
\chapter{Code listing}
a


%%% Bibliography  %%%%%%%%%%%%%%%%%%%%%%%%%%%%%%%%%%%%%%%%%%%%%%%%%%%%%%%%%%%%%%%%%%%%%%%%%%%%%%%%%

\printbibliography[title={LITERATURA CONSULTADA},heading=bibintoc]
\end{document} 