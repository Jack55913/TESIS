\newgeometry{left=4cm, right=2.5cm, top=2.5cm, bottom=2.5cm, marginparwidth=0pt, headsep=0pt}
\chapter{INTRODUCCIÓN}
\pagenumbering{arabic}
\setcounter{page}{1}
%TODO: SE HACE AL FINAL problema que se pretende resolver, justificación, hipótesis, objetivos, .


% Contexto 
\section{Planteamiento del problema}
% 2. Análisis del Movimiento de Nutrientes en Sistemas Hidropónicos Mediante Ecuaciones de Navier-Stokes

% Objetivo: Investigar cómo las ecuaciones de Navier-Stokes pueden mejorar la distribución de nutrientes en sistemas hidropónicos.
% Metodología: Modelado matemático y simulaciones para optimizar el diseño del sistema de riego.
% Impacto: Aumentar la eficiencia del uso de nutrientes y agua, promoviendo un crecimiento vegetal uniforme.


% 4. Estudio del Efecto de Diferentes Configuraciones de Riego en la Agricultura Vertical Mediante Navier-Stokes
% Objetivo: Analizar cómo distintas configuraciones de riego afectan la distribución de agua y nutrientes.
% Metodología: Simulaciones basadas en las ecuaciones de Navier-Stokes para evaluar la eficiencia de diferentes sistemas.
% Impacto: Optimización del riego, reducción del desperdicio de agua y mejora del crecimiento de las plantas.

%  8.Desarrollo de Algoritmos para el Control Automático del Riego en Agricultura Vertical Basados en Modelos CFD
% Objetivo: Crear algoritmos que automaticen el riego basándose en datos de simulaciones de flujo de fluidos.
% Metodología: Integración de modelos CFD con sistemas de control automático.
% Impacto: Incrementar la precisión del riego, optimizando el uso de agua y nutrientes.

\section{Contexto Histórico}


\begin{itemize}
    \item 1687: Isaac Newton - Publica la ley de la viscosidad en su obra Principia Mathematica, sentando las bases de la mecánica de fluidos.
    \item 1822: Claude-Louis Navier - Introduce las primeras ecuaciones que incorporan la viscosidad en la dinámica de fluidos.
    \item 1845: George Gabriel Stokes - Refina y formaliza las ecuaciones de Navier, produciendo las ecuaciones de Navier-Stokes tal como se conocen hoy.
    \item 1934: Jean Leray - Introduce el concepto de soluciones débiles y demuestra la existencia global de estas soluciones en tres dimensiones.
    \item 1950s-1960s: Olga Ladyzhenskaya - Realiza avances significativos en la teoría de existencia y unicidad de soluciones para las ecuaciones de Navier-Stokes.
    \item 1967: Lars Onsager - Propone teorías sobre la turbulencia y la energía en sistemas de fluidos.
    \item 1970s-1980s: Tosio Kato - Contribuye al estudio de la regularidad y la teoría de semigrupos aplicados a las ecuaciones de Navier-Stokes.
    \item 1970s: John von Neumann - Desarrolla métodos numéricos y computacionales para resolver las ecuaciones de Navier-Stokes.
    \item 1980s-1990s: Peter Constantin - Hace importantes contribuciones a la teoría de la turbulencia y el análisis de los flujos de fluidos.
    \item 1980s-1990s: Ronald Coifman - Trabaja en el análisis armónico y su aplicación a la dinámica de fluidos.
    % \item 2000: Problema del Milenio del Instituto Clay - Se plantea la pregunta sobre la existencia y regularidad de soluciones suaves para las ecuaciones de Navier-Stokes en tres dimensiones.
    \item 2000s-2010s: Cédric Villani - Trabaja en la teoría cinética de los gases y las ecuaciones de Boltzmann, relacionadas con las ecuaciones de Navier-Stokes.
    \item 2000s-2010s: Weinan E - Realiza avances en la teoría de la turbulencia y el análisis numérico de las ecuaciones de Navier-Stokes.
    \item 2010s: Terence Tao - Explora la regularidad y la estabilidad de las soluciones de las ecuaciones de Navier-Stokes.
    \item 2010s: Charles Fefferman - Trabaja en la teoría de la regularidad de las ecuaciones de Navier-Stokes.
    \item 2010s: Pierre-Louis Lions - Hace importantes contribuciones al estudio de las ecuaciones en derivadas parciales, incluidas las ecuaciones de Navier-Stokes.
    \item 2010s: Beatrice Riviere - Desarrolla métodos numéricos para resolver las ecuaciones de Navier-Stokes, aplicándolos a problemas prácticos.
    \item 2010s: Charles Peskin - Desarrolla el método del punto inmerso, una técnica numérica para simular fluidos y estructuras elásticas que interactúan.
    \item 2010s: Cédric Villani - Continuó trabajando en la teoría cinética de los gases y sus aplicaciones a las ecuaciones de Navier-Stokes. Recibió la Medalla Fields en 2010 por sus contribuciones a la teoría del transporte óptimo y sus aplicaciones a las ecuaciones en derivadas parciales.
    \item 2010s: Peter Constantin - Continuó sus investigaciones en turbulencia, estabilidad y métodos numéricos para fluidos, haciendo contribuciones significativas a la comprensión matemática de la dinámica de fluidos.
    \item 2014: Terence Tao - Publicó trabajos sobre problemas relacionados con la regularidad de las soluciones de las ecuaciones de Navier-Stokes, explorando métodos innovadores para abordar estos problemas.
    \item 2020: Yann Brenier - Avanzó en la comprensión de las ecuaciones de Navier-Stokes a través del estudio de flujos óptimos y transporte en fluidos, desarrollando nuevas técnicas en análisis y geometría para abordar estos problemas.
    \item 2020s: Edriss Titi - Realizó investigaciones sobre la regularidad y la estabilidad de las soluciones de las ecuaciones de Navier-Stokes, con aplicaciones a la modelización de fenómenos atmosféricos y oceánicos.
    \item 2020s: Alexei Borodin - Aplicó métodos probabilísticos y de teoría de matrices aleatorias al estudio de la dinámica de fluidos, proporcionando nuevas perspectivas sobre el comportamiento de las soluciones de las ecuaciones de Navier-Stokes.
    \item 2021: Carlos Kenig - Recibió el Premio Abel por sus contribuciones al estudio de ecuaciones en derivadas parciales, incluyendo resultados importantes sobre la regularidad y la dispersión de soluciones de las ecuaciones de Navier-Stokes.
    \item 2020s: Advances in Computational Fluid Dynamics (CFD) - Se realizaron avances significativos en la simulación numérica de fluidos mediante el uso de supercomputadoras y técnicas de aprendizaje automático, permitiendo simulaciones más precisas y detalladas de problemas complejos en dinámica de fluidos.
\end{itemize}

\section{Justificación}

\section{Hipótesis}


\section{Contribuciones de este trabajo}
% COLOCAR QUÉ APORTÓ LA CHAMBA AL MUNDO límites y alcances


\section{Esquema de la tesis}
Este trabajo está estructurado como sigue. La introducción va seguida de X capítulos independientes que se han ordenado de forma coherente con el proceso de desarrollo de las ecuaciones de Navier-Stokes. En gran medida, la notación es consistente a lo largo de esta tesis y cada excepción está claramente resaltada. Para facilitar la navegación, se incluyen apéndices al final de sus respectivos capítulos, mientras que la bibliografía acumulativa se adjunta al final de este documento. Los capítulos siguientes se resumen brevemente a continuación.
\begin{itemize}
    \item \textbf{Capítulo 1} Consiste en...
    \item \textbf{Capítulo 2} Consiste en...
    \item \textbf{Capítulo 3} Consiste en...
    \item \textbf{Capítulo 4} Consiste en...
    \item \textbf{Capítulo 5} Consiste en...
    \item \textbf{Capítulo 6} Consiste en...
\end{itemize}













% \section{Estado del arte}
% \subsection{Una formulación canónica hamiltoniana del problema de Navier-Stokes}
% En marzo de 2024, un grupo de matemáticos de Carolina del Sur, propusieron una formulación canónica hamiltoniana \cite{sanders2024canonical}, a continuación se muestra la ecuación:
% \begin{equation}
%     \int\,dx_2\left[\frac{1}{2} \frac{1}{\rho^2} \frac{\delta S^*}{\delta u_1}\frac{\delta S^*}{\delta u_1} - \frac{1}{\rho}\left(p_{,1} - \mu u_{1,22} \frac{\delta S^*}{\delta u_1} \right) \right] + \frac{\delta S^*}{\delta \partial t} = 0
% \label{eqea1}
% \end{equation}
% con $\delta S^*/Sp=0$. La solución a la ecuación (\ref{eqea1}) proporcionaría una transformación canónica a una nueva conjunto de coordenadas, dando expresiones analíticas para $(u_1,p)$

% EXPLOSIÓN EN TIEMPO FINITO PARA UNA ECUACIÓN TRIDIMENSIONAL PROMEDIO DE NAVIER-STOKES 2015

% \subsection{Una formulación cuantitativa del problema de regularidad global para el periódico Sistema Navier-Stokes, Terence Tao}
% 2007



% La gran incógnita es la turbulencia. Para escalas finas en tres dimensiones es mucho más no lineal.
% NO EXISTE UNA EXPLICACIÓN MATEMÁTICA FORMAL DE CÓMO SE PASA DE UN FLUJO REGULAR A UN FLUJO TURBULENTO



% EL TRABAJO MÁS NUEVO. PUEDE SER EL DE LARENCE TAO


