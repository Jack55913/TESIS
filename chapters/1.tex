\newgeometry{left=4cm, right=2.5cm, top=2.5cm, bottom=2.5cm, marginparwidth=0pt, headsep=0pt}
\chapter{INTRODUCCIÓN}
\pagenumbering{arabic}
\setcounter{page}{1}
%TODO: SE HACE AL FINAL problema que se pretende resolver, justificación, hipótesis, objetivos, .
% al final de cada capitulo poner conclusion

% Contextualización del tema
% DESARROLLO DEL CONTEXTO
% Planteamiento del problema

% Justificación



% Alcances y limitaciones

% Antecedentes  









% Contexto 

% \section{Flutter}
% \section{Firebase}
\section{Planteamiento del problema}

En México, las redes oficiales de monitoreo hidrometeorológico, como las operadas por la Comisión Nacional del Agua (CONAGUA), presentan una cobertura limitada en muchas regiones de montaña, donde los microclimas pueden variar significativamente en distancias cortas. % TODO: CITA BIBLIO


El Monte Tláloc, ubicado en la zona montañosa del oriente del Valle de México, es un ejemplo de ello: su importancia ambiental, histórica y cultural contrasta con la escasa información climática precisa y en tiempo real disponible para la comunidad local, investigadores y tomadores de decisiones. Esta falta de datos puntuales dificulta la \textbf{gestión sustentable del agua}, la prevención de riesgos y el análisis del cambio climático a escala local.
% TODO AGREGAR MÁS PROBLEMÁTICAS

Las aplicaciones disponibles para la recolección de datos meteorológicos suelen ser de uso profesional, poco accesibles o no están diseñadas para fomentar la participación ciudadana en contextos rurales o de baja conectividad. Esto genera una brecha entre el potencial de colaboración ciudadana y las herramientas disponibles para lograrlo.

Ante este panorama, surge la necesidad de desarrollar una aplicación multiplataforma intuitiva, accesible y robusta, que aproveche el poder de la ciencia ciudadana para llenar los vacíos de información sobre la precipitación en el Monte Tláloc. Dicha aplicación debe facilitar la recolección, visualización y validación de datos por parte de usuarios no expertos, promoviendo la generación de conocimiento colectivo, la educación ambiental y la participación activa de la comunidad en temas de gestión hídrica y climática.



% 2. Análisis del Movimiento de Nutrientes en Sistemas Hidropónicos Mediante Ecuaciones de Navier-Stokes

% Objetivo: Investigar cómo las ecuaciones de Navier-Stokes pueden mejorar la distribución de nutrientes en sistemas hidropónicos.
% Metodología: Modelado matemático y simulaciones para optimizar el diseño del sistema de riego.
% Impacto: Aumentar la eficiencia del uso de nutrientes y agua, promoviendo un crecimiento vegetal uniforme.


% 4. Estudio del Efecto de Diferentes Configuraciones de Riego en la Agricultura Vertical Mediante Navier-Stokes
% Objetivo: Analizar cómo distintas configuraciones de riego afectan la distribución de agua y nutrientes.
% Metodología: Simulaciones basadas en las ecuaciones de Navier-Stokes para evaluar la eficiencia de diferentes sistemas.
% Impacto: Optimización del riego, reducción del desperdicio de agua y mejora del crecimiento de las plantas.

%  8.Desarrollo de Algoritmos para el Control Automático del Riego en Agricultura Vertical Basados en Modelos CFD
% Objetivo: Crear algoritmos que automaticen el riego basándose en datos de simulaciones de flujo de fluidos.
% Metodología: Integración de modelos CFD con sistemas de control automático.
% Impacto: Incrementar la precisión del riego, optimizando el uso de agua y nutrientes.

\section{Contexto geográfico del Monte Tláloc}


El Monte Tláloc, es un volcán formado a partir de las capas de sucesivas erupciones basálticas fluidas; ubicado en el Eje Neovolcánico en el límite entre los municipios de Ixtapaluca y Texcoco al oriente del Estado de México. Forma parte de la Sierra Nevada y es el Área Natural Protegida “Parque Nacional Iztaccíhuatl-Popocatépetl” su ubicación hidrológica es al oriente de la cuenca de México. Con sus 4120 metros sobre el nivel del mar, el Tláloc es la novena cima más alta del país. Cuenta con un clima de montaña cuya designación oficial es semifrío subhúmedo con lluvias en verano, de humedad media \cite{inegi_texcoco}.


\section{Justificación}
Se identifica la necesidad de un instrumento para la captura y envío de datos pluviales que sea accesible, participativo y que garantice la disponibilidad de la información obtenida para su análisis y toma de decisiones. Este instrumento debe ser sencillo de usar y estar diseñado específicamente para el público objetivo: los ejidatarios. Ellos, a través de su conocimiento del territorio y participación activa, pueden convertirse en aliados estratégicos para la recolección continua y precisa de datos.
La aplicación desarrollada se plantea como una solución innovadora que responde a esta necesidad. Su diseño intuitivo permite que usuarios con conocimientos tecnológicos básicos puedan capturar y enviar información sobre las precipitaciones de manera rápida y eficiente. Además, al integrar elementos de ciencia ciudadana, se fomenta la colaboración activa de las comunidades locales, fortaleciendo su empoderamiento y compromiso con la conservación de los recursos hídricos.
Desde un enfoque técnico, el proyecto destaca por su carácter práctico y adaptable. La app aprovecha tecnologías modernas para registrar datos de lluvia, lo que no solo optimiza la recopilación de información en tiempo real, sino que también reduce los costos asociados a equipos de medición tradicionales. Al centralizar y analizar estos datos en una plataforma digital, se genera un repositorio de información confiable que puede ser utilizado por investigadores, autoridades locales y los mismos ejidatarios para tomar decisiones fundamentadas. 
Por último, la disponibilidad de esta información en un formato accesible y visualmente comprensible contribuye a sensibilizar a los usuarios sobre la importancia de monitorear los patrones de lluvia, facilitando su uso en estrategias de manejo hídrico, planificación agrícola y mitigación de riesgos climáticos. De esta forma, el proyecto no solo soluciona un problema técnico, sino que también tiene un impacto social y ambiental significativo.

\section{Hipótesis}

La implementación de una aplicación multiplataforma, basada en principios de ciencia ciudadana, mejora significativamente la precisión y frecuencia de los reportes de lluvia en la región del Monte Tláloc, al facilitar la participación activa de los habitantes locales mediante herramientas digitales accesibles e intuitivas; lo cual contribuye a la generación de datos meteorológicos complementarios a los obtenidos por estaciones profesionales, permitiendo una caracterización más detallada de los eventos de precipitación en zonas de difícil acceso.



\section{Contribuciones de este trabajo}

Este trabajo de tesis contribuye al campo del desarrollo tecnológico, la ciencia ciudadana y la meteorología local mediante la creación de una aplicación multiplataforma diseñada específicamente para el monitoreo participativo de lluvia en el Monte Tláloc. La solución propuesta integra tecnologías móviles modernas con servicios en la nube y diseño centrado en el usuario, permitiendo que cualquier ciudadano pueda registrar datos de precipitación de manera sencilla, segura y estructurada. Esta contribución tiene un impacto directo en la generación de datos alternativos en regiones donde la infraestructura meteorológica es escasa o limitada, y donde los fenómenos hidrometeorológicos presentan comportamientos complejos.

Desde el punto de vista técnico, la tesis presenta una arquitectura modular desarrollada con Flutter, integrando funcionalidades clave como, sincronización con Firebase, visualización gráfica de estadísticas y un sistema para validar la veracidad de las mediciones con base en algoritmos desarrollados para la interpretación de datos de pluviómetros caseros. Se propone también una metodología de evaluación del nivel de maduración tecnológica (TRL) aplicada a aplicaciones de ciencia ciudadana, lo cual permite medir de forma objetiva el avance y aplicabilidad real del sistema desarrollado.

Además, este trabajo representa un esfuerzo por brindar el acceso a las tecnologías de monitoreo ambiental, empoderando a las comunidades rurales al integrarlas como agentes activos en la recolección de datos climáticos, al tiempo que fortalece los vínculos entre el conocimiento científico y la sabiduría local. Finalmente, se generan aportes a futuras investigaciones en temas relacionados con aplicaciones móviles para monitoreo ambiental, ciencia abierta y educación en contextos rurales, abriendo camino a iniciativas de colaboración interdisciplinaria entre desarrolladores, científicos, comunidades y tomadores de decisiones.


\section{Esquema de la tesis}
Este trabajo está estructurado como sigue. La introducción va seguida de X capítulos independientes que se han ordenado de forma coherente con el proceso de desarrollo de las ecuaciones de Navier-Stokes. En gran medida, la notación es consistente a lo largo de esta tesis y cada excepción está claramente resaltada. Para facilitar la navegación, se incluyen apéndices al final de sus respectivos capítulos, mientras que la bibliografía acumulativa se adjunta al final de este documento. Los capítulos siguientes se resumen brevemente a continuación.
\begin{itemize}
    \item \textbf{Capítulo 1} Consiste en...
    \item \textbf{Capítulo 2} Consiste en...
    \item \textbf{Capítulo 3} Consiste en...
    \item \textbf{Capítulo 4} Consiste en...
    \item \textbf{Capítulo 5} Consiste en...
    \item \textbf{Capítulo 6} Consiste en...
\end{itemize}


\section{Alcance y limitaciones del estudio}









% \section{Estado del arte}
% \subsection{Una formulación canónica hamiltoniana del problema de Navier-Stokes}
% En marzo de 2024, un grupo de matemáticos de Carolina del Sur, propusieron una formulación canónica hamiltoniana \cite{sanders2024canonical}, a continuación se muestra la ecuación:
% \begin{equation}
%     \int\,dx_2\left[\frac{1}{2} \frac{1}{\rho^2} \frac{\delta S^*}{\delta u_1}\frac{\delta S^*}{\delta u_1} - \frac{1}{\rho}\left(p_{,1} - \mu u_{1,22} \frac{\delta S^*}{\delta u_1} \right) \right] + \frac{\delta S^*}{\delta \partial t} = 0
% \label{eqea1}
% \end{equation}
% con $\delta S^*/Sp=0$. La solución a la ecuación (\ref{eqea1}) proporcionaría una transformación canónica a una nueva conjunto de coordenadas, dando expresiones analíticas para $(u_1,p)$

% EXPLOSIÓN EN TIEMPO FINITO PARA UNA ECUACIÓN TRIDIMENSIONAL PROMEDIO DE NAVIER-STOKES 2015

% \subsection{Una formulación cuantitativa del problema de regularidad global para el periódico Sistema Navier-Stokes, Terence Tao}
% 2007



% La gran incógnita es la turbulencia. Para escalas finas en tres dimensiones es mucho más no lineal.
% NO EXISTE UNA EXPLICACIÓN MATEMÁTICA FORMAL DE CÓMO SE PASA DE UN FLUJO REGULAR A UN FLUJO TURBULENTO



% EL TRABAJO MÁS NUEVO. PUEDE SER EL DE LARENCE TAO


