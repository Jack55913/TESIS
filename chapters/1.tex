\newgeometry{left=4cm, right=2.5cm, top=2.5cm, bottom=2.5cm, marginparwidth=0pt, headsep=0pt}
\chapter{INTRODUCCIÓN}
\pagenumbering{arabic}
\setcounter{page}{1}
%TODO: problema que se pretende resolver, justificación, hipótesis, objetivos, límites y alcances del trabajo.
\section{Planteamiento del problema}
Las ecuaciones de Euler y Navier-Stokes describen el movimiento de un fluido en $\mathbb{R}^n$ tal que $(n=2\text{ o }3)$. Estas ecuaciones se resuelven por un vector desconocido de velocidad $u(x,t)= \left(u_i(x,t)\right)_{1\leq i \leq n}\in \mathbb{R}^n$ y presión $p(x,t)\in \mathbb{R}$, definido por la posición $x\in \mathbb{R}^n$ y tiempo $t\geq 0$. Se restringe a fluidos incompresibles que están en $\mathbb{R}^n$, dichas ecuaciones son:
\begin{align}
    &\frac{\partial}{\partial t} u_i + \sum_{j = 1}^n u_j \frac{\partial u_i}{\partial x_j} = \nu \Delta u_i - \frac{\partial p}{\partial x_i} + f_i(x,t)&& \left(x\in \mathbb{R}^n, t\geq 0\right),\label{eq1}\\
    &\text{div}\, u = \sum_{j = 1}^n \frac{\partial u_i}{\partial x_j} && \left(x\in \mathbb{R}^n, t\geq 0\right),\label{eq2}\\
    &\text{Condiciones iniciales:}\quad u(x,0) = u^{\circ}(x) && \left(x\in \mathbb{R}^n\right)\label{eq3}
\end{align}
\begin{notation}
De las ecuaciones (\ref{eq1}), (\ref{eq2}) y (\ref{eq3}) con $\nu=0$:
    \begin{itemize}
        \item $u^{\circ}(x)$ está dado
        \item $C^{\infty}$  es un campo vectorial libre de divergencia en $\mathbb{R}^n$,
        \item $f_i(x,t)$ son los componentes de una fuerza externa aplicada (ej. gravedad),
        \item La viscosidad $\nu$, es un coeficiente positivo
        \item $\Delta = \sum_{i=1}^n \frac{\partial^2}{\partial x_i^2}$ es el Laplaciano en espacios variables.
    \end{itemize}    
\end{notation}
\subsection{Descripción de las ecuaciones de Navier-Stokes}
La ecuación (\ref{eq1}), se basa en la Segunda Ley de Newton $f=m\cdot a$ para un fluido sujeto a una fuerza externa $f= \left(f_i(x,t)\right)_{1\leq i\leq n}$ y a las fuerzas que surgen de la presión y la fricción.

La ecuación (\ref{eq2}), define que el fluido es incompresible.  $u(x, t)$ no debe crecer a medida que $\left\lvert x\right\rvert \to \infty$. Por lo tanto, se restringirá la atención en las fuerzas $f$ y las condiciones iniciales $u^{\circ}$ que satisfacen las ecuaciones (\ref{eq4}) y (\ref{eq5})
\begin{align}
    &\left\lvert \partial_x^{\alpha}u^{\circ}(x) \right\rvert\leq C_{\alpha K}\left(1 + \left\lvert x \right\rvert  \right)^{-K}\in \mathbb{R}^n&& \forall\, \alpha\land K\label{eq4}\\
    &\left\lvert \partial_x^{\alpha}\partial_t^{m} f(x,t) \right\rvert\leq C_{\alpha m K}\left(1 + \left\lvert x \right\rvert + t \right)^{-K}\in \mathbb{R}^n\times \left[ 0,\infty \right)&& \forall\, \alpha, m \land K \label{eq5}
\end{align}
Se acepta una solución de (\ref{eq1}), (\ref{eq2}), (\ref{eq3}) como físicamente razonable sólo si satisfacen las ecuaciones (\ref{eq6}) y (\ref{eq7})
\begin{align}
    &p,u\in C^{\infty}\left( \mathbb{R}^n \times \left[ 0, \infty\right) \right)\label{eq6}\\
    & \int_{\mathbb{R}^n} \left\lvert u(x,t)\right\rvert^2\, dx < C\quad \forall t\geq 0\label{eq7}  
\end{align}
Alternativamente, para descartar problemas en el infinito, se buscan espacios espacialmente soluciones periódicas de (\ref{eq1}), (\ref{eq2}), (\ref{eq3}). Por lo tanto, asumimos que $u^{\circ}(x), f(x,t)$ satisfacen
\begin{align}
    u^{\circ}(x + e_j) = u^{\circ}(x),\quad f(x + e_j,t) = f(x,t),\quad p(x + e_j,t) = p(x,t) && \forall\, 1\leq j\leq n\label{eq8}
\end{align}
\begin{notation} De la ecuación (\ref{eq8})
    \begin{itemize}
        \item $e_j = j^{th}$
    \end{itemize}
\end{notation}
En lugar de (\ref{eq4}) y (\ref{eq5}), se supone que $u^{\circ}$ es suave y que se cumple la ecuación (\ref{eq9})
\begin{align}
    &\left\lvert \partial_x^{\alpha}\partial_t^{m} f(x,t) \right\rvert\leq C_{\alpha m K}\left(1 + \left\lvert t \right\rvert \right)^{-K}\in \mathbb{R}^3\times \left[ 0,\infty \right)&& \forall\, \alpha, m \land K 
    \label{eq9}
\end{align}
Se acepta una solución de (\ref{eq1}), (\ref{eq2}), (\ref{eq3}), físicamente satisfacen las ecuaciones (\ref{eq10}) y (\ref{eq11})
\begin{align}
    &u(x + t) = u(x + e_j,t)\in \mathbb{R}^3 \times  \left[0,\infty \right)&& \forall\, 1\leq j\leq n\label{eq10}\\
    & p,u\in C^{\infty}\left(\mathbb{R}^n \times \left[0,\infty \right) \right)
    \label{eq11}
\end{align}
Desarrollando la ecuación (\ref{eq10}), se tiene:
\begin{align*}
&- \iint_{\mathbb{R}^n \times \mathbb{R}} u \cdot \Delta \frac{\partial \theta}{\partial t} \, dx\, dt - \sum_{ij} \iint_{\mathbb{R}^n \times \mathbb{R}} u_i u_j \frac{\partial \theta_i}{\partial x_j}\,dx\,dt\\
&= \nu \iint_{\mathbb{R}^n \times \mathbb{R}} u \cdot  \Delta \theta\, dx\,dt + \iint_{\mathbb{R}^n \times \mathbb{R}} f \cdot \theta \, dx\, dt + \iint_{\mathbb{R}^n \times \mathbb{R}} p \cdot (\text{div}\theta)\,dx\,dt
\end{align*}
\subsection{Demostraciones sin resolver}
Se solicita una demostración de una de las siguientes cuatro afirmaciones.
\begin{enumerate}
    \item \textbf{Existencia y suavidad de las soluciones de Navier-Stokes en $\mathbb{R}^3$.}
    \begin{enumerate}
        \item Con $\nu>0$ y $n=3$, sea $u^{\circ}(x)$ cualquier campo vectorial suave y libre de divergencias que satisfaga la ecuación (\ref{eq4})
        \item Se toma $f(x,t)=0$, entonces existe una función suave $p(x,t),u_i(x,t)$ en $\mathbb{R}^3\times [0,\infty)$ que satisfagan las ecuaciones (\ref{eq1}),(\ref{eq2}),(\ref{eq3}),(\ref{eq6}),(\ref{eq7})
    \end{enumerate}
    \item \textbf{Existencia y fluidez de soluciones Navier-Stokes en $\mathbb{R}^3/\mathbb{Z}^3$} \begin{enumerate}
        \item Con $\nu > 0$ y $n = 3$. Sea $u^{\circ}(x)$ cualquier campo vectorial suave y libre de divergencia que satisfaga (\ref{eq8});
        \item Se toma $f(x, t)=0$. Entonces existen funciones suaves $p(x, t)$, $u_i(x, t)$ en $R^3\times [0,\infty)$ que satisfacen (\ref{eq1}),(\ref{eq2}),(\ref{eq3}),(\ref{eq10}),(\ref{eq11})
    \end{enumerate}
    \item \textbf{Desglose de las soluciones Navier-Stokes en $\mathbb{R}^3$} \begin{enumerate}
        \item  Con $\nu > 0$ y $n = 3$. Entonces existe un campo vectorial suave y libre de divergencia $u^{\circ}(x)$ en $\mathbb{R}^3$
        \item y existe un $f(x, t)$ suave en $\mathbb{R}^3 \times [0,\infty)$, que satisface (\ref{eq4}), (\ref{eq5}), para el cual no existen soluciones $(p, u)$ de (\ref{eq1}),(\ref{eq2}),(\ref{eq3}),(\ref{eq6}),(\ref{eq7}) en $\mathbb{R}^3\times [0,\infty)$
    \end{enumerate}
    \item \textbf{Desglose de las soluciones Navier-Stokes en $\mathbb{R}^3/\mathbb{Z}^3$.} \begin{enumerate}
        \item Con $\nu > 0$ y $n = 3$. Entonces existe un campo vectorial suave y libre de divergencia $u^{\circ}(x)$ en $\mathbb{R}^3$
        \item y existe un $f(x, t)$ suave en $\mathbb{R}^3 \times [0,\infty)$, que satisface (\ref{eq8}), (\ref{eq9}), para el cual no existen soluciones $(p, u)$ de (\ref{eq1}),(\ref{eq2}),(\ref{eq3}),(\ref{eq10}),(\ref{eq11}) en $\mathbb{R}^3 \times [0,\infty)$.
    \end{enumerate}
\end{enumerate}
\subsection{Resultados parciales conocidos sobre Euler y Navier- Stokes}
\subsubsection{Dos dimensiones}
\textbf{Resuelto.} En el capítulo (???) se demuestra para los análogos de las afirmaciones (1) y (2) son conocidos (Ladyzhenskaya \cite{ladyzhenskaya1969mathematical}), también para el caso más difícil de las ecuaciones de Euler. 
\subsubsection{Tres dimensiones}
\textbf{Sin resolver.} En tres dimensiones, se sabe que (1) y (2) se mantienen siempre que la velocidad inicial $u^{\circ}$ satisfaga una condición de pequeñez. Para los datos iniciales $u^{\circ}(x)$ que no se supone que sean pequeños, se sabe que (1) y (2) se cumplen (también para $\nu=0$) si el intervalo de tiempo $[0,\infty)$ se reemplaza por un intervalo de tiempo pequeño $\left[0, T\right)$, dependiendo $T$ de los datos iniciales.

Para un $u^{\circ}(x)$ inicial dado, el $T$ máximo permitido se denomina ``tiempo de explosión''. (1) y (2) se cumplen, o bien hay un $u^{\circ}(x)$ suave y sin divergencia para el cual (\ref{eq1}), (\ref{eq2}), (\ref{eq3}) tienen una solución con un tiempo de explosión finito. Para las ecuaciones de Navier-Stokes $(\nu > 0)$, si hay una solución con un tiempo de explosión finito $T$, entonces la velocidad $\left(u_i(x,t)\right)_{1\leq i \leq n}\in \mathbb{R}^n$ se vuelve ilimitado cerca del momento de la explosión.

Se sabe que suceden otras cosas desagradables en el momento de explosión $T$, si $T < \infty$. Para las ecuaciones de Euler ($\nu = 0$), si hay una solución (con $f \equiv  0$) con un tiempo de explosión finito $T$, entonces la vorticidad $\omega(x, t) = curl_x u(x, t)$ satisface
\begin{align}
    &\text{Baele-Katp-Majda}&&\int_0^T \left\{ \sup_{x\in \mathbb{R}^3}\left\lvert \omega(x,t)\right\rvert\right\}\, dt = \infty
\end{align}
para que la vorticidad explote rápidamente.

Muchos cálculos numéricos parecen mostrar una explosión en las soluciones de las ecuaciones de Euler, pero la extrema inestabilidad numérica de las ecuaciones hace que sea muy difícil sacar conclusiones confiables.

Los resultados anteriores están muy bien tratados en el libro de Bertozzi y Majda \cite{majda2002vorticity}. A partir de Leray \cite{leray1934mouvement}, se han logrado importantes avances en la comprensión de las soluciones débiles de las ecuaciones de Navier-Stokes. Para llegar a la idea de una \texttt{solución débil} de las ecuaciones de Navier-Stokes.

Para llegar a la idea de una solución débil en una Ecuación Diferencial Parcial (EDP), se integra la ecuación con una función de prueba y luego se integra por partes (formalmente) para hacer que las derivadas caigan en la función de prueba. Por ejemplo, si (\ref{eq1}) y (\ref{eq2}) se cumplen, entonces, para cualquier campo vectorial suave $\theta(x, t) = \left(\theta_i(x, t)\right)_{1\leq i\leq n}$ soportado de forma compacta en $\mathbb{R}^3 \times (0,\infty)$ , una integración formal por partes produce
\begin{equation}
\begin{split}
    &\iint_{\mathbb{R}^3\times \mathbb{R}}u \frac{\partial \theta}{\partial t}\,dx\,dt - \sum_{ij} \iint_{\mathbb{R}^r\times \mathbb{R}}u_i u_j \frac{\partial \theta_i}{\partial x_j} \,dx\,dt\\
    &=\nu \iint_{\mathbb{R}^r\times \mathbb{R}} u \cdot \Delta \theta\,dx\,dt + \iint_{\mathbb{R}^3\times \mathbb{R}} f \cdot \theta \,dx\,dt - \iint_{R^3\times \mathbb{R}}p \cdot \left(div\theta\right)\,dx\,dt
    \label{eq12}
\end{split}
\end{equation}
La ecuación (12) tiene sentido para $u\in L^2, f\in L^1, p\in L^1$, mientras que (\ref{eq1}) tiene sentido sólo si $u(x, t)$ es dos veces diferenciable en x. De manera similar, si $\varphi (x, t)$ es una función suave, soportada de manera compacta en $\mathbb{R}^3 \times (0,\infty)$, entonces una integración formal por partes y (\ref{eq2}) implica:
\begin{equation}
    \iint_{\mathbb{R}^3\times \mathbb{R}} u \cdot \nabla_x \varphi \,dx\,dt = 0
    \label{eq13}
\end{equation}
Una solución de (\ref{eq12}), (\ref{eq13}) se llama solución débil de las ecuaciones de Navier Stokes.

Una idea establecida desde hace mucho tiempo en el análisis es demostrar la existencia y regularidad de las soluciones de una EDP construyendo primero una solución débil y luego demostrando que cualquier solución débil es suave. Este programa se ha probado para Navier-Stokes con éxito parcial.

Leray en \cite{leray1934mouvement} demostró que las ecuaciones de Navier-Stokes (\ref{eq1}), (\ref{eq2}), (\ref{eq3}) en tres dimensiones espaciales siempre tienen una solución débil $(p, u)$ con propiedades de crecimiento adecuadas. Se desconoce la unicidad de las soluciones débiles de la ecuación de Navier-Stokes. Para la ecuación de Euler, la unicidad de las soluciones débiles es sorprendentemente falsa. Scheffer \cite{scheffer1993inviscid} y, más tarde, Schnirelman \cite{shnirelman1997nonuniqueness} exhibieron soluciones débiles de las ecuaciones de Euler en $\mathbb{R}^2 \times \mathbb{R}$ con soporte compacto en el espacio-tiempo. Esto corresponde a un fluido que parte del reposo en el instante $t = 0$, comienza a moverse en el instante $t = 1$ sin estímulo externo y vuelve al reposo en el instante $t = 2$, con su movimiento siempre confinado a una bola $B \subset  \mathbb{R}^3$.

Scheffer \cite{scheffer2006turbulence} aplicó ideas de la teoría de la medida geométrica para demostrar un teorema de regularidad parcial para soluciones débiles adecuadas de las ecuaciones de Navier-Stokes.

Caffarelli-Kohn-Nirenberg \cite{caffarelli1982partial} mejoraron los resultados de Scheffer y F.-H. Lin \cite{lin1998new} simplificó las pruebas de los resultados en Caffarelli-Kohn-Nirenberg \cite{caffarelli1982partial}. El teorema de regularidad parcial de \cite{caffarelli1982partial}, \cite{lin1998new} se refiere a un análogo parabólico de la dimensión de Hausdorff del conjunto singular de una solución débil adecuada de Navier-Stokes. Aquí, el conjunto singular de una solución débil u consta de todos los puntos $(x^{\circ}, t^{\circ}) \in \mathbb{R}^3 \times \mathbb{R}$ tales que $u$ es ilimitado en todas las vecindades de $(x^{\circ}, t^{\circ})$. (Si la fuerza $f$ es suave, y si $(x^{\circ}, t^{\circ})$ no pertenece al conjunto singular, entonces no es difícil demostrar que u puede corregirse en un conjunto de medida cero para volverse suave en una vecindad de $(x^{\circ}, t^{\circ})$.)

Para definir el análogo parabólico de la dimensión de Hausdorff, utilizamos cilindros parabólicos $Q_r = B_r \times I_r \subset \mathbb{R}^3 \times \mathbb{R}$, donde $B_r \subset \mathbb{R}^3$ es una bola de radio $r$, e $I_r \subset \mathbb{R}$ es un intervalo de longitud $r^2$. Dado $E \subset \mathbb{R} \times \mathbb{R}$ y $\delta  > 0$, establecemos
\begin{equation*}
    \mathcal{P}_{K,\delta}(E) = inf \left\{ \sum_{i = 1}^{\infty} r_{i}^{K}:Q_{r1},Q_{r2},\dots\text{Cubre }E, \text{ y cada }r_i<\delta \right\} 
\end{equation*}
y luego definir
\begin{equation*}
    \mathcal{P}_K(E) = \lim_{\delta \to 0+} \mathcal{P}_{K,\delta}(E)
\end{equation*}
Los principales resultados de \cite{caffarelli1982partial}, \cite{lin1998new} pueden expresarse aproximadamente de la siguiente manera:
\begin{theorem}
    Regularidad Parcial de Caffarelli-Kohn-Nirenberg y F.-H. Lin
    \begin{enumerate}
        \item (A) Sea u una solución débil de las ecuaciones de Navier-Stokes, que satisfaga condiciones de crecimiento adecuadas. Sea $E$ el conjunto singular de $u$. Entonces $\mathcal{P}_1(E)=0$.
        \item (B) Dado un campo vectorial libre de divergencia $u^{\circ}(x)$ y una fuerza $f(x, t)$ que satisface (\ref{eq4}) y (\ref{eq5}), existe una solución débil de Navier-Stokes (\ref{eq1}), (\ref{eq2}), (\ref{eq3}) que satisfacen las condiciones de crecimiento en (A).
    \end{enumerate}
\end{theorem}
En particular, el conjunto singular de $u$ no puede contener una curva espacio-temporal de la forma ${(x, t) \in \mathbb{R}^3 \times \mathbb{R}: x = \phi (t)}$. Este es el mejor teorema de regularidad parcial conocido hasta ahora para la ecuación de Navier-Stokes.

\section{Estado del arte}
\subsection{Una formulación cuantitativa del problema de regularidad global para el periódico Sistema Navier-Stokes, Terence Tao}



















% \subsection{Línea del tiempo}
% \begin{enumerate}
%     \item Navier-Stokes propone la ecuación en 1888
%     \item Leray resuelve la regularidad global de las ecuaciones en dos dimensiones en 1933
%     \item Kolmogorov
%     \item Leray: Soluciones débiles
%     \item Ladyzhenskaya: Soluciones en 2d
%     \item Sobolev: Espacios de sobolev
%     \item Shwartz
%     \item Terence Tao (ya)
% \end{enumerate}

% La gran incógnita es la turbulencia. Para escalas finas en tres dimensiones es mucho más no lineal.
% NO EXISTE UNA EXPLICACIÓN MATEMÁTICA FORMAL DE CÓMO SE PASA DE UN FLUJO REGULAR A UN FLUJO TURBULENTO



% EL TRABAJO MÁS NUEVO. PUEDE SER EL DE LARENCE TAO


\section{Contribuciones de este trabajo}
% COLOCAR QUÉ APORTÓ LA CHAMBA AL MUNDO


\section{Esquema de la tesis}
% EXPLICAR CÓMO SE LEE LA TESIS
% X = NÚMERO DE CAPITULOS DE LA TESIS
Este trabajo está estructurado como sigue. La introducción va seguida de X capítulos independientes que se han ordenado de forma coherente con el proceso de desarrollo de las ecuaciones de Navier-Stokes. En gran medida, la notación es consistente a lo largo de esta tesis y cada excepción está claramente resaltada. Para facilitar la navegación, se incluyen apéndices al final de sus respectivos capítulos, mientras que la bibliografía acumulativa se adjunta al final de este documento. Los capítulos siguientes se resumen brevemente a continuación.

\textbf{Capítulo 2} Consiste en...


\textbf{Capítulo 3} Consiste en...


\textbf{Capítulo 4} Consiste en...

