% \chapter{HIPÓTESIS}
% % Disponer de más agua, puede ser un incentivo para el desarrollo de sistemas hidropónicos y de agricultura vertical 

% % Hablar sobre la calidad del agua de lluvia y su tratamiento. saber si sirve pa regar
% \chapter{JUSTIFICACIÓN}

\chapter{REVISIÓN DE LITERATURA}

\section{Ecuación de Nevier-Stokes 2d, Ladyzhenskaya}
Hay dos direcciones principales en la vida científica del prof. Ladyzhenskaya. La primera: la existencia, unicidad y regularidad de las soluciones de las ecuaciones de Navier-Stokes. La segunda: teoría de la regularidad para ecuaciones elípticas y parabólicas no lineales.

En 1951, Olga A. Ladyzhenskaya demostró la segunda desigualdad fundamental para operadores elípticos L de segundo orden con coeficientes suaves y para cualquier condición de frontera homogénea clásica.
\begin{equation}
    \left\lVert u \right\rVert W_2^2 (\Omega)\leq c_{\Omega} \left( \left\lVert Lu\right\rVert_{L_2(\Omega)} +\left\lVert u\right\rVert_{L_2(\Omega)} \right)
\end{equation}
que es válido para cualquier $u\in W_2^2(\Omega)$.

En cuanto a la primera dirección, en 1958 en [3] Olga A. Ladyzhenskaya demostró la desigualdad multiplicativa
\begin{equation}
    \left\lVert u \right\rVert_{L_4(\Omega)}^4\leq c \left\lVert u\right\rVert_{L_2(\Omega)}^2\left\lVert \nabla u\right\rVert_{L_2(\Omega)}^2  
\end{equation}
que es válido para cualquier $u\in W_2^1(\Omega),\, \Omega\in \mathbb{R}^2$.

Esta desigualdad dio la posibilidad de demostrar la existencia de una solución única global. del sistema bidimensional Navier-Stokes