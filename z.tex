% los antecedentes relevantes al estudio en orden histórico
% \section{Conceptos básicos}
% Las ecuaciones de Naver-Stokes pretenden modelar la evolución de estas cantidades a partir de:
% \begin{itemize}
%     \item La segunda Ley de Newton
%     \item Ley de conservación de la masa y energía
% \end{itemize}
% \subsection{Fuerzas}
% Se consideran fuerzas en fluidos:
% \begin{itemize}
%     \item Variaciones espaciales de presión
%     \item Fuerzas de rozamiento entre moléculas
%     \item Viscosidad
%     \item Fuerza de gravedad
%     \item Fuerzas externas
% \end{itemize}



% \section{Enfoques}
% \subsection{Enfoque Euleriano}
% En el presente estudio se usará éste enfoque,
% \begin{equation}
%     u(x,t) = \left(u_1(x,t),u_2(x,t),u_3(x,t)\right)
% \end{equation}
% La conservación de momento, está dada por:
% \begin{equation}
%     \rho \left(u_t + (u \cdot \nabla)u\right) = -\nabla_{p} + \mu \Delta u + f_e
% \end{equation}
% Esta ecuación se resuelve con la ecuación de continuidad, \textbf{Conservación de masa}
% \begin{equation}
% \rho_t + u\cdot \nabla_{\rho} = 0
% \end{equation}
% \begin{equation}
%     \text{Incompresibilidad}\quad \nabla \cdot u = 0
% \end{equation}


% \subsection{Enfoque Lagrangiano}
% \begin{equation}
%     x = x(a,t)
% \end{equation}
% Es la trayectoria de la partícula que está en la posición del tiempo $t=0$

% Si se analiza el cambio una función $q$ según la trayectora, se calcula con la derivada material:
% \begin{equation}
%     D_tq = q_t + u \cdot \nabla_q
% \end{equation}
% Por la segunda ley de Newton, $D_t(\rho u)=$Fuerza, aunado a la conservación de masa y la incompresibilidad:
% \begin{equation}
%     D_t(\rho) = 0
% \end{equation}

% conservación del momento conservación de la masa
% Se investigó cada parte de la ecuación de Navier-Stokes, su evolución para dos y tres dimensiones, y los últimos trabajos para (tema de tesis) en la Agricultura Vertical


% campo de velocidad de un fluido viscoso incompresible
% flujo de impulso en flujo espacialmente no uniforme





