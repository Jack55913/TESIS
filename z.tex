% 3 REVISIÓN DE LITERATURA .................................................................................4
% 3.1 Evapotranspiración..........................................................................................4
% 3.1.1 Evaporación .............................................................................................4
% 3.1.2 Transpiración............................................................................................4
% 3.1.3 Proceso de Evapotranspiración (ET) ........................................................5
% 3.1.4 Evapotranspiración de referencia (ETo) ...................................................5
% 3.1.5 Método FAO Penman-Monteith para la determinación de la ETo .............6
% 3.2 Machine Learning .......................................................................................... 13
% 3.2.1 Aprendizaje supervisado ........................................................................ 14
% 3.2.2 Aprendizaje no supervisado.................................................................... 15
% 3.2.3 Aprendizaje reforzado............................................................................. 16
% 3.3 Redes Neuronales Artificiales........................................................................ 17
% 3.3.1 La neurona ............................................................................................. 17
% 3.3.2 Redes Neuronales.................................................................................. 18
% 3.3.3 Función de activación............................................................................. 20
% 3.3.4 Arquitectura............................................................................................ 24
% 3.3.5 Algoritmo backpropagation ..................................................................... 27
% 3.4 Python ........................................................................................................... 33
% 3.4.1 Google Colab ......................................................................................... 34
% 3.5 Parámetros estadísticos ................................................................................ 35
% 3.5.1 Coeficiente de correlación ...................................................................... 35
% 3.5.2 Error cuadrático medio ........................................................................... 35









\newpage
\section{Acceso a datos meteorológicos de zonas de montaña en México}
Actualmente, no existen estaciones meteorológicas instaladas en los montes (TODO: CITA, XXXX), sin embargo se encontraron estaciones como en el caso del valle de México, la cuál está ubicada en el Izta-Popo y es propiedad de la UNAM

Así mismo, se pueden encontrar radares meteorológicas como es el caso de las cruces. 

Esto indica que es muy escaza la información

















\newpage
\section{Revisión de estudios previos sobre monitoreo ciudadano meteorológico}

Un artículo publicado en RMetS por Samuel Michael Illingworth et al., titulado “Red de ciudadanos sobre precipitaciones del Reino Unido: un estudio piloto”, describe cómo se utilizó GoogleChart para llevar un registro colaborativo de las precipitaciones.(\cite{illingworth2021ukprecipitation}) 

Por otro lado, el artículo “Enhancing Engagement of Citizen Scientists to Monitor Precipitation Phase” menciona la aplicación Mountain Rain or Snow, una colaboración financiada por la NASA entre Lynker, Desert Research Institute y la Universidad de Nevada-Reno. Esta aplicación permite a los usuarios reportar si está lloviendo o nevando en un momento y lugar determinados.(\cite{lute2021enhancing})


En el contexto de África, el artículo “Evaluation of Factors Affecting the Quality of Citizen Science Rainfall Data in Akaki Catchment, Addis Ababa, Ethiopia” aborda los factores que influyen en la calidad de los datos sobre precipitaciones recolectados por científicos ciudadanos.(\cite{tedla2022evaluation}) 

Asimismo, la aplicación iFlood, mencionada en el estudio “Coastal Flooding Generated by Ocean Wave- and Surge-Driven Groundwater Fluctuations on a Sandy Barrier Island”, tiene un enfoque similar, pero está diseñada específicamente para reportar inundaciones.(\cite{elgar2021coastal}) 


Otras iniciativas destacan el uso de la ciencia ciudadana para monitorear la calidad del agua y llenar vacíos de datos para cumplir con los Objetivos de Desarrollo Sostenible de las Naciones Unidas, como se describe en el artículo “Using Citizen Science to Understand River Water Quality While Filling Data Gaps to Meet United Nations Sustainable Development Goal 6 Objectives”.(\cite{mcginn2021using})

En un enfoque relacionado, el desarrollo de aplicaciones móviles para el monitoreo de aguas subterráneas también ha sido promovido como una herramienta para involucrar a la ciencia ciudadana, según se menciona en el estudio “Groundwater Mobile App Development to Engage Citizen Science”.(\cite{dennis2019groundwater})



véase la figura 2 para ubicar sus categorías.















\newpage
\section{Tecnologías actuales en monitoreo climático}
La implementación de herramientas tecnológicas para el monitoreo de fenómenos climáticos ha demostrado ser una estrategia eficiente, especialmente cuando se combina con enfoques de ciencia ciudadana. Este estudio, al fomentar la colaboración comunitaria y el uso de tecnologías accesibles, tiene el potencial de generar información crítica para el manejo de los ecosistemas de montaña.



% \subsection{Pluviómetros con IoT}

\subsection{Estaciones meteorológicas}

Datos satelitales y radar metereológico: No existe la validación ni calibración. Hay información satelital que tiene dos cosas: El tamaño de pixel es muy grande y lo que está estimando() no midiendo, no se sabe qué tanto coincide con lo que hay en superficie lo real porque no hay datos, nosotros estamos generando estos datos.

En un futuro estos sirva para mejorar las estimaciones satelitales

Poner artículo donde pusieron en Canoas Altas, pero lo abandonaron, y no hay pase pública que hayan compartido


\subsection{Aplicaciones móviles}

Entre los avances más destacados está el proyecto Cooperative Open Online Landslide Repository (COOLR), que utiliza las aplicaciones \textbf{Landslide Reporter} y \textbf{Landslide Viewer}. Estas herramientas invitan a científicos ciudadanos de todo el mundo a contribuir con reportes de eventos de deslizamientos de tierra, mejorando la investigación y predicción de desastres.\cite{coolr2021} 

Además, la aplicación \textbf{Sense-it} ofrece un kit de herramientas de sensores para la investigación ciudadana, funcionando como una herramienta educativa en dispositivos Android.\cite{van2017senseit}


Otra categoría importante son los diarios de lluvia, como la aplicación \textbf{Rain Tracker} de Callum Hill, que permite a los usuarios gestionar sus propios datos de precipitaciones, aunque estos no son accesibles al público. \cite{hill2021raintracker}

Aplicaciones similares encontradas en el mercado de aplicaciones a junio de 2025, como \textbf{Pocket Rain Gauge}, \textbf{Rainlogger} y \textbf{Rain Recorder} registran las precipitaciones en función de la ubicación mediante GPS, pero tampoco ofrecen un sistema de registro público de los datos.


\subsection{Importancia hidrológica de las zonas de montaña}











\newpage
\section{Aportes del monitoreo ciudadano a la ciencia climática}



Aporta información para comprender diferentes aspectos: conocer la influencia que tiene la lluvia en los bosques, cultivos y ganado, para prevenir problemas de inundaciones y sequías, y para entender el efecto de procesos como el cambio de uso de suelo sobre el agua disponible. Todo esto es muy importante para los ecosistemas naturales y para las comunidades humanas.














% \section{Programación en Flutter}
