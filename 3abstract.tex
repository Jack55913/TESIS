% From mitthesis package
% Version: 1.01, 2023/06/19
% Documentation: https://ctan.org/pkg/mitthesis
%
% The abstract environment creates all the required headers and footnote. 
% You only need to add the text of the abstract itself.
%
% Approximately 500 words or less; try not to use formulas or special characters
% If you don't want an initial indentation, do \noindent at the start of the abstract
\begin{abstract}
% use \input rather than \include because we're inside an environment

En el contexto de una creciente necesidad por monitorear fenómenos hidrometeorológicos de manera precisa, económica y descentralizada, la ciencia ciudadana se posiciona como una estrategia clave para la recopilación de datos ambientales en tiempo real. Esta tesis presenta el desarrollo de Tláloc App, una aplicación móvil multiplataforma diseñada con Flutter, cuyo objetivo es facilitar el registro, análisis y visualización de datos de precipitación pluvial mediante la participación activa de comunidades locales en el Monte Tláloc, México.

La aplicación incorpora una arquitectura modular orientada a la escalabilidad y facilidad de mantenimiento, integrando tecnologías como Firebase Realtime Database para el almacenamiento de datos en la nube, y despliegue en Google Play Console para garantizar su accesibilidad en dispositivos Android. Además, se implementa un algoritmo especializado para calcular una medición real de precipitación a partir del análisis acumulado de registros y el estado de vaciado del pluviómetro, permitiendo así una mayor fidelidad en la interpretación de los datos.

El diseño técnico contempla interfaces intuitivas para la captura de datos, visualización estadística mediante gráficos interactivos y mecanismos para incentivar la constancia en la toma de mediciones. Los resultados obtenidos durante la fase de pruebas reflejan una alta viabilidad técnica y social de la plataforma, facilitando la generación de bases de datos confiables con bajo costo operativo.

En conclusión, Tláloc App representa un modelo replicable de integración entre tecnología móvil y ciencia participativa, promoviendo la soberanía tecnológica de comunidades rurales ante los retos del cambio climático y ofreciendo una herramienta concreta para el monitoreo ambiental colaborativo en regiones de alta variabilidad climática.



\textbf{Palabras-Clave:} Ciencia ciudadana, monitoreo de lluvia, Aplicaciones móviles.
\end{abstract}

\renewcommand\abstractname{ABSTRACT}
\begin{abstract}

	In the context of an increasing need to monitor hydrometeorological phenomena with precision, affordability, and decentralization, citizen science emerges as a key strategy for real-time environmental data collection. This thesis presents the development of Tláloc App, a cross-platform mobile application built with Flutter, designed to facilitate the recording, analysis, and visualization of rainfall data through the active participation of local communities in Monte Tláloc, Mexico.

The application features a modular architecture focused on scalability and maintainability, integrating technologies such as Firebase Realtime Database for cloud-based data storage and Google Play Console for distribution and accessibility on Android devices. Furthermore, it implements a specialized algorithm to calculate a real precipitation measurement based on the cumulative analysis of records and the emptied status of rain gauges, enhancing data accuracy and interpretability.

The technical design includes intuitive interfaces for data capture, interactive statistical visualizations, and features that encourage consistent data collection practices. Results from field testing demonstrate high technical and social feasibility, enabling the creation of reliable environmental datasets at low operational cost.

In conclusion, Tláloc App offers a replicable model of integration between mobile technology and participatory science, fostering technological empowerment of rural communities in the face of climate change and providing a concrete tool for collaborative environmental monitoring in regions with high climatic variability.

	
\textbf{Key-Words:} Citizen science, Rainfall monitoring, Mobile application
\end{abstract}



















% \begin{abstract}
% En respuesta a la creciente necesidad de contar con métodos precisos, económicos y descentralizados para registrar eventos hidrometeorológicos, esta tesis presenta el desarrollo de *Tláloc App*, una solución tecnológica que habilita la recopilación colaborativa de datos sobre precipitación en comunidades del Monte Tláloc, México.

% La aplicación, construida con Flutter, adopta una arquitectura modular que favorece su escalabilidad y mantenimiento. Integra servicios en la nube mediante Firebase y se distribuye a través de Google Play, lo cual garantiza su accesibilidad. Se incluye un algoritmo para calcular la precipitación efectiva a partir de registros acumulados y el estado del pluviómetro, mejorando así la precisión de los datos recolectados.

% El diseño contempla interfaces amigables para la captura y visualización de información, así como estrategias para fomentar la participación continua de los usuarios. Las pruebas de campo demuestran su viabilidad tanto técnica como social, permitiendo la construcción de bases de datos confiables con mínima inversión.

% Esta herramienta constituye un ejemplo replicable de innovación tecnológica al servicio de la gestión climática comunitaria, fomentando la autonomía digital en territorios rurales vulnerables al cambio climático.
% \vspace{1em}

% \textbf{Palabras clave:} participación comunitaria, precipitación pluvial, sensores, interfaz móvil, gestión climática.
% \end{abstract}