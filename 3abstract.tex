\renewcommand{\abstractheader}{APLICACIÓN MULTIPLATAFORMA BASADA EN CIENCIA CIUDADANA PARA EL MONITOREO DE LLUVIA EN EL MONTE TLÁLOC}

\begin{abstract}  

En el contexto de una creciente necesidad por monitorear fenómenos hidrometeorológicos de manera precisa, económica y descentralizada, la ciencia ciudadana se posiciona como una estrategia clave para la recopilación de datos ambientales en tiempo real. Esta tesis presenta el desarrollo de Tláloc App, una aplicación móvil multiplataforma diseñada con Flutter, cuyo objetivo es facilitar el registro, análisis y visualización de datos de precipitación pluvial mediante la participación de comunidades locales en el Monte Tláloc, México.

La aplicación incorpora una arquitectura modular orientada a la escalabilidad y facilidad de mantenimiento, integrando tecnologías como Firebase Realtime Database para el almacenamiento de datos en la nube, y despliegue en Google Play Console para garantizar su accesibilidad en dispositivos Android. Además, se implementa un algoritmo especializado para calcular una medición real de precipitación a partir del análisis acumulado de registros y el estado de vaciado del pluviómetro, permitiendo así una mayor fidelidad en la interpretación de los datos.

El diseño técnico contempla interfaces intuitivas para la captura de datos, visualización estadística mediante gráficos interactivos y mecanismos para incentivar la constancia en la toma de mediciones. Los resultados obtenidos durante la fase de pruebas reflejan una alta viabilidad técnica y social de la plataforma, facilitando la generación de bases de datos confiables con bajo costo operativo.

En conclusión, Tláloc App representa un modelo replicable de integración entre tecnología móvil y ciencia participativa, promoviendo la soberanía tecnológica de comunidades rurales ante los retos del cambio climático y ofreciendo una herramienta concreta para el monitoreo ambiental colaborativo en regiones de alta variabilidad climática.


\vfill
\textbf{Palabras-Clave:} Ciencia ciudadana, Monitoreo de lluvia, Aplicaciones móviles.
\end{abstract}


\renewcommand{\abstractheader}{A CITIZEN SCIENCE BASED MULTIPLATFORM APPLICATION FOR RAINFALL MONITORING ON MOUNT TLALOC}

\renewcommand\abstractname{ABSTRACT}
\begin{abstract}


In the context of a growing need to monitor hydrometeorological phenomena in an accurate, cost-effective, and decentralized manner, citizen science is positioned as a key strategy for real-time environmental data collection. This thesis presents the development of the Tláloc App, a multiplatform mobile application designed with Flutter, which aims to facilitate the recording, analysis, and visualization of rainfall data through the participation of local communities on Mount Tláloc, Mexico.

The application incorporates a modular architecture aimed at scalability and ease of maintenance, integrating technologies such as Firebase Realtime Database for cloud data storage, and deployment on the Google Play Console to ensure accessibility on Android devices. Additionally, a specialized algorithm is implemented to calculate a real-world rainfall measurement based on the cumulative analysis of records and the rain gauge's emptying status, thus enabling greater accuracy in data interpretation.

The technical design includes intuitive interfaces for data capture, statistical visualization through interactive graphics, and mechanisms to encourage consistency in measurement collection. The results obtained during the testing phase reflect the high technical and social viability of the platform, facilitating the generation of reliable databases with low operating costs.

In conclusion, the Tláloc App represents a replicable model of integration between mobile technology and participatory science, promoting the technological sovereignty of rural communities in the face of the challenges of climate change and offering a concrete tool for collaborative environmental monitoring in regions with high climate variability.

\vfill
\textbf{Key-Words:} Citizen science, Rainfall monitoring, Mobile application
\end{abstract}


